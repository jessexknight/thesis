\section{Research Questions}\label{intro.rqs}
Considering the issue of ``where to draw the line'' with respect to model complexity,
the original aim of this thesis was to offer generalizable insights as to
when different factors of risk heterogeneity matter most.%
\footnote{What is a PhD research proposal if not overambitious?}
However, preliminary work reviewing prior models and implementing a benchmark model
revealed another, equally important issue:
% TODO: (?) is this sudden pivot jarring?
\begin{quote}
  Models with the \textbf{same structure} and using the \textbf{same data}
  could differ further due to differences in
  \textbf{assumptions} made during data analyses and model implementation.
\end{quote}
For example:
survey data may give the numbers of sexual partners reported by respondents in the past month,
but these numbers may or may not be multiplied by 12
to define a ``yearly partnership rate'' (model input),
reflecting implicit assumptions of short \vs long partnership durations, respectively.
For another example:
a model may be stratified by age, but the model may or may not also consider
differences by age in rates of HIV testing, treatment initiation, etc.
Such assumptions therefore reflect another dimension to operationalized ``risk heterogeneity'',
which is especially relevant in applied modelling.
\par
Thus, the overarching research question of this thesis is:
\begin{quote}
  How do modelling assumptions influence outputs of compartmental HIV transmission models?
\end{quote}
where modelling assumptions are conceptualized as:
\begin{itemize}
  \item \textbf{model structure:} which
  stratifications of the population and/or processes are explicitly represented?
  \item \textbf{model parameterization:} which
  data are used as model inputs and calibration targets, and how?
  \item \textbf{model implementation:} which
  equations are used to define the rates of transmission and transition between compartments?
\end{itemize}
and model outputs include:
direct estimates (\eg HIV incidence over time),
counterfactual scenarios (\eg comparison of possible intervention impacts), and
qualitative interpretation of a collection of results.
\par
Of course, comprehensive study of these numerous assumptions and outputs
across epidemic contexts is beyond the scope of a single thesis.
Rather, I sought to explore these aspects mainly in the context of
heterosexual HIV transmission and sex work within Eswatini,
through the following specific aims, each reflecting a chapter of the thesis:
\begin{itemize}
  \item Chapter~\ref{sr} systematically reviews the structure and assumptions used in
  prior compartmental HIV transmission models exploring ART scale-up in Sub-Saharan Africa
  \item Chapter~\ref{model} details the design, parameterization, and calibration of
  a ``benchmark'' compartmental model of heterosexual HIV transmission in Eswatini,
  with specific focus on precise interpretation of input data, and adjustment for potential biases
  \item Chapter~\ref{foi} applies this model to examine
  the impact of different assumptions within the force of infection equation on
  modelled epidemic dynamics and the relative contribution of specific transmission pathways
  \item Chapter~\ref{art} further applies the model to explore
  how differences in who is assumed to be reached by ART scale-up
  can influence model-estimated ART prevention impacts
\end{itemize}
Finally, Chapter~\ref{conc} offers some concluding remarks.
