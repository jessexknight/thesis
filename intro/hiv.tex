\section{HIV}\label{intro.hiv}
An estimated 38~million people are living with HIV globally,
with 1.5~million new infections annually \cite{AIDSinfo}.
As discussed below, HIV often disproportionately affects specific populations
due to intersecting social, economic, and biological vulnerabilities \cite{WHO2016kp,Jin2021}.
Advances in prevention and treatment continue to reduce the rate of new infections and
increase the quality and quantity of life among people living with HIV (PLHIV) \cite{Eisinger2019tk}.
However, application of these tools must remain responsive to
the unique needs of different populations
in order to rapidly and equitably end the epidemic \cite{Eisinger2019tk}.
%===================================================================================================
\subsection{Infection \& Natural History}\label{intro.hiv.tinh}
HIV infection involves three key steps:
entry of the virus into target cells
(mainly cells bearing the CD4 receptor, especially T lympocytes);
reverse transcription and integration of viral RNA into host cell DNA; and
cellular production of HIV proteins to yield mature virions \cite{Maartens2014,Deeks2015}.
Transmission of HIV then requires
direct contact of infected bodily fluids with mucosal tissue, blood, or broken skin,
and transmission risk is mediated by the degree of viral exposure to target cells \cite{Deeks2015}.
Table~\ref{tab:beta} (adapted from \cite{Patel2014}) summarizes
the average probability of transmission for different unprotected exposures,
though transmission risk varies substantially with numerous factors \cite{Boily2009}.
\begin{table}
  \caption{Estimated probability of HIV transmission per 10,000 exposures}
  \label{tab:beta}
  \centering
  \begin{tabular}{lrcl}
  \toprule
  Exposure & Risk\tn{a} & (95\% CI) \\
  \midrule
  Blood transfusion     & 9250 &  (8900,~9610) \\
  Needle sharing        &   63 &    (41,~92)   \\
  Receptive anal sex    &  138 &   (102,~186)  \\
  Insertive anal sex    &   11 &   \d(4,~28)   \\
  Receptive vaginal sex &    8 &   \d(6,~11)   \\
  Insertive vaginal sex &    4 &   \d(1,~14)   \\
  Oral sex              &  --- &     (0,~4)    \\
  Mother-to-child       & 2260 &  (1700,~2900) \\ 
  \bottomrule
\end{tabular}
\floatfoot{\centering
  Adapted from \cite{Patel2014};
  \tnt[a]{per 10,000 exposures}}
\end{table}
\par
Following infection, HIV spreads rapidly through the lymphoid system
and becomes detectible in the blood within 10 days \cite{Deeks2015}.
During the subsequent acute phase ($<$1--2 months),
individuals may experience nonspecific symptoms
as plasma viral loads increase and circulating CD4 T cells decrease
\cite{Cohen2011ahi,Maartens2014,Deeks2015}.
These trends are then reversed by an adaptive immune response,
which temporarily suppresses infection,
marking the transition from acute to chronic phase \cite{Maartens2014,Deeks2015}.
Untreated asymptomatic chronic infection typically lasts multiple years,
as HIV evades clearance and CD4 T cells are progressively depleted \cite{Deeks2015}.
Progressive depletion of CD4 T cells then coincides with increasing risk of other infections
--- \ie immune deficiency, eventually reaching the clinical criteria for AIDS \cite{WHO2016art} ---
and with re-increasing viral load \cite{Maartens2014,Deeks2015}.
HIV infection also increases the risk of other diseases and mortality,
with multiple hypothesized mechanisms~\cite{Phillips2008}.
%===================================================================================================
\subsection{Epidemiology of HIV}\label{intro.hiv.epi}
While global adult HIV prevalence is estimated at 0.7\%,
national estimates range from $<$0.1~to~28\% \cite{AIDSinfo}.
Sub-Saharan Africa (SSA) bears the largest burden, with an estimated
two-thirds of all current infections and over half of all new infections \cite{AIDSinfo}.
Whereas transmission is often concentrated among specific key populations (see below),
widespread transmission beyond key populations is more common in SSA,
especially in East and Southern Africa \cite{Kilmarx2009,AIDSinfo}.
As such, national epidemics have historically been classified as
``concentrated'', ``generalized'', or ``mixed'' based on overall HIV prevalence;
however, the utility of this classification has been questioned because
it fails to reflect local drivers of transmission,
which can/should be used to guide epidemic response \cite{Mishra2012appr,Tanser2014,Boily2015}.
%---------------------------------------------------------------------------------------------------
\subsubsection{Key Populations \& Female Sex Workers}\label{intro.hiv.epi.kp}
Several key populations at highest risk of HIV acquisition have been identified, including
sex workers, men who have sex with men, transgender people, prisoners, and people who inject drugs
\cite{Mathers2008,Beyrer2012,Baral2012,Scorgie2012,Shannon2015,Rubenstein2016,WHO2016kp,Jin2021}.
This list is not exhaustive, and other populations at higher HIV risk --- particularly in SSA ---
include highly mobile populations, young women, and those engaged in transactional sex
\cite{Oldenburg2014,Wamoyi2016,Camlin2019,Cheuk2020,Day2020,Khalifa2022}.
HIV disproportionately affects these populations due to intersections of
behavioural risks, stigma, criminalization, violence, and poverty \cite{WHO2016kp}.
For example, women may enter sex work out of food/economic insecurity \cite{Scorgie2012}.
Then, criminalization of sex work leaves women vulnerable to sexual violence (including by police)
and arrest for carrying condoms, or condom confiscation \cite{Scorgie2012,Shannon2015}.
Moreover, women selling sex may be reluctant to engage in HIV treatment and/or prevention services
due to criminalization and experiences of stigma and discrimination \cite{Lancaster2016,Spyrelis2022}.
\par
The odds of HIV infection among female sex workers (FSW)
\vs women aged 15--49 was estimated as 13.5 globally, using data from 2007-11,
and 12.4 in SSA specifically \cite{Baral2012}.
Higher numbers of sexual partners increases risk directly via more potential exposures,
but also via co-transmission of other sexually transmitted infections (STIs) \cite{Scorgie2012}.
Some clients may also pay more for condomless and/or anal sex,
or perpetrate sexual violence \cite{Scorgie2012,Shannon2015}.
Women, especially young women, also have increased biological susceptibility to HIV
due to physical and immunological genital differences \cite{Yi2013}.
These vulnerabilities can be amplified through multiple structural factors noted above,
as well as patriarchal attitudes~\cite{Scorgie2012,Shannon2015}.
\par
The contribution of transmission among key populations and via sex work
is often suggested to be small in ``generalized'' epidemics \cite{Leclerc2008,Shubber2014}.
However, such conclusions typically rely on
biased household-based face-to-face survey data \cite{Langhaug2010,Lowndes2012},
generic behavioural assumptions \cite{Shubber2014}, and
simplistic methodology which fails to account for onward transmission \cite{Mishra2014mot}.
By contrast, mathematical transmission models can capture the large indirect benefits of
interventions prioritizating key populations \cite{Mishra2014mot,Long2021}
and such models often show that these prioritized interventions are most cost effective
\cite{Stuart2018,Maheu-Giroux2019,Johnson2019}.
\par
Understanding and meeting the unique needs of different key populations is thus
a core pillar of HIV response, which often overlaps with broader equity goals
\cite{Shannon2015,Beyrer2015,Beyrer2016}.
However, data collection and service delivery for key populations typically requires
distinct approaches from the population overall
due to the same factors that underpin vulnerabilities \cite{UNAIDS2010kps,WHO2016kp}.
For example, the sizes of key populations
are likely substantially underestimated in household-based surveys,
but could be estimated via respondent-driven or location-based sampling
\cite{UNAIDS2010kps,Lowndes2012,Abdul-Quader2014}.
Moreover, key populations' needs are rarely homogeneous within or between contexts.
Rather, emerging evidence indicates that differentiated service delivery
--- \ie increasing options and convenience for accessing treatment and prevention services ---
is critical for maximizing coverage and impact \cite{Grimsrud2016,Ehrenkranz2019,Huber2021}.
For example, a recent trial among South African FSW identified benefits of
decentralized HIV treatment access via a mobile van, including:
reduced tavel costs (direct costs, and opportunity costs of time away from work),
fewer administrative barriers (\eg proof of residence), and
confidentiality (with respect to both HIV status and sex work) \cite{Comins2022}.
Community empowerment and direct engagement with community members for service planning
can also improve treatment and prevention outcomes
among FSW and key populations in general~\cite{Atuhaire2021}.
% TODO: (?) STI screening, antenatal care, needle exchange, safe supply, male sex workers ...
%===================================================================================================
\subsection{Epidemic Response}\label{intro.hiv.resp}
HIV infection has been stigmatized since the beginning of the epidemic, and the epidemic response
--- including efforts to raise awareness and funding for HIV research and services ---
is rooted in activism against stigma and towards equity \cite{Merson2008}.
By now, over US\$\,20 billion is available globally each year to fight HIV, with approximately
60\% of funds provided by domestic and private sources (\eg ministries of health), and
40\% by international organizations \cite{UNAIDSFin}, including:
PEPFAR (the President's Emergency Plan for AIDS Relief),
the Global Fund (to Fight AIDS, Tuberculosis and Malaria), and
UNAIDS (Joint United Nations Programme on HIV and AIDS).
% TODO: (?) specific e.g. successes
These funds support numerous prevention and treatment tools \cite{Eisinger2019tk},
which are ideally prioritized according to epidemic drivers and cost effectiveness.
Understanding these drivers and estimating intervention impacts is rarely simple,
but can be supported by implementation science and mathematical modelling
\cite{Shelton2010,Mutevedzi2014,Geffen2018,Baral2019,Schwartz2022}.
%---------------------------------------------------------------------------------------------------
\subsubsection{Prevention}\label{intro.hiv.resp.prev}
The earliest efforts to prevent sexual transmission of HIV focused on
condom distribution and promotion, behaviour change, voluntary counselling and testing (VTC),
and treatment of other sexually transmitted infections (STIs) \cite{Royce1997,Marseille2002}.
Condoms continue to be a key tool in prevention,
including following HIV diagnosis \cite{Tiwari2020}.
By contrast, the effectiveness of population-level behaviour change interventions,
such as to reduce numbers of sexual partners, remains unclear \cite{Gregson2009}.
Current behavioural interventions mainly focus on
education to empower informed decision making \cite{Faust2018}.
In many cases, the feasibility of individual-level behaviour change
is constrained by structural factors, including
economic and political stability, gender equality, and criminalization \cite{Gupta2008}.
The role of structural factors were recognized early on \cite{Parker2000},
and efforts to understand and address these factors continue to inform novel prevention strategies
\cite{Gupta2008,Beyrer2012,McBride2021}.
Voluntary medical male circumcision (VMMC) is also protective \cite{Auvert2005},
and has been recommended in some contexts since 2007 \cite{WHO2020vmmc}.
More recent advancements in antiretroviral therapy (ART), especially multi-drug combinations,
have opened up new avenues of prevention, including
pre- and post-exposure prophylaxis (PEP, PrEP), as well as
``treatment as prevention'' \cite{Hosseinipour2002}.
While treatment of people living with HIV offers many benefits
(see \sref{intro.hiv.resp.treat} below),
prioritizating ART scale-up for preventing HIV
may pull focus and funds away from more cost-effective prevention strategies,
especially strategies which meet the needs of key populations
\cite{Shelton2010,Cohen2012,Baral2019}.
% TODO: (?) add more details
%---------------------------------------------------------------------------------------------------
\subsubsection{Treatment}\label{intro.hiv.resp.treat}
The goal of HIV \emph{cure} remains illusive due to
reverse transcription of viral DNA into host cell genomes and subsequent latency (viral inactivity),
which establishes a persistent reservoir comprising 0.01--1\% of CD4 T cells \cite{Ndungu2019}.
This reservoir is then maintained against clearance mechanisms via
viral replication and clonal expansion of latently infected cells
\cite{Maartens2014,Ndungu2019}.
Therefore, the goal of HIV \emph{treatment} has been to suppress viral replication,
and thereby restore normal immunologic function,
including mitigation of non-infectious disease pathways (\eg chronic inflammation)
\cite{Phillips2008,Deeks2015,Cihlar2016}.
\par
Numerous antiretroviral agents are now available,
which have multiple mechanisms of interrupting viral replication \cite{Maartens2014,WHO2016art}.
Standard therapy includes three agents in combination
--- also known as: combination ART (cART), highly active ART (HAART), or now simply ART ---
to reduce the risk of treatment failure and/or resistance mutations \cite{WHO2016art}.
Since turnover of free virus in blood occurs rapidly
(mean generation time estimated as 2.6 days) \cite{Perelson1996},
ART initially reduces blood viral loads rapidly \cite{Perelson1997,Maartens2014}.
Subsequent reductions are more gradual, and time to ``undetectable'' viral load
(defined here as $<$\,80 viral RNA copies per mL of blood)%
\footnote{Definitions of ``undetectable'' viral load can range from $<$\,50 to $<$\,1000 copies/mL.}
is estimated as median 3.1 [IQR: 2.8,~5.5] months \cite{Mujugira2016}.
Recovery of CD4 T cells is even slower, and complete recovery to baseline levels is rare
\cite{Battegay2006,Maartens2014}.
%---------------------------------------------------------------------------------------------------
\paragraph{Treatment Eligibility \& Benefits}
Initial ART eligibility criteria (especially for resource-constrained settings)
were focused on PLHIV with advanced disease
(defined by CD4 T cell counts per mL of blood and/or clinical staging) \cite{WHO2003art},
for whom ART is most beneficial \cite{Gabillard2013,Maartens2014}.%
\footnote{The cost of ART per person-year continues to vary widely,
  from less than \$100 for generic WHO-recommended first-line regimens
  to approximately \$30,000 for patented combinations in the USA \cite{MSF2016ART}.}
However, studies have since highlighted health benefits of earlier ART initiation
\cite{Cohen2011art,Grinsztejn2014,Lundgren2015,Danel2015},
and WHO has progressively expanded recommended eligibility criteria,
culminating in the 2016 ``treat all'' recommendation
\cite{WHO2003art,WHO2007art,WHO2013art,WHO2016art}.
Expanding eligibility has coincided with growing recognition of
additional benefits of ART for preventing transmission,
as demonstrated in several trials of serodiscordant couples
\cite{Anglemyer2013,Cohen2016,Rodger2019}.
This recognition, along with model-based predictions \cite{Granich2009,Eaton2012},
then motivated several large-scale studies of ``treatment as prevention''
designed to estimate the population-level incidence reduction achievable via ART scale-up
\cite{Havlir2019,Hayes2019,Iwuji2018}.
The results of these trials were mostly inconclusive,
prompting renewed calls to understand and address ``who is left behind'',
and other persistent drivers of transmission \cite{Akullian2017,Eisinger2019tk,Baral2019,Havlir2020}.
Improving ART coverage for individual-level and partnership-level benefits
nevertheless remains a distinct goal \cite{959595}.
%---------------------------------------------------------------------------------------------------
\paragraph{Treatment Cascade}
The HIV treatment cascade is conceptualized as key steps along the pathway to viral suppression.
Although more detailed steps can help identify specific service gaps \cite{Mountain2014exp},
the most basic cascade is defined as 3 steps:
HIV diagnosis, ART initiation, and viral suppression \cite{909090}.
These steps then form the basis of UNAIDS targets:
\mbox{90-90-90} by 2020 \cite{909090} and \mbox{95-95-95} by 2030 \cite{959595},
corresponding to the percentage of people living with HIV who know their HIV status,
of whom, the percentage who are on ART,
of whom, the percentage who have undetectable viral load.
National progress towards these goals is highly prioritized and commonly reported \cite{AIDSinfo},
although differences across risk groups
--- and the potential implications of these differences for ``treatment as prevention'' ---
are increasingly highlighted \cite{Akullian2017,Hakim2018,Green2020}.
