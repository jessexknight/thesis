Human immunodeficiency virus (HIV) is
the causative agent of acquired immune deficiency syndrome (AIDS)
and a leading cause of morbidity and mortality \cite{Maartens2014,GBD2019}.
Mathematical modelling of HIV transmission supports
the policy and programmatic response to the HIV epidemic in numerous ways.
Examples of model applications include
projecting the potential impact of interventions \cite{Eaton2012} and
generating fundamental insights about transmission dynamics \cite{Garnett1993}.
These models aim to mechanistically represent
the populations, behaviours, probabilities, and interventions
involved in transmission \cite{Garnett2011}.
Yet, mathematical models must make simplifying assumptions
for tractability of design, implementation, and interpretation,
and given the often limited quantity and quality of data \cite{Mishra2016,Garnett2011}.
Previous work has shown that certain modelling assumptions can influence
model-based answers to particular research questions
\cite{Garnett1993,Hontelez2013,Mishra2016,Johnson2016mf,Bernard2017,Knight2020},
with possible implications for how HIV resources are prioritized using model-based evidence.
\par
In is generally impossible to rank modelling assumptions
from ``most important'' to ``least important'' overall,
across the broad range of contexts and applications.
However, rigorous modelling analysis will include
a context/application-specific examination of assumptions,
which may offer generalizable insights \cite{Suen2017}.
To this end, this thesis explores
the epidemiological drivers of heterosexual HIV transmission in Eswatini,
and examines modelling assumptions related to
sexual partnership dynamics and unmet HIV prevention needs within sex work.
The focus on Eswatini is motivated by both high HIV prevalence
and recent examplary scale-up of universal HIV treatment.
\par
The remainder of this chapter introduces key concepts in HIV, including:
infection, epidemiology, treatment, and prevention;
followed by key concepts in mathematical modelling of HIV transmission, including:
types of models, applications, fundamental principles, and challenges in chosing assumptions.
\pagebreak % TEMP
