Human immunodeficiency virus (HIV) is
the causative agent of acquired immune deficiency syndrome (AIDS)
and a leading cause of morbidity and mortality \cite{Maartens2014,GBD2019}.
Mathematical modelling of HIV transmission supports epidemic response in numerous ways, such as
projecting the impact of interventions \cite{Eaton2012}, and
generating fundamental insights about transmission dynamics \cite{Garnett1993}.
These models mechanistically represent the populations, behaviours, probabilities, and interventions
involved in transmission \cite{Garnett2011}.
However, such models must make simplifying assumptions in order to remain tractable,
and given limited quantity and quality of data \cite{Mishra2016,Garnett2011}.
Previous work has shown that \emph{which} assumptions are used can influence model ouptuts
\cite{Garnett1993,Hontelez2013,Mishra2016,Johnson2016mf,Bernard2017,Knight2020}.
Yet, the potential influence of many modelling assumptions remain unexplored.
Therefore, this thesis aims to re-examine several key modelling assumptions
and explore their potential influence on model outputs.
\par
The remainder of this chapter introduces key concepts in HIV, including:
infection, epidemiology, treatment, and prevention;
plus key concepts in mathematical modelling of HIV transmission, including:
types of models, applications, fundamental principles, and challenges in chosing assumptions.
Special attention is given to
the HIV epidemic in Eswatini and among female sex workers specifically,
which are major foci in the thesis.
