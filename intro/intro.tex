Human immunodeficiency virus (HIV) is
the causative agent of acquired immune deficiency syndrome (AIDS)
and a leading cause of morbidity and mortality \cite{Maartens2014,GBD2019}.
Mathematical modelling of HIV transmission supports the policy and programmatic response to the HIV epidemic in numerous ways. 
Examples of the use of modelling include 
projecting the potential impact of interventions \cite{Eaton2012} and
generating fundamental insights about transmission dynamics \cite{Garnett1993}.
These models aim to mechanistically represent an abstraction of
the populations, behaviours, probabilities, and interventions
involved in transmission \cite{Garnett2011} in order to answer specific research questions.
These questions brodaly cover predictions (forecasting and estimations) and "what if" experiments via comparison of counterfactual scenarions (such as in the case of intervention impact).
Thus, mathematical models must make simplifying assumptions for tractability, %SM: tractability of what? (question reviewer/examiner may ask)
and given the often limited quantity and quality of data \cite{Mishra2016,Garnett2011}. %SM: is it only b/c of limitations on data and tractability? are there other reasons why math models are *always* a simplification?
Previous work has shown that certain modelling assumptions can influence
model-based answers to particular research questions
\cite{Garnett1993,Hontelez2013,Mishra2016,Johnson2016mf,Bernard2017,Knight2020},
with possible implications for how HIV resources are prioritized using model-based evidence.
\par
The broad range of modelling assumptions and applications
makes it impossible rank assumptions by influence in general,
but application-specific examination of assumptions
is a key step in rigorous modelling analysis
and may offer generalizable insights \cite{Suen2017}. %SM: rephrase sentence for more clarity
To this end, this thesis explores the epidemiological drivers of heterosexual HIV transmission in Eswatini, %SM: we generally avoid the term "drivers" d/t stigma (e.g. FSW driving transmission) unless expliciltly clarify what we mean by "drivers"...
and examines modelling assumptions related to
sexual partnership dynamics and unmet HIV prevention needs within sex work. % nice
I focus on Eswatini as the motivating setting becuase ... %SM: say why focusing on Eswatini as motivating setting. (remove the line about this could be anywhere b/c this early on - have not yet set up the stage for that?]
\par
The remainder of this chapter introduces key concepts in HIV, including:
infection %SM - why infection and not transmission?
, epidemiology, treatment, and prevention;
followed by key concepts in mathematical modelling of HIV transmission, including:
types of models, applications, fundamental principles, and challenges in chosing assumptions.
Special attention is given to the HIV epidemic in Eswatini and among female sex workers. %SM: sentence not needed
