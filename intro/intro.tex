Human immunodeficiency virus (HIV) is
the causative agent of acquired immune deficiency syndrome (AIDS)
and a leading cause of morbidity and mortality \cite{Maartens2014,GBD2019}.
Mathematical modelling of HIV transmission supports epidemic response in numerous ways, such as
projecting the impact of interventions \cite{Eaton2012} and
generating fundamental insights about transmission dynamics \cite{Garnett1993}.
These models aim to mechanistically represent
the populations, behaviours, probabilities, and interventions
involved in transmission \cite{Garnett2011}.
Such models must make simplifying assumptions in order to remain tractable,
and given the often limited quantity and quality of data \cite{Mishra2016,Garnett2011}.
Yet, previous work has shown that certain modelling assumptions can influence
model-based answers to particular research questions
\cite{Garnett1993,Hontelez2013,Mishra2016,Johnson2016mf,Bernard2017,Knight2020},
with possible implications for how HIV resources are prioritized using model-based evidence.
\par
The broad range of modelling assumptions and applications
makes it impossible rank assumptions by influence in general,
but application-specific examination of assumptions
is a key step in rigorous modelling analysis
and may offer generalizable insights \cite{Suen2017}.
To this end, this thesis explores drivers of heterosexual HIV transmission in Eswatini,
and examines modelling assumptions related to
sexual partnership dynamics and unmet HIV prevention needs within sex work.
Such assumptions and related findings are likely relevant to multiple epidemic contexts,
particularly across East and Southern Africa.
\par
The remainder of this chapter introduces key concepts in HIV, including:
infection, epidemiology, treatment, and prevention;
plus key concepts in mathematical modelling of HIV transmission, including:
types of models, applications, fundamental principles, and challenges in chosing assumptions.
Special attention is given to the HIV epidemic in Eswatini and among female sex workers.
