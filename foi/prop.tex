\clearpage % TEMP
\section{Proposed Model}\label{foi.prop}
In this section, I propose a new approach to
modelling HIV transmission within sexual partnerships in compartmental models.
The approach overcomes the limitations of prior approaches described in \sref{foi.prior.lims},
without the need to change modelling frameworks.
% ==================================================================================================
\subsection{Illustrative Toy Scenario}\label{foi.prop.toy}
Consider the moment of one transmission event
in a population of 16 monogamous partnerships, with 25\% infection prevalence
and random mixing by infection status (Figure~\ref{fig:foi.toy.pair}).
Initially, infection prevalence is equal among partners of
susceptible $S$ and infectious $I$ individuals: 6/24 and 2/8, respectively.
Immediately after transmission,
prevalence decreases to 5/23 among partners of $S$ but increases to 4/9 among partners of $I$,
decreasing the population-level transmission risk.
Next, three events are possible:
\begin{enumerate}[label=(\alph*)]
  \item \label{foi.toy.e.t}
  another transmission occurs among the remaining $S$-$I$ partnerships,
  yielding 4/22 prevalence among partners of $S$, and 6/10 among partners of $I$;
  population-level transmission risk decreases further
  \item \label{foi.toy.e.p}
  the partnership from the original transmission ends,
  and both individuals form new partnerships (assumed at random),
  yielding, on average, 9/32 prevalence among partners of both $S$ and $I$;
  population-level transmission risk increases \emph{above} the initial level ($9/32 > 6/24$)
  \item \label{foi.toy.e.q}
  any other partnership ends,
  and both individuals form new partnerships (assumed at random);
  infection prevalence among $S$ and $I$, and population-level transmission risk
  all remain unchanged, on average
\end{enumerate}
Prior compartmental models have effectively assumed that
event \ref{foi.toy.e.p} always occurs before \ref{foi.toy.e.t}
--- \ie the ``instantaneous partnership assumption''.
This assumption is reflected in Figure~\ref{fig:foi.toy.freq},
where the frequentist approximation does not explicitly model any individual partnerships.
Evidently, this assumption is worse for longer partnerships.
\begin{figure}
  \begin{subfigure}{.5\linewidth}
    \centering\includegraphics[scale=1]{foi.toy.pair}
    \caption{Pair-wise reality}
    \label{fig:foi.toy.pair}
  \end{subfigure}%
  \begin{subfigure}{.5\linewidth}
    \centering\includegraphics[scale=1]{foi.toy.freq}
    \caption{Frequentist approximation}
    \label{fig:foi.toy.freq}
  \end{subfigure}
  \caption{Comparison of pair-based reality and frequentist approximation
    for a population of 16 pairs with 25\% infection prevalence,
    at the moment of one transmission event}
  \label{fig:foi.toy}
  \floatfoot{
    $S$: susceptible; $I$: infectious; $\lambda$: force of infection;
    dashed circles: individuals involved in transmission event.}
\end{figure}
% ==================================================================================================
\subsection{Conceptual Development}\label{foi.prop.concept}
The toy scenario highlights how any partnership where transmission has occured
should be ``removed'' from the force of infection.
In a compartmental (non-pair-based) model,
these partnerships can be tracked as proportions of individuals: namely,
all individuals who recently acquired infection \emph{and}
all individuals who recently transmitted infection.
Here, I use ``recent'' to mean \emph{before individuals change partners}.
If some individuals have multiple concurrent partnerships ($K > 1$),
then these individuals should not be removed entirely,
but their effective numbers of partners should be reduced by 1.
If multiple types of partners are considered,
then only the partnership type involved in the transmission should be reduced.
This adjustment can then be applied until the individuals change partners
% TODO: but part of partnership has already passed, so delta_new < delta_total
--- at an effective rate inversely related to partnership duration: $\delta^{-1}$.
However, during this period, these individuals should still be modelled
to progress as usual through different stages of infection, activity group turnover, etc.
\par
Using this conceptual basis,
I propose a new stratification of the modelled infected population, denoted $\p$.
The stratum $\p = 0$ corresponds to no recent transmission,
or all ``new'' (potentially discordant) partners.
Other strata $\p \ne 0$ correspond to recent transmission via (to or from) partnership type $\p$.
Figure~\ref{fig:model.hiv.p} illustrates the new stratification
together with with the existing HIV infection stratification (Figure~\ref{fig:model.hiv}).
Following infection, all individuals enter a stratum $\p \ne 0$
corresponding to the partnership type $p$ by which they were infected.
Thus, the rate of entry to this stratum from $S_i$ is defined by
the incidence rate without aggregating across partnership types: $\lambda_{ip}$.
Individuals may then transition from $\p \ne 0$ to $\p = 0$
upon forming a new partnership, at a rate $\delta_p^{-1}$.
Finally, individuals may re-enter any stratum $\p \ne 0$
if they transmit infection via partnership type $p$.
I denote the corresponding rate as $\lambda'_{ip}$,
representing the per-person rate of \emph{transmission},
not \emph{acquisition} as in $\lambda_{ip}$.
This rate $\lambda'_{ip}$ is not defined or needed in prior models (\sref{foi.prior})
but I develop the necessary equations below in \sref{foi.prop.eq}.
The issue of transmission via multiple partnerships is discussed in \sref{foi.prop.mp}.
\begin{figure}
  \centering\includegraphics[scale=1]{model.hiv.p}
  \caption{Modelled states and transitions related to HIV infection,
    and a new stratification $\p$ to track
    the proportions of individuals in partnerships where transmission already occurred}
  \label{fig:model.hiv.p}
  \floatfoot{
    $S$: susceptible;
    $I_{h}$: infectious in stage $h$;
    $p$: partnership type;
    $\p$: new stratification, where
      $\p = 0$ reflects no recent transmission (all new partners), and
      $\p \ne 0$ reflects recent transmission via a type-$p$ partnership;
    $\lambda$: force of infection per susceptible;
    $\lambda'$: force of infection per infectious;
    $\eta$: rate of progression between infection stages;
    $\delta$: partnership duration.}
\end{figure}
% ==================================================================================================
\subsection[Equations]{Equations%
  \footnote{\shortquote{Enough talk. Show me the \$} --- \LaTeX\ users.}}\label{foi.prop.eq}
Since partnership duration is now considered separately and explicitly,
I do not define any adjusted per-partnership probability of transmission $B$.
Rather, I define the force of infection to directly include
the frequency of sex per partnership $F$ and probability of transmission per sex act $\beta$.
However, the mixing is now slightly more complicated,
since the effective number of partnerships ``offered'' depends on infection status.
In addition, these partnerships are now defined as numbers of concurrent partners $K$,
rather than partnership formation rates $Q$.
\par
Let $M_{pii'}$ be the total (population-level, not per-person)
number of type-$p$ partnerships between group $i$ and group $i'$.
As described \sref{model.par.mix}, this ``mixing matrix'' $M_{pii'}$ can be defined in several ways,
based on the total numbers of partnerships ``offered'' by each group: $M_{pi}, M_{pi'}$,
plus some parameter(s) specifying mixing patterns (\eg $\Phi$).
Working backwards, I start by defining $M_{pi}$ (and likewise $M_{pi'}$) via
the sum across health statuses --- \ie susceptible, and different stages of infection $h$:
\begin{equation}
  M_{pi} = M_{S,pi} + \sum_h M_{I,pih}
\end{equation}
I then define the total numbers of partnerships ``offered'' by susceptible individuals as:
\begin{equation}
  M_{S,pi} = S_{i} K_{pi} \label{eq:M.S}
\end{equation}
and likewise for individuals in infection stage $h$:
\begin{equation}
  M_{I,pih} = I_{ih,\p=p} (K_{pi}-1) + \sum\nolimits_{\p \ne p} I_{ih\p}\,K_{pi} \label{eq:M.I}
\end{equation}
\eqref{eq:M.I} is the key equation whereby
the effective numbers of type-$p$ partnerships ``offered'' by
individuals in stratum $\p$ are reduced by 1.
This reduction is then propagated through the mixing patterns when defining $M_{pii'}$.
Thus, we are now ready to construct the overall force of infection equation as follows.
I define the total (population-level, not per-person) rate of transmission
from group $i'$ and infection stage $h'$ to group $i$ via type-$p$ partnerships as:
\begin{equation}
  \Lambda_{pii'h'} = F_p \beta_{pii'h'} M_{pii'}
  \left(\frac{M_{S,pi}}{M_{pi}}\right)
  \left(\frac{M_{I,pi'h'}}{M_{pi'}}\right)
\end{equation}
where the two fractions represent the proportions of all partnerships $M_{pii'}$
formed between susceptible individuals from group $i$ ($M_{S,pi}$)
and infectious individuals in group/infection stage $i'h'$ ($M_{I,pi'h'}$).
The per-person transmission rates to group $i$, and from group $i'h'$ can then be defined as:
\begin{alignat}{1}
  \lambda_{pi} &= \sum_{i'h'} \Lambda_{pii'h'}\,{S_{i}}^{-1} \label{eq:foi.i} \\
  \lambda'_{pi'h'} &= \sum_{i} \Lambda_{pii'h'}\,{I_{i'h'}}^{-1} \label{eq:foi.jh}
\end{alignat}
However, for the purposes of solving the model,
division by $S_{i}$ and $I_{i'h'}$ can be skipped in Eqs.~(\ref{eq:foi.i}) and (\ref{eq:foi.jh}),
since $\lambda'_{pi}$ and $\lambda'_{pi'h'}$ are immediately multiplied by $S_{i}$ and $I_{i'h'}$,
respectively, in the system of differential equations.
\par
The final two model changes to reiterate are that infected individuals in stratum $I_{ih\p}$:
a) are assumed to form new partnerships at a rate $\delta_p^{-1}$,
and thereby transition to stratum $I_{ih\p_0}$ (``all new partners''3); and
b) otherwise transition between infection stages, cascade of care, activity groups, etc. as usual,
as illustrated in Figure~\ref{fig:model.hiv.p}.
% ==================================================================================================
\subsection{Transmission via Multiple Partnerships}\label{foi.prop.mp}
In the proposed approach (\ie new stratification and equations),
I do not explicitly model the proportions of infected individuals
who recently acquired and/or transmitted infection via two different partnership types,
or two partnerships of the same type.
To do so, the required size of the new dimension $\p$ would be at least $2^{P}$, not $P+1$,
where $P$ is the number of different types of partnerships modelled.
For transmission via three different partnerships, the required size would be at least $3^P$, etc.
Indeed, this exponential relationship is related to the challenge of
specifying all combinations of partnership states in pair-based models \cite{Kretzschmar2017}.
However, under frequentist assumptions, we can equivalently model
two transmissions by one individual as one transmission each by two individuals.
Thus, we can transfer two individuals from $I_{ih\p_0}$ to
$I_{ih\p_1}$ and $I_{ih\p_2}$ (one each) under the $P+1$ stratification,
instead of just one individual from $I_{ih\p_0}$ to
$I_{ih\p_{12}}$ under any of the exponential the stratifications.
\par
In fact, $I_{ih\p_0}$ can be \emph{negative} (but only for $\p = 0$),
because the dimension $\p$ is only relevant to \eqref{eq:M.I};
in all other contexts and equations,
we use $I_{ih} = \sum_{\p} I_{ih\p}$, which must be positive as usual.
Moreover, we can also have $I_{ih\p} > I_{ih}$, provided that:
\begin{equation}\label{eq:I.constr}
  I_{ih\p} \le I_{ih} K_{pi}
\end{equation}
reflecting the situation where 100\% of $I_{ih}$
have recently acquired and/or transmitted infection via at least one type-$p$ partnership,
or 50\% via at least two partnerships, etc.
This situation can therefore only arise in the context of
multiple concurrent type-$p$ partnerships: $K_{pi} > 1$.
If $I_{ih\p} > I_{ih}$, then $I_{ih\p_0}$ \emph{must} be negative,
but it can be shown that \eqref{eq:M.I} still yields the correct value of $M_{I,pih}$.
With this perspective, the constraint in \eqref{eq:I.constr} may be intuitive:
we cannot ``remove'' more than the total number of partnerships ``offered''.
This constraint should also be easy to guarantee for small enough timesteps,
because $M_{I,pih}$ approaches zero as $I_{ih\p}$ approaches $I_{ih} K_{pi}$
--- i.e. all type-$p$ partnerships become concordant,
and no more transmission can occur via these partnerships until partners change.
% TODO: discuss the possibility of being in multiple seroconcordant partnerships following transmission
% slightly more complicated modelling of transitions within "k" dimension,
% with main risk if we don't being: run out of people in k=0 (fully discordant
% serosorting refs: cross-sectional vs individual-cohort \cite{Wang2020,Kim2020}