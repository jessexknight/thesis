\section{Discussion}\label{foi.disc} 
In this chapter, I have critically reviewed
key assumptions and differences among various existing approaches to modelling
HIV transmission via sexual partnerships in compartmental models.
I formalized distinctions between:
within- \vs between-partnership heterogeneity
when calculating the average probability of transmission per partnership;
per partnership-year \vs per partnership-duration
when adjusting for ``post-transmission contacts'' (PTC); and
incidence rate \vs incidence proportion
when aggregating risk across multiple partnerships.
I also proposed a new approach: the \emph{Effective Partnerships Adjustment},
which can overcome some of the key limitations of prior approaches.
Finally, through comparison with the new approach in the Eswatini model,
I found that two popular prior approaches may overestimate
the relative contribution of longer \vs shorter partnerships to overall transmission,
even after recalibrating the model under each approach.
%===================================================================================================
\subsection{Insights for Transmission Modelling}\label{foi.disc.ins}
The critical review and experiment results in this chapter
offer new insights for modelling transmission of HIV, and possibly other infectious diseases.
This section discusses these insights
and draws connections to several prior model comparison studies.
%---------------------------------------------------------------------------------------------------
\subsubsection{Average Probabilities of Transmission per Partnership}
Models with instantaneous partnerships must compute
average probabilities of transmission per partnership.
The choice of averaging equation implies either
``within-partnership heterogeneity'' or ``between-partnership heterogeneity''
in the probability of transmission per act.
While I have shown that these two cases are mathematically related by an inequality,
I only found notable differences (\eg~$>10\%$) in the context of large cumulative risk
--- \eg a high volume of sex and/or risk-increasing transmission modifiers.
Indeed, approximations in HIV force of infection equations are often justified by noting that
the probabilities of transmission are typically small \cite{Kerr2015}.
Thus, the distinction between within- \vs between-partnership heterogeneity
may be of little practical consequence.
Moreover, ``correct'' interpretations of data on transmission modifiers are not always obvious.
For example, levels of condom use are often derived from questions like
\shortquote{Did you use a condom the last time you had sex?}
--- but this does not distinguish between
50\% condom use in 100\% partnerships \vs 100\% condom use in 50\% partnerships.
\par
On the other hand, two major challenges remain for
calculating average probabilities of transmission per partnership.
First, relative risks associated with transmission modifiers
are typically quantified at the per act (\vs per partnership) level, using
exposure-stratified individual-level data \cite{Jewell1990,Gray2001,Wawer2005,Boily2009}.
Yet, these relative risks have been directly applied to
average probabilities of transmission per partnership in several models.
This approach would then
underestimate the impact of risk-reducing modifiers (\eg condoms) and
overestimate the impact of risk-increassing modifiers (\eg GUD).%
\footnote{Modifying the transmission probability via $R$ ---
  per-act: $B_a = (1 - {(1 - R\beta)}^A)$ \vs per-partnership: $B_p = R\,(1 - {(1 - \beta)}^A)$;
  thus: $B_a > B_p$ if $R < 1$, and $B_a < B_p$ if $R > 1$.}
% TODO: (*) this belongs in foi.prior vs discussion
Second, it's not clear how \emph{dynamic} transmission modifiers (\eg condom use, infection stage)
should be modelled when aggregating sex acts across many years within longer partnerships.
%---------------------------------------------------------------------------------------------------
\subsubsection{Accumulation of Seroconcordance}
The concept of ``post-transmission contacts'' (PTC) is directly related to seroconcordance,
because transmission naturally generates seroconcordant ($I$-$I$) partnerships,
wherein ``post-transmission contacts'' occur.
Seroconcordant partnerships thus accumulate during an epidemic,
which decouples incidence from prevalence,
as shown in \sref{foi.prop.toy}~and~\ref{foi.exp.model.dyn}.
An ealier version of this work \cite{Knight2022smdm} therefore
described the accumulation of seroconcordant partnerships as
\shortquote{partnership level herd effects}, while
\citet{Eames2002} describe this accumulation in a network model as
\shortquote{correlation of infection statuses of neighboring individuals}.
There are two main implications of this seroconcordance perspective.
First, an alternate adjustment for ``post-transmission contacts'' in compartmental models
could make use of data on seroconcordant and/or serodiscordant partnerships.
Such an adjustment should consider potential biases in these data, and
be applied mainly over shorter time horizons
due to the dynamic nature of the seroconcordant proportion.
Second, efforts to quantify serosorting --- preferential selection of sexual partners
with matching (perceived) HIV serostatus \cite{Cassels2013} ---
may need to focus on new partnerships or longitudinal data \cite{Kim2020},
since it may not be possible to distinguish intentional seroconcordance
from transmission-driven seroconcordance in cross-sectional data \cite{Cassels2009,Wang2020}.
%---------------------------------------------------------------------------------------------------
\subsubsection{Influence of Instantaneous Partnerships on Epidemic Growth}
Prior model comparisons of instantaneous partnerships
with pair-based models \cite{Kretzschmar1998,Eames2002,Lloyd-Smith2004} and
with a stochastic dynamic network model \cite{Johnson2016mf}
have shown that instantaneous partnerships
can overestimate the initial epidemic growth rate and equilibrium prevalence.
Such findings seem intuitive.
However, in \sref{foi.exp.model} (Figure~\ref{fig:foi.ep.incidence}),
I showed how the epidemic growth rate under instantaneous partnerships depends
on the effective partnership duration used for the PTC adjustment
--- if such an adjustment is applied at all.
That is, when durations were capped at 1 year (approaches \iry,~\ipy),
this adjustment likely had little effect,
and modelled incidence was indeed overestimated relative to the \epa approach;
by contrast, when full partnership durations were used (approach \ird),
this adjustment reduced transmission immediately in anticipation of future PTC,
and modelled incidence was \emph{underestimated} relative to the \epa approach.
No PTC adjustments were described in \cite{Eames2002,Lloyd-Smith2004}, and
the PTC adjustments in \cite{Johnson2016mf} did not consider full partnership durations,%
\footnote{The PTC adjustments in \cite{Johnson2016mf} considered
  1 month for main/spousal, 6 months for casual, and nothing for sex work partnerships;
  thus, repeated sex work contacts were not considered.}
similar to the former 1-year case.
%---------------------------------------------------------------------------------------------------
\subsubsection{Partnership Types \vs Risk Groups}
The results in \sref{foi.exp.model.tpaf} suggest that models ignoring or underestimating PTC
have the potential to systematically overestimate
the relative contribution of longer \vs shorter partnerships to overall transmission.
Such results then extend to various risk groups,
depending on the average numbers and types of partnerships among each group.
These results are corroborated by \citet{Johnson2016mf}, who concluded:%
\footnote{The term ``frequency-dependent'' in \cite{Johnson2016mf} is synonymous with
  ``instantaneous partnerships'' here.}
\shortquote{Frequency-dependent models are likely to
  underestimate the importance of interventions that are targeted at high-risk groups,
  while overestimating the impact of interventions targeted at low-risk groups.}
While higher risk groups typically have more shorter partnerships,
higher risk is not necessarily synonymous with shorter partnerships.
Indeed, the Eswatini model includes both
shorter (casual) partnerships among the lowest risk groups and
longer (regular sex work) partnerships among the highest risk groups.
This distinction between risk groups and partnership types is therefore important to keep in mind
when interpreting these results and their potential implications.
%===================================================================================================
\subsection{Implications for Existing Model-Based Evidence}\label{foi.disc.evid}
The majority of existing compartmental HIV transmission models
have used an instantaneous partnerships approach,
with adjustments for PTC considering 1 year or less.
As noted above, it is possible that these models have systematically overestimated
the relative contribution of longer \vs shorter partnerships to overall transmission,
and thus, the relative impact of some prevention strategies \vs others.
I have illustrated these potential biases
in the context of the high-prevalence epidemic in Eswatini,
but these biases and implications could be different and possibly greater elsewhere.
Below I further discuss two specific models,
chosen due to their widespread use in numerous countries,
usually for the explicit purpose of guiding resource allocation.
%---------------------------------------------------------------------------------------------------
\paragraph{Optima HIV Model}\cite{Kerr2015,Stuart2018,Kerr2020,Optima2021}
The Optima model allows for flexible definitions of demographic/risk groups,
which usually include key populations.
Within the model, the force of infection is defined as an incidence proportion,
wherein all sex acts over a given time period (default $\Delta_t = 0.2$~years)%
\footnote{From: \hreftt{github.com/optimamodel/optima/blob/master/optima/parameters.py}}
are modelled as competing risks.
Thus, partnership durations are not considered.
While Optima results have often recommended
increased resources for key populations over country-specific current spending,
these recommendations may still undervalue
prevention in shorter partnerships and among highest risk groups
due to overestimation of transmission in longer partnerships.
%---------------------------------------------------------------------------------------------------
\paragraph{Goals Risk-Structured Model}\cite{Stover2014,Stover2021}
The Goals Model includes 11 total risk groups, including
FSW, their clients, MSM, and PWID, without age stratification \cite{Stover2014}.%
\footnote{An age stratified variant of Goals was recently developed for ``generalized'' epidemics,
  which subsumes key populations as proportions of age strata;
  the new model is called the Goals ``Age-Stratified Model'' (ASM), whereas
  the original model is now called the Goals ``Risk-Stratified Model'' (RSM) \cite{Stover2021}.
  To confuse matters further, a Goals ``Age-Risk-Stratified Model'' is also being developed.}
As in the Optima model, the force of infection is defined as an incidence proportion,
this time with $\Delta_t = 1$~year, and a default partnership change rate of
at least 1 per year among all risk groups.
Thus, for the same reasons as Optima, the relative impact of prioritizing
shorter partnerships and key populations for prevention
have likely been underestimated by the Goals Model.
%===================================================================================================
\subsection{Outlook \& Future Work}\label{foi.disc.fw}
% TODO: (~) explicitly mention limitations of IBMs again \cite{Hazelbag2020,Pellis2015}
The 2021 review by \citet{Rao2021} summarizes frameworks that have been used to
simulate partnership dynamics for modelling sexually transmitted infections
(see also \sref{foi.prior.alt} and Appendix~1 of \cite{Johnson2016mf}).
Besides pair-based models, the review does not identify another approach
which has extended the compartmental modelling framework beyond instantaneous partnerships.
However, several hybrid models have been developed \cite{Xiridou2003,Powers2011}
wherein long-term pairs are explicitly modelled,
but additional ``one-off'' partnerships are modelled as instantaneous.
When long-term partnership concurrency is low, such hybrid approaches
may offer substantial improvements over fully instantaneous partnerships
\cite{Kretzschmar1998,Eames2002,Lloyd-Smith2004}.
However, the high number of \emph{regular} clients reported by Swati FSW (\sref{model.par.fsw})
reflects precisely the kind of dense, persistent sexual network --- \ie high concurrency ---
which is difficult to model using a pair-based approach.%
\footnote{The importance of partnership concurrency in HIV transmission has been debated extensively
  \cite{Mah2010,Tanser2011,Goodreau2012,Boily2012,Sawers2013}.}
Thus, the \emph{Effective Partnerships Adjustment} represents
an alternative to hybrid / pair-based models for such contexts,
and a new solution to a longstanding modelling challenge \cite{Dietz1988a}.
%---------------------------------------------------------------------------------------------------
\paragraph{Future Work}
In considering potential integration of the \emph{Effective Partnerships Adjustment} approach
into new and existing models, four key areas for future work should be considered.
First, it would be helpful to explore in more detail which of the Eswatini model parameters
``adjusted'' during model recalibration under each approach,
in order to fit the same calibration targets.
Second, while I developed the approach in the context of the Eswatini model,
examining this approach in a simpler model
--- \eg with 2 risk groups and 2 partnerships types ---
may allow a more precise understanding of potential biases and mechanisms involved.
A simpler model may also be more practical as an example implemention of the approach
--- \eg as a reference for other modellers.%
\footnote{The main implemention of this approach in the Eswatini model is in:\\
 \hreftt{github.com/mishra-lab/hiv-fsw-art/blob/master/code/model/foi.py},
  where \texttt{foi_mode='base'}.}
Third, the approach should be compared and validated against
``gold standard'' individual-based models, similar to experiments in \cite{Johnson2016mf}.
Finally, while I developed the proposed approach in the context of HIV,
accurate simulation of partnership dynamics is also relevant for
models of other sexually transmitted infections.
However, the approach should be carefully adapted for
infections with recovery and/or re-infection,
as I have not considered how transitions \emph{out} of the newly proposed infected strata
should be conceptualized and parameterized.
% JK: omitting conclusion subsection for now, it felt repetitive
