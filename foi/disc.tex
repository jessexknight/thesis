\section{Discussion}\label{foi.disc}
Compartmental models of sexual HIV transmission continue to support HIV epidemic response globally,
including the Spectrum suite of models \cite{Stover2021,Spectrum2022},
the Optima HIV model \cite{Kerr2015,Optima2021},
the Asian Epidemic Model \cite{Brown2004},
the Thembisa model for South Africa \cite{Johnson2016cc,Johnson2022},
and numerous others (\eg Chapter~\ref{sr}).
Such models usually simulate sexual HIV transmission within risk- and/or age-stratified populations,
possibly considering multiple partnership types and/or transmission modifiers.
As I have shown in \sref{foi.prior},
several existing approaches (model structures and equations)
are used to define rates of HIV transmission via sexual partnerships in these models
--- especially with respect to heterogeneous populations, partnerships, and sex acts ---
each with implicit assumptions.
In \sref{foi.exp}, I explored the potential influence of these approaches/assumptions on:
the computed probability of transmission per partnership,
modelled epidemic dynamics, and on model-estimated prevention priorities.
Yet, many of these assumptions can be avoided altogether
using a new approach I developed in \sref{foi.prop},
representing an exciting opportunity to improve the quality of
compartmental model-based evidence for HIV response going forward.
%===================================================================================================
\subsection{Heterogeneity in Per-Act Probability of Transmission}\label{foi.disc.xph}
In \sref{foi.prior.xph}~and~\ref{foi.exp.xph}, I introduced and explored the distinction between
\emph{within}-partnership heterogeneity (WPH) \vs \emph{between}-partnership heterogeneity (BPH)
in the per-act probability of transmission $\beta$. Whereas
WPH reflects an assumption that all partnerships are identical, but comprise heterogeneous acts,
BPH reflects an assumption that partnerships are different, but each comprise identical acts.
This distinction --- \ie all of \eqrefr{eq:B}{eq:B.xph} and variants thereof ---
is unnecessary under the \emph{Effective Partnerships Adjustment} approach.
\par
Numerous variations on \eqrefr{eq:B}{eq:foi.ip} have been used in prior models.
% (see \sref{app.foi.prior}).
For example, the Optima \cite{Kerr2015} and Goals \cite{Stover2014} models
aggregate heterogeneity due to HIV infection stage
\emph{before} aggregating sex acts within each partnership or applying transmission modifiers.
Such an approach is difficult to justify, because
the prevalence of each infection stage evidently reflects distinct individuals
--- \emph{not} the distribution of infection stages within a given partnership
(see also footnote~\ref{foot:xph.future}.)
This approach then does not allow for
the \emph{multiplicative} interaction of infection stage and other modifiers,
while simultaneously allowing a high-infectivity stage with low prevalence (\eg acute HIV)
to increase transmission risk across all partnerships (see Figure~\ref{fig:binom.xph}).
As a result, this approach yields intermediate $B_{\wph} \ge B' \ge B_{\bph}$
(see \sref{app.foi.proof}).
It's not clear whether these variations have systematically biased existing model-based evidence,
but improved understanding of the assumptions and potential biases of each approach
can help guide interpretation of existing results,
and design of future models which do not adopt the proposed approach.
\par
In some cases, modellers have noted the discrepancies between equations between models,
but dismissed the differences as inconsequential because $\beta$ is usually small
\cite{Kerr2015}. % TODO: (*) more cite
This justification is fair when $\beta$ is indeed small.
However, several combinations of transmission modifiers
(\eg condomless anal sex with GUD and acute HIV infection) \cite{Boily2009,Fox2011}
can easily yield larger $\beta$, for which the discrepancies are \emph{not} inconsequential
(\eg Figure~\ref{fig:B.xph.surf}).
In fact, it is precisely these contexts of rapid transmission which define key epidemic dynamics,
as reflected in core group theory \cite{Watts2010}.
Moreover, since the prevalence of such modifiers
often varies across risk groups and transmission pathways,
differences in how heterogeneous $\beta$ is aggregated may ultimately yield
differences in the modelled contribution of risk groups and transmission pathways
to overall transmission --- although some differences might be reduced via model calibration.
\par
Lastly, statistical inference on modifiers of per-act transmission probability
--- \eg relative risk with condoms, GUD, etc. --- typically uses
exposure-stratified individual-level data \cite{Jewell1990,Gray2001,Wawer2005,Boily2009}.
Thus, these statistical models do not consider what \emph{proportion} of sex acts are exposed,
and need not distinguish between within \vs between partnership heterogeneity.
Yet, relative risks estimated from \emph{per-act} data
have been applied to the \emph{per-partnership} transmission probability in several models
\cite{TOOD}. % Anderson2014,Stover2021
Such an approach would then
underestimate the impact of risk-reducing modifiers (\eg condoms) and
overestimate the impact of risk-increassing modifiers (\eg GUD).%
\footnote{Modifying the transmission probability via $R$ ---
  per-act: $B_a = (1 - {(1 - R\beta)}^A)$ \vs per-partnership: $B_p = R\,(1 - {(1 - \beta)}^A)$;
  thus: $B_a > B_p$ if $R < 1$, and $B_a < B_p$ if $R > 1$.}
%===================================================================================================
\subsection{Beyond Instantaneous Partnerships}\label{foi.disc.bip}
The 2021 review by \citet{Rao2021} summarizes frameworks that have been used to
simulate partnership dynamics for modelling sexually transmitted infections
(see also \sref{foi.prior.alt} and Appendix~1 of \cite{Johnson2016mf}).
Besides pair-based models, the review does not identify another approach
which has extended the compartmental modelling framework beyond instantaneous partnerships.
Pair-based models have not seen widespread adoption, likely due to
exponential complexity (\ie numbers of required compartments) \cite{Kretzschmar2017}.
% TODO: (~) still not 100% sure what's going on in Thembisa
However, several hybrid models have been developed \cite{Xiridou2003,Powers2011}
wherein long-term pairs are explicitly modelled,
but additional ``one-off'' partnerships are modelled as instantaneous.
When long-term partnership concurrency is low, such hybrid approaches
likely offer substantial improvements over fully instantaneous partnerships
\cite{Kretzschmar1998,Eames2002,Lloyd-Smith2004}.
However, the high number of \emph{regular} clients reported by Swati FSW (\sref{model.par.fsw})
reflects precisely the kind of dense, persistent sexual network  --- \ie high concurrency ---
which is difficult to model via a pair-based approach.
The importance of partnership concurrency in HIV transmission has been debated extensively
\cite{Mah2010,Tanser2011,Goodreau2012,Boily2012,Sawers2013}.
Thus, the \emph{Effective Partnerships Adjustment} approach offers
an alternative to hybrid / pair-based models for such networks,
and thereby solves a 30-year old modelling challenge \cite{Dietz1988a}.
%---------------------------------------------------------------------------------------------------
\subsubsection{Prior Comparisons of Models with \vs Without Instantaneous Partnerships}\label{foi.disc.bip.prior}
The potential biases associated with instantaneous partnerships
have been explored previously, via comparison with
deterministic pair-based models \cite{Kretzschmar1998,Eames2002,Lloyd-Smith2004},
a stochastic pair-based model \cite{Eames2002},
a stochastic static network-based model \cite{Eames2002}, and
a stochastic dynamic network-based model \cite{Johnson2016mf}.
% TODO: (*) Watts1992: initial fast phase (all susceptible, then delayed by partnership change)
\par
\citet{Kretzschmar1998} highlight how biases associated with instantaneous partnership
increase with the true partnership durations,
and conclude that: \shortquote{the number of new partners per unit time
is not sufficient to predict the course of the epidemic,
but that partnership duration is a quantity that is equally influential.}
I agree and regret that data to directly inform sexual partnership durations,
especially for non-marital partnerships, remain lacking (see \sref{model.par.pdur}).
Efforts to fill this data gap will likely benefit from careful consideration of
different measurement approaches and sources of error \cite{Burington2010},
perhaps in conjunction with efforts to better quantify partnership formation rates,
as explored in \sref{model.par.pnum.adj}.
\par
\citet{Eames2002,Lloyd-Smith2004} both show that instantaneous partnerships
can result in overestimation of the initial epidemic growth rate and equilibrium prevalence.
Such findings seem intuitive.
However, in \sref{foi.exp.model} (Figure~\ref{fig:foi.ep.incidence}),
I showed how the rate of epidemic growth under instantaneous partnerships strongly depends
on the effective partnership duration used for the ``post-transmission contacts'' (PTC) adjustment
--- if such an adjustment is applied at all.
That is, when durations were capped at 1 year (approaches \iry,~\ipy),
this adjustment likely had little effect,
and modelled incidence was indeed overestimated relative to the \epa approach;
by contrast, when full partnership durations were used (approach \ird),
this adjustment reduced transmission immediately in anticipation of future PTC,
and modelled incidence was \emph{underestimated} relative to the \epa approach.
No adjustments for PTC were described in \cite{Eames2002,Lloyd-Smith2004},
reflecting the former case.
%---------------------------------------------------------------------------------------------------
\paragraph{\citet{Johnson2016mf}}
This landmark study compared modelling frameworks across 6 STIs.%
\footnote{The term ``frequency-dependent'' in \cite{Johnson2016mf} is synonymous with
  ``instantaneous partnerships'' here.}
The models explored were more complex than previous works, including:
population stratification by sex, age, and risk,
and three partnership types within a dynamic sexual network, in a South African context.
Although an adjustment for PTC in instantaneous partnerships was applied,
the adjustment considered different time period across partnership types:
1 month for main/spousal, 6 months for casual, and none for sex work partnerships;
thus, regular sex work partnerships were not considered.
Similar to experiments in \sref{foi.exp.model}, \cite{Johnson2016mf} first compared
model outputs from frameworks with equal parameters, and then again with recalibrated parameters.
\par
With equal parameters, findings echoed those above \cite{Eames2002,Lloyd-Smith2004},
although differences between frameworks were larger for
curable STIs with faster transmission, and smaller for HIV.
After recalibrating models to the same STI data from South Africa,
the best-fitting parameters differed significantly across modelling frameworks, as expected
--- \eg lower transmission probabilities were inferred with instantaneous partnerships.
More importantly, the relative impact (infections averted after 10 years)
of several illustrative intervention strategies also differed substantially.
These differences were summarized as:
\shortquote{\emph{[instantaneous partnership models]} are likely to
underestimate the importance of interventions that are targeted at high-risk groups, while
overestimating the impact of interventions targeted at low-risk groups} \cite{Johnson2016mf}.
Such findings are similar to those in \sref{foi.exp.model.tpaf}, where
the TPAFs of lower risk populations and main/spousal partnerships were overestimated, while
the TPAFs of casual partnerships were underestimated
under 1-year approaches (\iry,~\ipy) \vs the \emph{Effective Partnerships Adjustment} (\epa).
However, while \cite{Johnson2016mf} observed that instantaneous partnerships
\shortquote{may underestimate the contribution of commercial sex},
the TPAFs of FSW, clients, and sex work partnerships
were similar across approaches in \sref{foi.exp.model.tpaf}.
These different findings could be explained by the fact that
\cite{Johnson2016mf} only modelled one-off sex work partnerships,
whereas transmission via sex work in the Eswatini model
was dominated by regular partnerships (Figure~\ref{fig:foi.wiw.part}).
Indeed, many HIV models have yet to incorporate longer partnerships between higher risk groups
(see \sref{sr.res.f.net}).
Thus, I would offer to rephrase the conclusions of \cite{Johnson2016mf} as:
\shortquote{without complete adjustment for PTC,
  models with instantaneous partnerships may
  overestimate the contribution of longer partnerships, and
  underestimate the contribution of shorter partnerships.}
%---------------------------------------------------------------------------------------------------
\subsubsection{Implications for Existing Model-Based Evidence}\label{foi.disc.bip.evid}
The vast majority of existing compartmental HIV transmission models
have used an instantaneous partnerships approach,
with adjustments for PTC of 1-year or less.
The results in \sref{foi.exp.model.tpaf} suggest that such models
have likely \emph{systematically} and \emph{significantly} overestimated
the relative contributions of longer \vs shorter partnerships to overall transmission.
Such results are corroborated by \cite{Johnson2016mf},
and have substantial implications for the existing body of model-based evidence.
That is, models continue to help inform which interventions are prioritized and for whom,
and existing evidence may overestimate the importance of prevention within longer partnerships
for reducing overall transmission. For example:
%---------------------------------------------------------------------------------------------------
\paragraph{Anderson et al}\cite{Anderson2014,Anderson2017,Anderson2018}
These works explore ``optimal'' combinations of PrEP, early ART, behaviour change, and VMMC
for MSM, other men, FSW, and other women in Kenya, under various cost constraints.
Their modelling analyses indicate that
early ART for non-MSM men is usually more cost effective than PrEP for FSW.
Yet, for lowest risk men --- modelled median [IQR] 53~[37,~70]\% of non-MSM men ---
effective partnership duration was 4~[2.6,~6.2] years.%
\footnote{Median [IQR] estimated via Monte Carlo sampling of $\bar{c}, R, \varpi$
  from uniform prior distributions in Table~S7;
  posterior parameter distribution were not given in \cite{Anderson2014}.}
Moreover, only sex work partnerships were considered among FSW and their clients.
If longer partnership durations and overlapping partnership types were considered,
the relative impact of PrEP for FSW \vs early ART for non-MSM would likely increase.
%---------------------------------------------------------------------------------------------------
\paragraph{Optima HIV Model}\cite{Kerr2015,Stuart2018,Kerr2020,Optima2021}
As the name suggests, this model has similarly been applied to
``optimize resource allocation'' in over 20 countries.
Recommended allocations have generally increased resources for key populations over current spending.
However, the force of infection equation is defined as
an incidence proportion \cite{Kerr2020} --- \eqref{eq:foi.ip} ---
wherein all sex acts over a given time period (default $\Delta_t = 0.2$~years)%
\footnote{From: \hreftt{github.com/optimamodel/optima/blob/master/optima/parameters.py}}
are modelled as competing risks, and so partnership durations are \emph{completely ignored}.
Thus, the relative impact of prioritizing key populations
has likely been underestimated via both:
overestimation of transmission via longer partnerships due to ignoring partnership durations,
underestimation of transmission to higher risk groups due to the incidence proportion equation.
%---------------------------------------------------------------------------------------------------
\paragraph{Goals Model}\cite{Stover2014,Stover2021}
The Goals Model is part of the Spectrum suite of policy modelling tools \cite{Spectrum2022},
which have been widely applied in consultation with national ministries of health
to estimate yearly new infections and the impact of various interventions \cite{Stover2021}.
The Goals model includes mechanistic HIV transmission among 11 total risk groups, including
FSW, their clients, MSM, and PWID, without age stratification \cite{Stover2014}.%
\footnote{An age-stratified variant of Goals was recently developed for generalized epidemics,
  which subsumes key populations as proportions of age strata;
  the new model is called the Goals ``Age-Stratified Model'' (ASM), whereas
  the original model is now called the Goals ``Risk-Stratified Model'' (RSM) \cite{Stover2021}.
  To confuse matters further, a Goals ``Age-Risk-Stratified Model'' is also being developed.}
Yet, as in the Optima model, the force of infection is defined as an incidence proportion
--- using a transformation of \eqref{eq:foi.ip} --- with $\Delta_t = 1$~year
and an effective partnership change rate of at least 1 per year among all risk groups.
Thus, for the same reasons as Optima (and worse with $\Delta_t = {}$1 \vs 0.2~years),
the relative impact of prioritizing shorter partnerships and key populations for prevention
have likely been systematically underestimated by the Goals Model.
\par
In sum, decades of model-based HIV prevention evidence
--- for multiple countries and resource allocation questions ---
are built upon the instantaneous partnerships assumption and associated equations.
I have shown that this assumption and these equations can significantly bias
model-estimated contributions of different populations and partnerships to overall transmission,
and thus the model-estimated importance of different prevention strategies.
Specifically, the common practice of only adjusting for up to 1-year of PTC
(or equivalently: assuming that all individuals change partners at least once per year)
overestimates the importance of prevention in longer partnerships, and
underestimates the importance of prevention in shorter partnerships.
Moreover, defining the force of infection as an incidence proportion (\vs rate)
disproportionately reduces modelled incidence among populations at higher risk,
and thus underestimates the importance of prevention among these populations.
Finally, I illustrated these potential biases
in the context of a high prevalence HIV epidemic (Eswatini),
but these biases and implications for prevention could be even greater elsewhere.
% TODO: (?) comment on periods vs populations at higher risk i.e. sexual life course
%===================================================================================================
\subsection{Transmission-Driven Emergence of Serosorting Patterns}\label{foi.disc.ss}
% TODO: (?) this feels out of place, but not sure where to put instead ...
HIV serosorting is a controversial harm reduction strategy defined as
preferential selection of sexual partners with matching (perceived) HIV serostatus;
related strategies can involve modified sexual practices with a given partner on the same basis
\cite{Heymer2010,Cassels2010,Cassels2013}.
Serosorting is often quantified using cross-sectional data,
using the odds or excess fraction of seroconcordant partnerships
\vs random mixing by serostatus \cite{Cassels2009,Wang2020}.
However, using an illustrative toy scenario (\sref{foi.prop.toy}, Figure~\ref{fig:foi.toy}),
I have highlighted how transmission naturally generates
a disproportionate number of seroconcordant ($I$-$I$) partnerships
as compared to random mixing by serostatus.
This emergent property was also noted in \cite{Eames2002},
therein described as \shortquote{correlation of infection statuses of neighboring individuals}.
This disproportionate seroconcordance would be correlated with partnership duration.
Failure to consider this dynamic could then lead to
overestimation of the degree to which serosorting is intentional (from cross-sectional data).
Such biases in quantifying serosorting could be mitigated using longitudinal data
or consideration of only new partnerships \cite{Kim2020}.
%===================================================================================================
\subsection{Future Work}\label{foi.disc.fw}
Hopefully I have made a convincing case for the value added by the proposed approach.
Thus, an obvious area for future work would be to integrate this approach
into new and existing compartmental HIV transmission models,
including widely-used models like Spectrum Goals, Optima, Asian Epidemic Model, and Thembisa.
To this end, the complete implementation of the proposed approach
in the Eswatini model is available online,%
\footnote{See: \hreftt{github.com/mishra-lab/hiv-fsw-art/blob/master/code/model/foi.py},
  where \texttt{foi_mode='base'}.}
although perhaps it would also be useful to develop a simpler example model
with \vs without the proposed approach to illustrate the essential differences.
It may also be useful to compare key model outputs
before \vs after integrating the proposed approach in existing models
and highlight notable differences, similar to the experiments in \sref{foi.exp.model}.
Similar work could compare and indeed validate the proposed approach \vs individual-based models,
similar to the experiments in \cite{Johnson2016mf}.
\par
Additionally, while I developed the proposed approach in the context of HIV,
accurate simulation of partnership dynamics is also relevant for
compartmental models of other sexually transmitted infections \cite{Rao2021}, including
gonorrhea, chlamydia, syphilis, trichomoniasis, herpes, hepatitis, papillomavirus, and mpox.
However, the approach should be carefully adapted for curable infections,
as I have not considered how transitions \emph{out} of the newly proposed infected strata
(stratification $\p$ in Figure~\ref{fig:model.hiv.p}) should be conceptualized and parameterized.
