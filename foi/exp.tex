\section{Experiment}\label{foi.exp}
In this section, I describe some simple experiments (methods and results) to
highlight differences in the various equations and approaches
to modelling HIV transmission in sexual partnerships.
%===================================================================================================
\subsection{Within- \vs Between-Partnership Heterogeneity}\label{foi.exp.xph}
For computing an average per-partnership probability of transmission ($B$),
\sref{foi.prior.bhet} clarified the interpretations of
\eqref{eq:B.wph} \vs \eqref{eq:B.bph} as modelling
within-partnership heterogeneity (WPH) \vs between-partnership heterogeneity (BPH), respectively.
As shown in \sref{app.math.misc.xph} (proof), $B_{\wph} \ge B_{\bph}$.
Here I explore under what conditions the ratio $B_{\wph}~/~B_{\bph}$ is maximized
--- \ie when does choosing the correct approach matter most.
For simplicity, I considered a single illustrative factor $f$
affecting $\alpha_f \in [0,1]$ proportion of sex acts ($1-\alpha_f$ are unaffected),
with relative probability of transmission $R_f \in [0.01,~10]$.
I~then computed $B_{\wph}$ and $B_{\bph}$ for $A \in [1,1000]$ total sex acts,
using a base per-act probability of transmission $\beta = 0.34$\%
as a representative value for HIV \cite{Boily2009}.
\par
\begin{figure}
  \centering\includegraphics[scale=.6]{B.xph.surf.pdf}
  \caption{Average per-partnership probability of transmission $B$
    given heterogeneity in the per-act probability of transmission $\beta$
    within \vs between partnerships}
  \label{fig:B.xph.surf}
  \floatfoot{
    $B$: probability of transmission per partnership (log scale colourmap);
    $\beta = 0.34\%$: probability of transmission per sex act (fixed) \cite{Boily2009};
    $A$: total sex acts per partnership (log scale);
    $\alpha_f$: proportion of sex acts affected by factor $f$ (linear scale);
    $R_f$: relative $\beta$ given factor $f$ (log scale);
    WPH: within-partnership heterogeneity;
    BPH: between-partnership heterogeneity.}
\end{figure}
\begin{figure}
  \centering\includegraphics[scale=.6]{B.xph.max.pdf}
  \caption{Parameter values $(\alpha,A)$ which maximize the difference between
    the average per-partnership probability of transmission
    given within- \vs between-partnership heterogeneity}
  \label{fig:B.xph.max}
  \floatfoot{
    $B_{\wph}~/~B_{\bph}$: line colour;
    $\beta = 0.34\%$: probability of transmission per sex act (fixed) \cite{Boily2009};
    $A$: total sex acts per partnership (log scale);
    $\alpha_f$: proportion of sex acts affected by factor $f$ (linear scale);
    $R_f$: relative $\beta$ given factor $f$ (log scale);
    gray lines denote equivalent contours for $2\beta$ and $\frac12\beta$.}
\end{figure}
Figure~\ref{fig:B.xph.surf} illustrates four 2-dimensional cross sections of $B(R,\alpha,A)$
under WPH \vs BPH, and the ratio $B_{\wph} / B_{\bph}$;
the cross sections were at: $A = 32$, $\alpha = 0.5$, $R = 0.1$, and $R = 5$.
Based on these results, the difference between approaches can be summarized as:
\begin{itemize}
  \item negligible for $A < 10$, and small for $A < 100$
  \item increasing as $R$ gets farther from 1 ($R \rightarrow 0$ or $R \rightarrow \infty$)
  \item maximized by specific values of $(\alpha,A)$ for a given $R$, including
    $\alpha > \frac12$ for $R < 1$, and $\alpha < \frac12$ for $R > 1$
\end{itemize}
The specific values of $(\alpha,A)$ which maximize
the difference between approaches for a given $R$ and $\beta$
create a continuous curve (Figure~\ref{fig:B.xph.max}), which slowly tends towards
$\alpha \rightarrow 1, A \rightarrow \infty$ as $R \rightarrow 0$, and
$\alpha \rightarrow 0, A \rightarrow 0$ as $R \rightarrow \infty$.
The curve is sigmoidal for log-transformed $A$,
and shifts left with increasing $\beta$.
I did not derive an analytical expression, but it should be possible to do so.
In the context of HIV, the difference between approaches would be
larger for protective factors (\eg condoms)
affecting most of a large number of sex acts ($\alpha > 100$);
and likewise larger for risk-increasing factors (\eg anal sex)
affecting a minority of a moderate number of sex acts ($\alpha \approx 100$).
%===================================================================================================
\subsection{Partnership Durations}\label{foi.exp.dur}
As described in \sref{foi.prior.part}, multiple prior models have
implicitly assumed a maximum partnership duration $\delta \le 1$ year.
As such, the adjustment for ``wasted contacts'' \eqref{eq:B} would have less effect.
This reduced effect can be quantified via
the effective probability of transmission per sex act $\beta'$
--- \ie tangent slopes in Figure~\ref{fig:binom.dur} --- defined as:
\begin{equation}\label{eq:beta.eff}
  \beta' = \frac{B}{A} = \frac{1 - {(1 - \beta)}^{A}}{A}
\end{equation}
Figure~\ref{fig:dur.surf} illustrates the 1-year $\beta'_1$ \vs true-duration $\beta'_\delta$, for
different partnership durations $\delta \in [1, 30]$ and sex frequencies $F \in [1,~180]$ per year.
% Evidently $\beta'_1$ does not depend on $\delta$.
Assuming $\delta \le 1$ can considerably increase the modelled rate of transmission
for partnerships with $F \ge 52$ (\ie weekly) and a true duration $\delta \ge 5$ years,
including up to \emph{8-fold} difference with $F \approx 100$ and $\delta \approx 30$.
Thus, prior models using $\delta \le 1$ may have
substantially overestimated the relative contribution of
longer partnerships with frequent sex --- including main/spousal partnerships ---
to overall transmission.
\begin{figure}
  \centering\includegraphics[scale=.6]{dur.surf.pdf}
  \caption{Effective probability of transmission per sex act
    over 1 year \vs total partnership duration}
  \label{fig:dur.surf}
  \floatfoot{
    $\beta = 0.34\%$: probability of transmission per sex act (fixed) \cite{Boily2009};
    $F$: frequency of sex per partnership (per year, log scale);
    $\delta$: partnership duration (years, linear scale);
    $\beta_1, \beta_\delta$: effective probability of transmission per sex act,
      for 1 year \vs total partnership duration, respectively.}
\end{figure}
%===================================================================================================
\subsection{Full Model}\label{foi.exp.model}
TODO
Table~\ref{tab:foi.models}
\begin{table}[h]
  \centering
  \caption{Compared approaches to modelling HIV transmission via sexual partnerships}
  \label{tab:foi.models}
  \begin{tabular}{cclcl}
  \toprule
   & ID & Name & Key Eqs. & Key Parameters \\
  \midrule
  % TODO: double-check wph, bph ?
  \colorsquare{np} & \np & New Proposed        & (\ref{eq:M.SI})--(\ref{eq:foi.jh})  & $K,F,\delta$ \\
  \colorsquare{rd} & \rd & Rate-Duration       & (\ref{eq:B.bph}),~(\ref{eq:foi.ir}) & $A,Q$        \\
  \colorsquare{ry} & \ry & Rate-1-Year         & (\ref{eq:B.bph}),~(\ref{eq:foi.ir}) & $A_1,Q_1$    \\
  \colorsquare{pd} & \pd & Proportion-Duration & (\ref{eq:B.bph}),~(\ref{eq:foi.ip}) & $A,Q,$       \\
  \colorsquare{py} & \py & Proportion-1-Year   & (\ref{eq:B.bph}),~(\ref{eq:foi.ip}) & $A_1,Q_1$    \\
  \bottomrule
\end{tabular}
\floatfoot{
  $K$: number of concurrent partners;
  $F$: frequency of sex per partnership;
  $\delta$: partnership duration;
  $A = F\delta$: total sex acts per partnership;
  $Q = K/\delta$: partnership formation rate;
  $A_1 = F \delta_1$, $Q_1 = K/\delta_1$, where $\delta_1 = \min{(\delta,1)}$.}
\end{table}
