While developing the Eswatini model in Chapter~\ref{model},
and examining the models reviewed in Chapter~\ref{sr},
I noticed that the ``force of infection'' equation
--- the rate at which susceptible individuals acquire infection ---
has previously been defined in several different ways.
These differences were not due to differences in
which risk groups, partnership types, and/or intervention were considered,
but rather due to differences in
which mathematical approximations of sexual partnership dynamics were used.
It was not clear if or how such differences might influence overall model outputs.
Thus, this chapter explores different mathematical approaches to
modelling HIV transmission via sexual partnerships in compartmental models.
Specifically, I review prior approaches (\sref{foi.prior}),
develop a new approach (\sref{foi.prop}),
and compare the influence of selected approaches on model outputs (\sref{foi.exp}).
%---------------------------------------------------------------------------------------------------
\paragraph{First Principles \& Post-Transmission Contacts}
In the simplest compartmental transmission models
--- \ie without any population stratification, repeated contacts, etc. ---
the force of infection $\lambda$ is defined as:%
\footnote{\eqref{eq:foi.simple} also assumes
  frequency-dependent transmission \vs density-dependent transmission,
  which is almost always more appropriate for sexually-transmitted diseases \cite{Begon2002}.}
\begin{equation}\label{eq:foi.simple}
  \lambda = C \beta \frac{I}{N}
\end{equation} where:
$C$ is the average contact rate per-person;
$\beta$ is the average probability of transmission per contact; and
$I/N$ is the current prevalence of infection.
\par
If the population is stratified into multiple groups $i$,
the infection is stratified into multiple infectious stages $h$,
and contacts are stratified into multiple types $p$,
then \eqref{eq:foi.simple} can be generalized to:
\begin{equation}\label{eq:foi.strat}
  \lambda_i = \sum_{pi'h'} C_{pii'} \beta_{ph'} \frac{I_{i'h'}}{N_{i'}}
\end{equation}
\clearpage % TEMP
where:
$C_{pii'}$ is the average rate of type-$p$ contacts per-person among group $i$ with group $i'$,
$\beta_{ph'}$ is the average probability of transmission per type-$p$ contact given infection stage $h'$,
and $I_{i'h'}/N_{i'}$ is the prevalence of infection stage $h'$ among group $i'$.
Note that \eqref{eq:foi.strat} implicitly assumes that
contact rate and mixing by infection status/stage is random.
\par
The force of infection equation is further complicated by
repeated contacts with the same individuals, such as in sexual partnerships
(also household contacts, and other social relationships),
where each contact reflects a single ``sex act''.
With repeated \vs random contacts, individuals who recently acquired or transmitted infection
will continue to contact the same person, resulting in ``post-transmission contacts'' (PTC)
--- sometimes called ``wasted contacts'' (in terms of transmission) ---
and slower infection spread through the contact/partnership network.%
\footnote{Other conceptions of ``post-transmission contacts'' include:
  ``within-partnership competing risks'' and HIV seroconcordance.}
A major concern when modelling HIV transmission via sexual partnerships
is therefore to account for these PTC.
