A key equation in transmission models defines the force of infection $\lambda$:
the instantaneous rate at which susceptible individuals acquire infection.
In the simplest compartmental transmission models
--- \ie without any population stratification, repeated contacts, etc. ---
this rate is defined as:%
\footnote{\eqref{eq:foi.simple} also assumes
  frequency-dependent transmission \vs density-dependent transmission,
  which is almost always more appropriate for sexually-transmitted diseases \cite{Begon2002}.}
\begin{equation}\label{eq:foi.simple}
  \lambda = C \beta \frac{I}{N}
\end{equation}
where:
$C$ is the average contact rate per-person;
$\beta$ is the average probability of transmission per contact; and
$I/N$ is the current prevalence of infection.
\par
If the population is stratified into multiple groups $i$,
the infection is stratified into multiple infectious stages $h$,
and contacts are stratified into multiple types $p$,
then \eqref{eq:foi.simple} can be generalized to:
\begin{equation}\label{eq:foi.strat}
  \lambda_i = \sum_{pi'h'} C_{pii'} \beta_{ph'} \frac{I_{i'h'}}{N_{i'}}
\end{equation}
where:
$C_{pii'}$ is the average rate of type-$p$ contacts per-person among group $i$ with group $i'$,
$\beta_{ph'}$ is the average probability of transmission per type-$p$ contact given infection stage $h'$,
and $I_{i'h'}/N_{i'}$ is the prevalence of infection stage $h'$ among group $i'$.
Note that \eqref{eq:foi.strat} implicitly assumes that
contact rate and mixing by infection status/stage is random.
\par
The force of infection equation is further complicated by
repeated contacts with the same individuals, such as in sexual partnerships
(also household contacts, and other social relationships),
where each contact reflects a single ``sex act''.%
\footnote{Sex involving vaginal \emph{and} anal intercourse would be modelled as two sex acts.}
With repeated \vs random contacts, it is more likely that
individuals who recently acquired or transmitted infection will continue to contact the same person,
resulting in ``wasted contacts'' (in terms of transmission),%
\footnote{Another conception of ``wasted contacts'' is ``within-partnership competing risks''.}
and slower infection spread through the contact/partnership network.
\par
This chapter explores different mathematical approaches to
modelling HIV transmission within sexual partnerships in compartmental models,
considering these ``wasted contacts''.
In particular, I review prior approaches (\sref{foi.prior}),
develop a new approach (\sref{foi.prop}),
and compare the influence of each approach on selected model outputs (\sref{foi.exp}).
A preliminary version of this approach was presented in \cite{Knight2022-smdm}.
