\section{Parameterization}\label{model.par}
Model parameterization involves specification of parameter values (model inputs),
such as proportions, probabilities, rates, and ratios.
These parameters are used to define
the initial conditions and rates of transition between modelled states.
Some transition rates are fixed, such as rates of progression through HIV infection stages,
while other transition rates vary with time, such as rates of HIV diagnosis and ART initiation.
The transition rate for susceptible-to-infected --- \ie \emph{transmission} ---
is defined by a complex ``force of infection'' equation which mechanistically integrates
biological, behavioural, and interventional determinants of transmission;
Chapter~\ref{foi} develops this equation in detail,
including modifications to overcome limitations of prior equations.
\par
The true values of many parameters are uncertain due to several potential sources of error.
Thus, prior distributions are specified for 73 parameters (Table~\ref{tab:par.defs}),
and the joint posterior distributions of these uncertain parameters are then inferred
via model calibration, as described in \sref{model.cal}~and~\ref{model.res}.
Proportions and probabilities were generally modelled using
a beta approximation of the binomial distribution (BAB, see \sref{app.model.math.bab}),
while rates and ratios were generally modelled using
a gamma, skewnormal, or inverse gaussian distribution to guarantee positive values.
\section{Parameterization}\label{model.par}
As described in \sref{intro.model.param}, model parameterization involves
specification of model parameter values, such as proportions, probabilities, rates, and ratios,
including stratified values to reflect heterogeneity,
and sampling distributions to reflect uncertainty.
Proportions and probabilities were generally modelled using
a beta approximation of the binomial distribution (BAB, see \sref{app.math.distr.bab}),
while rates and ratios were generally modelled using
a gamma, skewnormal, or inverse gaussian distribution.
\paragraph{Notation}
If $X$ is a parameter stratified by dimensions $a,b,c$,
then $X_{ab_{1}c_{23}}$ denotes the values of $X$ for
a particular but \emph{unspecified} stratum of $a$,
the \emph{specific} stratum $b = 1$,
and the \emph{aggregated} strata $c = 2,3$
(the aggregating operation is context-dependent, \eg sum for probabilities).
Additionally, the indices $sihc$ from Table~\ref{tab:model.dims} denote ``self'' strata,
whereas $s'i'h'c'$ denote ``other'' strata --- \ie individuals' partners.%
\footnote{\label{foot:code.note}%
  In the code: R uses one-based indexing, which match the notation here directly,
  while Python uses zero-based indexing, which therefore appear as $i \rightarrow i-1$ in the code.
  Also, the model code reorders states in the ART Cascade dimension for computational efficiency,
  with $c={}$1:~Undiagnosed; 2:~Diagnosed; 3:~Virally~Un-suppressed; 4:~On~ART; 5:~Virally~Suppressed.}
Finally, I re-use several dummy variables throughout the chapter:
$\rho$ for proportions, $\lambda$ for rates, $T$ for time periods, and $f$ for constants.
%===================================================================================================
\subsection{Risk Heterogeneity Among FSW}\label{model.par.fsw}
Existing HIV transmission models which include FSW
have rarely sub-stratified this population, such as to reflect
differential HIV risk or distinct typologies of sex work \cite{Blanchard2008,Scorgie2012};
yet such heterogeneities may influence transmission dynamics.
Among the studies identified in Chapter~\ref{sr},
only three sub-stratified FSW by risk-related factors:
\citet{Cremin2017} defined three levels of risk via regression analysis,
\citet{Low2015} distinguished between occasional and full-time FSW, while
\citet{Shannon2015} sub-stratified FSW by
work environment, violence exposure, and context-specific structural factors.
Seven other studies, reflecting two unique models \cite{Johnson2012,Maheu-Giroux2017},
employed age stratification of all activity groups, including FSW;
these models had several risk-related parameters which varied by age.
\par
The model structure here (Figure~\ref{fig:model.risk})
was designed to capture \emph{within}-FSW risk heterogeneity.
The objective of the following analysis was therefore to parameterize
lower \vs higher risk FSW.
I sought to define these groups based on biobehavioural and/or contextual factors
which are demonstrably associated with HIV risk,
and which can be mechanistically incorporated into a transmission model ---
\ie through the force of infection equation.
Later, the parameterization of these groups was validated through model fitting
to relative differences in HIV prevalence \sref{model.cal.targ.prev}.
\par
Many cross-sectional studies of HIV among FSW quantify
the association of risk factors with HIV serostatus
\cite{Aklilu2001,Dunkle2005,Scorgie2012,Jonas2020}.
However, serostatus reflects cumulative risk exposure,
whereas sexual risk behaviour is dynamic \cite{Watts2010,vanWees2020},
as is use of prevention resources \cite{Roberts2020}.
For example, while HIV prevalence often increases with age,
HIV incidence among women can peak before age 25 \cite{Dellar2015}.
Thus, risk factors associated with HIV serostatus are not necessarily
mechanistically related to HIV acquisition.
Indeed, FSW may reduce risk behaviours in response to seroconversion \cite{McClelland2006}.
Cohort studies that measure incidence
can help identify risk factors for HIV acquisition \cite{McKinnon2015,Nouaman2022},
but large sample sizes are often required to accurately estimate overall incidence rate,
let alone risk factors \cite{Priddy2011}.
%---------------------------------------------------------------------------------------------------
\subsubsection{FSW Survey Data}\label{model.par.fsw.data}
Three biobehavioural surveys, in
2011 \cite{Baral2014} (N = 325),
2014 \cite{EswKP2014} (N = 781), and
2021 \cite{EswIBBS2022} (N = 676)
provide HIV status and biobehavioural data on FSW in Eswatini.
The 2011 and 2021 surveys featured serologic HIV testing,
and employed respondent driven sampling (RDS, details in \cite{Yam2013}).
The 2014 survey relied on self-reported HIV status,
andd employed venue-based snowball sampling, based on the
Priorities for Local AIDS Control Efforts (PLACE) methodology,
which aims to identify areas of higher incidence \cite{Weir2005}.
More details about each study are given in \sref{intro.esw.hiv.data} and Table~\ref{tab:esw.data}.
I analyzed the individual-level data from 2011 and 2014 (data from 2021 not yet available)
to explore the potential association of biobehavioural factors with HIV risk,
so that such factors could then be used to distinguish between
lower risk \vs higher risk FSW.
% TODO: (*) add descriptive table
%---------------------------------------------------------------------------------------------------
\subsubsection{HIV Status}\label{model.par.fsw.hiv}
Only the 2011 and 2021 studies included serologic testing for HIV.
Among those tested in 2011 (N = 317, 98\%), 70\% were \hivp,
yielding RDS-adjusted prevalence estimate of 61\% (CI: 51--71\%) \cite{Baral2014}.
Among serologically \hivn, 11\% self-reported \hivp status (false positive), and
among serologically \hivp, 26\% self-reported \hivn status (false negative or undiagnosed).
Overall, self-reported HIV status underestimated HIV prevalence in 2011
by a factor of approximately 0.78 (55~vs~70\%).
Unadjusted HIV prevalence in 2021 was 58.8\%,
with 88\% (363/411) reporting previous awareness of \hivp status.
\par
In 2014, self-reported HIV prevalence was 38\% among respondents who reported (85\%).
This 38\% is surprisingly low considering that
the PLACE methodology explicitly aimed to sample venues
with higher HIV incidence \cite{Weir2005}, and 2014 \vs 2011 respondents
were older (median 27 \vs 25 years), % 2021 median: 28
had been selling sex longer (median 5 \vs 4 years), % 2021: 6
and tested more frequently (87 \vs 75\% tested at least once in the past year, % 2021: 75
82 \vs 63\% among self-reported \hivn).
Perhaps the differences are attributable to the sampling methodology.
Among respondents who self-reported \hivp status,
the 2014 survey also asked for age of HIV diagnosis (6\% missing).
Age of HIV diagnosis supports crude time-to-event analysis (next section),
which can account for confounding by age and censoring,
as compared to logistic regression on HIV status,
keeping in mind the limitations of self-reported HIV status.
%---------------------------------------------------------------------------------------------------
\subsubsection{Risk Factors for HIV}\label{model.par.fsw.fac}
Next, I explored the potential association of risk factors with HIV
via the following three models:%
\footnote{Logistic regression models were implemented using \texttt{lrm} from:
  \hreftt{cran.r-project.org/package=rms}.\\
Cox proportional hazards models were implemented using \texttt{coxaalen} from:
  \hreftt{cran.r-project.org/package=coxinterval}.}
\begin{enumerate}
  \item Logistic regression on serologic HIV status (2011 data)
  \item Logistic regression on self-reported HIV status (2014 data)
  \item Cox proportional hazards for interval-censored time to HIV infection,
    with interval from self-reported sex work debut 
    to either self-reported time of HIV diagnosis or survey date (2014 data);
    Figure~\ref{fig:fsw.tte.interval} illustrates
    the four potential censoring cases in this framework.
\end{enumerate}
An important limitation to all models is that
risk factors reported by FSW at the time of survey
are assumed to be fixed characteristics of the respondents,
rather than dynamic characteristics that vary over time.
Additionally, respondents with any missing variables for each individual model
were excluded from that model. % TODO: (%)
\begin{figure}
  \centering
  \includegraphics[scale=1]{diag.tte}
  \caption{Illustration of time-to-event analysis framework
    for cross-sectional FSW survey data}
  \label{fig:fsw.tte.interval}
  \floatfoot{
    $\bm{\times}$: HIV infection;
    SW: time of sex work debut;
    Dx: time of HIV diagnosis.}
\end{figure}
\par
Risk factors were selected based on
prior knowledge of plausible mechanistic influence on HIV incidence and/or prevalence.
The risk factors explored are summarized in Table~\ref{tab:fsw.stats},
including univariate and multivariable association under each model.
Variable selection for multivariable models
was performed using backward selection as described by \citet{Lawless1978},
using a $p \le 0.1$ (per variable) threshold for stepwise variable retention.
Estimated conditional effects of
variables retained in the multivariable logistic regression models
are illustrated in Figure~\ref{fig:fsw.lr}.
\begin{table}
  \centering
  \caption{Risk factors explored for association with \hivp status among FSW in Eswatini}
  \label{tab:fsw.stats}
  \centerline{%
\small%
\begin{tabular}{lcccccccccccc}
  \toprule
  & \multicolumn{4}{c}{2011 LR}
  & \multicolumn{4}{c}{2014 LR}
  & \multicolumn{4}{c}{2014 CPH} \\
  \cmidrule(rl){2-5}\cmidrule(rl){6-9}\cmidrule(rl){10-13}
  & \multicolumn{2}{c}{Univar} & \multicolumn{2}{c}{Multivar}
  & \multicolumn{2}{c}{Univar} & \multicolumn{2}{c}{Multivar}
  & \multicolumn{2}{c}{Univar} & \multicolumn{2}{c}{Multivar} \\
  \cmidrule(rl){2-3}\cmidrule(rl){4-5}\cmidrule(rl){6-7}\cmidrule(rl){8-9}\cmidrule(rl){10-11}\cmidrule(rl){12-13}
  Factor                          &  OR  &   p   &  OR  &   p    &  OR  &   p    &  OR  &   p    &  HR  &   p    &  HR  &   p    \\
  \midrule                        % 2011 LR uni  % 2011 LR multi % 2014 LR uni   % 2014 LR multi % 2014 CPH uni  % 2014 CPH multi
  Age\tn{a}                       & 1.11 & \vsig & ---  &  ---   & 1.14 & \vsig  & 1.15 & \vsig  & 1.09 & \vsig  & 1.09 & \vsig  \\
  Years selling sex\tn{a}         & 1.13 & \vsig & 1.13 & \vsig  & 1.12 & \vsig  & ---  &  ---   & 1.08 & \vsig  & ---  &  ---   \\
  Monthly sex work income\tn{b}   & 0.98 & 0.155 & ---  &  ---   & 0.98 & 0.097  & 0.97 & 0.084  & 0.98 & 0.019\s& 0.97 & 0.001\s\\[1ex]
  Non-paying partners\tn{c}       & 0.88 & 0.307 & ---  &  ---   & 1.07 & 0.233  & ---  &  ---   & 1.05 & 0.312  & ---  &  ---   \\
  Monthly new clients\tn{c}       & 1.01 & 0.412 & ---  &  ---   & 1.05 & \vsig  & 1.07 & \vsig  & 1.04 & \vsig  & 1.04 & \vsig  \\
  Monthly regular clients\tn{c}   & 1.01 & 0.351 & ---  &  ---   & 1.03 & 0.002  & ---  &  ---   & 1.02 & \vsig  & 1.02 & 0.034\s\\[1ex]
  Non-paying condom use\tn{d}     & 0.90 & 0.703 & ---  &  ---   & 0.90 & 0.673  & ---  &  ---   & 0.92 & 0.677  & ---  &  ---   \\
  New client condom use\tn{d}     & 0.60 & 0.100 & ---  &  ---   & 0.48 & 0.006\s& 1.25 & 0.599  & 0.56 & 0.004\s& ---  &  ---   \\
  Regular client condom use\tn{d} & 0.58 & 0.110 & ---  &  ---   & 0.39 & \vsig  & 0.35 & 0.004\s& 0.49 & \vsig  & 0.50 & \vsig  \\[1ex]
  Any anal sex past month         & 0.97 & 0.896 & ---  &  ---   & 1.89 & 0.015\s& ---  &  ---   & 1.57 & 0.015\s& 1.27 & 0.260  \\
  Any STI symptoms past year      & 2.29 & \vsig & 2.41 & \vsig  & 2.75 & \vsig  & 2.80 & \vsig  & 2.17 & \vsig  & 2.05 & \vsig  \\
  \bottomrule
\end{tabular}}
% TODO: HIV status?
\floatfoot{\raggedright
  \tnt[a]{OR per year};
  \tnt[b]{OR per Swazi lilangeni per month};
  \tnt[c]{OR per partner};
  \tnt[d]{2011: always vs not always, 2014: at last sex}.
  --- indicates variable was not selected in the multivariate model.
  LR: logistic regression on HIV$+/-$ status;
  CPH: Cox proportional hazards on time to self-reported HIV seroconversion.
  OR: odds ratio; HR: hazard ratio; p: p-value.
  2011 data based on serologic HIV test;
  2014 data based on self-reported HIV status, age of sex work debut, and age of HIV diagnosis.
}
\end{table}
\begin{figure}[h]
  \subcapoverlap
  \foreach \year/\var/\nvar in {2011/f/1,2011/c/2,2014/f/3,2014/c/3}{
  \begin{subfigure}{\nvar\linewidth/5+\linewidth/5}
    \includegraphics[scale=.7]{fsw.\year.lr.hiv.\var}
    \caption{\raggedright}
    \label{fig:fsw.lr.\year.\var}
  \end{subfigure}}
  \caption{Predicted conditional effects (probability)
    of variables in multivariable logistic regression models for HIV status}
  \label{fig:fsw.lr}
  \floatfoot{\fffsw{fig:fsw.lr}
    conditional probabilities shown for fixed covariates at arbitrary values.}
\end{figure}
\par
Following variable selection, each multivariable model was used to estimate
the total \hivp status odds ratio (logistic) or HIV incidence hazard ratio (Cox)
for each respondent in the respective survey ---
\ie $e^{X_i\,\beta}$ for respondent $i$ ---
representing an overall ``risk score'' under each model.
Respondents were then stratified into the top 20\% and bottom 80\% by these risk scores.
The values of each variable were compared between these two strata
using a test for the ratio of the means \cite{Tamhane2004} to support model parameterization;
these ratios are summarized in Table~\ref{tab:fsw.ratios},
and the distributions of variable values across the two strata
are illustrated in Figure~\ref{fig:fsw.f}.
\begin{table}
  \centering
  \caption{Ratios of HIV risk factor variables among higher \vs lower risk FSW in Eswatini}
  \label{tab:fsw.ratios}
  \centerline{\footnotesize%
\begin{tabular}{lcccccc}
  \toprule
  & \multicolumn{2}{c}{2011 LR}
  & \multicolumn{2}{c}{2014 LR}
  & \multicolumn{2}{c}{2014 CPH} \\
  \cmidrule(rl){2-3}\cmidrule(rl){4-5}\cmidrule(rl){6-7}
  Factor                            &  High / Low   &   Ratio (95\% CI)   &  High / Low   &   Ratio (95\% CI)   &   High / Low   &   Ratio (95\% CI)   \\
  \midrule
  Age                               & 31.8  / 24.7  & 1.29 (1.22, 1.36)\s & 32.6  / 26.2  & 1.24 (1.20, 1.28)\s &  33.5  / 26.6  & 1.26 (1.21, 1.31)\s \\
  Years selling sex                 & 11.3  /  4.03 & 2.81 (2.41, 3.25)\s & 10.0  /  5.47 & 1.83 (1.64, 2.03)\s &  10.2  /  5.83 & 1.75 (1.54, 1.98)\s \\
  Monthly sex work income\tn{a}     & 15.1  / 15.2  & 1.00 (0.86, 1.15)   &  6.77 /  7.06 & 0.96 (0.82, 1.11)   &   6.32 /  7.28 & 0.87 (0.73, 1.02)   \\[1ex]
  Non-paying partners               &  1.42 /  1.43 & 0.99 (0.81, 1.19)   &  1.56 /  1.11 & 1.40 (1.11, 1.72)\s &   1.53 /  1.19 & 1.29 (0.98, 1.62)   \\
  Monthly new clients               &  5.50 /  6.98 & 0.79 (0.49, 1.15)   &  8.39 /  4.15 & 2.02 (1.63, 2.44)\s &   8.36 /  4.41 & 1.90 (1.43, 2.39)\s \\
  Monthly regular clients           &  9.35 /  9.05 & 1.03 (0.69, 1.42)   & 11.1  /  8.25 & 1.35 (1.13, 1.57)\s &  12.4  /  8.61 & 1.44 (1.18, 1.71)\s \\[1ex]
  Non-paying condom use\tn{bc}      &  0.26 /  0.35 & 0.73 (0.40, 1.11)   &  0.77 /  0.81 & 0.95 (0.84, 1.06)   &   0.76 /  0.81 & 0.95 (0.81, 1.08)   \\
  New client condom use\tn{bc}      &  0.68 /  0.76 & 0.89 (0.73, 1.06)   &  0.79 /  0.91 & 0.86 (0.79, 0.94)\s &   0.74 /  0.94 & 0.79 (0.69, 0.88)\s \\
  Regular client condom use\tn{bc}  &  0.38 /  0.46 & 0.83 (0.45, 1.28)   &  0.67 /  0.91 & 0.74 (0.65, 0.82)\s &   0.60 /  0.92 & 0.65 (0.55, 0.75)\s \\[1ex]
  Any anal sex past month           &  0.59 /  0.41 & 1.41 (1.06, 1.84)\s &  0.17 /  0.07 & 2.43 (1.47, 3.85)\s &   0.23 /  0.07 & 3.24 (1.95, 5.34)\s \\
  Any STI symptoms past year\tn{c}  &  0.79 /  0.43 & 1.86 (1.54, 2.25)\s &  0.59 /  0.15 & 3.94 (3.15, 5.03)\s &   0.61 /  0.17 & 3.67 (2.87, 4.79)\s \\[1ex]
  HIV prevalence\tn{d}              &  0.94 /  0.64 & 1.46 (1.30, 1.63)\s &  0.66 /  0.29 & 2.29 (1.92, 2.75)\s &   0.71 /  0.31 & 2.32 (1.94, 2.80)\s \\
  \bottomrule
\end{tabular}}
% TODO: HIV status?
\floatfoot{\raggedright
  High / Low: mean variable value among higher / lower risk groups, as defined by
  the top 20\% / bottom 80\% in multivariable model-predicted risk score:
  odds ratio from logistic regression (LR);
  hazards ratio from Cox proportional hazards (CPH).
  \tnt[a]{Swati lilangeni per month};
  \tnt[b]{2011: always \vs not always, 2014: did use condom at last sex};
  \tnt[c]{proportion of respondents};
  \tnt[d]{2011: serologic HIV status; 2014: self-reported HIV status};
  \tnt[*]{statistically significant, $p < 0.05$}.
}
\end{table}

\section{Parameterization}\label{model.par}
As described in \sref{intro.model.param}, model parameterization involves
specification of model parameter values, such as proportions, probabilities, rates, and ratios,
including stratified values to reflect heterogeneity,
and sampling distributions to reflect uncertainty.
Proportions and probabilities were generally modelled using
a beta approximation of the binomial distribution (BAB, see \sref{app.math.distr.bab}),
while rates and ratios were generally modelled using
a gamma, skewnormal, or inverse gaussian distribution.
\paragraph{Notation}
If $X$ is a parameter stratified by dimensions $a,b,c$,
then $X_{ab_{1}c_{23}}$ denotes the values of $X$ for
a particular but \emph{unspecified} stratum of $a$,
the \emph{specific} stratum $b = 1$,
and the \emph{aggregated} strata $c = 2,3$
(the aggregating operation is context-dependent, \eg sum for probabilities).
Additionally, the indices $sihc$ from Table~\ref{tab:model.dims} denote ``self'' strata,
whereas $s'i'h'c'$ denote ``other'' strata --- \ie individuals' partners.%
\footnote{\label{foot:code.note}%
  In the code: R uses one-based indexing, which match the notation here directly,
  while Python uses zero-based indexing, which therefore appear as $i \rightarrow i-1$ in the code.
  Also, the model code reorders states in the ART Cascade dimension for computational efficiency,
  with $c={}$1:~Undiagnosed; 2:~Diagnosed; 3:~Virally~Un-suppressed; 4:~On~ART; 5:~Virally~Suppressed.}
Finally, I re-use several dummy variables throughout the chapter:
$\rho$ for proportions, $\lambda$ for rates, $T$ for time periods, and $f$ for constants.
%===================================================================================================
\subsection{Risk Heterogeneity Among FSW}\label{model.par.fsw}
Existing HIV transmission models which include FSW
have rarely sub-stratified this population, such as to reflect
differential HIV risk or distinct typologies of sex work \cite{Blanchard2008,Scorgie2012};
yet such heterogeneities may influence transmission dynamics.
Among the studies identified in Chapter~\ref{sr},
only three sub-stratified FSW by risk-related factors:
\citet{Cremin2017} defined three levels of risk via regression analysis,
\citet{Low2015} distinguished between occasional and full-time FSW, while
\citet{Shannon2015} sub-stratified FSW by
work environment, violence exposure, and context-specific structural factors.
Seven other studies, reflecting two unique models \cite{Johnson2012,Maheu-Giroux2017},
employed age stratification of all activity groups, including FSW;
these models had several risk-related parameters which varied by age.
\par
The model structure here (Figure~\ref{fig:model.risk})
was designed to capture \emph{within}-FSW risk heterogeneity.
The objective of the following analysis was therefore to parameterize
lower \vs higher risk FSW.
I sought to define these groups based on biobehavioural and/or contextual factors
which are demonstrably associated with HIV risk,
and which can be mechanistically incorporated into a transmission model ---
\ie through the force of infection equation.
Later, the parameterization of these groups was validated through model fitting
to relative differences in HIV prevalence \sref{model.cal.targ.prev}.
\par
Many cross-sectional studies of HIV among FSW quantify
the association of risk factors with HIV serostatus
\cite{Aklilu2001,Dunkle2005,Scorgie2012,Jonas2020}.
However, serostatus reflects cumulative risk exposure,
whereas sexual risk behaviour is dynamic \cite{Watts2010,vanWees2020},
as is use of prevention resources \cite{Roberts2020}.
For example, while HIV prevalence often increases with age,
HIV incidence among women can peak before age 25 \cite{Dellar2015}.
Thus, risk factors associated with HIV serostatus are not necessarily
mechanistically related to HIV acquisition.
Indeed, FSW may reduce risk behaviours in response to seroconversion \cite{McClelland2006}.
Cohort studies that measure incidence
can help identify risk factors for HIV acquisition \cite{McKinnon2015,Nouaman2022},
but large sample sizes are often required to accurately estimate overall incidence rate,
let alone risk factors \cite{Priddy2011}.
%---------------------------------------------------------------------------------------------------
\subsubsection{FSW Survey Data}\label{model.par.fsw.data}
Three biobehavioural surveys, in
2011 \cite{Baral2014} (N = 325),
2014 \cite{EswKP2014} (N = 781), and
2021 \cite{EswIBBS2022} (N = 676)
provide HIV status and biobehavioural data on FSW in Eswatini.
The 2011 and 2021 surveys featured serologic HIV testing,
and employed respondent driven sampling (RDS, details in \cite{Yam2013}).
The 2014 survey relied on self-reported HIV status,
andd employed venue-based snowball sampling, based on the
Priorities for Local AIDS Control Efforts (PLACE) methodology,
which aims to identify areas of higher incidence \cite{Weir2005}.
More details about each study are given in \sref{intro.esw.hiv.data} and Table~\ref{tab:esw.data}.
I analyzed the individual-level data from 2011 and 2014 (data from 2021 not yet available)
to explore the potential association of biobehavioural factors with HIV risk,
so that such factors could then be used to distinguish between
lower risk \vs higher risk FSW.
% TODO: (*) add descriptive table
%---------------------------------------------------------------------------------------------------
\subsubsection{HIV Status}\label{model.par.fsw.hiv}
Only the 2011 and 2021 studies included serologic testing for HIV.
Among those tested in 2011 (N = 317, 98\%), 70\% were \hivp,
yielding RDS-adjusted prevalence estimate of 61\% (CI: 51--71\%) \cite{Baral2014}.
Among serologically \hivn, 11\% self-reported \hivp status (false positive), and
among serologically \hivp, 26\% self-reported \hivn status (false negative or undiagnosed).
Overall, self-reported HIV status underestimated HIV prevalence in 2011
by a factor of approximately 0.78 (55~vs~70\%).
Unadjusted HIV prevalence in 2021 was 58.8\%,
with 88\% (363/411) reporting previous awareness of \hivp status.
\par
In 2014, self-reported HIV prevalence was 38\% among respondents who reported (85\%).
This 38\% is surprisingly low considering that
the PLACE methodology explicitly aimed to sample venues
with higher HIV incidence \cite{Weir2005}, and 2014 \vs 2011 respondents
were older (median 27 \vs 25 years), % 2021 median: 28
had been selling sex longer (median 5 \vs 4 years), % 2021: 6
and tested more frequently (87 \vs 75\% tested at least once in the past year, % 2021: 75
82 \vs 63\% among self-reported \hivn).
Perhaps the differences are attributable to the sampling methodology.
Among respondents who self-reported \hivp status,
the 2014 survey also asked for age of HIV diagnosis (6\% missing).
Age of HIV diagnosis supports crude time-to-event analysis (next section),
which can account for confounding by age and censoring,
as compared to logistic regression on HIV status,
keeping in mind the limitations of self-reported HIV status.
%---------------------------------------------------------------------------------------------------
\subsubsection{Risk Factors for HIV}\label{model.par.fsw.fac}
Next, I explored the potential association of risk factors with HIV
via the following three models:%
\footnote{Logistic regression models were implemented using \texttt{lrm} from:
  \hreftt{cran.r-project.org/package=rms}.\\
Cox proportional hazards models were implemented using \texttt{coxaalen} from:
  \hreftt{cran.r-project.org/package=coxinterval}.}
\begin{enumerate}
  \item Logistic regression on serologic HIV status (2011 data)
  \item Logistic regression on self-reported HIV status (2014 data)
  \item Cox proportional hazards for interval-censored time to HIV infection,
    with interval from self-reported sex work debut 
    to either self-reported time of HIV diagnosis or survey date (2014 data);
    Figure~\ref{fig:fsw.tte.interval} illustrates
    the four potential censoring cases in this framework.
\end{enumerate}
An important limitation to all models is that
risk factors reported by FSW at the time of survey
are assumed to be fixed characteristics of the respondents,
rather than dynamic characteristics that vary over time.
Additionally, respondents with any missing variables for each individual model
were excluded from that model. % TODO: (%)
\begin{figure}
  \centering
  \includegraphics[scale=1]{diag.tte}
  \caption{Illustration of time-to-event analysis framework
    for cross-sectional FSW survey data}
  \label{fig:fsw.tte.interval}
  \floatfoot{
    $\bm{\times}$: HIV infection;
    SW: time of sex work debut;
    Dx: time of HIV diagnosis.}
\end{figure}
\par
Risk factors were selected based on
prior knowledge of plausible mechanistic influence on HIV incidence and/or prevalence.
The risk factors explored are summarized in Table~\ref{tab:fsw.stats},
including univariate and multivariable association under each model.
Variable selection for multivariable models
was performed using backward selection as described by \citet{Lawless1978},
using a $p \le 0.1$ (per variable) threshold for stepwise variable retention.
Estimated conditional effects of
variables retained in the multivariable logistic regression models
are illustrated in Figure~\ref{fig:fsw.lr}.
\begin{table}
  \centering
  \caption{Risk factors explored for association with \hivp status among FSW in Eswatini}
  \label{tab:fsw.stats}
  \input{model/tab.fsw.factor.stats}
\end{table}
\begin{figure}[h]
  \subcapoverlap
  \foreach \year/\var/\nvar in {2011/f/1,2011/c/2,2014/f/3,2014/c/3}{
  \begin{subfigure}{\nvar\linewidth/5+\linewidth/5}
    \includegraphics[scale=.7]{fsw.\year.lr.hiv.\var}
    \caption{\raggedright}
    \label{fig:fsw.lr.\year.\var}
  \end{subfigure}}
  \caption{Predicted conditional effects (probability)
    of variables in multivariable logistic regression models for HIV status}
  \label{fig:fsw.lr}
  \floatfoot{\fffsw{fig:fsw.lr}
    conditional probabilities shown for fixed covariates at arbitrary values.}
\end{figure}
\par
Following variable selection, each multivariable model was used to estimate
the total \hivp status odds ratio (logistic) or HIV incidence hazard ratio (Cox)
for each respondent in the respective survey ---
\ie $e^{X_i\,\beta}$ for respondent $i$ ---
representing an overall ``risk score'' under each model.
Respondents were then stratified into the top 20\% and bottom 80\% by these risk scores.
The values of each variable were compared between these two strata
using a test for the ratio of the means \cite{Tamhane2004} to support model parameterization;
these ratios are summarized in Table~\ref{tab:fsw.ratios},
and the distributions of variable values across the two strata
are illustrated in Figure~\ref{fig:fsw.f}.
\begin{table}
  \centering
  \caption{Ratios of HIV risk factor variables among higher \vs lower risk FSW in Eswatini}
  \label{tab:fsw.ratios}
  \input{model/tab.fsw.factor.ratios}
\end{table}

\section{Parameterization}\label{model.par}
As described in \sref{intro.model.param}, model parameterization involves
specification of model parameter values, such as proportions, probabilities, rates, and ratios,
including stratified values to reflect heterogeneity,
and sampling distributions to reflect uncertainty.
Proportions and probabilities were generally modelled using
a beta approximation of the binomial distribution (BAB, see \sref{app.math.distr.bab}),
while rates and ratios were generally modelled using
a gamma, skewnormal, or inverse gaussian distribution.
\paragraph{Notation}
If $X$ is a parameter stratified by dimensions $a,b,c$,
then $X_{ab_{1}c_{23}}$ denotes the values of $X$ for
a particular but \emph{unspecified} stratum of $a$,
the \emph{specific} stratum $b = 1$,
and the \emph{aggregated} strata $c = 2,3$
(the aggregating operation is context-dependent, \eg sum for probabilities).
Additionally, the indices $sihc$ from Table~\ref{tab:model.dims} denote ``self'' strata,
whereas $s'i'h'c'$ denote ``other'' strata --- \ie individuals' partners.%
\footnote{\label{foot:code.note}%
  In the code: R uses one-based indexing, which match the notation here directly,
  while Python uses zero-based indexing, which therefore appear as $i \rightarrow i-1$ in the code.
  Also, the model code reorders states in the ART Cascade dimension for computational efficiency,
  with $c={}$1:~Undiagnosed; 2:~Diagnosed; 3:~Virally~Un-suppressed; 4:~On~ART; 5:~Virally~Suppressed.}
Finally, I re-use several dummy variables throughout the chapter:
$\rho$ for proportions, $\lambda$ for rates, $T$ for time periods, and $f$ for constants.
\input{model/par.fsw}
\input{model/par.beta}
\input{model/par.hiv}
\input{model/par.popsex}
%===================================================================================================
\subsection{HIV Progression \& Mortality}\label{model.par.hiv}
%---------------------------------------------------------------------------------------------------
\subsubsection{HIV Progression}\label{model.par.hiv.dur}
The length of time spent in each HIV stage is related to
rates of progression between stages $\eta_{h}$,
rates of additional HIV-attributable mortality by stage $\mu_{\textsc{hiv},h}$,
and treatment via antiretroviral therapy (ART).
\citet{Lodi2011} estimate median times from seroconversion to
CD4 $<$ 500, $<$ 350, and $<$ 200 cells/mm\tsup{3}, while
\citet{Mangal2017} directly estimate the rates of progression between CD4 states $\eta_{h}$
in a simple compartmental model.
Based on these data, I modelled mean durations ($1/\eta_{h}$) of:%
\footnote{Assuming exponential distributions for durations in each CD4 state
  (see \sref{app.model.math.comp} for more details).}
0.142 years in acute infection ($h=2$, from \sref{model.par.beta.hiv});
3.35 years in CD4~$>$~500 ($h=3$);
3.74 years in 350~$<$~CD4~$<$~500 ($h=4$); and
5.26 years in 200~$<$~CD4~$<$~350 ($h=5$); plus
the remaining time until death in CD4~$<$~200 ($h=6$, AIDS).
Since the duration in acute infection ($h=2$) is randomly sampled,
the remaining duration in CD4~$>$~500 ($h=3$) is adjusted accordingly.
%---------------------------------------------------------------------------------------------------
\subsubsection{HIV Mortality}\label{model.par.hiv.mort}
Mortality rates by CD4-count in the absence of ART were estimated in
multiple African studies \cite{Badri2006,Anglaret2012,Mangal2017};
based on these data, I estimated yearly HIV-attributable mortality rates $\mu_{\textsc{hiv},h}$ as:
0 during acute phase ($h=2$);
0.4\% during CD4~$>$~500 ($h=3$);
2\% during 350~$<$~CD4~$<$~500 ($h=4$);
4\% during 200~$<$~CD4~$<$~350 ($h=5$); and 
20\% during CD4~$<$~200 ($h=6$, AIDS).
%===================================================================================================
\subsection{Antiretroviral Therapy}\label{model.par.art}
Viral suppression via antiretroviral therapy (ART) influences
the probability of HIV transmission, as well as rates of HIV progression and HIV-related mortality.
The model considers individuals on ART before ($c=4$) and after ($c=5$)
achieving full viral load suppression (VLS), as defined by undetectable HIV RNA in blood samples.
Among retained patients initiating ART, time to VLS
is usually described as ``within 6 months'' \cite{Thompson2012}.
More specifically, \citet{Mujugira2016} estimate the median time to VLS as 3 months,
yielding an estimated \emph{mean} duration for $c=4$ of 4.3 months (see \sref{app.model.math.comp}).
%---------------------------------------------------------------------------------------------------
\subsubsection{Probability of HIV Transmission}\label{model.par.art.beta}
All available evidence suggests that viral suppression by ART to undetectable levels
prevents HIV transmission, \ie undetectable = untransmittable (``U=U'') \cite{Eisinger2019}.
Thus, I assumed zero HIV transmission from individuals with VLS ($c=5$).
However, HIV transmission may still occur
during the period between ART initiation to viral suppression ($c=4$) \cite{Mujugira2016}.
\citet{Donnell2010} estimate an adjusted incidence ratio of 0.08~(0.0,~0.57) for all individuals on ART.
However, in \cite{Donnell2010} and \cite{Cohen2016}, the 1 and 4 (respectively)
genetically linked infections from individuals on ART all occurred within 90 days of ART initiation,
suggesting that risk of transmission only persists before viral suppression.
Adjusting the incidence denominator (person-time)
to 90 days per individual who initiated ART in \cite{Donnell2010}
results in approximately 3.13 times higher estimated incidence ratio: 0.25 for this specific period.%
\footnote{In \cite{Donnell2010}, individuals who initiated ART contributed
  approximately 9.4 months per-person (273 persons / 349 person-years, Tables~2~and~3);
  thus the first 3 months of each individual represent
  3/9.4 = 0.319 fewer person-months of follow-up.}
Thus, I sampled relative infectiousness on ART but before viral suppression ($c=4$)
from a beta distribution with mean (95\%~CI) of 0.25~(0.01,~0.67).
%---------------------------------------------------------------------------------------------------
\subsubsection{HIV Progression \& Mortality}\label{model.par.art.hiv}
\def\hunprog{$h = 6 \rightarrow 5 \rightarrow 4 \rightarrow 3$\xspace}
Effective ART stops CD4 cell decline and results in some CD4 recovery \cite{Battegay2006,Lawn2006}.
Most CD4 recovery occurs within the first year of treatment \cite{Battegay2006}.
Due to the limited number of modelled treatment states,
I model this initial recovery to be associated with the 4.3-month pre-VLS ART state ($c=4$).
\citet{Lawn2006,Gabillard2013} estimate an increase of between 25--39 cells/mm\tsup{3} per month
during the first 3 months of treatment.
Since HIV states $h=4,5,6$ correspond to 150, 150, and 200-wide CD4 strata,
I model rates of movement along \hunprog during pre-VLS ART ($c=4$) as
0.20, 0.20, 0.17 per month, respectively.
After initial increases, CD4 recovery is modest and plateaus.
\citet{Battegay2006} report approximate increases of
22.4 cells/mm\tsup{3} per year between years 1 and 5 on ART.
Thus, I model rates of movement along \hunprog after VLS ($c=5$) as 0.15 per year.
\par
Since higher CD4 states are modelled to have lower mortality rates (see \sref{model.par.hiv.mort}),
the modelled recovery of CD4 cells via ART described above implicitly affords a mortality benefit.
However, HIV infection is associated with increased risk of death by non-AIDS causes
--- \ie unrelated to CD4 count ---
including cardiovascular disease and renal disease \cite{Phillips2008}.
\citet{Lundgren2015} estimated 61\% reduction in non-AIDS life-threatening events due to ART.
For the same CD4 strata, \citet{Gabillard2013} also report approximately 2-times higher
mortality rates within the first year of ART versus thereafter,
suggesting that VLS is associated with 50\% mortality reduction independent of CD4 increase.
Thus, I modelled an additional 50\% reduction in mortality among individuals with VLS ($c=5$),
and half this (25\%) reduction before achieving VLS ($c=4$).
% TODO: rates of diagnosis, testing, vls
\section{Parameterization}\label{model.par}
As described in \sref{intro.model.param}, model parameterization involves
specification of model parameter values, such as proportions, probabilities, rates, and ratios,
including stratified values to reflect heterogeneity,
and sampling distributions to reflect uncertainty.
Proportions and probabilities were generally modelled using
a beta approximation of the binomial distribution (BAB, see \sref{app.math.distr.bab}),
while rates and ratios were generally modelled using
a gamma, skewnormal, or inverse gaussian distribution.
\paragraph{Notation}
If $X$ is a parameter stratified by dimensions $a,b,c$,
then $X_{ab_{1}c_{23}}$ denotes the values of $X$ for
a particular but \emph{unspecified} stratum of $a$,
the \emph{specific} stratum $b = 1$,
and the \emph{aggregated} strata $c = 2,3$
(the aggregating operation is context-dependent, \eg sum for probabilities).
Additionally, the indices $sihc$ from Table~\ref{tab:model.dims} denote ``self'' strata,
whereas $s'i'h'c'$ denote ``other'' strata --- \ie individuals' partners.%
\footnote{\label{foot:code.note}%
  In the code: R uses one-based indexing, which match the notation here directly,
  while Python uses zero-based indexing, which therefore appear as $i \rightarrow i-1$ in the code.
  Also, the model code reorders states in the ART Cascade dimension for computational efficiency,
  with $c={}$1:~Undiagnosed; 2:~Diagnosed; 3:~Virally~Un-suppressed; 4:~On~ART; 5:~Virally~Suppressed.}
Finally, I re-use several dummy variables throughout the chapter:
$\rho$ for proportions, $\lambda$ for rates, $T$ for time periods, and $f$ for constants.
\input{model/par.fsw}
\input{model/par.beta}
\input{model/par.hiv}
\input{model/par.popsex}
%===================================================================================================
\subsection{HIV Progression \& Mortality}\label{model.par.hiv}
%---------------------------------------------------------------------------------------------------
\subsubsection{HIV Progression}\label{model.par.hiv.dur}
The length of time spent in each HIV stage is related to
rates of progression between stages $\eta_{h}$,
rates of additional HIV-attributable mortality by stage $\mu_{\textsc{hiv},h}$,
and treatment via antiretroviral therapy (ART).
\citet{Lodi2011} estimate median times from seroconversion to
CD4 $<$ 500, $<$ 350, and $<$ 200 cells/mm\tsup{3}, while
\citet{Mangal2017} directly estimate the rates of progression between CD4 states $\eta_{h}$
in a simple compartmental model.
Based on these data, I modelled mean durations ($1/\eta_{h}$) of:%
\footnote{Assuming exponential distributions for durations in each CD4 state
  (see \sref{app.model.math.comp} for more details).}
0.142 years in acute infection ($h=2$, from \sref{model.par.beta.hiv});
3.35 years in CD4~$>$~500 ($h=3$);
3.74 years in 350~$<$~CD4~$<$~500 ($h=4$); and
5.26 years in 200~$<$~CD4~$<$~350 ($h=5$); plus
the remaining time until death in CD4~$<$~200 ($h=6$, AIDS).
Since the duration in acute infection ($h=2$) is randomly sampled,
the remaining duration in CD4~$>$~500 ($h=3$) is adjusted accordingly.
%---------------------------------------------------------------------------------------------------
\subsubsection{HIV Mortality}\label{model.par.hiv.mort}
Mortality rates by CD4-count in the absence of ART were estimated in
multiple African studies \cite{Badri2006,Anglaret2012,Mangal2017};
based on these data, I estimated yearly HIV-attributable mortality rates $\mu_{\textsc{hiv},h}$ as:
0 during acute phase ($h=2$);
0.4\% during CD4~$>$~500 ($h=3$);
2\% during 350~$<$~CD4~$<$~500 ($h=4$);
4\% during 200~$<$~CD4~$<$~350 ($h=5$); and 
20\% during CD4~$<$~200 ($h=6$, AIDS).
%===================================================================================================
\subsection{Antiretroviral Therapy}\label{model.par.art}
Viral suppression via antiretroviral therapy (ART) influences
the probability of HIV transmission, as well as rates of HIV progression and HIV-related mortality.
The model considers individuals on ART before ($c=4$) and after ($c=5$)
achieving full viral load suppression (VLS), as defined by undetectable HIV RNA in blood samples.
Among retained patients initiating ART, time to VLS
is usually described as ``within 6 months'' \cite{Thompson2012}.
More specifically, \citet{Mujugira2016} estimate the median time to VLS as 3 months,
yielding an estimated \emph{mean} duration for $c=4$ of 4.3 months (see \sref{app.model.math.comp}).
%---------------------------------------------------------------------------------------------------
\subsubsection{Probability of HIV Transmission}\label{model.par.art.beta}
All available evidence suggests that viral suppression by ART to undetectable levels
prevents HIV transmission, \ie undetectable = untransmittable (``U=U'') \cite{Eisinger2019}.
Thus, I assumed zero HIV transmission from individuals with VLS ($c=5$).
However, HIV transmission may still occur
during the period between ART initiation to viral suppression ($c=4$) \cite{Mujugira2016}.
\citet{Donnell2010} estimate an adjusted incidence ratio of 0.08~(0.0,~0.57) for all individuals on ART.
However, in \cite{Donnell2010} and \cite{Cohen2016}, the 1 and 4 (respectively)
genetically linked infections from individuals on ART all occurred within 90 days of ART initiation,
suggesting that risk of transmission only persists before viral suppression.
Adjusting the incidence denominator (person-time)
to 90 days per individual who initiated ART in \cite{Donnell2010}
results in approximately 3.13 times higher estimated incidence ratio: 0.25 for this specific period.%
\footnote{In \cite{Donnell2010}, individuals who initiated ART contributed
  approximately 9.4 months per-person (273 persons / 349 person-years, Tables~2~and~3);
  thus the first 3 months of each individual represent
  3/9.4 = 0.319 fewer person-months of follow-up.}
Thus, I sampled relative infectiousness on ART but before viral suppression ($c=4$)
from a beta distribution with mean (95\%~CI) of 0.25~(0.01,~0.67).
%---------------------------------------------------------------------------------------------------
\subsubsection{HIV Progression \& Mortality}\label{model.par.art.hiv}
\def\hunprog{$h = 6 \rightarrow 5 \rightarrow 4 \rightarrow 3$\xspace}
Effective ART stops CD4 cell decline and results in some CD4 recovery \cite{Battegay2006,Lawn2006}.
Most CD4 recovery occurs within the first year of treatment \cite{Battegay2006}.
Due to the limited number of modelled treatment states,
I model this initial recovery to be associated with the 4.3-month pre-VLS ART state ($c=4$).
\citet{Lawn2006,Gabillard2013} estimate an increase of between 25--39 cells/mm\tsup{3} per month
during the first 3 months of treatment.
Since HIV states $h=4,5,6$ correspond to 150, 150, and 200-wide CD4 strata,
I model rates of movement along \hunprog during pre-VLS ART ($c=4$) as
0.20, 0.20, 0.17 per month, respectively.
After initial increases, CD4 recovery is modest and plateaus.
\citet{Battegay2006} report approximate increases of
22.4 cells/mm\tsup{3} per year between years 1 and 5 on ART.
Thus, I model rates of movement along \hunprog after VLS ($c=5$) as 0.15 per year.
\par
Since higher CD4 states are modelled to have lower mortality rates (see \sref{model.par.hiv.mort}),
the modelled recovery of CD4 cells via ART described above implicitly affords a mortality benefit.
However, HIV infection is associated with increased risk of death by non-AIDS causes
--- \ie unrelated to CD4 count ---
including cardiovascular disease and renal disease \cite{Phillips2008}.
\citet{Lundgren2015} estimated 61\% reduction in non-AIDS life-threatening events due to ART.
For the same CD4 strata, \citet{Gabillard2013} also report approximately 2-times higher
mortality rates within the first year of ART versus thereafter,
suggesting that VLS is associated with 50\% mortality reduction independent of CD4 increase.
Thus, I modelled an additional 50\% reduction in mortality among individuals with VLS ($c=5$),
and half this (25\%) reduction before achieving VLS ($c=4$).
% TODO: rates of diagnosis, testing, vls
\section{Parameterization}\label{model.par}
As described in \sref{intro.model.param}, model parameterization involves
specification of model parameter values, such as proportions, probabilities, rates, and ratios,
including stratified values to reflect heterogeneity,
and sampling distributions to reflect uncertainty.
Proportions and probabilities were generally modelled using
a beta approximation of the binomial distribution (BAB, see \sref{app.math.distr.bab}),
while rates and ratios were generally modelled using
a gamma, skewnormal, or inverse gaussian distribution.
\paragraph{Notation}
If $X$ is a parameter stratified by dimensions $a,b,c$,
then $X_{ab_{1}c_{23}}$ denotes the values of $X$ for
a particular but \emph{unspecified} stratum of $a$,
the \emph{specific} stratum $b = 1$,
and the \emph{aggregated} strata $c = 2,3$
(the aggregating operation is context-dependent, \eg sum for probabilities).
Additionally, the indices $sihc$ from Table~\ref{tab:model.dims} denote ``self'' strata,
whereas $s'i'h'c'$ denote ``other'' strata --- \ie individuals' partners.%
\footnote{\label{foot:code.note}%
  In the code: R uses one-based indexing, which match the notation here directly,
  while Python uses zero-based indexing, which therefore appear as $i \rightarrow i-1$ in the code.
  Also, the model code reorders states in the ART Cascade dimension for computational efficiency,
  with $c={}$1:~Undiagnosed; 2:~Diagnosed; 3:~Virally~Un-suppressed; 4:~On~ART; 5:~Virally~Suppressed.}
Finally, I re-use several dummy variables throughout the chapter:
$\rho$ for proportions, $\lambda$ for rates, $T$ for time periods, and $f$ for constants.
%===================================================================================================
\subsection{Risk Heterogeneity Among FSW}\label{model.par.fsw}
Existing HIV transmission models which include FSW
have rarely sub-stratified this population, such as to reflect
differential HIV risk or distinct typologies of sex work \cite{Blanchard2008,Scorgie2012};
yet such heterogeneities may influence transmission dynamics.
Among the studies identified in Chapter~\ref{sr},
only three sub-stratified FSW by risk-related factors:
\citet{Cremin2017} defined three levels of risk via regression analysis,
\citet{Low2015} distinguished between occasional and full-time FSW, while
\citet{Shannon2015} sub-stratified FSW by
work environment, violence exposure, and context-specific structural factors.
Seven other studies, reflecting two unique models \cite{Johnson2012,Maheu-Giroux2017},
employed age stratification of all activity groups, including FSW;
these models had several risk-related parameters which varied by age.
\par
The model structure here (Figure~\ref{fig:model.risk})
was designed to capture \emph{within}-FSW risk heterogeneity.
The objective of the following analysis was therefore to parameterize
lower \vs higher risk FSW.
I sought to define these groups based on biobehavioural and/or contextual factors
which are demonstrably associated with HIV risk,
and which can be mechanistically incorporated into a transmission model ---
\ie through the force of infection equation.
Later, the parameterization of these groups was validated through model fitting
to relative differences in HIV prevalence \sref{model.cal.targ.prev}.
\par
Many cross-sectional studies of HIV among FSW quantify
the association of risk factors with HIV serostatus
\cite{Aklilu2001,Dunkle2005,Scorgie2012,Jonas2020}.
However, serostatus reflects cumulative risk exposure,
whereas sexual risk behaviour is dynamic \cite{Watts2010,vanWees2020},
as is use of prevention resources \cite{Roberts2020}.
For example, while HIV prevalence often increases with age,
HIV incidence among women can peak before age 25 \cite{Dellar2015}.
Thus, risk factors associated with HIV serostatus are not necessarily
mechanistically related to HIV acquisition.
Indeed, FSW may reduce risk behaviours in response to seroconversion \cite{McClelland2006}.
Cohort studies that measure incidence
can help identify risk factors for HIV acquisition \cite{McKinnon2015,Nouaman2022},
but large sample sizes are often required to accurately estimate overall incidence rate,
let alone risk factors \cite{Priddy2011}.
%---------------------------------------------------------------------------------------------------
\subsubsection{FSW Survey Data}\label{model.par.fsw.data}
Three biobehavioural surveys, in
2011 \cite{Baral2014} (N = 325),
2014 \cite{EswKP2014} (N = 781), and
2021 \cite{EswIBBS2022} (N = 676)
provide HIV status and biobehavioural data on FSW in Eswatini.
The 2011 and 2021 surveys featured serologic HIV testing,
and employed respondent driven sampling (RDS, details in \cite{Yam2013}).
The 2014 survey relied on self-reported HIV status,
andd employed venue-based snowball sampling, based on the
Priorities for Local AIDS Control Efforts (PLACE) methodology,
which aims to identify areas of higher incidence \cite{Weir2005}.
More details about each study are given in \sref{intro.esw.hiv.data} and Table~\ref{tab:esw.data}.
I analyzed the individual-level data from 2011 and 2014 (data from 2021 not yet available)
to explore the potential association of biobehavioural factors with HIV risk,
so that such factors could then be used to distinguish between
lower risk \vs higher risk FSW.
% TODO: (*) add descriptive table
%---------------------------------------------------------------------------------------------------
\subsubsection{HIV Status}\label{model.par.fsw.hiv}
Only the 2011 and 2021 studies included serologic testing for HIV.
Among those tested in 2011 (N = 317, 98\%), 70\% were \hivp,
yielding RDS-adjusted prevalence estimate of 61\% (CI: 51--71\%) \cite{Baral2014}.
Among serologically \hivn, 11\% self-reported \hivp status (false positive), and
among serologically \hivp, 26\% self-reported \hivn status (false negative or undiagnosed).
Overall, self-reported HIV status underestimated HIV prevalence in 2011
by a factor of approximately 0.78 (55~vs~70\%).
Unadjusted HIV prevalence in 2021 was 58.8\%,
with 88\% (363/411) reporting previous awareness of \hivp status.
\par
In 2014, self-reported HIV prevalence was 38\% among respondents who reported (85\%).
This 38\% is surprisingly low considering that
the PLACE methodology explicitly aimed to sample venues
with higher HIV incidence \cite{Weir2005}, and 2014 \vs 2011 respondents
were older (median 27 \vs 25 years), % 2021 median: 28
had been selling sex longer (median 5 \vs 4 years), % 2021: 6
and tested more frequently (87 \vs 75\% tested at least once in the past year, % 2021: 75
82 \vs 63\% among self-reported \hivn).
Perhaps the differences are attributable to the sampling methodology.
Among respondents who self-reported \hivp status,
the 2014 survey also asked for age of HIV diagnosis (6\% missing).
Age of HIV diagnosis supports crude time-to-event analysis (next section),
which can account for confounding by age and censoring,
as compared to logistic regression on HIV status,
keeping in mind the limitations of self-reported HIV status.
%---------------------------------------------------------------------------------------------------
\subsubsection{Risk Factors for HIV}\label{model.par.fsw.fac}
Next, I explored the potential association of risk factors with HIV
via the following three models:%
\footnote{Logistic regression models were implemented using \texttt{lrm} from:
  \hreftt{cran.r-project.org/package=rms}.\\
Cox proportional hazards models were implemented using \texttt{coxaalen} from:
  \hreftt{cran.r-project.org/package=coxinterval}.}
\begin{enumerate}
  \item Logistic regression on serologic HIV status (2011 data)
  \item Logistic regression on self-reported HIV status (2014 data)
  \item Cox proportional hazards for interval-censored time to HIV infection,
    with interval from self-reported sex work debut 
    to either self-reported time of HIV diagnosis or survey date (2014 data);
    Figure~\ref{fig:fsw.tte.interval} illustrates
    the four potential censoring cases in this framework.
\end{enumerate}
An important limitation to all models is that
risk factors reported by FSW at the time of survey
are assumed to be fixed characteristics of the respondents,
rather than dynamic characteristics that vary over time.
Additionally, respondents with any missing variables for each individual model
were excluded from that model. % TODO: (%)
\begin{figure}
  \centering
  \includegraphics[scale=1]{diag.tte}
  \caption{Illustration of time-to-event analysis framework
    for cross-sectional FSW survey data}
  \label{fig:fsw.tte.interval}
  \floatfoot{
    $\bm{\times}$: HIV infection;
    SW: time of sex work debut;
    Dx: time of HIV diagnosis.}
\end{figure}
\par
Risk factors were selected based on
prior knowledge of plausible mechanistic influence on HIV incidence and/or prevalence.
The risk factors explored are summarized in Table~\ref{tab:fsw.stats},
including univariate and multivariable association under each model.
Variable selection for multivariable models
was performed using backward selection as described by \citet{Lawless1978},
using a $p \le 0.1$ (per variable) threshold for stepwise variable retention.
Estimated conditional effects of
variables retained in the multivariable logistic regression models
are illustrated in Figure~\ref{fig:fsw.lr}.
\begin{table}
  \centering
  \caption{Risk factors explored for association with \hivp status among FSW in Eswatini}
  \label{tab:fsw.stats}
  \input{model/tab.fsw.factor.stats}
\end{table}
\begin{figure}[h]
  \subcapoverlap
  \foreach \year/\var/\nvar in {2011/f/1,2011/c/2,2014/f/3,2014/c/3}{
  \begin{subfigure}{\nvar\linewidth/5+\linewidth/5}
    \includegraphics[scale=.7]{fsw.\year.lr.hiv.\var}
    \caption{\raggedright}
    \label{fig:fsw.lr.\year.\var}
  \end{subfigure}}
  \caption{Predicted conditional effects (probability)
    of variables in multivariable logistic regression models for HIV status}
  \label{fig:fsw.lr}
  \floatfoot{\fffsw{fig:fsw.lr}
    conditional probabilities shown for fixed covariates at arbitrary values.}
\end{figure}
\par
Following variable selection, each multivariable model was used to estimate
the total \hivp status odds ratio (logistic) or HIV incidence hazard ratio (Cox)
for each respondent in the respective survey ---
\ie $e^{X_i\,\beta}$ for respondent $i$ ---
representing an overall ``risk score'' under each model.
Respondents were then stratified into the top 20\% and bottom 80\% by these risk scores.
The values of each variable were compared between these two strata
using a test for the ratio of the means \cite{Tamhane2004} to support model parameterization;
these ratios are summarized in Table~\ref{tab:fsw.ratios},
and the distributions of variable values across the two strata
are illustrated in Figure~\ref{fig:fsw.f}.
\begin{table}
  \centering
  \caption{Ratios of HIV risk factor variables among higher \vs lower risk FSW in Eswatini}
  \label{tab:fsw.ratios}
  \input{model/tab.fsw.factor.ratios}
\end{table}

\section{Parameterization}\label{model.par}
As described in \sref{intro.model.param}, model parameterization involves
specification of model parameter values, such as proportions, probabilities, rates, and ratios,
including stratified values to reflect heterogeneity,
and sampling distributions to reflect uncertainty.
Proportions and probabilities were generally modelled using
a beta approximation of the binomial distribution (BAB, see \sref{app.math.distr.bab}),
while rates and ratios were generally modelled using
a gamma, skewnormal, or inverse gaussian distribution.
\paragraph{Notation}
If $X$ is a parameter stratified by dimensions $a,b,c$,
then $X_{ab_{1}c_{23}}$ denotes the values of $X$ for
a particular but \emph{unspecified} stratum of $a$,
the \emph{specific} stratum $b = 1$,
and the \emph{aggregated} strata $c = 2,3$
(the aggregating operation is context-dependent, \eg sum for probabilities).
Additionally, the indices $sihc$ from Table~\ref{tab:model.dims} denote ``self'' strata,
whereas $s'i'h'c'$ denote ``other'' strata --- \ie individuals' partners.%
\footnote{\label{foot:code.note}%
  In the code: R uses one-based indexing, which match the notation here directly,
  while Python uses zero-based indexing, which therefore appear as $i \rightarrow i-1$ in the code.
  Also, the model code reorders states in the ART Cascade dimension for computational efficiency,
  with $c={}$1:~Undiagnosed; 2:~Diagnosed; 3:~Virally~Un-suppressed; 4:~On~ART; 5:~Virally~Suppressed.}
Finally, I re-use several dummy variables throughout the chapter:
$\rho$ for proportions, $\lambda$ for rates, $T$ for time periods, and $f$ for constants.
\input{model/par.fsw}
\input{model/par.beta}
\input{model/par.hiv}
\input{model/par.popsex}
%===================================================================================================
\subsection{HIV Progression \& Mortality}\label{model.par.hiv}
%---------------------------------------------------------------------------------------------------
\subsubsection{HIV Progression}\label{model.par.hiv.dur}
The length of time spent in each HIV stage is related to
rates of progression between stages $\eta_{h}$,
rates of additional HIV-attributable mortality by stage $\mu_{\textsc{hiv},h}$,
and treatment via antiretroviral therapy (ART).
\citet{Lodi2011} estimate median times from seroconversion to
CD4 $<$ 500, $<$ 350, and $<$ 200 cells/mm\tsup{3}, while
\citet{Mangal2017} directly estimate the rates of progression between CD4 states $\eta_{h}$
in a simple compartmental model.
Based on these data, I modelled mean durations ($1/\eta_{h}$) of:%
\footnote{Assuming exponential distributions for durations in each CD4 state
  (see \sref{app.model.math.comp} for more details).}
0.142 years in acute infection ($h=2$, from \sref{model.par.beta.hiv});
3.35 years in CD4~$>$~500 ($h=3$);
3.74 years in 350~$<$~CD4~$<$~500 ($h=4$); and
5.26 years in 200~$<$~CD4~$<$~350 ($h=5$); plus
the remaining time until death in CD4~$<$~200 ($h=6$, AIDS).
Since the duration in acute infection ($h=2$) is randomly sampled,
the remaining duration in CD4~$>$~500 ($h=3$) is adjusted accordingly.
%---------------------------------------------------------------------------------------------------
\subsubsection{HIV Mortality}\label{model.par.hiv.mort}
Mortality rates by CD4-count in the absence of ART were estimated in
multiple African studies \cite{Badri2006,Anglaret2012,Mangal2017};
based on these data, I estimated yearly HIV-attributable mortality rates $\mu_{\textsc{hiv},h}$ as:
0 during acute phase ($h=2$);
0.4\% during CD4~$>$~500 ($h=3$);
2\% during 350~$<$~CD4~$<$~500 ($h=4$);
4\% during 200~$<$~CD4~$<$~350 ($h=5$); and 
20\% during CD4~$<$~200 ($h=6$, AIDS).
%===================================================================================================
\subsection{Antiretroviral Therapy}\label{model.par.art}
Viral suppression via antiretroviral therapy (ART) influences
the probability of HIV transmission, as well as rates of HIV progression and HIV-related mortality.
The model considers individuals on ART before ($c=4$) and after ($c=5$)
achieving full viral load suppression (VLS), as defined by undetectable HIV RNA in blood samples.
Among retained patients initiating ART, time to VLS
is usually described as ``within 6 months'' \cite{Thompson2012}.
More specifically, \citet{Mujugira2016} estimate the median time to VLS as 3 months,
yielding an estimated \emph{mean} duration for $c=4$ of 4.3 months (see \sref{app.model.math.comp}).
%---------------------------------------------------------------------------------------------------
\subsubsection{Probability of HIV Transmission}\label{model.par.art.beta}
All available evidence suggests that viral suppression by ART to undetectable levels
prevents HIV transmission, \ie undetectable = untransmittable (``U=U'') \cite{Eisinger2019}.
Thus, I assumed zero HIV transmission from individuals with VLS ($c=5$).
However, HIV transmission may still occur
during the period between ART initiation to viral suppression ($c=4$) \cite{Mujugira2016}.
\citet{Donnell2010} estimate an adjusted incidence ratio of 0.08~(0.0,~0.57) for all individuals on ART.
However, in \cite{Donnell2010} and \cite{Cohen2016}, the 1 and 4 (respectively)
genetically linked infections from individuals on ART all occurred within 90 days of ART initiation,
suggesting that risk of transmission only persists before viral suppression.
Adjusting the incidence denominator (person-time)
to 90 days per individual who initiated ART in \cite{Donnell2010}
results in approximately 3.13 times higher estimated incidence ratio: 0.25 for this specific period.%
\footnote{In \cite{Donnell2010}, individuals who initiated ART contributed
  approximately 9.4 months per-person (273 persons / 349 person-years, Tables~2~and~3);
  thus the first 3 months of each individual represent
  3/9.4 = 0.319 fewer person-months of follow-up.}
Thus, I sampled relative infectiousness on ART but before viral suppression ($c=4$)
from a beta distribution with mean (95\%~CI) of 0.25~(0.01,~0.67).
%---------------------------------------------------------------------------------------------------
\subsubsection{HIV Progression \& Mortality}\label{model.par.art.hiv}
\def\hunprog{$h = 6 \rightarrow 5 \rightarrow 4 \rightarrow 3$\xspace}
Effective ART stops CD4 cell decline and results in some CD4 recovery \cite{Battegay2006,Lawn2006}.
Most CD4 recovery occurs within the first year of treatment \cite{Battegay2006}.
Due to the limited number of modelled treatment states,
I model this initial recovery to be associated with the 4.3-month pre-VLS ART state ($c=4$).
\citet{Lawn2006,Gabillard2013} estimate an increase of between 25--39 cells/mm\tsup{3} per month
during the first 3 months of treatment.
Since HIV states $h=4,5,6$ correspond to 150, 150, and 200-wide CD4 strata,
I model rates of movement along \hunprog during pre-VLS ART ($c=4$) as
0.20, 0.20, 0.17 per month, respectively.
After initial increases, CD4 recovery is modest and plateaus.
\citet{Battegay2006} report approximate increases of
22.4 cells/mm\tsup{3} per year between years 1 and 5 on ART.
Thus, I model rates of movement along \hunprog after VLS ($c=5$) as 0.15 per year.
\par
Since higher CD4 states are modelled to have lower mortality rates (see \sref{model.par.hiv.mort}),
the modelled recovery of CD4 cells via ART described above implicitly affords a mortality benefit.
However, HIV infection is associated with increased risk of death by non-AIDS causes
--- \ie unrelated to CD4 count ---
including cardiovascular disease and renal disease \cite{Phillips2008}.
\citet{Lundgren2015} estimated 61\% reduction in non-AIDS life-threatening events due to ART.
For the same CD4 strata, \citet{Gabillard2013} also report approximately 2-times higher
mortality rates within the first year of ART versus thereafter,
suggesting that VLS is associated with 50\% mortality reduction independent of CD4 increase.
Thus, I modelled an additional 50\% reduction in mortality among individuals with VLS ($c=5$),
and half this (25\%) reduction before achieving VLS ($c=4$).
% TODO: rates of diagnosis, testing, vls
\section{Parameterization}\label{model.par}
As described in \sref{intro.model.param}, model parameterization involves
specification of model parameter values, such as proportions, probabilities, rates, and ratios,
including stratified values to reflect heterogeneity,
and sampling distributions to reflect uncertainty.
Proportions and probabilities were generally modelled using
a beta approximation of the binomial distribution (BAB, see \sref{app.math.distr.bab}),
while rates and ratios were generally modelled using
a gamma, skewnormal, or inverse gaussian distribution.
\paragraph{Notation}
If $X$ is a parameter stratified by dimensions $a,b,c$,
then $X_{ab_{1}c_{23}}$ denotes the values of $X$ for
a particular but \emph{unspecified} stratum of $a$,
the \emph{specific} stratum $b = 1$,
and the \emph{aggregated} strata $c = 2,3$
(the aggregating operation is context-dependent, \eg sum for probabilities).
Additionally, the indices $sihc$ from Table~\ref{tab:model.dims} denote ``self'' strata,
whereas $s'i'h'c'$ denote ``other'' strata --- \ie individuals' partners.%
\footnote{\label{foot:code.note}%
  In the code: R uses one-based indexing, which match the notation here directly,
  while Python uses zero-based indexing, which therefore appear as $i \rightarrow i-1$ in the code.
  Also, the model code reorders states in the ART Cascade dimension for computational efficiency,
  with $c={}$1:~Undiagnosed; 2:~Diagnosed; 3:~Virally~Un-suppressed; 4:~On~ART; 5:~Virally~Suppressed.}
Finally, I re-use several dummy variables throughout the chapter:
$\rho$ for proportions, $\lambda$ for rates, $T$ for time periods, and $f$ for constants.
\input{model/par.fsw}
\input{model/par.beta}
\input{model/par.hiv}
\input{model/par.popsex}
\section{Parameterization}\label{model.par}
As described in \sref{intro.model.param}, model parameterization involves
specification of model parameter values, such as proportions, probabilities, rates, and ratios,
including stratified values to reflect heterogeneity,
and sampling distributions to reflect uncertainty.
Proportions and probabilities were generally modelled using
a beta approximation of the binomial distribution (BAB, see \sref{app.math.distr.bab}),
while rates and ratios were generally modelled using
a gamma, skewnormal, or inverse gaussian distribution.
\paragraph{Notation}
If $X$ is a parameter stratified by dimensions $a,b,c$,
then $X_{ab_{1}c_{23}}$ denotes the values of $X$ for
a particular but \emph{unspecified} stratum of $a$,
the \emph{specific} stratum $b = 1$,
and the \emph{aggregated} strata $c = 2,3$
(the aggregating operation is context-dependent, \eg sum for probabilities).
Additionally, the indices $sihc$ from Table~\ref{tab:model.dims} denote ``self'' strata,
whereas $s'i'h'c'$ denote ``other'' strata --- \ie individuals' partners.%
\footnote{\label{foot:code.note}%
  In the code: R uses one-based indexing, which match the notation here directly,
  while Python uses zero-based indexing, which therefore appear as $i \rightarrow i-1$ in the code.
  Also, the model code reorders states in the ART Cascade dimension for computational efficiency,
  with $c={}$1:~Undiagnosed; 2:~Diagnosed; 3:~Virally~Un-suppressed; 4:~On~ART; 5:~Virally~Suppressed.}
Finally, I re-use several dummy variables throughout the chapter:
$\rho$ for proportions, $\lambda$ for rates, $T$ for time periods, and $f$ for constants.
%===================================================================================================
\subsection{Risk Heterogeneity Among FSW}\label{model.par.fsw}
Existing HIV transmission models which include FSW
have rarely sub-stratified this population, such as to reflect
differential HIV risk or distinct typologies of sex work \cite{Blanchard2008,Scorgie2012};
yet such heterogeneities may influence transmission dynamics.
Among the studies identified in Chapter~\ref{sr},
only three sub-stratified FSW by risk-related factors:
\citet{Cremin2017} defined three levels of risk via regression analysis,
\citet{Low2015} distinguished between occasional and full-time FSW, while
\citet{Shannon2015} sub-stratified FSW by
work environment, violence exposure, and context-specific structural factors.
Seven other studies, reflecting two unique models \cite{Johnson2012,Maheu-Giroux2017},
employed age stratification of all activity groups, including FSW;
these models had several risk-related parameters which varied by age.
\par
The model structure here (Figure~\ref{fig:model.risk})
was designed to capture \emph{within}-FSW risk heterogeneity.
The objective of the following analysis was therefore to parameterize
lower \vs higher risk FSW.
I sought to define these groups based on biobehavioural and/or contextual factors
which are demonstrably associated with HIV risk,
and which can be mechanistically incorporated into a transmission model ---
\ie through the force of infection equation.
Later, the parameterization of these groups was validated through model fitting
to relative differences in HIV prevalence \sref{model.cal.targ.prev}.
\par
Many cross-sectional studies of HIV among FSW quantify
the association of risk factors with HIV serostatus
\cite{Aklilu2001,Dunkle2005,Scorgie2012,Jonas2020}.
However, serostatus reflects cumulative risk exposure,
whereas sexual risk behaviour is dynamic \cite{Watts2010,vanWees2020},
as is use of prevention resources \cite{Roberts2020}.
For example, while HIV prevalence often increases with age,
HIV incidence among women can peak before age 25 \cite{Dellar2015}.
Thus, risk factors associated with HIV serostatus are not necessarily
mechanistically related to HIV acquisition.
Indeed, FSW may reduce risk behaviours in response to seroconversion \cite{McClelland2006}.
Cohort studies that measure incidence
can help identify risk factors for HIV acquisition \cite{McKinnon2015,Nouaman2022},
but large sample sizes are often required to accurately estimate overall incidence rate,
let alone risk factors \cite{Priddy2011}.
%---------------------------------------------------------------------------------------------------
\subsubsection{FSW Survey Data}\label{model.par.fsw.data}
Three biobehavioural surveys, in
2011 \cite{Baral2014} (N = 325),
2014 \cite{EswKP2014} (N = 781), and
2021 \cite{EswIBBS2022} (N = 676)
provide HIV status and biobehavioural data on FSW in Eswatini.
The 2011 and 2021 surveys featured serologic HIV testing,
and employed respondent driven sampling (RDS, details in \cite{Yam2013}).
The 2014 survey relied on self-reported HIV status,
andd employed venue-based snowball sampling, based on the
Priorities for Local AIDS Control Efforts (PLACE) methodology,
which aims to identify areas of higher incidence \cite{Weir2005}.
More details about each study are given in \sref{intro.esw.hiv.data} and Table~\ref{tab:esw.data}.
I analyzed the individual-level data from 2011 and 2014 (data from 2021 not yet available)
to explore the potential association of biobehavioural factors with HIV risk,
so that such factors could then be used to distinguish between
lower risk \vs higher risk FSW.
% TODO: (*) add descriptive table
%---------------------------------------------------------------------------------------------------
\subsubsection{HIV Status}\label{model.par.fsw.hiv}
Only the 2011 and 2021 studies included serologic testing for HIV.
Among those tested in 2011 (N = 317, 98\%), 70\% were \hivp,
yielding RDS-adjusted prevalence estimate of 61\% (CI: 51--71\%) \cite{Baral2014}.
Among serologically \hivn, 11\% self-reported \hivp status (false positive), and
among serologically \hivp, 26\% self-reported \hivn status (false negative or undiagnosed).
Overall, self-reported HIV status underestimated HIV prevalence in 2011
by a factor of approximately 0.78 (55~vs~70\%).
Unadjusted HIV prevalence in 2021 was 58.8\%,
with 88\% (363/411) reporting previous awareness of \hivp status.
\par
In 2014, self-reported HIV prevalence was 38\% among respondents who reported (85\%).
This 38\% is surprisingly low considering that
the PLACE methodology explicitly aimed to sample venues
with higher HIV incidence \cite{Weir2005}, and 2014 \vs 2011 respondents
were older (median 27 \vs 25 years), % 2021 median: 28
had been selling sex longer (median 5 \vs 4 years), % 2021: 6
and tested more frequently (87 \vs 75\% tested at least once in the past year, % 2021: 75
82 \vs 63\% among self-reported \hivn).
Perhaps the differences are attributable to the sampling methodology.
Among respondents who self-reported \hivp status,
the 2014 survey also asked for age of HIV diagnosis (6\% missing).
Age of HIV diagnosis supports crude time-to-event analysis (next section),
which can account for confounding by age and censoring,
as compared to logistic regression on HIV status,
keeping in mind the limitations of self-reported HIV status.
%---------------------------------------------------------------------------------------------------
\subsubsection{Risk Factors for HIV}\label{model.par.fsw.fac}
Next, I explored the potential association of risk factors with HIV
via the following three models:%
\footnote{Logistic regression models were implemented using \texttt{lrm} from:
  \hreftt{cran.r-project.org/package=rms}.\\
Cox proportional hazards models were implemented using \texttt{coxaalen} from:
  \hreftt{cran.r-project.org/package=coxinterval}.}
\begin{enumerate}
  \item Logistic regression on serologic HIV status (2011 data)
  \item Logistic regression on self-reported HIV status (2014 data)
  \item Cox proportional hazards for interval-censored time to HIV infection,
    with interval from self-reported sex work debut 
    to either self-reported time of HIV diagnosis or survey date (2014 data);
    Figure~\ref{fig:fsw.tte.interval} illustrates
    the four potential censoring cases in this framework.
\end{enumerate}
An important limitation to all models is that
risk factors reported by FSW at the time of survey
are assumed to be fixed characteristics of the respondents,
rather than dynamic characteristics that vary over time.
Additionally, respondents with any missing variables for each individual model
were excluded from that model. % TODO: (%)
\begin{figure}
  \centering
  \includegraphics[scale=1]{diag.tte}
  \caption{Illustration of time-to-event analysis framework
    for cross-sectional FSW survey data}
  \label{fig:fsw.tte.interval}
  \floatfoot{
    $\bm{\times}$: HIV infection;
    SW: time of sex work debut;
    Dx: time of HIV diagnosis.}
\end{figure}
\par
Risk factors were selected based on
prior knowledge of plausible mechanistic influence on HIV incidence and/or prevalence.
The risk factors explored are summarized in Table~\ref{tab:fsw.stats},
including univariate and multivariable association under each model.
Variable selection for multivariable models
was performed using backward selection as described by \citet{Lawless1978},
using a $p \le 0.1$ (per variable) threshold for stepwise variable retention.
Estimated conditional effects of
variables retained in the multivariable logistic regression models
are illustrated in Figure~\ref{fig:fsw.lr}.
\begin{table}
  \centering
  \caption{Risk factors explored for association with \hivp status among FSW in Eswatini}
  \label{tab:fsw.stats}
  \centerline{%
\small%
\begin{tabular}{lcccccccccccc}
  \toprule
  & \multicolumn{4}{c}{2011 LR}
  & \multicolumn{4}{c}{2014 LR}
  & \multicolumn{4}{c}{2014 CPH} \\
  \cmidrule(rl){2-5}\cmidrule(rl){6-9}\cmidrule(rl){10-13}
  & \multicolumn{2}{c}{Univar} & \multicolumn{2}{c}{Multivar}
  & \multicolumn{2}{c}{Univar} & \multicolumn{2}{c}{Multivar}
  & \multicolumn{2}{c}{Univar} & \multicolumn{2}{c}{Multivar} \\
  \cmidrule(rl){2-3}\cmidrule(rl){4-5}\cmidrule(rl){6-7}\cmidrule(rl){8-9}\cmidrule(rl){10-11}\cmidrule(rl){12-13}
  Factor                          &  OR  &   p   &  OR  &   p    &  OR  &   p    &  OR  &   p    &  HR  &   p    &  HR  &   p    \\
  \midrule                        % 2011 LR uni  % 2011 LR multi % 2014 LR uni   % 2014 LR multi % 2014 CPH uni  % 2014 CPH multi
  Age\tn{a}                       & 1.11 & \vsig & ---  &  ---   & 1.14 & \vsig  & 1.15 & \vsig  & 1.09 & \vsig  & 1.09 & \vsig  \\
  Years selling sex\tn{a}         & 1.13 & \vsig & 1.13 & \vsig  & 1.12 & \vsig  & ---  &  ---   & 1.08 & \vsig  & ---  &  ---   \\
  Monthly sex work income\tn{b}   & 0.98 & 0.155 & ---  &  ---   & 0.98 & 0.097  & 0.97 & 0.084  & 0.98 & 0.019\s& 0.97 & 0.001\s\\[1ex]
  Non-paying partners\tn{c}       & 0.88 & 0.307 & ---  &  ---   & 1.07 & 0.233  & ---  &  ---   & 1.05 & 0.312  & ---  &  ---   \\
  Monthly new clients\tn{c}       & 1.01 & 0.412 & ---  &  ---   & 1.05 & \vsig  & 1.07 & \vsig  & 1.04 & \vsig  & 1.04 & \vsig  \\
  Monthly regular clients\tn{c}   & 1.01 & 0.351 & ---  &  ---   & 1.03 & 0.002  & ---  &  ---   & 1.02 & \vsig  & 1.02 & 0.034\s\\[1ex]
  Non-paying condom use\tn{d}     & 0.90 & 0.703 & ---  &  ---   & 0.90 & 0.673  & ---  &  ---   & 0.92 & 0.677  & ---  &  ---   \\
  New client condom use\tn{d}     & 0.60 & 0.100 & ---  &  ---   & 0.48 & 0.006\s& 1.25 & 0.599  & 0.56 & 0.004\s& ---  &  ---   \\
  Regular client condom use\tn{d} & 0.58 & 0.110 & ---  &  ---   & 0.39 & \vsig  & 0.35 & 0.004\s& 0.49 & \vsig  & 0.50 & \vsig  \\[1ex]
  Any anal sex past month         & 0.97 & 0.896 & ---  &  ---   & 1.89 & 0.015\s& ---  &  ---   & 1.57 & 0.015\s& 1.27 & 0.260  \\
  Any STI symptoms past year      & 2.29 & \vsig & 2.41 & \vsig  & 2.75 & \vsig  & 2.80 & \vsig  & 2.17 & \vsig  & 2.05 & \vsig  \\
  \bottomrule
\end{tabular}}
% TODO: HIV status?
\floatfoot{\raggedright
  \tnt[a]{OR per year};
  \tnt[b]{OR per Swazi lilangeni per month};
  \tnt[c]{OR per partner};
  \tnt[d]{2011: always vs not always, 2014: at last sex}.
  --- indicates variable was not selected in the multivariate model.
  LR: logistic regression on HIV$+/-$ status;
  CPH: Cox proportional hazards on time to self-reported HIV seroconversion.
  OR: odds ratio; HR: hazard ratio; p: p-value.
  2011 data based on serologic HIV test;
  2014 data based on self-reported HIV status, age of sex work debut, and age of HIV diagnosis.
}
\end{table}
\begin{figure}[h]
  \subcapoverlap
  \foreach \year/\var/\nvar in {2011/f/1,2011/c/2,2014/f/3,2014/c/3}{
  \begin{subfigure}{\nvar\linewidth/5+\linewidth/5}
    \includegraphics[scale=.7]{fsw.\year.lr.hiv.\var}
    \caption{\raggedright}
    \label{fig:fsw.lr.\year.\var}
  \end{subfigure}}
  \caption{Predicted conditional effects (probability)
    of variables in multivariable logistic regression models for HIV status}
  \label{fig:fsw.lr}
  \floatfoot{\fffsw{fig:fsw.lr}
    conditional probabilities shown for fixed covariates at arbitrary values.}
\end{figure}
\par
Following variable selection, each multivariable model was used to estimate
the total \hivp status odds ratio (logistic) or HIV incidence hazard ratio (Cox)
for each respondent in the respective survey ---
\ie $e^{X_i\,\beta}$ for respondent $i$ ---
representing an overall ``risk score'' under each model.
Respondents were then stratified into the top 20\% and bottom 80\% by these risk scores.
The values of each variable were compared between these two strata
using a test for the ratio of the means \cite{Tamhane2004} to support model parameterization;
these ratios are summarized in Table~\ref{tab:fsw.ratios},
and the distributions of variable values across the two strata
are illustrated in Figure~\ref{fig:fsw.f}.
\begin{table}
  \centering
  \caption{Ratios of HIV risk factor variables among higher \vs lower risk FSW in Eswatini}
  \label{tab:fsw.ratios}
  \centerline{\footnotesize%
\begin{tabular}{lcccccc}
  \toprule
  & \multicolumn{2}{c}{2011 LR}
  & \multicolumn{2}{c}{2014 LR}
  & \multicolumn{2}{c}{2014 CPH} \\
  \cmidrule(rl){2-3}\cmidrule(rl){4-5}\cmidrule(rl){6-7}
  Factor                            &  High / Low   &   Ratio (95\% CI)   &  High / Low   &   Ratio (95\% CI)   &   High / Low   &   Ratio (95\% CI)   \\
  \midrule
  Age                               & 31.8  / 24.7  & 1.29 (1.22, 1.36)\s & 32.6  / 26.2  & 1.24 (1.20, 1.28)\s &  33.5  / 26.6  & 1.26 (1.21, 1.31)\s \\
  Years selling sex                 & 11.3  /  4.03 & 2.81 (2.41, 3.25)\s & 10.0  /  5.47 & 1.83 (1.64, 2.03)\s &  10.2  /  5.83 & 1.75 (1.54, 1.98)\s \\
  Monthly sex work income\tn{a}     & 15.1  / 15.2  & 1.00 (0.86, 1.15)   &  6.77 /  7.06 & 0.96 (0.82, 1.11)   &   6.32 /  7.28 & 0.87 (0.73, 1.02)   \\[1ex]
  Non-paying partners               &  1.42 /  1.43 & 0.99 (0.81, 1.19)   &  1.56 /  1.11 & 1.40 (1.11, 1.72)\s &   1.53 /  1.19 & 1.29 (0.98, 1.62)   \\
  Monthly new clients               &  5.50 /  6.98 & 0.79 (0.49, 1.15)   &  8.39 /  4.15 & 2.02 (1.63, 2.44)\s &   8.36 /  4.41 & 1.90 (1.43, 2.39)\s \\
  Monthly regular clients           &  9.35 /  9.05 & 1.03 (0.69, 1.42)   & 11.1  /  8.25 & 1.35 (1.13, 1.57)\s &  12.4  /  8.61 & 1.44 (1.18, 1.71)\s \\[1ex]
  Non-paying condom use\tn{bc}      &  0.26 /  0.35 & 0.73 (0.40, 1.11)   &  0.77 /  0.81 & 0.95 (0.84, 1.06)   &   0.76 /  0.81 & 0.95 (0.81, 1.08)   \\
  New client condom use\tn{bc}      &  0.68 /  0.76 & 0.89 (0.73, 1.06)   &  0.79 /  0.91 & 0.86 (0.79, 0.94)\s &   0.74 /  0.94 & 0.79 (0.69, 0.88)\s \\
  Regular client condom use\tn{bc}  &  0.38 /  0.46 & 0.83 (0.45, 1.28)   &  0.67 /  0.91 & 0.74 (0.65, 0.82)\s &   0.60 /  0.92 & 0.65 (0.55, 0.75)\s \\[1ex]
  Any anal sex past month           &  0.59 /  0.41 & 1.41 (1.06, 1.84)\s &  0.17 /  0.07 & 2.43 (1.47, 3.85)\s &   0.23 /  0.07 & 3.24 (1.95, 5.34)\s \\
  Any STI symptoms past year\tn{c}  &  0.79 /  0.43 & 1.86 (1.54, 2.25)\s &  0.59 /  0.15 & 3.94 (3.15, 5.03)\s &   0.61 /  0.17 & 3.67 (2.87, 4.79)\s \\[1ex]
  HIV prevalence\tn{d}              &  0.94 /  0.64 & 1.46 (1.30, 1.63)\s &  0.66 /  0.29 & 2.29 (1.92, 2.75)\s &   0.71 /  0.31 & 2.32 (1.94, 2.80)\s \\
  \bottomrule
\end{tabular}}
% TODO: HIV status?
\floatfoot{\raggedright
  High / Low: mean variable value among higher / lower risk groups, as defined by
  the top 20\% / bottom 80\% in multivariable model-predicted risk score:
  odds ratio from logistic regression (LR);
  hazards ratio from Cox proportional hazards (CPH).
  \tnt[a]{Swati lilangeni per month};
  \tnt[b]{2011: always \vs not always, 2014: did use condom at last sex};
  \tnt[c]{proportion of respondents};
  \tnt[d]{2011: serologic HIV status; 2014: self-reported HIV status};
  \tnt[*]{statistically significant, $p < 0.05$}.
}
\end{table}

\section{Parameterization}\label{model.par}
As described in \sref{intro.model.param}, model parameterization involves
specification of model parameter values, such as proportions, probabilities, rates, and ratios,
including stratified values to reflect heterogeneity,
and sampling distributions to reflect uncertainty.
Proportions and probabilities were generally modelled using
a beta approximation of the binomial distribution (BAB, see \sref{app.math.distr.bab}),
while rates and ratios were generally modelled using
a gamma, skewnormal, or inverse gaussian distribution.
\paragraph{Notation}
If $X$ is a parameter stratified by dimensions $a,b,c$,
then $X_{ab_{1}c_{23}}$ denotes the values of $X$ for
a particular but \emph{unspecified} stratum of $a$,
the \emph{specific} stratum $b = 1$,
and the \emph{aggregated} strata $c = 2,3$
(the aggregating operation is context-dependent, \eg sum for probabilities).
Additionally, the indices $sihc$ from Table~\ref{tab:model.dims} denote ``self'' strata,
whereas $s'i'h'c'$ denote ``other'' strata --- \ie individuals' partners.%
\footnote{\label{foot:code.note}%
  In the code: R uses one-based indexing, which match the notation here directly,
  while Python uses zero-based indexing, which therefore appear as $i \rightarrow i-1$ in the code.
  Also, the model code reorders states in the ART Cascade dimension for computational efficiency,
  with $c={}$1:~Undiagnosed; 2:~Diagnosed; 3:~Virally~Un-suppressed; 4:~On~ART; 5:~Virally~Suppressed.}
Finally, I re-use several dummy variables throughout the chapter:
$\rho$ for proportions, $\lambda$ for rates, $T$ for time periods, and $f$ for constants.
%===================================================================================================
\subsection{Risk Heterogeneity Among FSW}\label{model.par.fsw}
Existing HIV transmission models which include FSW
have rarely sub-stratified this population, such as to reflect
differential HIV risk or distinct typologies of sex work \cite{Blanchard2008,Scorgie2012};
yet such heterogeneities may influence transmission dynamics.
Among the studies identified in Chapter~\ref{sr},
only three sub-stratified FSW by risk-related factors:
\citet{Cremin2017} defined three levels of risk via regression analysis,
\citet{Low2015} distinguished between occasional and full-time FSW, while
\citet{Shannon2015} sub-stratified FSW by
work environment, violence exposure, and context-specific structural factors.
Seven other studies, reflecting two unique models \cite{Johnson2012,Maheu-Giroux2017},
employed age stratification of all activity groups, including FSW;
these models had several risk-related parameters which varied by age.
\par
The model structure here (Figure~\ref{fig:model.risk})
was designed to capture \emph{within}-FSW risk heterogeneity.
The objective of the following analysis was therefore to parameterize
lower \vs higher risk FSW.
I sought to define these groups based on biobehavioural and/or contextual factors
which are demonstrably associated with HIV risk,
and which can be mechanistically incorporated into a transmission model ---
\ie through the force of infection equation.
Later, the parameterization of these groups was validated through model fitting
to relative differences in HIV prevalence \sref{model.cal.targ.prev}.
\par
Many cross-sectional studies of HIV among FSW quantify
the association of risk factors with HIV serostatus
\cite{Aklilu2001,Dunkle2005,Scorgie2012,Jonas2020}.
However, serostatus reflects cumulative risk exposure,
whereas sexual risk behaviour is dynamic \cite{Watts2010,vanWees2020},
as is use of prevention resources \cite{Roberts2020}.
For example, while HIV prevalence often increases with age,
HIV incidence among women can peak before age 25 \cite{Dellar2015}.
Thus, risk factors associated with HIV serostatus are not necessarily
mechanistically related to HIV acquisition.
Indeed, FSW may reduce risk behaviours in response to seroconversion \cite{McClelland2006}.
Cohort studies that measure incidence
can help identify risk factors for HIV acquisition \cite{McKinnon2015,Nouaman2022},
but large sample sizes are often required to accurately estimate overall incidence rate,
let alone risk factors \cite{Priddy2011}.
%---------------------------------------------------------------------------------------------------
\subsubsection{FSW Survey Data}\label{model.par.fsw.data}
Three biobehavioural surveys, in
2011 \cite{Baral2014} (N = 325),
2014 \cite{EswKP2014} (N = 781), and
2021 \cite{EswIBBS2022} (N = 676)
provide HIV status and biobehavioural data on FSW in Eswatini.
The 2011 and 2021 surveys featured serologic HIV testing,
and employed respondent driven sampling (RDS, details in \cite{Yam2013}).
The 2014 survey relied on self-reported HIV status,
andd employed venue-based snowball sampling, based on the
Priorities for Local AIDS Control Efforts (PLACE) methodology,
which aims to identify areas of higher incidence \cite{Weir2005}.
More details about each study are given in \sref{intro.esw.hiv.data} and Table~\ref{tab:esw.data}.
I analyzed the individual-level data from 2011 and 2014 (data from 2021 not yet available)
to explore the potential association of biobehavioural factors with HIV risk,
so that such factors could then be used to distinguish between
lower risk \vs higher risk FSW.
% TODO: (*) add descriptive table
%---------------------------------------------------------------------------------------------------
\subsubsection{HIV Status}\label{model.par.fsw.hiv}
Only the 2011 and 2021 studies included serologic testing for HIV.
Among those tested in 2011 (N = 317, 98\%), 70\% were \hivp,
yielding RDS-adjusted prevalence estimate of 61\% (CI: 51--71\%) \cite{Baral2014}.
Among serologically \hivn, 11\% self-reported \hivp status (false positive), and
among serologically \hivp, 26\% self-reported \hivn status (false negative or undiagnosed).
Overall, self-reported HIV status underestimated HIV prevalence in 2011
by a factor of approximately 0.78 (55~vs~70\%).
Unadjusted HIV prevalence in 2021 was 58.8\%,
with 88\% (363/411) reporting previous awareness of \hivp status.
\par
In 2014, self-reported HIV prevalence was 38\% among respondents who reported (85\%).
This 38\% is surprisingly low considering that
the PLACE methodology explicitly aimed to sample venues
with higher HIV incidence \cite{Weir2005}, and 2014 \vs 2011 respondents
were older (median 27 \vs 25 years), % 2021 median: 28
had been selling sex longer (median 5 \vs 4 years), % 2021: 6
and tested more frequently (87 \vs 75\% tested at least once in the past year, % 2021: 75
82 \vs 63\% among self-reported \hivn).
Perhaps the differences are attributable to the sampling methodology.
Among respondents who self-reported \hivp status,
the 2014 survey also asked for age of HIV diagnosis (6\% missing).
Age of HIV diagnosis supports crude time-to-event analysis (next section),
which can account for confounding by age and censoring,
as compared to logistic regression on HIV status,
keeping in mind the limitations of self-reported HIV status.
%---------------------------------------------------------------------------------------------------
\subsubsection{Risk Factors for HIV}\label{model.par.fsw.fac}
Next, I explored the potential association of risk factors with HIV
via the following three models:%
\footnote{Logistic regression models were implemented using \texttt{lrm} from:
  \hreftt{cran.r-project.org/package=rms}.\\
Cox proportional hazards models were implemented using \texttt{coxaalen} from:
  \hreftt{cran.r-project.org/package=coxinterval}.}
\begin{enumerate}
  \item Logistic regression on serologic HIV status (2011 data)
  \item Logistic regression on self-reported HIV status (2014 data)
  \item Cox proportional hazards for interval-censored time to HIV infection,
    with interval from self-reported sex work debut 
    to either self-reported time of HIV diagnosis or survey date (2014 data);
    Figure~\ref{fig:fsw.tte.interval} illustrates
    the four potential censoring cases in this framework.
\end{enumerate}
An important limitation to all models is that
risk factors reported by FSW at the time of survey
are assumed to be fixed characteristics of the respondents,
rather than dynamic characteristics that vary over time.
Additionally, respondents with any missing variables for each individual model
were excluded from that model. % TODO: (%)
\begin{figure}
  \centering
  \includegraphics[scale=1]{diag.tte}
  \caption{Illustration of time-to-event analysis framework
    for cross-sectional FSW survey data}
  \label{fig:fsw.tte.interval}
  \floatfoot{
    $\bm{\times}$: HIV infection;
    SW: time of sex work debut;
    Dx: time of HIV diagnosis.}
\end{figure}
\par
Risk factors were selected based on
prior knowledge of plausible mechanistic influence on HIV incidence and/or prevalence.
The risk factors explored are summarized in Table~\ref{tab:fsw.stats},
including univariate and multivariable association under each model.
Variable selection for multivariable models
was performed using backward selection as described by \citet{Lawless1978},
using a $p \le 0.1$ (per variable) threshold for stepwise variable retention.
Estimated conditional effects of
variables retained in the multivariable logistic regression models
are illustrated in Figure~\ref{fig:fsw.lr}.
\begin{table}
  \centering
  \caption{Risk factors explored for association with \hivp status among FSW in Eswatini}
  \label{tab:fsw.stats}
  \input{model/tab.fsw.factor.stats}
\end{table}
\begin{figure}[h]
  \subcapoverlap
  \foreach \year/\var/\nvar in {2011/f/1,2011/c/2,2014/f/3,2014/c/3}{
  \begin{subfigure}{\nvar\linewidth/5+\linewidth/5}
    \includegraphics[scale=.7]{fsw.\year.lr.hiv.\var}
    \caption{\raggedright}
    \label{fig:fsw.lr.\year.\var}
  \end{subfigure}}
  \caption{Predicted conditional effects (probability)
    of variables in multivariable logistic regression models for HIV status}
  \label{fig:fsw.lr}
  \floatfoot{\fffsw{fig:fsw.lr}
    conditional probabilities shown for fixed covariates at arbitrary values.}
\end{figure}
\par
Following variable selection, each multivariable model was used to estimate
the total \hivp status odds ratio (logistic) or HIV incidence hazard ratio (Cox)
for each respondent in the respective survey ---
\ie $e^{X_i\,\beta}$ for respondent $i$ ---
representing an overall ``risk score'' under each model.
Respondents were then stratified into the top 20\% and bottom 80\% by these risk scores.
The values of each variable were compared between these two strata
using a test for the ratio of the means \cite{Tamhane2004} to support model parameterization;
these ratios are summarized in Table~\ref{tab:fsw.ratios},
and the distributions of variable values across the two strata
are illustrated in Figure~\ref{fig:fsw.f}.
\begin{table}
  \centering
  \caption{Ratios of HIV risk factor variables among higher \vs lower risk FSW in Eswatini}
  \label{tab:fsw.ratios}
  \input{model/tab.fsw.factor.ratios}
\end{table}

\section{Parameterization}\label{model.par}
As described in \sref{intro.model.param}, model parameterization involves
specification of model parameter values, such as proportions, probabilities, rates, and ratios,
including stratified values to reflect heterogeneity,
and sampling distributions to reflect uncertainty.
Proportions and probabilities were generally modelled using
a beta approximation of the binomial distribution (BAB, see \sref{app.math.distr.bab}),
while rates and ratios were generally modelled using
a gamma, skewnormal, or inverse gaussian distribution.
\paragraph{Notation}
If $X$ is a parameter stratified by dimensions $a,b,c$,
then $X_{ab_{1}c_{23}}$ denotes the values of $X$ for
a particular but \emph{unspecified} stratum of $a$,
the \emph{specific} stratum $b = 1$,
and the \emph{aggregated} strata $c = 2,3$
(the aggregating operation is context-dependent, \eg sum for probabilities).
Additionally, the indices $sihc$ from Table~\ref{tab:model.dims} denote ``self'' strata,
whereas $s'i'h'c'$ denote ``other'' strata --- \ie individuals' partners.%
\footnote{\label{foot:code.note}%
  In the code: R uses one-based indexing, which match the notation here directly,
  while Python uses zero-based indexing, which therefore appear as $i \rightarrow i-1$ in the code.
  Also, the model code reorders states in the ART Cascade dimension for computational efficiency,
  with $c={}$1:~Undiagnosed; 2:~Diagnosed; 3:~Virally~Un-suppressed; 4:~On~ART; 5:~Virally~Suppressed.}
Finally, I re-use several dummy variables throughout the chapter:
$\rho$ for proportions, $\lambda$ for rates, $T$ for time periods, and $f$ for constants.
\input{model/par.fsw}
\input{model/par.beta}
\input{model/par.hiv}
\input{model/par.popsex}
%===================================================================================================
\subsection{HIV Progression \& Mortality}\label{model.par.hiv}
%---------------------------------------------------------------------------------------------------
\subsubsection{HIV Progression}\label{model.par.hiv.dur}
The length of time spent in each HIV stage is related to
rates of progression between stages $\eta_{h}$,
rates of additional HIV-attributable mortality by stage $\mu_{\textsc{hiv},h}$,
and treatment via antiretroviral therapy (ART).
\citet{Lodi2011} estimate median times from seroconversion to
CD4 $<$ 500, $<$ 350, and $<$ 200 cells/mm\tsup{3}, while
\citet{Mangal2017} directly estimate the rates of progression between CD4 states $\eta_{h}$
in a simple compartmental model.
Based on these data, I modelled mean durations ($1/\eta_{h}$) of:%
\footnote{Assuming exponential distributions for durations in each CD4 state
  (see \sref{app.model.math.comp} for more details).}
0.142 years in acute infection ($h=2$, from \sref{model.par.beta.hiv});
3.35 years in CD4~$>$~500 ($h=3$);
3.74 years in 350~$<$~CD4~$<$~500 ($h=4$); and
5.26 years in 200~$<$~CD4~$<$~350 ($h=5$); plus
the remaining time until death in CD4~$<$~200 ($h=6$, AIDS).
Since the duration in acute infection ($h=2$) is randomly sampled,
the remaining duration in CD4~$>$~500 ($h=3$) is adjusted accordingly.
%---------------------------------------------------------------------------------------------------
\subsubsection{HIV Mortality}\label{model.par.hiv.mort}
Mortality rates by CD4-count in the absence of ART were estimated in
multiple African studies \cite{Badri2006,Anglaret2012,Mangal2017};
based on these data, I estimated yearly HIV-attributable mortality rates $\mu_{\textsc{hiv},h}$ as:
0 during acute phase ($h=2$);
0.4\% during CD4~$>$~500 ($h=3$);
2\% during 350~$<$~CD4~$<$~500 ($h=4$);
4\% during 200~$<$~CD4~$<$~350 ($h=5$); and 
20\% during CD4~$<$~200 ($h=6$, AIDS).
%===================================================================================================
\subsection{Antiretroviral Therapy}\label{model.par.art}
Viral suppression via antiretroviral therapy (ART) influences
the probability of HIV transmission, as well as rates of HIV progression and HIV-related mortality.
The model considers individuals on ART before ($c=4$) and after ($c=5$)
achieving full viral load suppression (VLS), as defined by undetectable HIV RNA in blood samples.
Among retained patients initiating ART, time to VLS
is usually described as ``within 6 months'' \cite{Thompson2012}.
More specifically, \citet{Mujugira2016} estimate the median time to VLS as 3 months,
yielding an estimated \emph{mean} duration for $c=4$ of 4.3 months (see \sref{app.model.math.comp}).
%---------------------------------------------------------------------------------------------------
\subsubsection{Probability of HIV Transmission}\label{model.par.art.beta}
All available evidence suggests that viral suppression by ART to undetectable levels
prevents HIV transmission, \ie undetectable = untransmittable (``U=U'') \cite{Eisinger2019}.
Thus, I assumed zero HIV transmission from individuals with VLS ($c=5$).
However, HIV transmission may still occur
during the period between ART initiation to viral suppression ($c=4$) \cite{Mujugira2016}.
\citet{Donnell2010} estimate an adjusted incidence ratio of 0.08~(0.0,~0.57) for all individuals on ART.
However, in \cite{Donnell2010} and \cite{Cohen2016}, the 1 and 4 (respectively)
genetically linked infections from individuals on ART all occurred within 90 days of ART initiation,
suggesting that risk of transmission only persists before viral suppression.
Adjusting the incidence denominator (person-time)
to 90 days per individual who initiated ART in \cite{Donnell2010}
results in approximately 3.13 times higher estimated incidence ratio: 0.25 for this specific period.%
\footnote{In \cite{Donnell2010}, individuals who initiated ART contributed
  approximately 9.4 months per-person (273 persons / 349 person-years, Tables~2~and~3);
  thus the first 3 months of each individual represent
  3/9.4 = 0.319 fewer person-months of follow-up.}
Thus, I sampled relative infectiousness on ART but before viral suppression ($c=4$)
from a beta distribution with mean (95\%~CI) of 0.25~(0.01,~0.67).
%---------------------------------------------------------------------------------------------------
\subsubsection{HIV Progression \& Mortality}\label{model.par.art.hiv}
\def\hunprog{$h = 6 \rightarrow 5 \rightarrow 4 \rightarrow 3$\xspace}
Effective ART stops CD4 cell decline and results in some CD4 recovery \cite{Battegay2006,Lawn2006}.
Most CD4 recovery occurs within the first year of treatment \cite{Battegay2006}.
Due to the limited number of modelled treatment states,
I model this initial recovery to be associated with the 4.3-month pre-VLS ART state ($c=4$).
\citet{Lawn2006,Gabillard2013} estimate an increase of between 25--39 cells/mm\tsup{3} per month
during the first 3 months of treatment.
Since HIV states $h=4,5,6$ correspond to 150, 150, and 200-wide CD4 strata,
I model rates of movement along \hunprog during pre-VLS ART ($c=4$) as
0.20, 0.20, 0.17 per month, respectively.
After initial increases, CD4 recovery is modest and plateaus.
\citet{Battegay2006} report approximate increases of
22.4 cells/mm\tsup{3} per year between years 1 and 5 on ART.
Thus, I model rates of movement along \hunprog after VLS ($c=5$) as 0.15 per year.
\par
Since higher CD4 states are modelled to have lower mortality rates (see \sref{model.par.hiv.mort}),
the modelled recovery of CD4 cells via ART described above implicitly affords a mortality benefit.
However, HIV infection is associated with increased risk of death by non-AIDS causes
--- \ie unrelated to CD4 count ---
including cardiovascular disease and renal disease \cite{Phillips2008}.
\citet{Lundgren2015} estimated 61\% reduction in non-AIDS life-threatening events due to ART.
For the same CD4 strata, \citet{Gabillard2013} also report approximately 2-times higher
mortality rates within the first year of ART versus thereafter,
suggesting that VLS is associated with 50\% mortality reduction independent of CD4 increase.
Thus, I modelled an additional 50\% reduction in mortality among individuals with VLS ($c=5$),
and half this (25\%) reduction before achieving VLS ($c=4$).
% TODO: rates of diagnosis, testing, vls
\section{Parameterization}\label{model.par}
As described in \sref{intro.model.param}, model parameterization involves
specification of model parameter values, such as proportions, probabilities, rates, and ratios,
including stratified values to reflect heterogeneity,
and sampling distributions to reflect uncertainty.
Proportions and probabilities were generally modelled using
a beta approximation of the binomial distribution (BAB, see \sref{app.math.distr.bab}),
while rates and ratios were generally modelled using
a gamma, skewnormal, or inverse gaussian distribution.
\paragraph{Notation}
If $X$ is a parameter stratified by dimensions $a,b,c$,
then $X_{ab_{1}c_{23}}$ denotes the values of $X$ for
a particular but \emph{unspecified} stratum of $a$,
the \emph{specific} stratum $b = 1$,
and the \emph{aggregated} strata $c = 2,3$
(the aggregating operation is context-dependent, \eg sum for probabilities).
Additionally, the indices $sihc$ from Table~\ref{tab:model.dims} denote ``self'' strata,
whereas $s'i'h'c'$ denote ``other'' strata --- \ie individuals' partners.%
\footnote{\label{foot:code.note}%
  In the code: R uses one-based indexing, which match the notation here directly,
  while Python uses zero-based indexing, which therefore appear as $i \rightarrow i-1$ in the code.
  Also, the model code reorders states in the ART Cascade dimension for computational efficiency,
  with $c={}$1:~Undiagnosed; 2:~Diagnosed; 3:~Virally~Un-suppressed; 4:~On~ART; 5:~Virally~Suppressed.}
Finally, I re-use several dummy variables throughout the chapter:
$\rho$ for proportions, $\lambda$ for rates, $T$ for time periods, and $f$ for constants.
\input{model/par.fsw}
\input{model/par.beta}
\input{model/par.hiv}
\input{model/par.popsex}
%===================================================================================================
\subsection{HIV Progression \& Mortality}\label{model.par.hiv}
%---------------------------------------------------------------------------------------------------
\subsubsection{HIV Progression}\label{model.par.hiv.dur}
The length of time spent in each HIV stage is related to
rates of progression between stages $\eta_{h}$,
rates of additional HIV-attributable mortality by stage $\mu_{\textsc{hiv},h}$,
and treatment via antiretroviral therapy (ART).
\citet{Lodi2011} estimate median times from seroconversion to
CD4 $<$ 500, $<$ 350, and $<$ 200 cells/mm\tsup{3}, while
\citet{Mangal2017} directly estimate the rates of progression between CD4 states $\eta_{h}$
in a simple compartmental model.
Based on these data, I modelled mean durations ($1/\eta_{h}$) of:%
\footnote{Assuming exponential distributions for durations in each CD4 state
  (see \sref{app.model.math.comp} for more details).}
0.142 years in acute infection ($h=2$, from \sref{model.par.beta.hiv});
3.35 years in CD4~$>$~500 ($h=3$);
3.74 years in 350~$<$~CD4~$<$~500 ($h=4$); and
5.26 years in 200~$<$~CD4~$<$~350 ($h=5$); plus
the remaining time until death in CD4~$<$~200 ($h=6$, AIDS).
Since the duration in acute infection ($h=2$) is randomly sampled,
the remaining duration in CD4~$>$~500 ($h=3$) is adjusted accordingly.
%---------------------------------------------------------------------------------------------------
\subsubsection{HIV Mortality}\label{model.par.hiv.mort}
Mortality rates by CD4-count in the absence of ART were estimated in
multiple African studies \cite{Badri2006,Anglaret2012,Mangal2017};
based on these data, I estimated yearly HIV-attributable mortality rates $\mu_{\textsc{hiv},h}$ as:
0 during acute phase ($h=2$);
0.4\% during CD4~$>$~500 ($h=3$);
2\% during 350~$<$~CD4~$<$~500 ($h=4$);
4\% during 200~$<$~CD4~$<$~350 ($h=5$); and 
20\% during CD4~$<$~200 ($h=6$, AIDS).
%===================================================================================================
\subsection{Antiretroviral Therapy}\label{model.par.art}
Viral suppression via antiretroviral therapy (ART) influences
the probability of HIV transmission, as well as rates of HIV progression and HIV-related mortality.
The model considers individuals on ART before ($c=4$) and after ($c=5$)
achieving full viral load suppression (VLS), as defined by undetectable HIV RNA in blood samples.
Among retained patients initiating ART, time to VLS
is usually described as ``within 6 months'' \cite{Thompson2012}.
More specifically, \citet{Mujugira2016} estimate the median time to VLS as 3 months,
yielding an estimated \emph{mean} duration for $c=4$ of 4.3 months (see \sref{app.model.math.comp}).
%---------------------------------------------------------------------------------------------------
\subsubsection{Probability of HIV Transmission}\label{model.par.art.beta}
All available evidence suggests that viral suppression by ART to undetectable levels
prevents HIV transmission, \ie undetectable = untransmittable (``U=U'') \cite{Eisinger2019}.
Thus, I assumed zero HIV transmission from individuals with VLS ($c=5$).
However, HIV transmission may still occur
during the period between ART initiation to viral suppression ($c=4$) \cite{Mujugira2016}.
\citet{Donnell2010} estimate an adjusted incidence ratio of 0.08~(0.0,~0.57) for all individuals on ART.
However, in \cite{Donnell2010} and \cite{Cohen2016}, the 1 and 4 (respectively)
genetically linked infections from individuals on ART all occurred within 90 days of ART initiation,
suggesting that risk of transmission only persists before viral suppression.
Adjusting the incidence denominator (person-time)
to 90 days per individual who initiated ART in \cite{Donnell2010}
results in approximately 3.13 times higher estimated incidence ratio: 0.25 for this specific period.%
\footnote{In \cite{Donnell2010}, individuals who initiated ART contributed
  approximately 9.4 months per-person (273 persons / 349 person-years, Tables~2~and~3);
  thus the first 3 months of each individual represent
  3/9.4 = 0.319 fewer person-months of follow-up.}
Thus, I sampled relative infectiousness on ART but before viral suppression ($c=4$)
from a beta distribution with mean (95\%~CI) of 0.25~(0.01,~0.67).
%---------------------------------------------------------------------------------------------------
\subsubsection{HIV Progression \& Mortality}\label{model.par.art.hiv}
\def\hunprog{$h = 6 \rightarrow 5 \rightarrow 4 \rightarrow 3$\xspace}
Effective ART stops CD4 cell decline and results in some CD4 recovery \cite{Battegay2006,Lawn2006}.
Most CD4 recovery occurs within the first year of treatment \cite{Battegay2006}.
Due to the limited number of modelled treatment states,
I model this initial recovery to be associated with the 4.3-month pre-VLS ART state ($c=4$).
\citet{Lawn2006,Gabillard2013} estimate an increase of between 25--39 cells/mm\tsup{3} per month
during the first 3 months of treatment.
Since HIV states $h=4,5,6$ correspond to 150, 150, and 200-wide CD4 strata,
I model rates of movement along \hunprog during pre-VLS ART ($c=4$) as
0.20, 0.20, 0.17 per month, respectively.
After initial increases, CD4 recovery is modest and plateaus.
\citet{Battegay2006} report approximate increases of
22.4 cells/mm\tsup{3} per year between years 1 and 5 on ART.
Thus, I model rates of movement along \hunprog after VLS ($c=5$) as 0.15 per year.
\par
Since higher CD4 states are modelled to have lower mortality rates (see \sref{model.par.hiv.mort}),
the modelled recovery of CD4 cells via ART described above implicitly affords a mortality benefit.
However, HIV infection is associated with increased risk of death by non-AIDS causes
--- \ie unrelated to CD4 count ---
including cardiovascular disease and renal disease \cite{Phillips2008}.
\citet{Lundgren2015} estimated 61\% reduction in non-AIDS life-threatening events due to ART.
For the same CD4 strata, \citet{Gabillard2013} also report approximately 2-times higher
mortality rates within the first year of ART versus thereafter,
suggesting that VLS is associated with 50\% mortality reduction independent of CD4 increase.
Thus, I modelled an additional 50\% reduction in mortality among individuals with VLS ($c=5$),
and half this (25\%) reduction before achieving VLS ($c=4$).
% TODO: rates of diagnosis, testing, vls
\section{Parameterization}\label{model.par}
As described in \sref{intro.model.param}, model parameterization involves
specification of model parameter values, such as proportions, probabilities, rates, and ratios,
including stratified values to reflect heterogeneity,
and sampling distributions to reflect uncertainty.
Proportions and probabilities were generally modelled using
a beta approximation of the binomial distribution (BAB, see \sref{app.math.distr.bab}),
while rates and ratios were generally modelled using
a gamma, skewnormal, or inverse gaussian distribution.
\paragraph{Notation}
If $X$ is a parameter stratified by dimensions $a,b,c$,
then $X_{ab_{1}c_{23}}$ denotes the values of $X$ for
a particular but \emph{unspecified} stratum of $a$,
the \emph{specific} stratum $b = 1$,
and the \emph{aggregated} strata $c = 2,3$
(the aggregating operation is context-dependent, \eg sum for probabilities).
Additionally, the indices $sihc$ from Table~\ref{tab:model.dims} denote ``self'' strata,
whereas $s'i'h'c'$ denote ``other'' strata --- \ie individuals' partners.%
\footnote{\label{foot:code.note}%
  In the code: R uses one-based indexing, which match the notation here directly,
  while Python uses zero-based indexing, which therefore appear as $i \rightarrow i-1$ in the code.
  Also, the model code reorders states in the ART Cascade dimension for computational efficiency,
  with $c={}$1:~Undiagnosed; 2:~Diagnosed; 3:~Virally~Un-suppressed; 4:~On~ART; 5:~Virally~Suppressed.}
Finally, I re-use several dummy variables throughout the chapter:
$\rho$ for proportions, $\lambda$ for rates, $T$ for time periods, and $f$ for constants.
%===================================================================================================
\subsection{Risk Heterogeneity Among FSW}\label{model.par.fsw}
Existing HIV transmission models which include FSW
have rarely sub-stratified this population, such as to reflect
differential HIV risk or distinct typologies of sex work \cite{Blanchard2008,Scorgie2012};
yet such heterogeneities may influence transmission dynamics.
Among the studies identified in Chapter~\ref{sr},
only three sub-stratified FSW by risk-related factors:
\citet{Cremin2017} defined three levels of risk via regression analysis,
\citet{Low2015} distinguished between occasional and full-time FSW, while
\citet{Shannon2015} sub-stratified FSW by
work environment, violence exposure, and context-specific structural factors.
Seven other studies, reflecting two unique models \cite{Johnson2012,Maheu-Giroux2017},
employed age stratification of all activity groups, including FSW;
these models had several risk-related parameters which varied by age.
\par
The model structure here (Figure~\ref{fig:model.risk})
was designed to capture \emph{within}-FSW risk heterogeneity.
The objective of the following analysis was therefore to parameterize
lower \vs higher risk FSW.
I sought to define these groups based on biobehavioural and/or contextual factors
which are demonstrably associated with HIV risk,
and which can be mechanistically incorporated into a transmission model ---
\ie through the force of infection equation.
Later, the parameterization of these groups was validated through model fitting
to relative differences in HIV prevalence \sref{model.cal.targ.prev}.
\par
Many cross-sectional studies of HIV among FSW quantify
the association of risk factors with HIV serostatus
\cite{Aklilu2001,Dunkle2005,Scorgie2012,Jonas2020}.
However, serostatus reflects cumulative risk exposure,
whereas sexual risk behaviour is dynamic \cite{Watts2010,vanWees2020},
as is use of prevention resources \cite{Roberts2020}.
For example, while HIV prevalence often increases with age,
HIV incidence among women can peak before age 25 \cite{Dellar2015}.
Thus, risk factors associated with HIV serostatus are not necessarily
mechanistically related to HIV acquisition.
Indeed, FSW may reduce risk behaviours in response to seroconversion \cite{McClelland2006}.
Cohort studies that measure incidence
can help identify risk factors for HIV acquisition \cite{McKinnon2015,Nouaman2022},
but large sample sizes are often required to accurately estimate overall incidence rate,
let alone risk factors \cite{Priddy2011}.
%---------------------------------------------------------------------------------------------------
\subsubsection{FSW Survey Data}\label{model.par.fsw.data}
Three biobehavioural surveys, in
2011 \cite{Baral2014} (N = 325),
2014 \cite{EswKP2014} (N = 781), and
2021 \cite{EswIBBS2022} (N = 676)
provide HIV status and biobehavioural data on FSW in Eswatini.
The 2011 and 2021 surveys featured serologic HIV testing,
and employed respondent driven sampling (RDS, details in \cite{Yam2013}).
The 2014 survey relied on self-reported HIV status,
andd employed venue-based snowball sampling, based on the
Priorities for Local AIDS Control Efforts (PLACE) methodology,
which aims to identify areas of higher incidence \cite{Weir2005}.
More details about each study are given in \sref{intro.esw.hiv.data} and Table~\ref{tab:esw.data}.
I analyzed the individual-level data from 2011 and 2014 (data from 2021 not yet available)
to explore the potential association of biobehavioural factors with HIV risk,
so that such factors could then be used to distinguish between
lower risk \vs higher risk FSW.
% TODO: (*) add descriptive table
%---------------------------------------------------------------------------------------------------
\subsubsection{HIV Status}\label{model.par.fsw.hiv}
Only the 2011 and 2021 studies included serologic testing for HIV.
Among those tested in 2011 (N = 317, 98\%), 70\% were \hivp,
yielding RDS-adjusted prevalence estimate of 61\% (CI: 51--71\%) \cite{Baral2014}.
Among serologically \hivn, 11\% self-reported \hivp status (false positive), and
among serologically \hivp, 26\% self-reported \hivn status (false negative or undiagnosed).
Overall, self-reported HIV status underestimated HIV prevalence in 2011
by a factor of approximately 0.78 (55~vs~70\%).
Unadjusted HIV prevalence in 2021 was 58.8\%,
with 88\% (363/411) reporting previous awareness of \hivp status.
\par
In 2014, self-reported HIV prevalence was 38\% among respondents who reported (85\%).
This 38\% is surprisingly low considering that
the PLACE methodology explicitly aimed to sample venues
with higher HIV incidence \cite{Weir2005}, and 2014 \vs 2011 respondents
were older (median 27 \vs 25 years), % 2021 median: 28
had been selling sex longer (median 5 \vs 4 years), % 2021: 6
and tested more frequently (87 \vs 75\% tested at least once in the past year, % 2021: 75
82 \vs 63\% among self-reported \hivn).
Perhaps the differences are attributable to the sampling methodology.
Among respondents who self-reported \hivp status,
the 2014 survey also asked for age of HIV diagnosis (6\% missing).
Age of HIV diagnosis supports crude time-to-event analysis (next section),
which can account for confounding by age and censoring,
as compared to logistic regression on HIV status,
keeping in mind the limitations of self-reported HIV status.
%---------------------------------------------------------------------------------------------------
\subsubsection{Risk Factors for HIV}\label{model.par.fsw.fac}
Next, I explored the potential association of risk factors with HIV
via the following three models:%
\footnote{Logistic regression models were implemented using \texttt{lrm} from:
  \hreftt{cran.r-project.org/package=rms}.\\
Cox proportional hazards models were implemented using \texttt{coxaalen} from:
  \hreftt{cran.r-project.org/package=coxinterval}.}
\begin{enumerate}
  \item Logistic regression on serologic HIV status (2011 data)
  \item Logistic regression on self-reported HIV status (2014 data)
  \item Cox proportional hazards for interval-censored time to HIV infection,
    with interval from self-reported sex work debut 
    to either self-reported time of HIV diagnosis or survey date (2014 data);
    Figure~\ref{fig:fsw.tte.interval} illustrates
    the four potential censoring cases in this framework.
\end{enumerate}
An important limitation to all models is that
risk factors reported by FSW at the time of survey
are assumed to be fixed characteristics of the respondents,
rather than dynamic characteristics that vary over time.
Additionally, respondents with any missing variables for each individual model
were excluded from that model. % TODO: (%)
\begin{figure}
  \centering
  \includegraphics[scale=1]{diag.tte}
  \caption{Illustration of time-to-event analysis framework
    for cross-sectional FSW survey data}
  \label{fig:fsw.tte.interval}
  \floatfoot{
    $\bm{\times}$: HIV infection;
    SW: time of sex work debut;
    Dx: time of HIV diagnosis.}
\end{figure}
\par
Risk factors were selected based on
prior knowledge of plausible mechanistic influence on HIV incidence and/or prevalence.
The risk factors explored are summarized in Table~\ref{tab:fsw.stats},
including univariate and multivariable association under each model.
Variable selection for multivariable models
was performed using backward selection as described by \citet{Lawless1978},
using a $p \le 0.1$ (per variable) threshold for stepwise variable retention.
Estimated conditional effects of
variables retained in the multivariable logistic regression models
are illustrated in Figure~\ref{fig:fsw.lr}.
\begin{table}
  \centering
  \caption{Risk factors explored for association with \hivp status among FSW in Eswatini}
  \label{tab:fsw.stats}
  \input{model/tab.fsw.factor.stats}
\end{table}
\begin{figure}[h]
  \subcapoverlap
  \foreach \year/\var/\nvar in {2011/f/1,2011/c/2,2014/f/3,2014/c/3}{
  \begin{subfigure}{\nvar\linewidth/5+\linewidth/5}
    \includegraphics[scale=.7]{fsw.\year.lr.hiv.\var}
    \caption{\raggedright}
    \label{fig:fsw.lr.\year.\var}
  \end{subfigure}}
  \caption{Predicted conditional effects (probability)
    of variables in multivariable logistic regression models for HIV status}
  \label{fig:fsw.lr}
  \floatfoot{\fffsw{fig:fsw.lr}
    conditional probabilities shown for fixed covariates at arbitrary values.}
\end{figure}
\par
Following variable selection, each multivariable model was used to estimate
the total \hivp status odds ratio (logistic) or HIV incidence hazard ratio (Cox)
for each respondent in the respective survey ---
\ie $e^{X_i\,\beta}$ for respondent $i$ ---
representing an overall ``risk score'' under each model.
Respondents were then stratified into the top 20\% and bottom 80\% by these risk scores.
The values of each variable were compared between these two strata
using a test for the ratio of the means \cite{Tamhane2004} to support model parameterization;
these ratios are summarized in Table~\ref{tab:fsw.ratios},
and the distributions of variable values across the two strata
are illustrated in Figure~\ref{fig:fsw.f}.
\begin{table}
  \centering
  \caption{Ratios of HIV risk factor variables among higher \vs lower risk FSW in Eswatini}
  \label{tab:fsw.ratios}
  \input{model/tab.fsw.factor.ratios}
\end{table}

\section{Parameterization}\label{model.par}
As described in \sref{intro.model.param}, model parameterization involves
specification of model parameter values, such as proportions, probabilities, rates, and ratios,
including stratified values to reflect heterogeneity,
and sampling distributions to reflect uncertainty.
Proportions and probabilities were generally modelled using
a beta approximation of the binomial distribution (BAB, see \sref{app.math.distr.bab}),
while rates and ratios were generally modelled using
a gamma, skewnormal, or inverse gaussian distribution.
\paragraph{Notation}
If $X$ is a parameter stratified by dimensions $a,b,c$,
then $X_{ab_{1}c_{23}}$ denotes the values of $X$ for
a particular but \emph{unspecified} stratum of $a$,
the \emph{specific} stratum $b = 1$,
and the \emph{aggregated} strata $c = 2,3$
(the aggregating operation is context-dependent, \eg sum for probabilities).
Additionally, the indices $sihc$ from Table~\ref{tab:model.dims} denote ``self'' strata,
whereas $s'i'h'c'$ denote ``other'' strata --- \ie individuals' partners.%
\footnote{\label{foot:code.note}%
  In the code: R uses one-based indexing, which match the notation here directly,
  while Python uses zero-based indexing, which therefore appear as $i \rightarrow i-1$ in the code.
  Also, the model code reorders states in the ART Cascade dimension for computational efficiency,
  with $c={}$1:~Undiagnosed; 2:~Diagnosed; 3:~Virally~Un-suppressed; 4:~On~ART; 5:~Virally~Suppressed.}
Finally, I re-use several dummy variables throughout the chapter:
$\rho$ for proportions, $\lambda$ for rates, $T$ for time periods, and $f$ for constants.
\input{model/par.fsw}
\input{model/par.beta}
\input{model/par.hiv}
\input{model/par.popsex}
%===================================================================================================
\subsection{HIV Progression \& Mortality}\label{model.par.hiv}
%---------------------------------------------------------------------------------------------------
\subsubsection{HIV Progression}\label{model.par.hiv.dur}
The length of time spent in each HIV stage is related to
rates of progression between stages $\eta_{h}$,
rates of additional HIV-attributable mortality by stage $\mu_{\textsc{hiv},h}$,
and treatment via antiretroviral therapy (ART).
\citet{Lodi2011} estimate median times from seroconversion to
CD4 $<$ 500, $<$ 350, and $<$ 200 cells/mm\tsup{3}, while
\citet{Mangal2017} directly estimate the rates of progression between CD4 states $\eta_{h}$
in a simple compartmental model.
Based on these data, I modelled mean durations ($1/\eta_{h}$) of:%
\footnote{Assuming exponential distributions for durations in each CD4 state
  (see \sref{app.model.math.comp} for more details).}
0.142 years in acute infection ($h=2$, from \sref{model.par.beta.hiv});
3.35 years in CD4~$>$~500 ($h=3$);
3.74 years in 350~$<$~CD4~$<$~500 ($h=4$); and
5.26 years in 200~$<$~CD4~$<$~350 ($h=5$); plus
the remaining time until death in CD4~$<$~200 ($h=6$, AIDS).
Since the duration in acute infection ($h=2$) is randomly sampled,
the remaining duration in CD4~$>$~500 ($h=3$) is adjusted accordingly.
%---------------------------------------------------------------------------------------------------
\subsubsection{HIV Mortality}\label{model.par.hiv.mort}
Mortality rates by CD4-count in the absence of ART were estimated in
multiple African studies \cite{Badri2006,Anglaret2012,Mangal2017};
based on these data, I estimated yearly HIV-attributable mortality rates $\mu_{\textsc{hiv},h}$ as:
0 during acute phase ($h=2$);
0.4\% during CD4~$>$~500 ($h=3$);
2\% during 350~$<$~CD4~$<$~500 ($h=4$);
4\% during 200~$<$~CD4~$<$~350 ($h=5$); and 
20\% during CD4~$<$~200 ($h=6$, AIDS).
%===================================================================================================
\subsection{Antiretroviral Therapy}\label{model.par.art}
Viral suppression via antiretroviral therapy (ART) influences
the probability of HIV transmission, as well as rates of HIV progression and HIV-related mortality.
The model considers individuals on ART before ($c=4$) and after ($c=5$)
achieving full viral load suppression (VLS), as defined by undetectable HIV RNA in blood samples.
Among retained patients initiating ART, time to VLS
is usually described as ``within 6 months'' \cite{Thompson2012}.
More specifically, \citet{Mujugira2016} estimate the median time to VLS as 3 months,
yielding an estimated \emph{mean} duration for $c=4$ of 4.3 months (see \sref{app.model.math.comp}).
%---------------------------------------------------------------------------------------------------
\subsubsection{Probability of HIV Transmission}\label{model.par.art.beta}
All available evidence suggests that viral suppression by ART to undetectable levels
prevents HIV transmission, \ie undetectable = untransmittable (``U=U'') \cite{Eisinger2019}.
Thus, I assumed zero HIV transmission from individuals with VLS ($c=5$).
However, HIV transmission may still occur
during the period between ART initiation to viral suppression ($c=4$) \cite{Mujugira2016}.
\citet{Donnell2010} estimate an adjusted incidence ratio of 0.08~(0.0,~0.57) for all individuals on ART.
However, in \cite{Donnell2010} and \cite{Cohen2016}, the 1 and 4 (respectively)
genetically linked infections from individuals on ART all occurred within 90 days of ART initiation,
suggesting that risk of transmission only persists before viral suppression.
Adjusting the incidence denominator (person-time)
to 90 days per individual who initiated ART in \cite{Donnell2010}
results in approximately 3.13 times higher estimated incidence ratio: 0.25 for this specific period.%
\footnote{In \cite{Donnell2010}, individuals who initiated ART contributed
  approximately 9.4 months per-person (273 persons / 349 person-years, Tables~2~and~3);
  thus the first 3 months of each individual represent
  3/9.4 = 0.319 fewer person-months of follow-up.}
Thus, I sampled relative infectiousness on ART but before viral suppression ($c=4$)
from a beta distribution with mean (95\%~CI) of 0.25~(0.01,~0.67).
%---------------------------------------------------------------------------------------------------
\subsubsection{HIV Progression \& Mortality}\label{model.par.art.hiv}
\def\hunprog{$h = 6 \rightarrow 5 \rightarrow 4 \rightarrow 3$\xspace}
Effective ART stops CD4 cell decline and results in some CD4 recovery \cite{Battegay2006,Lawn2006}.
Most CD4 recovery occurs within the first year of treatment \cite{Battegay2006}.
Due to the limited number of modelled treatment states,
I model this initial recovery to be associated with the 4.3-month pre-VLS ART state ($c=4$).
\citet{Lawn2006,Gabillard2013} estimate an increase of between 25--39 cells/mm\tsup{3} per month
during the first 3 months of treatment.
Since HIV states $h=4,5,6$ correspond to 150, 150, and 200-wide CD4 strata,
I model rates of movement along \hunprog during pre-VLS ART ($c=4$) as
0.20, 0.20, 0.17 per month, respectively.
After initial increases, CD4 recovery is modest and plateaus.
\citet{Battegay2006} report approximate increases of
22.4 cells/mm\tsup{3} per year between years 1 and 5 on ART.
Thus, I model rates of movement along \hunprog after VLS ($c=5$) as 0.15 per year.
\par
Since higher CD4 states are modelled to have lower mortality rates (see \sref{model.par.hiv.mort}),
the modelled recovery of CD4 cells via ART described above implicitly affords a mortality benefit.
However, HIV infection is associated with increased risk of death by non-AIDS causes
--- \ie unrelated to CD4 count ---
including cardiovascular disease and renal disease \cite{Phillips2008}.
\citet{Lundgren2015} estimated 61\% reduction in non-AIDS life-threatening events due to ART.
For the same CD4 strata, \citet{Gabillard2013} also report approximately 2-times higher
mortality rates within the first year of ART versus thereafter,
suggesting that VLS is associated with 50\% mortality reduction independent of CD4 increase.
Thus, I modelled an additional 50\% reduction in mortality among individuals with VLS ($c=5$),
and half this (25\%) reduction before achieving VLS ($c=4$).
% TODO: rates of diagnosis, testing, vls
\section{Parameterization}\label{model.par}
As described in \sref{intro.model.param}, model parameterization involves
specification of model parameter values, such as proportions, probabilities, rates, and ratios,
including stratified values to reflect heterogeneity,
and sampling distributions to reflect uncertainty.
Proportions and probabilities were generally modelled using
a beta approximation of the binomial distribution (BAB, see \sref{app.math.distr.bab}),
while rates and ratios were generally modelled using
a gamma, skewnormal, or inverse gaussian distribution.
\paragraph{Notation}
If $X$ is a parameter stratified by dimensions $a,b,c$,
then $X_{ab_{1}c_{23}}$ denotes the values of $X$ for
a particular but \emph{unspecified} stratum of $a$,
the \emph{specific} stratum $b = 1$,
and the \emph{aggregated} strata $c = 2,3$
(the aggregating operation is context-dependent, \eg sum for probabilities).
Additionally, the indices $sihc$ from Table~\ref{tab:model.dims} denote ``self'' strata,
whereas $s'i'h'c'$ denote ``other'' strata --- \ie individuals' partners.%
\footnote{\label{foot:code.note}%
  In the code: R uses one-based indexing, which match the notation here directly,
  while Python uses zero-based indexing, which therefore appear as $i \rightarrow i-1$ in the code.
  Also, the model code reorders states in the ART Cascade dimension for computational efficiency,
  with $c={}$1:~Undiagnosed; 2:~Diagnosed; 3:~Virally~Un-suppressed; 4:~On~ART; 5:~Virally~Suppressed.}
Finally, I re-use several dummy variables throughout the chapter:
$\rho$ for proportions, $\lambda$ for rates, $T$ for time periods, and $f$ for constants.
\input{model/par.fsw}
\input{model/par.beta}
\input{model/par.hiv}
\input{model/par.popsex}
%===================================================================================================
\subsection{HIV Progression \& Mortality}\label{model.par.hiv}
%---------------------------------------------------------------------------------------------------
\subsubsection{HIV Progression}\label{model.par.hiv.dur}
The length of time spent in each HIV stage is related to
rates of progression between stages $\eta_{h}$,
rates of additional HIV-attributable mortality by stage $\mu_{\textsc{hiv},h}$,
and treatment via antiretroviral therapy (ART).
\citet{Lodi2011} estimate median times from seroconversion to
CD4 $<$ 500, $<$ 350, and $<$ 200 cells/mm\tsup{3}, while
\citet{Mangal2017} directly estimate the rates of progression between CD4 states $\eta_{h}$
in a simple compartmental model.
Based on these data, I modelled mean durations ($1/\eta_{h}$) of:%
\footnote{Assuming exponential distributions for durations in each CD4 state
  (see \sref{app.model.math.comp} for more details).}
0.142 years in acute infection ($h=2$, from \sref{model.par.beta.hiv});
3.35 years in CD4~$>$~500 ($h=3$);
3.74 years in 350~$<$~CD4~$<$~500 ($h=4$); and
5.26 years in 200~$<$~CD4~$<$~350 ($h=5$); plus
the remaining time until death in CD4~$<$~200 ($h=6$, AIDS).
Since the duration in acute infection ($h=2$) is randomly sampled,
the remaining duration in CD4~$>$~500 ($h=3$) is adjusted accordingly.
%---------------------------------------------------------------------------------------------------
\subsubsection{HIV Mortality}\label{model.par.hiv.mort}
Mortality rates by CD4-count in the absence of ART were estimated in
multiple African studies \cite{Badri2006,Anglaret2012,Mangal2017};
based on these data, I estimated yearly HIV-attributable mortality rates $\mu_{\textsc{hiv},h}$ as:
0 during acute phase ($h=2$);
0.4\% during CD4~$>$~500 ($h=3$);
2\% during 350~$<$~CD4~$<$~500 ($h=4$);
4\% during 200~$<$~CD4~$<$~350 ($h=5$); and 
20\% during CD4~$<$~200 ($h=6$, AIDS).
%===================================================================================================
\subsection{Antiretroviral Therapy}\label{model.par.art}
Viral suppression via antiretroviral therapy (ART) influences
the probability of HIV transmission, as well as rates of HIV progression and HIV-related mortality.
The model considers individuals on ART before ($c=4$) and after ($c=5$)
achieving full viral load suppression (VLS), as defined by undetectable HIV RNA in blood samples.
Among retained patients initiating ART, time to VLS
is usually described as ``within 6 months'' \cite{Thompson2012}.
More specifically, \citet{Mujugira2016} estimate the median time to VLS as 3 months,
yielding an estimated \emph{mean} duration for $c=4$ of 4.3 months (see \sref{app.model.math.comp}).
%---------------------------------------------------------------------------------------------------
\subsubsection{Probability of HIV Transmission}\label{model.par.art.beta}
All available evidence suggests that viral suppression by ART to undetectable levels
prevents HIV transmission, \ie undetectable = untransmittable (``U=U'') \cite{Eisinger2019}.
Thus, I assumed zero HIV transmission from individuals with VLS ($c=5$).
However, HIV transmission may still occur
during the period between ART initiation to viral suppression ($c=4$) \cite{Mujugira2016}.
\citet{Donnell2010} estimate an adjusted incidence ratio of 0.08~(0.0,~0.57) for all individuals on ART.
However, in \cite{Donnell2010} and \cite{Cohen2016}, the 1 and 4 (respectively)
genetically linked infections from individuals on ART all occurred within 90 days of ART initiation,
suggesting that risk of transmission only persists before viral suppression.
Adjusting the incidence denominator (person-time)
to 90 days per individual who initiated ART in \cite{Donnell2010}
results in approximately 3.13 times higher estimated incidence ratio: 0.25 for this specific period.%
\footnote{In \cite{Donnell2010}, individuals who initiated ART contributed
  approximately 9.4 months per-person (273 persons / 349 person-years, Tables~2~and~3);
  thus the first 3 months of each individual represent
  3/9.4 = 0.319 fewer person-months of follow-up.}
Thus, I sampled relative infectiousness on ART but before viral suppression ($c=4$)
from a beta distribution with mean (95\%~CI) of 0.25~(0.01,~0.67).
%---------------------------------------------------------------------------------------------------
\subsubsection{HIV Progression \& Mortality}\label{model.par.art.hiv}
\def\hunprog{$h = 6 \rightarrow 5 \rightarrow 4 \rightarrow 3$\xspace}
Effective ART stops CD4 cell decline and results in some CD4 recovery \cite{Battegay2006,Lawn2006}.
Most CD4 recovery occurs within the first year of treatment \cite{Battegay2006}.
Due to the limited number of modelled treatment states,
I model this initial recovery to be associated with the 4.3-month pre-VLS ART state ($c=4$).
\citet{Lawn2006,Gabillard2013} estimate an increase of between 25--39 cells/mm\tsup{3} per month
during the first 3 months of treatment.
Since HIV states $h=4,5,6$ correspond to 150, 150, and 200-wide CD4 strata,
I model rates of movement along \hunprog during pre-VLS ART ($c=4$) as
0.20, 0.20, 0.17 per month, respectively.
After initial increases, CD4 recovery is modest and plateaus.
\citet{Battegay2006} report approximate increases of
22.4 cells/mm\tsup{3} per year between years 1 and 5 on ART.
Thus, I model rates of movement along \hunprog after VLS ($c=5$) as 0.15 per year.
\par
Since higher CD4 states are modelled to have lower mortality rates (see \sref{model.par.hiv.mort}),
the modelled recovery of CD4 cells via ART described above implicitly affords a mortality benefit.
However, HIV infection is associated with increased risk of death by non-AIDS causes
--- \ie unrelated to CD4 count ---
including cardiovascular disease and renal disease \cite{Phillips2008}.
\citet{Lundgren2015} estimated 61\% reduction in non-AIDS life-threatening events due to ART.
For the same CD4 strata, \citet{Gabillard2013} also report approximately 2-times higher
mortality rates within the first year of ART versus thereafter,
suggesting that VLS is associated with 50\% mortality reduction independent of CD4 increase.
Thus, I modelled an additional 50\% reduction in mortality among individuals with VLS ($c=5$),
and half this (25\%) reduction before achieving VLS ($c=4$).
% TODO: rates of diagnosis, testing, vls
%===================================================================================================
\subsection{Risk Heterogeneity Among FSW}\label{model.par.fsw}
Existing HIV transmission models which include FSW
have rarely sub-stratified this population, such as to reflect
differential HIV risk or distinct typologies of sex work \cite{Blanchard2008,Scorgie2012};
yet such heterogeneities may influence transmission dynamics.
Among the studies identified in Chapter~\ref{sr},
only three sub-stratified FSW by risk-related factors:
\citet{Cremin2017} defined three levels of risk via regression analysis,
\citet{Low2015} distinguished between occasional and full-time FSW, while
\citet{Shannon2015} sub-stratified FSW by
work environment, violence exposure, and context-specific structural factors.
Seven other studies, reflecting two unique models \cite{Johnson2012,Maheu-Giroux2017},
employed age stratification of all activity groups, including FSW;
these models had several risk-related parameters which varied by age.
\par
The model structure here (Figure~\ref{fig:model.risk})
was designed to capture \emph{within}-FSW risk heterogeneity.
The objective of the following analysis was therefore to parameterize
lower \vs higher risk FSW.
I sought to define these groups based on biobehavioural and/or contextual factors
which are demonstrably associated with HIV risk,
and which can be mechanistically incorporated into a transmission model ---
\ie through the force of infection equation.
Later, the parameterization of these groups was validated through model fitting
to relative differences in HIV prevalence \sref{model.cal.targ.prev}.
\par
Many cross-sectional studies of HIV among FSW quantify
the association of risk factors with HIV serostatus
\cite{Aklilu2001,Dunkle2005,Scorgie2012,Jonas2020}.
However, serostatus reflects cumulative risk exposure,
whereas sexual risk behaviour is dynamic \cite{Watts2010,vanWees2020},
as is use of prevention resources \cite{Roberts2020}.
For example, while HIV prevalence often increases with age,
HIV incidence among women can peak before age 25 \cite{Dellar2015}.
Thus, risk factors associated with HIV serostatus are not necessarily
mechanistically related to HIV acquisition.
Indeed, FSW may reduce risk behaviours in response to seroconversion \cite{McClelland2006}.
Cohort studies that measure incidence
can help identify risk factors for HIV acquisition \cite{McKinnon2015,Nouaman2022},
but large sample sizes are often required to accurately estimate overall incidence rate,
let alone risk factors \cite{Priddy2011}.
%---------------------------------------------------------------------------------------------------
\subsubsection{FSW Survey Data}\label{model.par.fsw.data}
Three biobehavioural surveys, in
2011 \cite{Baral2014} (N = 325),
2014 \cite{EswKP2014} (N = 781), and
2021 \cite{EswIBBS2022} (N = 676)
provide HIV status and biobehavioural data on FSW in Eswatini.
The 2011 and 2021 surveys featured serologic HIV testing,
and employed respondent driven sampling (RDS, details in \cite{Yam2013}).
The 2014 survey relied on self-reported HIV status,
andd employed venue-based snowball sampling, based on the
Priorities for Local AIDS Control Efforts (PLACE) methodology,
which aims to identify areas of higher incidence \cite{Weir2005}.
More details about each study are given in \sref{intro.esw.hiv.data} and Table~\ref{tab:esw.data}.
I analyzed the individual-level data from 2011 and 2014 (data from 2021 not yet available)
to explore the potential association of biobehavioural factors with HIV risk,
so that such factors could then be used to distinguish between
lower risk \vs higher risk FSW.
% TODO: (*) add descriptive table
%---------------------------------------------------------------------------------------------------
\subsubsection{HIV Status}\label{model.par.fsw.hiv}
Only the 2011 and 2021 studies included serologic testing for HIV.
Among those tested in 2011 (N = 317, 98\%), 70\% were \hivp,
yielding RDS-adjusted prevalence estimate of 61\% (CI: 51--71\%) \cite{Baral2014}.
Among serologically \hivn, 11\% self-reported \hivp status (false positive), and
among serologically \hivp, 26\% self-reported \hivn status (false negative or undiagnosed).
Overall, self-reported HIV status underestimated HIV prevalence in 2011
by a factor of approximately 0.78 (55~vs~70\%).
Unadjusted HIV prevalence in 2021 was 58.8\%,
with 88\% (363/411) reporting previous awareness of \hivp status.
\par
In 2014, self-reported HIV prevalence was 38\% among respondents who reported (85\%).
This 38\% is surprisingly low considering that
the PLACE methodology explicitly aimed to sample venues
with higher HIV incidence \cite{Weir2005}, and 2014 \vs 2011 respondents
were older (median 27 \vs 25 years), % 2021 median: 28
had been selling sex longer (median 5 \vs 4 years), % 2021: 6
and tested more frequently (87 \vs 75\% tested at least once in the past year, % 2021: 75
82 \vs 63\% among self-reported \hivn).
Perhaps the differences are attributable to the sampling methodology.
Among respondents who self-reported \hivp status,
the 2014 survey also asked for age of HIV diagnosis (6\% missing).
Age of HIV diagnosis supports crude time-to-event analysis (next section),
which can account for confounding by age and censoring,
as compared to logistic regression on HIV status,
keeping in mind the limitations of self-reported HIV status.
%---------------------------------------------------------------------------------------------------
\subsubsection{Risk Factors for HIV}\label{model.par.fsw.fac}
Next, I explored the potential association of risk factors with HIV
via the following three models:%
\footnote{Logistic regression models were implemented using \texttt{lrm} from:
  \hreftt{cran.r-project.org/package=rms}.\\
Cox proportional hazards models were implemented using \texttt{coxaalen} from:
  \hreftt{cran.r-project.org/package=coxinterval}.}
\begin{enumerate}
  \item Logistic regression on serologic HIV status (2011 data)
  \item Logistic regression on self-reported HIV status (2014 data)
  \item Cox proportional hazards for interval-censored time to HIV infection,
    with interval from self-reported sex work debut 
    to either self-reported time of HIV diagnosis or survey date (2014 data);
    Figure~\ref{fig:fsw.tte.interval} illustrates
    the four potential censoring cases in this framework.
\end{enumerate}
An important limitation to all models is that
risk factors reported by FSW at the time of survey
are assumed to be fixed characteristics of the respondents,
rather than dynamic characteristics that vary over time.
Additionally, respondents with any missing variables for each individual model
were excluded from that model. % TODO: (%)
\begin{figure}
  \centering
  \includegraphics[scale=1]{diag.tte}
  \caption{Illustration of time-to-event analysis framework
    for cross-sectional FSW survey data}
  \label{fig:fsw.tte.interval}
  \floatfoot{
    $\bm{\times}$: HIV infection;
    SW: time of sex work debut;
    Dx: time of HIV diagnosis.}
\end{figure}
\par
Risk factors were selected based on
prior knowledge of plausible mechanistic influence on HIV incidence and/or prevalence.
The risk factors explored are summarized in Table~\ref{tab:fsw.stats},
including univariate and multivariable association under each model.
Variable selection for multivariable models
was performed using backward selection as described by \citet{Lawless1978},
using a $p \le 0.1$ (per variable) threshold for stepwise variable retention.
Estimated conditional effects of
variables retained in the multivariable logistic regression models
are illustrated in Figure~\ref{fig:fsw.lr}.
\begin{table}
  \centering
  \caption{Risk factors explored for association with \hivp status among FSW in Eswatini}
  \label{tab:fsw.stats}
  \centerline{%
\small%
\begin{tabular}{lcccccccccccc}
  \toprule
  & \multicolumn{4}{c}{2011 LR}
  & \multicolumn{4}{c}{2014 LR}
  & \multicolumn{4}{c}{2014 CPH} \\
  \cmidrule(rl){2-5}\cmidrule(rl){6-9}\cmidrule(rl){10-13}
  & \multicolumn{2}{c}{Univar} & \multicolumn{2}{c}{Multivar}
  & \multicolumn{2}{c}{Univar} & \multicolumn{2}{c}{Multivar}
  & \multicolumn{2}{c}{Univar} & \multicolumn{2}{c}{Multivar} \\
  \cmidrule(rl){2-3}\cmidrule(rl){4-5}\cmidrule(rl){6-7}\cmidrule(rl){8-9}\cmidrule(rl){10-11}\cmidrule(rl){12-13}
  Factor                          &  OR  &   p   &  OR  &   p    &  OR  &   p    &  OR  &   p    &  HR  &   p    &  HR  &   p    \\
  \midrule                        % 2011 LR uni  % 2011 LR multi % 2014 LR uni   % 2014 LR multi % 2014 CPH uni  % 2014 CPH multi
  Age\tn{a}                       & 1.11 & \vsig & ---  &  ---   & 1.14 & \vsig  & 1.15 & \vsig  & 1.09 & \vsig  & 1.09 & \vsig  \\
  Years selling sex\tn{a}         & 1.13 & \vsig & 1.13 & \vsig  & 1.12 & \vsig  & ---  &  ---   & 1.08 & \vsig  & ---  &  ---   \\
  Monthly sex work income\tn{b}   & 0.98 & 0.155 & ---  &  ---   & 0.98 & 0.097  & 0.97 & 0.084  & 0.98 & 0.019\s& 0.97 & 0.001\s\\[1ex]
  Non-paying partners\tn{c}       & 0.88 & 0.307 & ---  &  ---   & 1.07 & 0.233  & ---  &  ---   & 1.05 & 0.312  & ---  &  ---   \\
  Monthly new clients\tn{c}       & 1.01 & 0.412 & ---  &  ---   & 1.05 & \vsig  & 1.07 & \vsig  & 1.04 & \vsig  & 1.04 & \vsig  \\
  Monthly regular clients\tn{c}   & 1.01 & 0.351 & ---  &  ---   & 1.03 & 0.002  & ---  &  ---   & 1.02 & \vsig  & 1.02 & 0.034\s\\[1ex]
  Non-paying condom use\tn{d}     & 0.90 & 0.703 & ---  &  ---   & 0.90 & 0.673  & ---  &  ---   & 0.92 & 0.677  & ---  &  ---   \\
  New client condom use\tn{d}     & 0.60 & 0.100 & ---  &  ---   & 0.48 & 0.006\s& 1.25 & 0.599  & 0.56 & 0.004\s& ---  &  ---   \\
  Regular client condom use\tn{d} & 0.58 & 0.110 & ---  &  ---   & 0.39 & \vsig  & 0.35 & 0.004\s& 0.49 & \vsig  & 0.50 & \vsig  \\[1ex]
  Any anal sex past month         & 0.97 & 0.896 & ---  &  ---   & 1.89 & 0.015\s& ---  &  ---   & 1.57 & 0.015\s& 1.27 & 0.260  \\
  Any STI symptoms past year      & 2.29 & \vsig & 2.41 & \vsig  & 2.75 & \vsig  & 2.80 & \vsig  & 2.17 & \vsig  & 2.05 & \vsig  \\
  \bottomrule
\end{tabular}}
% TODO: HIV status?
\floatfoot{\raggedright
  \tnt[a]{OR per year};
  \tnt[b]{OR per Swazi lilangeni per month};
  \tnt[c]{OR per partner};
  \tnt[d]{2011: always vs not always, 2014: at last sex}.
  --- indicates variable was not selected in the multivariate model.
  LR: logistic regression on HIV$+/-$ status;
  CPH: Cox proportional hazards on time to self-reported HIV seroconversion.
  OR: odds ratio; HR: hazard ratio; p: p-value.
  2011 data based on serologic HIV test;
  2014 data based on self-reported HIV status, age of sex work debut, and age of HIV diagnosis.
}
\end{table}
\begin{figure}[h]
  \subcapoverlap
  \foreach \year/\var/\nvar in {2011/f/1,2011/c/2,2014/f/3,2014/c/3}{
  \begin{subfigure}{\nvar\linewidth/5+\linewidth/5}
    \includegraphics[scale=.7]{fsw.\year.lr.hiv.\var}
    \caption{\raggedright}
    \label{fig:fsw.lr.\year.\var}
  \end{subfigure}}
  \caption{Predicted conditional effects (probability)
    of variables in multivariable logistic regression models for HIV status}
  \label{fig:fsw.lr}
  \floatfoot{\fffsw{fig:fsw.lr}
    conditional probabilities shown for fixed covariates at arbitrary values.}
\end{figure}
\par
Following variable selection, each multivariable model was used to estimate
the total \hivp status odds ratio (logistic) or HIV incidence hazard ratio (Cox)
for each respondent in the respective survey ---
\ie $e^{X_i\,\beta}$ for respondent $i$ ---
representing an overall ``risk score'' under each model.
Respondents were then stratified into the top 20\% and bottom 80\% by these risk scores.
The values of each variable were compared between these two strata
using a test for the ratio of the means \cite{Tamhane2004} to support model parameterization;
these ratios are summarized in Table~\ref{tab:fsw.ratios},
and the distributions of variable values across the two strata
are illustrated in Figure~\ref{fig:fsw.f}.
\begin{table}
  \centering
  \caption{Ratios of HIV risk factor variables among higher \vs lower risk FSW in Eswatini}
  \label{tab:fsw.ratios}
  \centerline{\footnotesize%
\begin{tabular}{lcccccc}
  \toprule
  & \multicolumn{2}{c}{2011 LR}
  & \multicolumn{2}{c}{2014 LR}
  & \multicolumn{2}{c}{2014 CPH} \\
  \cmidrule(rl){2-3}\cmidrule(rl){4-5}\cmidrule(rl){6-7}
  Factor                            &  High / Low   &   Ratio (95\% CI)   &  High / Low   &   Ratio (95\% CI)   &   High / Low   &   Ratio (95\% CI)   \\
  \midrule
  Age                               & 31.8  / 24.7  & 1.29 (1.22, 1.36)\s & 32.6  / 26.2  & 1.24 (1.20, 1.28)\s &  33.5  / 26.6  & 1.26 (1.21, 1.31)\s \\
  Years selling sex                 & 11.3  /  4.03 & 2.81 (2.41, 3.25)\s & 10.0  /  5.47 & 1.83 (1.64, 2.03)\s &  10.2  /  5.83 & 1.75 (1.54, 1.98)\s \\
  Monthly sex work income\tn{a}     & 15.1  / 15.2  & 1.00 (0.86, 1.15)   &  6.77 /  7.06 & 0.96 (0.82, 1.11)   &   6.32 /  7.28 & 0.87 (0.73, 1.02)   \\[1ex]
  Non-paying partners               &  1.42 /  1.43 & 0.99 (0.81, 1.19)   &  1.56 /  1.11 & 1.40 (1.11, 1.72)\s &   1.53 /  1.19 & 1.29 (0.98, 1.62)   \\
  Monthly new clients               &  5.50 /  6.98 & 0.79 (0.49, 1.15)   &  8.39 /  4.15 & 2.02 (1.63, 2.44)\s &   8.36 /  4.41 & 1.90 (1.43, 2.39)\s \\
  Monthly regular clients           &  9.35 /  9.05 & 1.03 (0.69, 1.42)   & 11.1  /  8.25 & 1.35 (1.13, 1.57)\s &  12.4  /  8.61 & 1.44 (1.18, 1.71)\s \\[1ex]
  Non-paying condom use\tn{bc}      &  0.26 /  0.35 & 0.73 (0.40, 1.11)   &  0.77 /  0.81 & 0.95 (0.84, 1.06)   &   0.76 /  0.81 & 0.95 (0.81, 1.08)   \\
  New client condom use\tn{bc}      &  0.68 /  0.76 & 0.89 (0.73, 1.06)   &  0.79 /  0.91 & 0.86 (0.79, 0.94)\s &   0.74 /  0.94 & 0.79 (0.69, 0.88)\s \\
  Regular client condom use\tn{bc}  &  0.38 /  0.46 & 0.83 (0.45, 1.28)   &  0.67 /  0.91 & 0.74 (0.65, 0.82)\s &   0.60 /  0.92 & 0.65 (0.55, 0.75)\s \\[1ex]
  Any anal sex past month           &  0.59 /  0.41 & 1.41 (1.06, 1.84)\s &  0.17 /  0.07 & 2.43 (1.47, 3.85)\s &   0.23 /  0.07 & 3.24 (1.95, 5.34)\s \\
  Any STI symptoms past year\tn{c}  &  0.79 /  0.43 & 1.86 (1.54, 2.25)\s &  0.59 /  0.15 & 3.94 (3.15, 5.03)\s &   0.61 /  0.17 & 3.67 (2.87, 4.79)\s \\[1ex]
  HIV prevalence\tn{d}              &  0.94 /  0.64 & 1.46 (1.30, 1.63)\s &  0.66 /  0.29 & 2.29 (1.92, 2.75)\s &   0.71 /  0.31 & 2.32 (1.94, 2.80)\s \\
  \bottomrule
\end{tabular}}
% TODO: HIV status?
\floatfoot{\raggedright
  High / Low: mean variable value among higher / lower risk groups, as defined by
  the top 20\% / bottom 80\% in multivariable model-predicted risk score:
  odds ratio from logistic regression (LR);
  hazards ratio from Cox proportional hazards (CPH).
  \tnt[a]{Swati lilangeni per month};
  \tnt[b]{2011: always \vs not always, 2014: did use condom at last sex};
  \tnt[c]{proportion of respondents};
  \tnt[d]{2011: serologic HIV status; 2014: self-reported HIV status};
  \tnt[*]{statistically significant, $p < 0.05$}.
}
\end{table}

\section{Parameterization}\label{model.par}
As described in \sref{intro.model.param}, model parameterization involves
specification of model parameter values, such as proportions, probabilities, rates, and ratios,
including stratified values to reflect heterogeneity,
and sampling distributions to reflect uncertainty.
Proportions and probabilities were generally modelled using
a beta approximation of the binomial distribution (BAB, see \sref{app.math.distr.bab}),
while rates and ratios were generally modelled using
a gamma, skewnormal, or inverse gaussian distribution.
\paragraph{Notation}
If $X$ is a parameter stratified by dimensions $a,b,c$,
then $X_{ab_{1}c_{23}}$ denotes the values of $X$ for
a particular but \emph{unspecified} stratum of $a$,
the \emph{specific} stratum $b = 1$,
and the \emph{aggregated} strata $c = 2,3$
(the aggregating operation is context-dependent, \eg sum for probabilities).
Additionally, the indices $sihc$ from Table~\ref{tab:model.dims} denote ``self'' strata,
whereas $s'i'h'c'$ denote ``other'' strata --- \ie individuals' partners.%
\footnote{\label{foot:code.note}%
  In the code: R uses one-based indexing, which match the notation here directly,
  while Python uses zero-based indexing, which therefore appear as $i \rightarrow i-1$ in the code.
  Also, the model code reorders states in the ART Cascade dimension for computational efficiency,
  with $c={}$1:~Undiagnosed; 2:~Diagnosed; 3:~Virally~Un-suppressed; 4:~On~ART; 5:~Virally~Suppressed.}
Finally, I re-use several dummy variables throughout the chapter:
$\rho$ for proportions, $\lambda$ for rates, $T$ for time periods, and $f$ for constants.
%===================================================================================================
\subsection{Risk Heterogeneity Among FSW}\label{model.par.fsw}
Existing HIV transmission models which include FSW
have rarely sub-stratified this population, such as to reflect
differential HIV risk or distinct typologies of sex work \cite{Blanchard2008,Scorgie2012};
yet such heterogeneities may influence transmission dynamics.
Among the studies identified in Chapter~\ref{sr},
only three sub-stratified FSW by risk-related factors:
\citet{Cremin2017} defined three levels of risk via regression analysis,
\citet{Low2015} distinguished between occasional and full-time FSW, while
\citet{Shannon2015} sub-stratified FSW by
work environment, violence exposure, and context-specific structural factors.
Seven other studies, reflecting two unique models \cite{Johnson2012,Maheu-Giroux2017},
employed age stratification of all activity groups, including FSW;
these models had several risk-related parameters which varied by age.
\par
The model structure here (Figure~\ref{fig:model.risk})
was designed to capture \emph{within}-FSW risk heterogeneity.
The objective of the following analysis was therefore to parameterize
lower \vs higher risk FSW.
I sought to define these groups based on biobehavioural and/or contextual factors
which are demonstrably associated with HIV risk,
and which can be mechanistically incorporated into a transmission model ---
\ie through the force of infection equation.
Later, the parameterization of these groups was validated through model fitting
to relative differences in HIV prevalence \sref{model.cal.targ.prev}.
\par
Many cross-sectional studies of HIV among FSW quantify
the association of risk factors with HIV serostatus
\cite{Aklilu2001,Dunkle2005,Scorgie2012,Jonas2020}.
However, serostatus reflects cumulative risk exposure,
whereas sexual risk behaviour is dynamic \cite{Watts2010,vanWees2020},
as is use of prevention resources \cite{Roberts2020}.
For example, while HIV prevalence often increases with age,
HIV incidence among women can peak before age 25 \cite{Dellar2015}.
Thus, risk factors associated with HIV serostatus are not necessarily
mechanistically related to HIV acquisition.
Indeed, FSW may reduce risk behaviours in response to seroconversion \cite{McClelland2006}.
Cohort studies that measure incidence
can help identify risk factors for HIV acquisition \cite{McKinnon2015,Nouaman2022},
but large sample sizes are often required to accurately estimate overall incidence rate,
let alone risk factors \cite{Priddy2011}.
%---------------------------------------------------------------------------------------------------
\subsubsection{FSW Survey Data}\label{model.par.fsw.data}
Three biobehavioural surveys, in
2011 \cite{Baral2014} (N = 325),
2014 \cite{EswKP2014} (N = 781), and
2021 \cite{EswIBBS2022} (N = 676)
provide HIV status and biobehavioural data on FSW in Eswatini.
The 2011 and 2021 surveys featured serologic HIV testing,
and employed respondent driven sampling (RDS, details in \cite{Yam2013}).
The 2014 survey relied on self-reported HIV status,
andd employed venue-based snowball sampling, based on the
Priorities for Local AIDS Control Efforts (PLACE) methodology,
which aims to identify areas of higher incidence \cite{Weir2005}.
More details about each study are given in \sref{intro.esw.hiv.data} and Table~\ref{tab:esw.data}.
I analyzed the individual-level data from 2011 and 2014 (data from 2021 not yet available)
to explore the potential association of biobehavioural factors with HIV risk,
so that such factors could then be used to distinguish between
lower risk \vs higher risk FSW.
% TODO: (*) add descriptive table
%---------------------------------------------------------------------------------------------------
\subsubsection{HIV Status}\label{model.par.fsw.hiv}
Only the 2011 and 2021 studies included serologic testing for HIV.
Among those tested in 2011 (N = 317, 98\%), 70\% were \hivp,
yielding RDS-adjusted prevalence estimate of 61\% (CI: 51--71\%) \cite{Baral2014}.
Among serologically \hivn, 11\% self-reported \hivp status (false positive), and
among serologically \hivp, 26\% self-reported \hivn status (false negative or undiagnosed).
Overall, self-reported HIV status underestimated HIV prevalence in 2011
by a factor of approximately 0.78 (55~vs~70\%).
Unadjusted HIV prevalence in 2021 was 58.8\%,
with 88\% (363/411) reporting previous awareness of \hivp status.
\par
In 2014, self-reported HIV prevalence was 38\% among respondents who reported (85\%).
This 38\% is surprisingly low considering that
the PLACE methodology explicitly aimed to sample venues
with higher HIV incidence \cite{Weir2005}, and 2014 \vs 2011 respondents
were older (median 27 \vs 25 years), % 2021 median: 28
had been selling sex longer (median 5 \vs 4 years), % 2021: 6
and tested more frequently (87 \vs 75\% tested at least once in the past year, % 2021: 75
82 \vs 63\% among self-reported \hivn).
Perhaps the differences are attributable to the sampling methodology.
Among respondents who self-reported \hivp status,
the 2014 survey also asked for age of HIV diagnosis (6\% missing).
Age of HIV diagnosis supports crude time-to-event analysis (next section),
which can account for confounding by age and censoring,
as compared to logistic regression on HIV status,
keeping in mind the limitations of self-reported HIV status.
%---------------------------------------------------------------------------------------------------
\subsubsection{Risk Factors for HIV}\label{model.par.fsw.fac}
Next, I explored the potential association of risk factors with HIV
via the following three models:%
\footnote{Logistic regression models were implemented using \texttt{lrm} from:
  \hreftt{cran.r-project.org/package=rms}.\\
Cox proportional hazards models were implemented using \texttt{coxaalen} from:
  \hreftt{cran.r-project.org/package=coxinterval}.}
\begin{enumerate}
  \item Logistic regression on serologic HIV status (2011 data)
  \item Logistic regression on self-reported HIV status (2014 data)
  \item Cox proportional hazards for interval-censored time to HIV infection,
    with interval from self-reported sex work debut 
    to either self-reported time of HIV diagnosis or survey date (2014 data);
    Figure~\ref{fig:fsw.tte.interval} illustrates
    the four potential censoring cases in this framework.
\end{enumerate}
An important limitation to all models is that
risk factors reported by FSW at the time of survey
are assumed to be fixed characteristics of the respondents,
rather than dynamic characteristics that vary over time.
Additionally, respondents with any missing variables for each individual model
were excluded from that model. % TODO: (%)
\begin{figure}
  \centering
  \includegraphics[scale=1]{diag.tte}
  \caption{Illustration of time-to-event analysis framework
    for cross-sectional FSW survey data}
  \label{fig:fsw.tte.interval}
  \floatfoot{
    $\bm{\times}$: HIV infection;
    SW: time of sex work debut;
    Dx: time of HIV diagnosis.}
\end{figure}
\par
Risk factors were selected based on
prior knowledge of plausible mechanistic influence on HIV incidence and/or prevalence.
The risk factors explored are summarized in Table~\ref{tab:fsw.stats},
including univariate and multivariable association under each model.
Variable selection for multivariable models
was performed using backward selection as described by \citet{Lawless1978},
using a $p \le 0.1$ (per variable) threshold for stepwise variable retention.
Estimated conditional effects of
variables retained in the multivariable logistic regression models
are illustrated in Figure~\ref{fig:fsw.lr}.
\begin{table}
  \centering
  \caption{Risk factors explored for association with \hivp status among FSW in Eswatini}
  \label{tab:fsw.stats}
  \centerline{%
\small%
\begin{tabular}{lcccccccccccc}
  \toprule
  & \multicolumn{4}{c}{2011 LR}
  & \multicolumn{4}{c}{2014 LR}
  & \multicolumn{4}{c}{2014 CPH} \\
  \cmidrule(rl){2-5}\cmidrule(rl){6-9}\cmidrule(rl){10-13}
  & \multicolumn{2}{c}{Univar} & \multicolumn{2}{c}{Multivar}
  & \multicolumn{2}{c}{Univar} & \multicolumn{2}{c}{Multivar}
  & \multicolumn{2}{c}{Univar} & \multicolumn{2}{c}{Multivar} \\
  \cmidrule(rl){2-3}\cmidrule(rl){4-5}\cmidrule(rl){6-7}\cmidrule(rl){8-9}\cmidrule(rl){10-11}\cmidrule(rl){12-13}
  Factor                          &  OR  &   p   &  OR  &   p    &  OR  &   p    &  OR  &   p    &  HR  &   p    &  HR  &   p    \\
  \midrule                        % 2011 LR uni  % 2011 LR multi % 2014 LR uni   % 2014 LR multi % 2014 CPH uni  % 2014 CPH multi
  Age\tn{a}                       & 1.11 & \vsig & ---  &  ---   & 1.14 & \vsig  & 1.15 & \vsig  & 1.09 & \vsig  & 1.09 & \vsig  \\
  Years selling sex\tn{a}         & 1.13 & \vsig & 1.13 & \vsig  & 1.12 & \vsig  & ---  &  ---   & 1.08 & \vsig  & ---  &  ---   \\
  Monthly sex work income\tn{b}   & 0.98 & 0.155 & ---  &  ---   & 0.98 & 0.097  & 0.97 & 0.084  & 0.98 & 0.019\s& 0.97 & 0.001\s\\[1ex]
  Non-paying partners\tn{c}       & 0.88 & 0.307 & ---  &  ---   & 1.07 & 0.233  & ---  &  ---   & 1.05 & 0.312  & ---  &  ---   \\
  Monthly new clients\tn{c}       & 1.01 & 0.412 & ---  &  ---   & 1.05 & \vsig  & 1.07 & \vsig  & 1.04 & \vsig  & 1.04 & \vsig  \\
  Monthly regular clients\tn{c}   & 1.01 & 0.351 & ---  &  ---   & 1.03 & 0.002  & ---  &  ---   & 1.02 & \vsig  & 1.02 & 0.034\s\\[1ex]
  Non-paying condom use\tn{d}     & 0.90 & 0.703 & ---  &  ---   & 0.90 & 0.673  & ---  &  ---   & 0.92 & 0.677  & ---  &  ---   \\
  New client condom use\tn{d}     & 0.60 & 0.100 & ---  &  ---   & 0.48 & 0.006\s& 1.25 & 0.599  & 0.56 & 0.004\s& ---  &  ---   \\
  Regular client condom use\tn{d} & 0.58 & 0.110 & ---  &  ---   & 0.39 & \vsig  & 0.35 & 0.004\s& 0.49 & \vsig  & 0.50 & \vsig  \\[1ex]
  Any anal sex past month         & 0.97 & 0.896 & ---  &  ---   & 1.89 & 0.015\s& ---  &  ---   & 1.57 & 0.015\s& 1.27 & 0.260  \\
  Any STI symptoms past year      & 2.29 & \vsig & 2.41 & \vsig  & 2.75 & \vsig  & 2.80 & \vsig  & 2.17 & \vsig  & 2.05 & \vsig  \\
  \bottomrule
\end{tabular}}
% TODO: HIV status?
\floatfoot{\raggedright
  \tnt[a]{OR per year};
  \tnt[b]{OR per Swazi lilangeni per month};
  \tnt[c]{OR per partner};
  \tnt[d]{2011: always vs not always, 2014: at last sex}.
  --- indicates variable was not selected in the multivariate model.
  LR: logistic regression on HIV$+/-$ status;
  CPH: Cox proportional hazards on time to self-reported HIV seroconversion.
  OR: odds ratio; HR: hazard ratio; p: p-value.
  2011 data based on serologic HIV test;
  2014 data based on self-reported HIV status, age of sex work debut, and age of HIV diagnosis.
}
\end{table}
\begin{figure}[h]
  \subcapoverlap
  \foreach \year/\var/\nvar in {2011/f/1,2011/c/2,2014/f/3,2014/c/3}{
  \begin{subfigure}{\nvar\linewidth/5+\linewidth/5}
    \includegraphics[scale=.7]{fsw.\year.lr.hiv.\var}
    \caption{\raggedright}
    \label{fig:fsw.lr.\year.\var}
  \end{subfigure}}
  \caption{Predicted conditional effects (probability)
    of variables in multivariable logistic regression models for HIV status}
  \label{fig:fsw.lr}
  \floatfoot{\fffsw{fig:fsw.lr}
    conditional probabilities shown for fixed covariates at arbitrary values.}
\end{figure}
\par
Following variable selection, each multivariable model was used to estimate
the total \hivp status odds ratio (logistic) or HIV incidence hazard ratio (Cox)
for each respondent in the respective survey ---
\ie $e^{X_i\,\beta}$ for respondent $i$ ---
representing an overall ``risk score'' under each model.
Respondents were then stratified into the top 20\% and bottom 80\% by these risk scores.
The values of each variable were compared between these two strata
using a test for the ratio of the means \cite{Tamhane2004} to support model parameterization;
these ratios are summarized in Table~\ref{tab:fsw.ratios},
and the distributions of variable values across the two strata
are illustrated in Figure~\ref{fig:fsw.f}.
\begin{table}
  \centering
  \caption{Ratios of HIV risk factor variables among higher \vs lower risk FSW in Eswatini}
  \label{tab:fsw.ratios}
  \centerline{\footnotesize%
\begin{tabular}{lcccccc}
  \toprule
  & \multicolumn{2}{c}{2011 LR}
  & \multicolumn{2}{c}{2014 LR}
  & \multicolumn{2}{c}{2014 CPH} \\
  \cmidrule(rl){2-3}\cmidrule(rl){4-5}\cmidrule(rl){6-7}
  Factor                            &  High / Low   &   Ratio (95\% CI)   &  High / Low   &   Ratio (95\% CI)   &   High / Low   &   Ratio (95\% CI)   \\
  \midrule
  Age                               & 31.8  / 24.7  & 1.29 (1.22, 1.36)\s & 32.6  / 26.2  & 1.24 (1.20, 1.28)\s &  33.5  / 26.6  & 1.26 (1.21, 1.31)\s \\
  Years selling sex                 & 11.3  /  4.03 & 2.81 (2.41, 3.25)\s & 10.0  /  5.47 & 1.83 (1.64, 2.03)\s &  10.2  /  5.83 & 1.75 (1.54, 1.98)\s \\
  Monthly sex work income\tn{a}     & 15.1  / 15.2  & 1.00 (0.86, 1.15)   &  6.77 /  7.06 & 0.96 (0.82, 1.11)   &   6.32 /  7.28 & 0.87 (0.73, 1.02)   \\[1ex]
  Non-paying partners               &  1.42 /  1.43 & 0.99 (0.81, 1.19)   &  1.56 /  1.11 & 1.40 (1.11, 1.72)\s &   1.53 /  1.19 & 1.29 (0.98, 1.62)   \\
  Monthly new clients               &  5.50 /  6.98 & 0.79 (0.49, 1.15)   &  8.39 /  4.15 & 2.02 (1.63, 2.44)\s &   8.36 /  4.41 & 1.90 (1.43, 2.39)\s \\
  Monthly regular clients           &  9.35 /  9.05 & 1.03 (0.69, 1.42)   & 11.1  /  8.25 & 1.35 (1.13, 1.57)\s &  12.4  /  8.61 & 1.44 (1.18, 1.71)\s \\[1ex]
  Non-paying condom use\tn{bc}      &  0.26 /  0.35 & 0.73 (0.40, 1.11)   &  0.77 /  0.81 & 0.95 (0.84, 1.06)   &   0.76 /  0.81 & 0.95 (0.81, 1.08)   \\
  New client condom use\tn{bc}      &  0.68 /  0.76 & 0.89 (0.73, 1.06)   &  0.79 /  0.91 & 0.86 (0.79, 0.94)\s &   0.74 /  0.94 & 0.79 (0.69, 0.88)\s \\
  Regular client condom use\tn{bc}  &  0.38 /  0.46 & 0.83 (0.45, 1.28)   &  0.67 /  0.91 & 0.74 (0.65, 0.82)\s &   0.60 /  0.92 & 0.65 (0.55, 0.75)\s \\[1ex]
  Any anal sex past month           &  0.59 /  0.41 & 1.41 (1.06, 1.84)\s &  0.17 /  0.07 & 2.43 (1.47, 3.85)\s &   0.23 /  0.07 & 3.24 (1.95, 5.34)\s \\
  Any STI symptoms past year\tn{c}  &  0.79 /  0.43 & 1.86 (1.54, 2.25)\s &  0.59 /  0.15 & 3.94 (3.15, 5.03)\s &   0.61 /  0.17 & 3.67 (2.87, 4.79)\s \\[1ex]
  HIV prevalence\tn{d}              &  0.94 /  0.64 & 1.46 (1.30, 1.63)\s &  0.66 /  0.29 & 2.29 (1.92, 2.75)\s &   0.71 /  0.31 & 2.32 (1.94, 2.80)\s \\
  \bottomrule
\end{tabular}}
% TODO: HIV status?
\floatfoot{\raggedright
  High / Low: mean variable value among higher / lower risk groups, as defined by
  the top 20\% / bottom 80\% in multivariable model-predicted risk score:
  odds ratio from logistic regression (LR);
  hazards ratio from Cox proportional hazards (CPH).
  \tnt[a]{Swati lilangeni per month};
  \tnt[b]{2011: always \vs not always, 2014: did use condom at last sex};
  \tnt[c]{proportion of respondents};
  \tnt[d]{2011: serologic HIV status; 2014: self-reported HIV status};
  \tnt[*]{statistically significant, $p < 0.05$}.
}
\end{table}

\section{Parameterization}\label{model.par}
As described in \sref{intro.model.param}, model parameterization involves
specification of model parameter values, such as proportions, probabilities, rates, and ratios,
including stratified values to reflect heterogeneity,
and sampling distributions to reflect uncertainty.
Proportions and probabilities were generally modelled using
a beta approximation of the binomial distribution (BAB, see \sref{app.math.distr.bab}),
while rates and ratios were generally modelled using
a gamma, skewnormal, or inverse gaussian distribution.
\paragraph{Notation}
If $X$ is a parameter stratified by dimensions $a,b,c$,
then $X_{ab_{1}c_{23}}$ denotes the values of $X$ for
a particular but \emph{unspecified} stratum of $a$,
the \emph{specific} stratum $b = 1$,
and the \emph{aggregated} strata $c = 2,3$
(the aggregating operation is context-dependent, \eg sum for probabilities).
Additionally, the indices $sihc$ from Table~\ref{tab:model.dims} denote ``self'' strata,
whereas $s'i'h'c'$ denote ``other'' strata --- \ie individuals' partners.%
\footnote{\label{foot:code.note}%
  In the code: R uses one-based indexing, which match the notation here directly,
  while Python uses zero-based indexing, which therefore appear as $i \rightarrow i-1$ in the code.
  Also, the model code reorders states in the ART Cascade dimension for computational efficiency,
  with $c={}$1:~Undiagnosed; 2:~Diagnosed; 3:~Virally~Un-suppressed; 4:~On~ART; 5:~Virally~Suppressed.}
Finally, I re-use several dummy variables throughout the chapter:
$\rho$ for proportions, $\lambda$ for rates, $T$ for time periods, and $f$ for constants.
%===================================================================================================
\subsection{Risk Heterogeneity Among FSW}\label{model.par.fsw}
Existing HIV transmission models which include FSW
have rarely sub-stratified this population, such as to reflect
differential HIV risk or distinct typologies of sex work \cite{Blanchard2008,Scorgie2012};
yet such heterogeneities may influence transmission dynamics.
Among the studies identified in Chapter~\ref{sr},
only three sub-stratified FSW by risk-related factors:
\citet{Cremin2017} defined three levels of risk via regression analysis,
\citet{Low2015} distinguished between occasional and full-time FSW, while
\citet{Shannon2015} sub-stratified FSW by
work environment, violence exposure, and context-specific structural factors.
Seven other studies, reflecting two unique models \cite{Johnson2012,Maheu-Giroux2017},
employed age stratification of all activity groups, including FSW;
these models had several risk-related parameters which varied by age.
\par
The model structure here (Figure~\ref{fig:model.risk})
was designed to capture \emph{within}-FSW risk heterogeneity.
The objective of the following analysis was therefore to parameterize
lower \vs higher risk FSW.
I sought to define these groups based on biobehavioural and/or contextual factors
which are demonstrably associated with HIV risk,
and which can be mechanistically incorporated into a transmission model ---
\ie through the force of infection equation.
Later, the parameterization of these groups was validated through model fitting
to relative differences in HIV prevalence \sref{model.cal.targ.prev}.
\par
Many cross-sectional studies of HIV among FSW quantify
the association of risk factors with HIV serostatus
\cite{Aklilu2001,Dunkle2005,Scorgie2012,Jonas2020}.
However, serostatus reflects cumulative risk exposure,
whereas sexual risk behaviour is dynamic \cite{Watts2010,vanWees2020},
as is use of prevention resources \cite{Roberts2020}.
For example, while HIV prevalence often increases with age,
HIV incidence among women can peak before age 25 \cite{Dellar2015}.
Thus, risk factors associated with HIV serostatus are not necessarily
mechanistically related to HIV acquisition.
Indeed, FSW may reduce risk behaviours in response to seroconversion \cite{McClelland2006}.
Cohort studies that measure incidence
can help identify risk factors for HIV acquisition \cite{McKinnon2015,Nouaman2022},
but large sample sizes are often required to accurately estimate overall incidence rate,
let alone risk factors \cite{Priddy2011}.
%---------------------------------------------------------------------------------------------------
\subsubsection{FSW Survey Data}\label{model.par.fsw.data}
Three biobehavioural surveys, in
2011 \cite{Baral2014} (N = 325),
2014 \cite{EswKP2014} (N = 781), and
2021 \cite{EswIBBS2022} (N = 676)
provide HIV status and biobehavioural data on FSW in Eswatini.
The 2011 and 2021 surveys featured serologic HIV testing,
and employed respondent driven sampling (RDS, details in \cite{Yam2013}).
The 2014 survey relied on self-reported HIV status,
andd employed venue-based snowball sampling, based on the
Priorities for Local AIDS Control Efforts (PLACE) methodology,
which aims to identify areas of higher incidence \cite{Weir2005}.
More details about each study are given in \sref{intro.esw.hiv.data} and Table~\ref{tab:esw.data}.
I analyzed the individual-level data from 2011 and 2014 (data from 2021 not yet available)
to explore the potential association of biobehavioural factors with HIV risk,
so that such factors could then be used to distinguish between
lower risk \vs higher risk FSW.
% TODO: (*) add descriptive table
%---------------------------------------------------------------------------------------------------
\subsubsection{HIV Status}\label{model.par.fsw.hiv}
Only the 2011 and 2021 studies included serologic testing for HIV.
Among those tested in 2011 (N = 317, 98\%), 70\% were \hivp,
yielding RDS-adjusted prevalence estimate of 61\% (CI: 51--71\%) \cite{Baral2014}.
Among serologically \hivn, 11\% self-reported \hivp status (false positive), and
among serologically \hivp, 26\% self-reported \hivn status (false negative or undiagnosed).
Overall, self-reported HIV status underestimated HIV prevalence in 2011
by a factor of approximately 0.78 (55~vs~70\%).
Unadjusted HIV prevalence in 2021 was 58.8\%,
with 88\% (363/411) reporting previous awareness of \hivp status.
\par
In 2014, self-reported HIV prevalence was 38\% among respondents who reported (85\%).
This 38\% is surprisingly low considering that
the PLACE methodology explicitly aimed to sample venues
with higher HIV incidence \cite{Weir2005}, and 2014 \vs 2011 respondents
were older (median 27 \vs 25 years), % 2021 median: 28
had been selling sex longer (median 5 \vs 4 years), % 2021: 6
and tested more frequently (87 \vs 75\% tested at least once in the past year, % 2021: 75
82 \vs 63\% among self-reported \hivn).
Perhaps the differences are attributable to the sampling methodology.
Among respondents who self-reported \hivp status,
the 2014 survey also asked for age of HIV diagnosis (6\% missing).
Age of HIV diagnosis supports crude time-to-event analysis (next section),
which can account for confounding by age and censoring,
as compared to logistic regression on HIV status,
keeping in mind the limitations of self-reported HIV status.
%---------------------------------------------------------------------------------------------------
\subsubsection{Risk Factors for HIV}\label{model.par.fsw.fac}
Next, I explored the potential association of risk factors with HIV
via the following three models:%
\footnote{Logistic regression models were implemented using \texttt{lrm} from:
  \hreftt{cran.r-project.org/package=rms}.\\
Cox proportional hazards models were implemented using \texttt{coxaalen} from:
  \hreftt{cran.r-project.org/package=coxinterval}.}
\begin{enumerate}
  \item Logistic regression on serologic HIV status (2011 data)
  \item Logistic regression on self-reported HIV status (2014 data)
  \item Cox proportional hazards for interval-censored time to HIV infection,
    with interval from self-reported sex work debut 
    to either self-reported time of HIV diagnosis or survey date (2014 data);
    Figure~\ref{fig:fsw.tte.interval} illustrates
    the four potential censoring cases in this framework.
\end{enumerate}
An important limitation to all models is that
risk factors reported by FSW at the time of survey
are assumed to be fixed characteristics of the respondents,
rather than dynamic characteristics that vary over time.
Additionally, respondents with any missing variables for each individual model
were excluded from that model. % TODO: (%)
\begin{figure}
  \centering
  \includegraphics[scale=1]{diag.tte}
  \caption{Illustration of time-to-event analysis framework
    for cross-sectional FSW survey data}
  \label{fig:fsw.tte.interval}
  \floatfoot{
    $\bm{\times}$: HIV infection;
    SW: time of sex work debut;
    Dx: time of HIV diagnosis.}
\end{figure}
\par
Risk factors were selected based on
prior knowledge of plausible mechanistic influence on HIV incidence and/or prevalence.
The risk factors explored are summarized in Table~\ref{tab:fsw.stats},
including univariate and multivariable association under each model.
Variable selection for multivariable models
was performed using backward selection as described by \citet{Lawless1978},
using a $p \le 0.1$ (per variable) threshold for stepwise variable retention.
Estimated conditional effects of
variables retained in the multivariable logistic regression models
are illustrated in Figure~\ref{fig:fsw.lr}.
\begin{table}
  \centering
  \caption{Risk factors explored for association with \hivp status among FSW in Eswatini}
  \label{tab:fsw.stats}
  \input{model/tab.fsw.factor.stats}
\end{table}
\begin{figure}[h]
  \subcapoverlap
  \foreach \year/\var/\nvar in {2011/f/1,2011/c/2,2014/f/3,2014/c/3}{
  \begin{subfigure}{\nvar\linewidth/5+\linewidth/5}
    \includegraphics[scale=.7]{fsw.\year.lr.hiv.\var}
    \caption{\raggedright}
    \label{fig:fsw.lr.\year.\var}
  \end{subfigure}}
  \caption{Predicted conditional effects (probability)
    of variables in multivariable logistic regression models for HIV status}
  \label{fig:fsw.lr}
  \floatfoot{\fffsw{fig:fsw.lr}
    conditional probabilities shown for fixed covariates at arbitrary values.}
\end{figure}
\par
Following variable selection, each multivariable model was used to estimate
the total \hivp status odds ratio (logistic) or HIV incidence hazard ratio (Cox)
for each respondent in the respective survey ---
\ie $e^{X_i\,\beta}$ for respondent $i$ ---
representing an overall ``risk score'' under each model.
Respondents were then stratified into the top 20\% and bottom 80\% by these risk scores.
The values of each variable were compared between these two strata
using a test for the ratio of the means \cite{Tamhane2004} to support model parameterization;
these ratios are summarized in Table~\ref{tab:fsw.ratios},
and the distributions of variable values across the two strata
are illustrated in Figure~\ref{fig:fsw.f}.
\begin{table}
  \centering
  \caption{Ratios of HIV risk factor variables among higher \vs lower risk FSW in Eswatini}
  \label{tab:fsw.ratios}
  \input{model/tab.fsw.factor.ratios}
\end{table}

\section{Parameterization}\label{model.par}
As described in \sref{intro.model.param}, model parameterization involves
specification of model parameter values, such as proportions, probabilities, rates, and ratios,
including stratified values to reflect heterogeneity,
and sampling distributions to reflect uncertainty.
Proportions and probabilities were generally modelled using
a beta approximation of the binomial distribution (BAB, see \sref{app.math.distr.bab}),
while rates and ratios were generally modelled using
a gamma, skewnormal, or inverse gaussian distribution.
\paragraph{Notation}
If $X$ is a parameter stratified by dimensions $a,b,c$,
then $X_{ab_{1}c_{23}}$ denotes the values of $X$ for
a particular but \emph{unspecified} stratum of $a$,
the \emph{specific} stratum $b = 1$,
and the \emph{aggregated} strata $c = 2,3$
(the aggregating operation is context-dependent, \eg sum for probabilities).
Additionally, the indices $sihc$ from Table~\ref{tab:model.dims} denote ``self'' strata,
whereas $s'i'h'c'$ denote ``other'' strata --- \ie individuals' partners.%
\footnote{\label{foot:code.note}%
  In the code: R uses one-based indexing, which match the notation here directly,
  while Python uses zero-based indexing, which therefore appear as $i \rightarrow i-1$ in the code.
  Also, the model code reorders states in the ART Cascade dimension for computational efficiency,
  with $c={}$1:~Undiagnosed; 2:~Diagnosed; 3:~Virally~Un-suppressed; 4:~On~ART; 5:~Virally~Suppressed.}
Finally, I re-use several dummy variables throughout the chapter:
$\rho$ for proportions, $\lambda$ for rates, $T$ for time periods, and $f$ for constants.
\input{model/par.fsw}
\input{model/par.beta}
\input{model/par.hiv}
\input{model/par.popsex}
%===================================================================================================
\subsection{HIV Progression \& Mortality}\label{model.par.hiv}
%---------------------------------------------------------------------------------------------------
\subsubsection{HIV Progression}\label{model.par.hiv.dur}
The length of time spent in each HIV stage is related to
rates of progression between stages $\eta_{h}$,
rates of additional HIV-attributable mortality by stage $\mu_{\textsc{hiv},h}$,
and treatment via antiretroviral therapy (ART).
\citet{Lodi2011} estimate median times from seroconversion to
CD4 $<$ 500, $<$ 350, and $<$ 200 cells/mm\tsup{3}, while
\citet{Mangal2017} directly estimate the rates of progression between CD4 states $\eta_{h}$
in a simple compartmental model.
Based on these data, I modelled mean durations ($1/\eta_{h}$) of:%
\footnote{Assuming exponential distributions for durations in each CD4 state
  (see \sref{app.model.math.comp} for more details).}
0.142 years in acute infection ($h=2$, from \sref{model.par.beta.hiv});
3.35 years in CD4~$>$~500 ($h=3$);
3.74 years in 350~$<$~CD4~$<$~500 ($h=4$); and
5.26 years in 200~$<$~CD4~$<$~350 ($h=5$); plus
the remaining time until death in CD4~$<$~200 ($h=6$, AIDS).
Since the duration in acute infection ($h=2$) is randomly sampled,
the remaining duration in CD4~$>$~500 ($h=3$) is adjusted accordingly.
%---------------------------------------------------------------------------------------------------
\subsubsection{HIV Mortality}\label{model.par.hiv.mort}
Mortality rates by CD4-count in the absence of ART were estimated in
multiple African studies \cite{Badri2006,Anglaret2012,Mangal2017};
based on these data, I estimated yearly HIV-attributable mortality rates $\mu_{\textsc{hiv},h}$ as:
0 during acute phase ($h=2$);
0.4\% during CD4~$>$~500 ($h=3$);
2\% during 350~$<$~CD4~$<$~500 ($h=4$);
4\% during 200~$<$~CD4~$<$~350 ($h=5$); and 
20\% during CD4~$<$~200 ($h=6$, AIDS).
%===================================================================================================
\subsection{Antiretroviral Therapy}\label{model.par.art}
Viral suppression via antiretroviral therapy (ART) influences
the probability of HIV transmission, as well as rates of HIV progression and HIV-related mortality.
The model considers individuals on ART before ($c=4$) and after ($c=5$)
achieving full viral load suppression (VLS), as defined by undetectable HIV RNA in blood samples.
Among retained patients initiating ART, time to VLS
is usually described as ``within 6 months'' \cite{Thompson2012}.
More specifically, \citet{Mujugira2016} estimate the median time to VLS as 3 months,
yielding an estimated \emph{mean} duration for $c=4$ of 4.3 months (see \sref{app.model.math.comp}).
%---------------------------------------------------------------------------------------------------
\subsubsection{Probability of HIV Transmission}\label{model.par.art.beta}
All available evidence suggests that viral suppression by ART to undetectable levels
prevents HIV transmission, \ie undetectable = untransmittable (``U=U'') \cite{Eisinger2019}.
Thus, I assumed zero HIV transmission from individuals with VLS ($c=5$).
However, HIV transmission may still occur
during the period between ART initiation to viral suppression ($c=4$) \cite{Mujugira2016}.
\citet{Donnell2010} estimate an adjusted incidence ratio of 0.08~(0.0,~0.57) for all individuals on ART.
However, in \cite{Donnell2010} and \cite{Cohen2016}, the 1 and 4 (respectively)
genetically linked infections from individuals on ART all occurred within 90 days of ART initiation,
suggesting that risk of transmission only persists before viral suppression.
Adjusting the incidence denominator (person-time)
to 90 days per individual who initiated ART in \cite{Donnell2010}
results in approximately 3.13 times higher estimated incidence ratio: 0.25 for this specific period.%
\footnote{In \cite{Donnell2010}, individuals who initiated ART contributed
  approximately 9.4 months per-person (273 persons / 349 person-years, Tables~2~and~3);
  thus the first 3 months of each individual represent
  3/9.4 = 0.319 fewer person-months of follow-up.}
Thus, I sampled relative infectiousness on ART but before viral suppression ($c=4$)
from a beta distribution with mean (95\%~CI) of 0.25~(0.01,~0.67).
%---------------------------------------------------------------------------------------------------
\subsubsection{HIV Progression \& Mortality}\label{model.par.art.hiv}
\def\hunprog{$h = 6 \rightarrow 5 \rightarrow 4 \rightarrow 3$\xspace}
Effective ART stops CD4 cell decline and results in some CD4 recovery \cite{Battegay2006,Lawn2006}.
Most CD4 recovery occurs within the first year of treatment \cite{Battegay2006}.
Due to the limited number of modelled treatment states,
I model this initial recovery to be associated with the 4.3-month pre-VLS ART state ($c=4$).
\citet{Lawn2006,Gabillard2013} estimate an increase of between 25--39 cells/mm\tsup{3} per month
during the first 3 months of treatment.
Since HIV states $h=4,5,6$ correspond to 150, 150, and 200-wide CD4 strata,
I model rates of movement along \hunprog during pre-VLS ART ($c=4$) as
0.20, 0.20, 0.17 per month, respectively.
After initial increases, CD4 recovery is modest and plateaus.
\citet{Battegay2006} report approximate increases of
22.4 cells/mm\tsup{3} per year between years 1 and 5 on ART.
Thus, I model rates of movement along \hunprog after VLS ($c=5$) as 0.15 per year.
\par
Since higher CD4 states are modelled to have lower mortality rates (see \sref{model.par.hiv.mort}),
the modelled recovery of CD4 cells via ART described above implicitly affords a mortality benefit.
However, HIV infection is associated with increased risk of death by non-AIDS causes
--- \ie unrelated to CD4 count ---
including cardiovascular disease and renal disease \cite{Phillips2008}.
\citet{Lundgren2015} estimated 61\% reduction in non-AIDS life-threatening events due to ART.
For the same CD4 strata, \citet{Gabillard2013} also report approximately 2-times higher
mortality rates within the first year of ART versus thereafter,
suggesting that VLS is associated with 50\% mortality reduction independent of CD4 increase.
Thus, I modelled an additional 50\% reduction in mortality among individuals with VLS ($c=5$),
and half this (25\%) reduction before achieving VLS ($c=4$).
% TODO: rates of diagnosis, testing, vls
\section{Parameterization}\label{model.par}
As described in \sref{intro.model.param}, model parameterization involves
specification of model parameter values, such as proportions, probabilities, rates, and ratios,
including stratified values to reflect heterogeneity,
and sampling distributions to reflect uncertainty.
Proportions and probabilities were generally modelled using
a beta approximation of the binomial distribution (BAB, see \sref{app.math.distr.bab}),
while rates and ratios were generally modelled using
a gamma, skewnormal, or inverse gaussian distribution.
\paragraph{Notation}
If $X$ is a parameter stratified by dimensions $a,b,c$,
then $X_{ab_{1}c_{23}}$ denotes the values of $X$ for
a particular but \emph{unspecified} stratum of $a$,
the \emph{specific} stratum $b = 1$,
and the \emph{aggregated} strata $c = 2,3$
(the aggregating operation is context-dependent, \eg sum for probabilities).
Additionally, the indices $sihc$ from Table~\ref{tab:model.dims} denote ``self'' strata,
whereas $s'i'h'c'$ denote ``other'' strata --- \ie individuals' partners.%
\footnote{\label{foot:code.note}%
  In the code: R uses one-based indexing, which match the notation here directly,
  while Python uses zero-based indexing, which therefore appear as $i \rightarrow i-1$ in the code.
  Also, the model code reorders states in the ART Cascade dimension for computational efficiency,
  with $c={}$1:~Undiagnosed; 2:~Diagnosed; 3:~Virally~Un-suppressed; 4:~On~ART; 5:~Virally~Suppressed.}
Finally, I re-use several dummy variables throughout the chapter:
$\rho$ for proportions, $\lambda$ for rates, $T$ for time periods, and $f$ for constants.
\input{model/par.fsw}
\input{model/par.beta}
\input{model/par.hiv}
\input{model/par.popsex}
%===================================================================================================
\subsection{HIV Progression \& Mortality}\label{model.par.hiv}
%---------------------------------------------------------------------------------------------------
\subsubsection{HIV Progression}\label{model.par.hiv.dur}
The length of time spent in each HIV stage is related to
rates of progression between stages $\eta_{h}$,
rates of additional HIV-attributable mortality by stage $\mu_{\textsc{hiv},h}$,
and treatment via antiretroviral therapy (ART).
\citet{Lodi2011} estimate median times from seroconversion to
CD4 $<$ 500, $<$ 350, and $<$ 200 cells/mm\tsup{3}, while
\citet{Mangal2017} directly estimate the rates of progression between CD4 states $\eta_{h}$
in a simple compartmental model.
Based on these data, I modelled mean durations ($1/\eta_{h}$) of:%
\footnote{Assuming exponential distributions for durations in each CD4 state
  (see \sref{app.model.math.comp} for more details).}
0.142 years in acute infection ($h=2$, from \sref{model.par.beta.hiv});
3.35 years in CD4~$>$~500 ($h=3$);
3.74 years in 350~$<$~CD4~$<$~500 ($h=4$); and
5.26 years in 200~$<$~CD4~$<$~350 ($h=5$); plus
the remaining time until death in CD4~$<$~200 ($h=6$, AIDS).
Since the duration in acute infection ($h=2$) is randomly sampled,
the remaining duration in CD4~$>$~500 ($h=3$) is adjusted accordingly.
%---------------------------------------------------------------------------------------------------
\subsubsection{HIV Mortality}\label{model.par.hiv.mort}
Mortality rates by CD4-count in the absence of ART were estimated in
multiple African studies \cite{Badri2006,Anglaret2012,Mangal2017};
based on these data, I estimated yearly HIV-attributable mortality rates $\mu_{\textsc{hiv},h}$ as:
0 during acute phase ($h=2$);
0.4\% during CD4~$>$~500 ($h=3$);
2\% during 350~$<$~CD4~$<$~500 ($h=4$);
4\% during 200~$<$~CD4~$<$~350 ($h=5$); and 
20\% during CD4~$<$~200 ($h=6$, AIDS).
%===================================================================================================
\subsection{Antiretroviral Therapy}\label{model.par.art}
Viral suppression via antiretroviral therapy (ART) influences
the probability of HIV transmission, as well as rates of HIV progression and HIV-related mortality.
The model considers individuals on ART before ($c=4$) and after ($c=5$)
achieving full viral load suppression (VLS), as defined by undetectable HIV RNA in blood samples.
Among retained patients initiating ART, time to VLS
is usually described as ``within 6 months'' \cite{Thompson2012}.
More specifically, \citet{Mujugira2016} estimate the median time to VLS as 3 months,
yielding an estimated \emph{mean} duration for $c=4$ of 4.3 months (see \sref{app.model.math.comp}).
%---------------------------------------------------------------------------------------------------
\subsubsection{Probability of HIV Transmission}\label{model.par.art.beta}
All available evidence suggests that viral suppression by ART to undetectable levels
prevents HIV transmission, \ie undetectable = untransmittable (``U=U'') \cite{Eisinger2019}.
Thus, I assumed zero HIV transmission from individuals with VLS ($c=5$).
However, HIV transmission may still occur
during the period between ART initiation to viral suppression ($c=4$) \cite{Mujugira2016}.
\citet{Donnell2010} estimate an adjusted incidence ratio of 0.08~(0.0,~0.57) for all individuals on ART.
However, in \cite{Donnell2010} and \cite{Cohen2016}, the 1 and 4 (respectively)
genetically linked infections from individuals on ART all occurred within 90 days of ART initiation,
suggesting that risk of transmission only persists before viral suppression.
Adjusting the incidence denominator (person-time)
to 90 days per individual who initiated ART in \cite{Donnell2010}
results in approximately 3.13 times higher estimated incidence ratio: 0.25 for this specific period.%
\footnote{In \cite{Donnell2010}, individuals who initiated ART contributed
  approximately 9.4 months per-person (273 persons / 349 person-years, Tables~2~and~3);
  thus the first 3 months of each individual represent
  3/9.4 = 0.319 fewer person-months of follow-up.}
Thus, I sampled relative infectiousness on ART but before viral suppression ($c=4$)
from a beta distribution with mean (95\%~CI) of 0.25~(0.01,~0.67).
%---------------------------------------------------------------------------------------------------
\subsubsection{HIV Progression \& Mortality}\label{model.par.art.hiv}
\def\hunprog{$h = 6 \rightarrow 5 \rightarrow 4 \rightarrow 3$\xspace}
Effective ART stops CD4 cell decline and results in some CD4 recovery \cite{Battegay2006,Lawn2006}.
Most CD4 recovery occurs within the first year of treatment \cite{Battegay2006}.
Due to the limited number of modelled treatment states,
I model this initial recovery to be associated with the 4.3-month pre-VLS ART state ($c=4$).
\citet{Lawn2006,Gabillard2013} estimate an increase of between 25--39 cells/mm\tsup{3} per month
during the first 3 months of treatment.
Since HIV states $h=4,5,6$ correspond to 150, 150, and 200-wide CD4 strata,
I model rates of movement along \hunprog during pre-VLS ART ($c=4$) as
0.20, 0.20, 0.17 per month, respectively.
After initial increases, CD4 recovery is modest and plateaus.
\citet{Battegay2006} report approximate increases of
22.4 cells/mm\tsup{3} per year between years 1 and 5 on ART.
Thus, I model rates of movement along \hunprog after VLS ($c=5$) as 0.15 per year.
\par
Since higher CD4 states are modelled to have lower mortality rates (see \sref{model.par.hiv.mort}),
the modelled recovery of CD4 cells via ART described above implicitly affords a mortality benefit.
However, HIV infection is associated with increased risk of death by non-AIDS causes
--- \ie unrelated to CD4 count ---
including cardiovascular disease and renal disease \cite{Phillips2008}.
\citet{Lundgren2015} estimated 61\% reduction in non-AIDS life-threatening events due to ART.
For the same CD4 strata, \citet{Gabillard2013} also report approximately 2-times higher
mortality rates within the first year of ART versus thereafter,
suggesting that VLS is associated with 50\% mortality reduction independent of CD4 increase.
Thus, I modelled an additional 50\% reduction in mortality among individuals with VLS ($c=5$),
and half this (25\%) reduction before achieving VLS ($c=4$).
% TODO: rates of diagnosis, testing, vls
\section{Parameterization}\label{model.par}
As described in \sref{intro.model.param}, model parameterization involves
specification of model parameter values, such as proportions, probabilities, rates, and ratios,
including stratified values to reflect heterogeneity,
and sampling distributions to reflect uncertainty.
Proportions and probabilities were generally modelled using
a beta approximation of the binomial distribution (BAB, see \sref{app.math.distr.bab}),
while rates and ratios were generally modelled using
a gamma, skewnormal, or inverse gaussian distribution.
\paragraph{Notation}
If $X$ is a parameter stratified by dimensions $a,b,c$,
then $X_{ab_{1}c_{23}}$ denotes the values of $X$ for
a particular but \emph{unspecified} stratum of $a$,
the \emph{specific} stratum $b = 1$,
and the \emph{aggregated} strata $c = 2,3$
(the aggregating operation is context-dependent, \eg sum for probabilities).
Additionally, the indices $sihc$ from Table~\ref{tab:model.dims} denote ``self'' strata,
whereas $s'i'h'c'$ denote ``other'' strata --- \ie individuals' partners.%
\footnote{\label{foot:code.note}%
  In the code: R uses one-based indexing, which match the notation here directly,
  while Python uses zero-based indexing, which therefore appear as $i \rightarrow i-1$ in the code.
  Also, the model code reorders states in the ART Cascade dimension for computational efficiency,
  with $c={}$1:~Undiagnosed; 2:~Diagnosed; 3:~Virally~Un-suppressed; 4:~On~ART; 5:~Virally~Suppressed.}
Finally, I re-use several dummy variables throughout the chapter:
$\rho$ for proportions, $\lambda$ for rates, $T$ for time periods, and $f$ for constants.
%===================================================================================================
\subsection{Risk Heterogeneity Among FSW}\label{model.par.fsw}
Existing HIV transmission models which include FSW
have rarely sub-stratified this population, such as to reflect
differential HIV risk or distinct typologies of sex work \cite{Blanchard2008,Scorgie2012};
yet such heterogeneities may influence transmission dynamics.
Among the studies identified in Chapter~\ref{sr},
only three sub-stratified FSW by risk-related factors:
\citet{Cremin2017} defined three levels of risk via regression analysis,
\citet{Low2015} distinguished between occasional and full-time FSW, while
\citet{Shannon2015} sub-stratified FSW by
work environment, violence exposure, and context-specific structural factors.
Seven other studies, reflecting two unique models \cite{Johnson2012,Maheu-Giroux2017},
employed age stratification of all activity groups, including FSW;
these models had several risk-related parameters which varied by age.
\par
The model structure here (Figure~\ref{fig:model.risk})
was designed to capture \emph{within}-FSW risk heterogeneity.
The objective of the following analysis was therefore to parameterize
lower \vs higher risk FSW.
I sought to define these groups based on biobehavioural and/or contextual factors
which are demonstrably associated with HIV risk,
and which can be mechanistically incorporated into a transmission model ---
\ie through the force of infection equation.
Later, the parameterization of these groups was validated through model fitting
to relative differences in HIV prevalence \sref{model.cal.targ.prev}.
\par
Many cross-sectional studies of HIV among FSW quantify
the association of risk factors with HIV serostatus
\cite{Aklilu2001,Dunkle2005,Scorgie2012,Jonas2020}.
However, serostatus reflects cumulative risk exposure,
whereas sexual risk behaviour is dynamic \cite{Watts2010,vanWees2020},
as is use of prevention resources \cite{Roberts2020}.
For example, while HIV prevalence often increases with age,
HIV incidence among women can peak before age 25 \cite{Dellar2015}.
Thus, risk factors associated with HIV serostatus are not necessarily
mechanistically related to HIV acquisition.
Indeed, FSW may reduce risk behaviours in response to seroconversion \cite{McClelland2006}.
Cohort studies that measure incidence
can help identify risk factors for HIV acquisition \cite{McKinnon2015,Nouaman2022},
but large sample sizes are often required to accurately estimate overall incidence rate,
let alone risk factors \cite{Priddy2011}.
%---------------------------------------------------------------------------------------------------
\subsubsection{FSW Survey Data}\label{model.par.fsw.data}
Three biobehavioural surveys, in
2011 \cite{Baral2014} (N = 325),
2014 \cite{EswKP2014} (N = 781), and
2021 \cite{EswIBBS2022} (N = 676)
provide HIV status and biobehavioural data on FSW in Eswatini.
The 2011 and 2021 surveys featured serologic HIV testing,
and employed respondent driven sampling (RDS, details in \cite{Yam2013}).
The 2014 survey relied on self-reported HIV status,
andd employed venue-based snowball sampling, based on the
Priorities for Local AIDS Control Efforts (PLACE) methodology,
which aims to identify areas of higher incidence \cite{Weir2005}.
More details about each study are given in \sref{intro.esw.hiv.data} and Table~\ref{tab:esw.data}.
I analyzed the individual-level data from 2011 and 2014 (data from 2021 not yet available)
to explore the potential association of biobehavioural factors with HIV risk,
so that such factors could then be used to distinguish between
lower risk \vs higher risk FSW.
% TODO: (*) add descriptive table
%---------------------------------------------------------------------------------------------------
\subsubsection{HIV Status}\label{model.par.fsw.hiv}
Only the 2011 and 2021 studies included serologic testing for HIV.
Among those tested in 2011 (N = 317, 98\%), 70\% were \hivp,
yielding RDS-adjusted prevalence estimate of 61\% (CI: 51--71\%) \cite{Baral2014}.
Among serologically \hivn, 11\% self-reported \hivp status (false positive), and
among serologically \hivp, 26\% self-reported \hivn status (false negative or undiagnosed).
Overall, self-reported HIV status underestimated HIV prevalence in 2011
by a factor of approximately 0.78 (55~vs~70\%).
Unadjusted HIV prevalence in 2021 was 58.8\%,
with 88\% (363/411) reporting previous awareness of \hivp status.
\par
In 2014, self-reported HIV prevalence was 38\% among respondents who reported (85\%).
This 38\% is surprisingly low considering that
the PLACE methodology explicitly aimed to sample venues
with higher HIV incidence \cite{Weir2005}, and 2014 \vs 2011 respondents
were older (median 27 \vs 25 years), % 2021 median: 28
had been selling sex longer (median 5 \vs 4 years), % 2021: 6
and tested more frequently (87 \vs 75\% tested at least once in the past year, % 2021: 75
82 \vs 63\% among self-reported \hivn).
Perhaps the differences are attributable to the sampling methodology.
Among respondents who self-reported \hivp status,
the 2014 survey also asked for age of HIV diagnosis (6\% missing).
Age of HIV diagnosis supports crude time-to-event analysis (next section),
which can account for confounding by age and censoring,
as compared to logistic regression on HIV status,
keeping in mind the limitations of self-reported HIV status.
%---------------------------------------------------------------------------------------------------
\subsubsection{Risk Factors for HIV}\label{model.par.fsw.fac}
Next, I explored the potential association of risk factors with HIV
via the following three models:%
\footnote{Logistic regression models were implemented using \texttt{lrm} from:
  \hreftt{cran.r-project.org/package=rms}.\\
Cox proportional hazards models were implemented using \texttt{coxaalen} from:
  \hreftt{cran.r-project.org/package=coxinterval}.}
\begin{enumerate}
  \item Logistic regression on serologic HIV status (2011 data)
  \item Logistic regression on self-reported HIV status (2014 data)
  \item Cox proportional hazards for interval-censored time to HIV infection,
    with interval from self-reported sex work debut 
    to either self-reported time of HIV diagnosis or survey date (2014 data);
    Figure~\ref{fig:fsw.tte.interval} illustrates
    the four potential censoring cases in this framework.
\end{enumerate}
An important limitation to all models is that
risk factors reported by FSW at the time of survey
are assumed to be fixed characteristics of the respondents,
rather than dynamic characteristics that vary over time.
Additionally, respondents with any missing variables for each individual model
were excluded from that model. % TODO: (%)
\begin{figure}
  \centering
  \includegraphics[scale=1]{diag.tte}
  \caption{Illustration of time-to-event analysis framework
    for cross-sectional FSW survey data}
  \label{fig:fsw.tte.interval}
  \floatfoot{
    $\bm{\times}$: HIV infection;
    SW: time of sex work debut;
    Dx: time of HIV diagnosis.}
\end{figure}
\par
Risk factors were selected based on
prior knowledge of plausible mechanistic influence on HIV incidence and/or prevalence.
The risk factors explored are summarized in Table~\ref{tab:fsw.stats},
including univariate and multivariable association under each model.
Variable selection for multivariable models
was performed using backward selection as described by \citet{Lawless1978},
using a $p \le 0.1$ (per variable) threshold for stepwise variable retention.
Estimated conditional effects of
variables retained in the multivariable logistic regression models
are illustrated in Figure~\ref{fig:fsw.lr}.
\begin{table}
  \centering
  \caption{Risk factors explored for association with \hivp status among FSW in Eswatini}
  \label{tab:fsw.stats}
  \input{model/tab.fsw.factor.stats}
\end{table}
\begin{figure}[h]
  \subcapoverlap
  \foreach \year/\var/\nvar in {2011/f/1,2011/c/2,2014/f/3,2014/c/3}{
  \begin{subfigure}{\nvar\linewidth/5+\linewidth/5}
    \includegraphics[scale=.7]{fsw.\year.lr.hiv.\var}
    \caption{\raggedright}
    \label{fig:fsw.lr.\year.\var}
  \end{subfigure}}
  \caption{Predicted conditional effects (probability)
    of variables in multivariable logistic regression models for HIV status}
  \label{fig:fsw.lr}
  \floatfoot{\fffsw{fig:fsw.lr}
    conditional probabilities shown for fixed covariates at arbitrary values.}
\end{figure}
\par
Following variable selection, each multivariable model was used to estimate
the total \hivp status odds ratio (logistic) or HIV incidence hazard ratio (Cox)
for each respondent in the respective survey ---
\ie $e^{X_i\,\beta}$ for respondent $i$ ---
representing an overall ``risk score'' under each model.
Respondents were then stratified into the top 20\% and bottom 80\% by these risk scores.
The values of each variable were compared between these two strata
using a test for the ratio of the means \cite{Tamhane2004} to support model parameterization;
these ratios are summarized in Table~\ref{tab:fsw.ratios},
and the distributions of variable values across the two strata
are illustrated in Figure~\ref{fig:fsw.f}.
\begin{table}
  \centering
  \caption{Ratios of HIV risk factor variables among higher \vs lower risk FSW in Eswatini}
  \label{tab:fsw.ratios}
  \input{model/tab.fsw.factor.ratios}
\end{table}

\section{Parameterization}\label{model.par}
As described in \sref{intro.model.param}, model parameterization involves
specification of model parameter values, such as proportions, probabilities, rates, and ratios,
including stratified values to reflect heterogeneity,
and sampling distributions to reflect uncertainty.
Proportions and probabilities were generally modelled using
a beta approximation of the binomial distribution (BAB, see \sref{app.math.distr.bab}),
while rates and ratios were generally modelled using
a gamma, skewnormal, or inverse gaussian distribution.
\paragraph{Notation}
If $X$ is a parameter stratified by dimensions $a,b,c$,
then $X_{ab_{1}c_{23}}$ denotes the values of $X$ for
a particular but \emph{unspecified} stratum of $a$,
the \emph{specific} stratum $b = 1$,
and the \emph{aggregated} strata $c = 2,3$
(the aggregating operation is context-dependent, \eg sum for probabilities).
Additionally, the indices $sihc$ from Table~\ref{tab:model.dims} denote ``self'' strata,
whereas $s'i'h'c'$ denote ``other'' strata --- \ie individuals' partners.%
\footnote{\label{foot:code.note}%
  In the code: R uses one-based indexing, which match the notation here directly,
  while Python uses zero-based indexing, which therefore appear as $i \rightarrow i-1$ in the code.
  Also, the model code reorders states in the ART Cascade dimension for computational efficiency,
  with $c={}$1:~Undiagnosed; 2:~Diagnosed; 3:~Virally~Un-suppressed; 4:~On~ART; 5:~Virally~Suppressed.}
Finally, I re-use several dummy variables throughout the chapter:
$\rho$ for proportions, $\lambda$ for rates, $T$ for time periods, and $f$ for constants.
\input{model/par.fsw}
\input{model/par.beta}
\input{model/par.hiv}
\input{model/par.popsex}
%===================================================================================================
\subsection{HIV Progression \& Mortality}\label{model.par.hiv}
%---------------------------------------------------------------------------------------------------
\subsubsection{HIV Progression}\label{model.par.hiv.dur}
The length of time spent in each HIV stage is related to
rates of progression between stages $\eta_{h}$,
rates of additional HIV-attributable mortality by stage $\mu_{\textsc{hiv},h}$,
and treatment via antiretroviral therapy (ART).
\citet{Lodi2011} estimate median times from seroconversion to
CD4 $<$ 500, $<$ 350, and $<$ 200 cells/mm\tsup{3}, while
\citet{Mangal2017} directly estimate the rates of progression between CD4 states $\eta_{h}$
in a simple compartmental model.
Based on these data, I modelled mean durations ($1/\eta_{h}$) of:%
\footnote{Assuming exponential distributions for durations in each CD4 state
  (see \sref{app.model.math.comp} for more details).}
0.142 years in acute infection ($h=2$, from \sref{model.par.beta.hiv});
3.35 years in CD4~$>$~500 ($h=3$);
3.74 years in 350~$<$~CD4~$<$~500 ($h=4$); and
5.26 years in 200~$<$~CD4~$<$~350 ($h=5$); plus
the remaining time until death in CD4~$<$~200 ($h=6$, AIDS).
Since the duration in acute infection ($h=2$) is randomly sampled,
the remaining duration in CD4~$>$~500 ($h=3$) is adjusted accordingly.
%---------------------------------------------------------------------------------------------------
\subsubsection{HIV Mortality}\label{model.par.hiv.mort}
Mortality rates by CD4-count in the absence of ART were estimated in
multiple African studies \cite{Badri2006,Anglaret2012,Mangal2017};
based on these data, I estimated yearly HIV-attributable mortality rates $\mu_{\textsc{hiv},h}$ as:
0 during acute phase ($h=2$);
0.4\% during CD4~$>$~500 ($h=3$);
2\% during 350~$<$~CD4~$<$~500 ($h=4$);
4\% during 200~$<$~CD4~$<$~350 ($h=5$); and 
20\% during CD4~$<$~200 ($h=6$, AIDS).
%===================================================================================================
\subsection{Antiretroviral Therapy}\label{model.par.art}
Viral suppression via antiretroviral therapy (ART) influences
the probability of HIV transmission, as well as rates of HIV progression and HIV-related mortality.
The model considers individuals on ART before ($c=4$) and after ($c=5$)
achieving full viral load suppression (VLS), as defined by undetectable HIV RNA in blood samples.
Among retained patients initiating ART, time to VLS
is usually described as ``within 6 months'' \cite{Thompson2012}.
More specifically, \citet{Mujugira2016} estimate the median time to VLS as 3 months,
yielding an estimated \emph{mean} duration for $c=4$ of 4.3 months (see \sref{app.model.math.comp}).
%---------------------------------------------------------------------------------------------------
\subsubsection{Probability of HIV Transmission}\label{model.par.art.beta}
All available evidence suggests that viral suppression by ART to undetectable levels
prevents HIV transmission, \ie undetectable = untransmittable (``U=U'') \cite{Eisinger2019}.
Thus, I assumed zero HIV transmission from individuals with VLS ($c=5$).
However, HIV transmission may still occur
during the period between ART initiation to viral suppression ($c=4$) \cite{Mujugira2016}.
\citet{Donnell2010} estimate an adjusted incidence ratio of 0.08~(0.0,~0.57) for all individuals on ART.
However, in \cite{Donnell2010} and \cite{Cohen2016}, the 1 and 4 (respectively)
genetically linked infections from individuals on ART all occurred within 90 days of ART initiation,
suggesting that risk of transmission only persists before viral suppression.
Adjusting the incidence denominator (person-time)
to 90 days per individual who initiated ART in \cite{Donnell2010}
results in approximately 3.13 times higher estimated incidence ratio: 0.25 for this specific period.%
\footnote{In \cite{Donnell2010}, individuals who initiated ART contributed
  approximately 9.4 months per-person (273 persons / 349 person-years, Tables~2~and~3);
  thus the first 3 months of each individual represent
  3/9.4 = 0.319 fewer person-months of follow-up.}
Thus, I sampled relative infectiousness on ART but before viral suppression ($c=4$)
from a beta distribution with mean (95\%~CI) of 0.25~(0.01,~0.67).
%---------------------------------------------------------------------------------------------------
\subsubsection{HIV Progression \& Mortality}\label{model.par.art.hiv}
\def\hunprog{$h = 6 \rightarrow 5 \rightarrow 4 \rightarrow 3$\xspace}
Effective ART stops CD4 cell decline and results in some CD4 recovery \cite{Battegay2006,Lawn2006}.
Most CD4 recovery occurs within the first year of treatment \cite{Battegay2006}.
Due to the limited number of modelled treatment states,
I model this initial recovery to be associated with the 4.3-month pre-VLS ART state ($c=4$).
\citet{Lawn2006,Gabillard2013} estimate an increase of between 25--39 cells/mm\tsup{3} per month
during the first 3 months of treatment.
Since HIV states $h=4,5,6$ correspond to 150, 150, and 200-wide CD4 strata,
I model rates of movement along \hunprog during pre-VLS ART ($c=4$) as
0.20, 0.20, 0.17 per month, respectively.
After initial increases, CD4 recovery is modest and plateaus.
\citet{Battegay2006} report approximate increases of
22.4 cells/mm\tsup{3} per year between years 1 and 5 on ART.
Thus, I model rates of movement along \hunprog after VLS ($c=5$) as 0.15 per year.
\par
Since higher CD4 states are modelled to have lower mortality rates (see \sref{model.par.hiv.mort}),
the modelled recovery of CD4 cells via ART described above implicitly affords a mortality benefit.
However, HIV infection is associated with increased risk of death by non-AIDS causes
--- \ie unrelated to CD4 count ---
including cardiovascular disease and renal disease \cite{Phillips2008}.
\citet{Lundgren2015} estimated 61\% reduction in non-AIDS life-threatening events due to ART.
For the same CD4 strata, \citet{Gabillard2013} also report approximately 2-times higher
mortality rates within the first year of ART versus thereafter,
suggesting that VLS is associated with 50\% mortality reduction independent of CD4 increase.
Thus, I modelled an additional 50\% reduction in mortality among individuals with VLS ($c=5$),
and half this (25\%) reduction before achieving VLS ($c=4$).
% TODO: rates of diagnosis, testing, vls
\section{Parameterization}\label{model.par}
As described in \sref{intro.model.param}, model parameterization involves
specification of model parameter values, such as proportions, probabilities, rates, and ratios,
including stratified values to reflect heterogeneity,
and sampling distributions to reflect uncertainty.
Proportions and probabilities were generally modelled using
a beta approximation of the binomial distribution (BAB, see \sref{app.math.distr.bab}),
while rates and ratios were generally modelled using
a gamma, skewnormal, or inverse gaussian distribution.
\paragraph{Notation}
If $X$ is a parameter stratified by dimensions $a,b,c$,
then $X_{ab_{1}c_{23}}$ denotes the values of $X$ for
a particular but \emph{unspecified} stratum of $a$,
the \emph{specific} stratum $b = 1$,
and the \emph{aggregated} strata $c = 2,3$
(the aggregating operation is context-dependent, \eg sum for probabilities).
Additionally, the indices $sihc$ from Table~\ref{tab:model.dims} denote ``self'' strata,
whereas $s'i'h'c'$ denote ``other'' strata --- \ie individuals' partners.%
\footnote{\label{foot:code.note}%
  In the code: R uses one-based indexing, which match the notation here directly,
  while Python uses zero-based indexing, which therefore appear as $i \rightarrow i-1$ in the code.
  Also, the model code reorders states in the ART Cascade dimension for computational efficiency,
  with $c={}$1:~Undiagnosed; 2:~Diagnosed; 3:~Virally~Un-suppressed; 4:~On~ART; 5:~Virally~Suppressed.}
Finally, I re-use several dummy variables throughout the chapter:
$\rho$ for proportions, $\lambda$ for rates, $T$ for time periods, and $f$ for constants.
\input{model/par.fsw}
\input{model/par.beta}
\input{model/par.hiv}
\input{model/par.popsex}
%===================================================================================================
\subsection{Activity Group Sizes}\label{model.par.size}
Population sizes of all activity groups are modelled as proportions of the total population,
which are assumed to remain roughly constant.
Individuals can, however, move between groups (see \sref{model.par.turn.act})
--- \ie groups are open populations ---
and disproportionate mortality due to HIV between groups
may cause higher risk groups to shrink over time.
Overall population growth is discussed in \sref{model.par.turn.bd}.
%---------------------------------------------------------------------------------------------------
\subsubsection{Female Sex Workers}\label{model.par.size.fsw}
The proportion of women who report sex work in national demographic and health surveys
is generally considered unreliable due to social desirability bias,
particularly if the survey is face-to-face and household-based
\cite{Konings1995,Gregson2002,Gregson2004,Lowndes2012,Behanzin2013}.
Therefore, FSW population size estimates require
targeted surveys and unique methodologies \cite{UNAIDS2010kps,Abdul-Quader2014}.
In both \cite{EswKP2014} and \cite{EswIBBS2022}, the Swati FSW population size
was estimated using a combination of
unique object method, service multiplier method, prior survey participation,
and network scale-up method (NSUM) \cite{UNAIDS2010kps}.
In 2011 \cite{EswKP2014}, regional FSW population size estimates
ranged from 0.7\% to 6.5\% of all women,
with overall population-weighted mean across regions of 2.9\%;
in 2021 \cite{EswIBBS2022}, the mean (95\%~CI) estimates were 2.43~(1.17,~5.02)\%.
To reflect this uncertainty in the model, a BAB distribution was fitted
such that 95\% of the probability fell between 0.7\% and 6.5\%,
and used as the prior distribution for the proportion of women who are FSW:
$P_{s_{1}i_{34}} / P_{s_{1}}$.
Then, following the analysis in \sref{model.par.fsw},
the proportion of all FSW in the higher risk FSW group was fixed at 20\%,
and likewise the lower risk group at 80\%.
%---------------------------------------------------------------------------------------------------
\subsubsection{Clients of FSW}\label{model.par.size.cli}
Similar to FSW, household-based surveys are not considered reliable data sources
for estimating the population size of clients of FSW \cite{Behanzin2013}.
However, few surveys are designed to reach clients of FSW,
and no direct estimates of FSW size exist for Eswatini.
So, I use a common approach for inferring the FSW client size \cite{Cote2004},
similar to the ``multiplier method'' \cite{Morison2001}.
Given the FSW population proportion $P_{s_{1}i_{34}}$,
the number of average yearly new and regular sex work clients per FSW $Q_{p_{34}s_{1}i_{34}}$,
the frequency of sex per partnership-year $F_{p_{34}}$, and
the total number of yearly commercial sex acts per client year $Q_{p_{34}s_{2}i_{34}}\,F_{p_{34}}$,
the total client population $P_{s_{2}i_{34}}$ is defined as:
\begin{equation}
  {\textstyle\sum_{i}} P_{s_{2}i_{34}} =
  \frac{\sum_{pi} P_{s_{1}i}\,Q_{p_{34}s_{1}i_{34}}\,F_{p_{34}}}
       {\sum_{pi}             Q_{p_{34}s_{2}i_{34}}\,F_{p_{34}}}
  \label{eq:model.fsw.cli.tot}
\end{equation}
Then, as with FSW, the proportion of total clients in the higher risk client group
is defined as 20\% of all clients, and likewise for the lower risk group at 80\%.
Using $Q_{p_{34}s_{1}i_{34}}$, $Q_{p_{34}s_{2}i_{34}}$, and $F_{p_{34}}$
as defined below in \sref{model.par.pnum.sw}, the client population size $P_{s_{2}i_{34}}$
estimated by this method was 14.5~(5.2,~33.9)\% of men.
%---------------------------------------------------------------------------------------------------
\subsubsection{Wider Population}\label{model.par.size.wp}
Based on the results of \sref{model.par.wp},
I defined the sizes of the modelled lower and medium activity groups,
and the average numbers of main/spousal partnerships per person.
I assumed that $W'_{2+}$ and $M'_{2+}$ included FSW and client population sizes, respectively.
Thus, the populations size of medium activity women was defined as
$P_{s_{1}i_{2}} = W'_{2+} - P_{s_{1}i_{34}}$.
Sampling $W'_{2+}$ from a BAB distribution with 95\%~CI (10,~27)\%,
the resulting 95\%~CI for medium activity women $P_{s_{1}i_{2}}$ was (6,~25)\% of women.
The lowest activity women population size was then defined as $1 - P_{s_{1}i_{234}}$,
representing (73,~90)\% of women.
Since there is greater uncertainty in the client population size,
the same approach for the medium activity men population size $P_{s_{2}i_{2}}$
could yield negative values.
Instead, I sampled $P_{s_{2}i_{2}}$ directly from
a BAB distribution with 95\%~CI (10,~17)\%, yielding
95\%~CI for $P_{s_{2}i_{234}}$ of (15,~50)\% of men,
which is close to (15,~44)\% from $M_{2+}$.
The lowest activity men were then then defined as $1 - P_{s_{2}i_{234}}$,
representing (50,~85)\% of men.

%===================================================================================================
\subsection{Turnover}\label{model.par.turn}
%---------------------------------------------------------------------------------------------------
\subsubsection{Births \& Deaths}\label{model.par.turn.bd}
The modelled population considers ages 15--49,
reflecting commonly reported data and the majority of sexual activity.
In the absence of mortality, individuals would therefore
remain within the modelled ``open cohort'' population for 35 years.
The estimated average yearly mortality rate for these ages was 1.44\% around 2006
\cite[Table~15.2]{SDHS2006}.
However, this estimate includes HIV/AIDS-attributable mortality,
which I model separately (see \sref{model.par.hiv.mort}),
accounting for approximately 64\% of deaths around that time \cite{WHO2006esw}.
Thus, the overall exit rate from the modelled cohort
due to reaching age 50 (``aging out'') and non-HIV-attributable mortality was:
$\mu = 1/35 + (1-.64) 1.44\% = 3.78\%$.
\par
I estimated the rate of entry into the modelled population $\nu$
to fit population size of ages 15--49 in Eswatini \cite{WorldBank},
and approximate population growth rates \cite{UNWPP2019},
given that I model HIV/AIDS-attributable mortality separately.
Specifically, I assumed a population growth rate $g = \nu - \mu$ in the absence of HIV/AIDS of
4\% in 1980, 3\% in 2000, 1.5\% in 2010, and 1.5\% in 2020 (monotonic cubic interpolation).
I sampled $g$ in 2050 from a uniform prior with 95\% CI (0.7\%,~1.5\%),
reflecting uncertainty in estimated projections \cite{UNWPP2019}.
Finally, I calculated the population entry rate as $\nu = g + \mu$.
These parameter values were informally validated by comparison of model outputs with
Swati population sizes for ages 15--49 from \cite{WorldBank}.
The distribution of activity groups among individuals \emph{entering} the model, denoted $E_{si}$,
is different from the distribution among individuals \emph{currently} in the model $P_{si}$,
but $E_{si}$ is computed automatically as described below in \sref{model.par.turn.act}.
%---------------------------------------------------------------------------------------------------
\subsubsection{Activity Group Turnover}\label{model.par.turn.act}
In addition to overall population turnover (entry/exit from the open population),
I model movement of individuals between activity groups within the model.
Activity group turnover reflects the fact that risk is not constant over sexual life course,
and reported duration in higher activity contexts can be short \cite{Scorgie2012}.
Previous modelling has shown that activity group turnover (sometimes called ``episodic risk'')
can strongly influence parameter fitting and intervention impact \cite{Henry2015,Knight2020}.
I model turnover from activity group $si$ to $si'$ as a constant rate $\theta_{sii'}$,
which implies an assumption that (in the absence of HIV) duration in group $si$ is
exponentially distributed with mean $D_{si}$ \cite{Roberts2015}:
\begin{equation}\label{eq:model.par.dur}
  D_{si} = \frac{1}{\mu + \sum_{i'}\theta_{sii'}}
\end{equation}
where $\mu$ is the overall exit rate from \sref{model.par.turn.bd}.
As shown previously \cite{Knight2020}, the relative sizes of each sex-activity group $P_{si}$
can be maintained at fixed values by satisfying the following ``mass-balance'' equation:
\begin{equation}
  \nu P_{si} = \nu E_{si} + \sum_{i'} \theta_{si'i} P_{si'} - \sum_{i} \theta_{sii'} P_{si}
\end{equation}
Specific turnover rates $\theta_{sii'}$ and entrant activity group sizes $E_{si}$
can then be uniquely resolved by specifying
$N_i\,(N_i-1) = 12$ non-redundant and compatible constraints,
where specifying each $D_{si}$ is one such constraint.
%---------------------------------------------------------------------------------------------------
\paragraph{Censored Durations}
Cross sectional sex work surveys will often ask respondents about their duration in sex work.
These durations might then be taken to be the average durations in sex work;
however, this will be an underestimate,
because respondents will continue selling sex after the survey \cite{Fazito2012}.%
\footnote{An alternate example would be
  to take the mean age of a population as the life expectancy!
  Thanks to Saulius Simcikas and Dr. Jarle Tufto
  for help identifying and discussing this bias:
  \hreftt{stats.stackexchange.com/questions/298828}.}
\par
Figure~\ref{fig:diag.xdur} illustrates a steady-state population
with 5 women selling sex at any given time.
The steady-state assumption implies that a women leaving sex work after $D$ years
will be immediately replaced by a women entering sex work
whose eventual duration will also be $D$ years.
Let $D$ be this true duration, and $D_s$ be the duration reported in the survey.
If we assume that the survey reaches women at a random time point during the duration $D$,
then $D_s \sim \opname{Unif}(0,D)$,
and the mean reported duration is $\E[D_s] = \frac{1}{2}\E[D]$.
Thus, $\E[D] = 2\,\E[D_s]$ would be an estimate of the true mean duration from the sample.
In reality, sex work surveys may be more likely to reach
women who have already been selling sex for several months or years,
due to delayed self-identification as sex worker \cite{Cheuk2020}.
Thus, we would expect that $f = \E[D] / \E[D_s] \in (1,2)$,
which we can use to compute the mean exit rate as described in \sref{app.model.math.exp}.
\begin{figure}[h]
  \centering
  \includegraphics[scale=1]{diag.xdur}
  \caption{Illustrative steady-state population of 5 FSW,
    with varying true durations in sex work $D$,
    \vs the observed durations in sex work $D_s$ via cross-sectional survey.}
  \label{fig:diag.xdur}
\end{figure}
\par
Another observation we can make from Figure~\ref{fig:diag.xdur} is that
women who sell sex longer are more likely to be captured in the survey.
That is, while the sampled durations are representative of women who \emph{currently} sell sex,
these durations are biased high \vs the population of women who \emph{ever} sell sex.
It's not clear whether this observation is widely understood
and kept in mind when interpreting sex work survey data.
%---------------------------------------------------------------------------------------------------
\paragraph{Duration Selling Sex}
The FSW survey data for 2011 \cite{Baral2014}, 2014 \cite{EswKP2014}, and 2021 \cite{EswIBBS2022}
include questions on the respondent's current age, and age of first selling sex;
the difference between these ages can then define a ``duration selling sex''.
Using this approach, the unadjusted years selling sex among Swati FSW were
median [IQR]: 4~[2,~7] in 2011 and 5~[3,~9] in 2014,
with histograms shown in Figure~\ref{fig:fsw.yss.raw}.
However, such estimates have three sources of bias:
sampling error, censoring, and measurement error.
\par
Sampling error was addressed through RDS-adjustment in 2011 and 2021,
yielding estimates of the proportions of FSW
who have been selling sex for 0--2, 3--5, 6--10, and 10+ years.
The adjusted proportions are not significantly different between 2011 and 2021, and
indicate fewer years selling sex \vs the unadjusted proportions, which would be consistent with
challenges in reaching women in the first year(s) of sex work \cite{Cheuk2020}.
I fit an exponential distribution to the cumulative adjusted proportions
(Figure~\ref{fig:fsw.yss.adj}), yielding an estimated distribution mean
${\lambda}^{-1}$ of 4.2~(3.5,~5.3) years.
However, as discussed above, these reported years are still right censored, and
thereby underestimate the eventual duration in sex work among respondents by a factor $f \le 2$.
Thus, the overall mean duration in sex work would be given by $\bar{D} = f\,\lambda^{-1}$.
Yet, additionally, the current definition of duration selling sex
includes a hidden assumption that FSW sell sex continuously after starting.
In fact, 348/777 (45\%) FSW reported having ever stopped selling sex
in the 2014 survey \cite{EswKP2014} (other surveys did not ask).
Among these FSW, the expected duration selling sex in the current period
(\ie since re-starting most recently)
must be less than half ($\rho < 1/2$) of the durations calculated above.
Thus, an adjusted overall mean duration can be calculated as
$\bar{D} = (0.45\,\rho + 0.55)\,f\,\lambda^{-1}$.
Taking $\rho \sim \opname{Unif}(0.2,0.4)$ and $f \sim \opname{Unif}(1.5,2)$,
we obtain $\bar{D}$ with mean (95\%~CI): 5.13~(3.87,~6.72),
similar to the pooled estimate for African FSW up to 2010: 5.5~years~\cite{Fazito2012}.
\par
Finally, I assumed that higher risk FSW stay in sex work longer by a factor of
$R_{D}$ with 95\%~CI (1.54,~3.25) (gamma prior, Table~\ref{tab:fsw.ratios}).
Thus, durations in sex work among higher risk ($D_{HR}$) and lower risk ($D_{LR}$) FSW
can be resolved using:
\begin{equation}
  \begin{aligned}
    \bar{D} &= 0.2\,D_{HR} + 0.8\,D_{LR} \\
    R_{D} &= D_{HR} / D_{LR}
  \end{aligned}
\end{equation}
yielding mean (95\%~CI) $D_{LR}$: 4.07~(2.96,~5.48) and $D_{HR}$: 9.33~(6.30,~13.13) (gamma priors).
%---------------------------------------------------------------------------------------------------
\paragraph{Duration Buying Sex}
Data to inform the average duration spent buying sex among clients is limited.
\citet{Fazito2012} estimated mean durations of 4.6--5.5 years
based on studies in Benin \cite{Lowndes2000} and Kenya \cite{Voeten2002}.
\citet[Table~G]{Hodgins2022} also gives pooled estimates for
the proportions of men in Sub-Saharan Africa
who paid for sex \emph{ever} \vs in \emph{p12m} during 2000--2020.
Estimates ranged from 8.8~(6.5,~11.7)\% of men aged 25--34 who ever bought sex,
to 2.2~(1.5,~3.2)\% of men aged 35--54 who bought sex in p12m.
Based on these data, I defined a gamma prior distribution for the duration buying sex
with 95\%~CI (4,~15) years, applied to both higher and lower risk clients.
%---------------------------------------------------------------------------------------------------
\paragraph{Lowest \& Medium Activity Groups}
Data on individual-level changes to numbers of non-sex work partners in p12m
is even more sparse than data related to sex work;
so, it's unclear to what extent individuals move between the lowest and medium activity groups
throughout their sexual life course.
Data from Uganda, Zimbabwe, and South Africa \cite{Todd2009}
suggested that sexual activity (proportion sexually active and mean numbers of partners)
was approximately stable with age (after sexual debut and and before age 49),
with modest trends toward lower activity at older age.
However, these population-level data do not necessarily suggest that
the \emph{same} individuals have multiple partnerships each year.
Reflecting this uncertainty, I sampled
the rate of turnover from medium to lowest activity for both women and men
from a gamma prior with 95\%~CI (5,~50)\% per year.
%---------------------------------------------------------------------------------------------------
\paragraph{Additional Turnover Assumptions}
The above assumptions specify 3 key constraints for each sex:
two durations $D_{si}$ and one turnover rate $\theta_{sii'}$.
Since higher and lower risk FSW (and clients) are conceptualized as mutually exclusive groups,
I modelled no turnover between these groups:
$\theta_{si_{3}i'_{4}} = \theta_{si_{4}i'_{3}} = 0$ (+2~constraints).
Next, since FSW often enter sex work shortly after sexual debut \cite{Cheuk2020,Ma2020},
and sexual activity is roughly constant or slightly declining with age \cite{Todd2009},
I assumed that $E_{si} = f\,P_{si}$,%
\footnote{Subject to $f \le (\nu - \mu + D_{si}^{-1})\,\nu^{-1}$,
  which can be derived from Eq.~(10) in \cite{Knight2020}.}
with $f = 2$ for FSW, $f = 1.5$ for clients,
and $f = 1$ for medium activity women and men (+3~constraints);
then $f < 1$ for the lowest activity women and men is computed automatically.
Finally, since exiting sex work is unlikely to be
an abrupt transition to monogamous or zero sexual activity \cite{Scorgie2012,Learmonth2015},
I further assumed that (50,~90)\% of women exiting sex work
transition to the medium activity group (BAB prior) (+1~constraint);
in the absence of relevant data, I made a similar assumption regarding clients,
with (25,~90)\% former clients transitioning to the medium activity group (+1~constraint).
These $10 < 12$ total constraints then allow two degrees of freedom to resolve
the values of $\theta_{sii'}$ and $E_{si}$.
A non-negative solution to the system of constraints is solved as described in \cite{Knight2020},%
\footnote{Using \hreftt{docs.scipy.org/doc/scipy/reference/generated/scipy.optimize.nnls.html}}
repeated at each timestep as $\nu$ varies with time.

%===================================================================================================
\subsection{Partnership Numbers}\label{model.par.pnum}
This section summarizes the numbers of partnerships modelled among activity groups.
Similar to group sizes, I draw on the
analysis of FSW data in \sref{model.par.fsw} and
bias adjustment for wider population in \sref{model.par.wp}.
However, first in \sref{model.par.pnum.adj},
I critically review the interpretation of reported partner numbers
with regards to partnership duration, and develop a related adjustment.
%---------------------------------------------------------------------------------------------------
\subsubsection{Adjusting for Partnership Duration}\label{model.par.pnum.adj}
Sexual partnerships are usually quantified using cross-sectional surveys.
In this case, respondents are typically asked to report the numbers of unique partners ``$x$''
during a standardized recall period $\omega$
--- \eg \shortquote{How many different people have you had sex with during the past year?}
Such data can then be used to define
a rate of partnership change $Q$ and/or a number of concurrrent partnerships $K$,
depending on the modelled force of infection equation (see Chapter~\ref{foi}).
\par
If partnership duration is long and the recall period is short
--- including $\omega \approx 0$ for
\shortquote{Are you currently in a long-term sexual partnership?} ---
the reported partnerships mostly reflect \emph{ongoing} partnerships,
and thus $x \approx K$.
If partnership duration is short and the recall period is long,
--- including $\delta \approx 0$ for
\shortquote{How many one-off sexual partners have you had during the past year?} ---
the reported partnerships mostly reflect \emph{complete} partnerships,
and thus $x/\omega \approx Q$.
However, if partnership duration and recall period are similar in length,
the reported partnerships reflect a mixture of tail-ends, complete, and ongoing partnerships,
and thus $x$ overestimates $K$, but $x/\omega$ also overestimates $Q$.
In summary:
\begin{itemize}
  \item $\omega \ll \delta$: mostly ongoing partnerships;
  $x \approx K$ (concurrrent)
  \item $\omega \gg \delta$: mostly complete partnerships;
  $x/\omega \approx Q$ (change rate)
  \item $\omega\,\approx\,\delta$: some tail-ends, some complete, some ongoing;
  $x > K$, $x/\omega > Q$ (neither)
\end{itemize}
\par
I developed an approach to estimate $Q$ and $K$ from $x$,
for a known recall period $\omega$ and partnership duration $\delta$.
The average durations of each type of partnership are estimated in \sref{model.par.pdur}.
The approach draws on a similar assumption as in \sref{model.par.turn.act}:
that survey timing is effectively random with respect to partnership duration.
Then, if either end of the recall period would capture an ongoing partnership,
the intersection point would be, on average, at the partnership mid-point.
Thus, the recall period is effectively extended
by half the partnership duration $\delta/2$ on each end, and $\delta$ overall,
as illustrated in Figure~\ref{fig:diag.recall}.
As such, we can define $Q$ and $K$ as:
\begin{alignat}{1}
  Q &= \frac{x}{\omega+\delta} \label{eq:x2Q}\\
  K &= \frac{x\delta}{\omega+\delta} = Q\delta \label{eq:x2K}
\end{alignat}
\begin{figure}
  \centering
  \includegraphics[scale=1]{diag.recall}
  \caption{Illustration of conceptual framework for quantifying partnerships
    from the number reported during a given recall period}
  \label{fig:diag.recall}
  \floatfoot{
    Circle: partnership start; line: ongoing partnership; cross: partnership end;
    $\omega$/red: recall period;
    $\delta$: partnership duration;
    $x$: number of reported partnerships for $\omega$.}
\end{figure}
\par
As an example, Figure~\ref{fig:diag.recall} illustrates
a recall period of $\omega = 1$ year,
for which $x = 9$ partnerships are reported,
having durations of $\delta = 9$ months.
Thus, we can compute $Q = 9/(1+0.75) = 5.14$ and $K = 5.14(0.75) = 3.86$,
which is a slight underestimate of the true values $Q = 5.33, K = 4$,
due to the randomness in the exact ``location'' of the recall period.
%---------------------------------------------------------------------------------------------------
\subsubsection{Sex Work Partnerships}\label{model.par.pnum.sw}
%---------------------------------------------------------------------------------------------------
\paragraph{Female Sex Workers}
Table~\ref{tab:fsw.ratios} summarizes
the numbers of new and regular clients \emph{per month} reported by Swati FSW,
stratified by higher \vs lower risk per the analysis in \sref{model.par.fsw.fac}.
These data thus would provide $x$ for $\omega = 1$ month.
However, based on the survey questions,%
\footnote{The survey questions were: \shortquote{In the last 30 days,
  how many (new/regular) clients have you had sex with?}, or similar.}
it's not clear whether these reported partner numbers
represent the numbers of unique men or unique client visits.
\par
I assumed that all \emph{new} clients were one-off visits;
thus the reported partner numbers effectively represented
$1/12$th of the total numbers of yearly partnerships $Q_{p_{3}}$.
As such, I sampled the yearly rate of new sex work partnerships among lower risk FSW
from a gamma distribution with mean (95\%~CI) as 4.1~(2.5,~6.0) $\times$ 12,
and the \emph{relative} rate among higher risk FSW from 2.0~(1.6,~2.5).
Since each partnership is assumed to include only one sex act,
the partnership duration $\delta_{p_{3}}$, frequency of sex $F_{p_{3}}$,
and number of concurrent partnerships $K_{p_{3}}$ are ill-defined,
but can be defined for convenience as
$\delta_{p_{3}} = 1/12$ (years), $F_{p_{3}} = 12$ (per year),
and $K_{p_{3}} = Q_{p_{3}} / 12$.
\par
For \emph{regular} sex work partnerships, uncertainties remain regarding
partnership duration $\delta_{p_{4}}$ (see \sref{model.par.pdur}),
frequency of sex per month $F_{p_{4}}/12$, and
survey responses $x$ reflecting unique clients or total client visits per month.
If $x$ reflects the numbers of unique clients, then
$Q_{p_{4}s_{1}i_{34}}$ can be defined via \eqref{eq:x2Q} using $x$ directly;
whereas if $x$ reflects the numbers of unique visits, then
$Q_{p_{4}s_{1}i_{34}}$ should be defined using $x/(F_{p_{4}}/12)$.
I assumed that $\rho = 2/3$ of respondents interpreted the question as in the former case,
and $1-\rho = 1/3$ as in the latter, such that:
\begin{equation}\label{eq:x.swr}
  x' = \rho\,x + (1-\rho)\,x/(F_{p_{4}}/12)
\end{equation}
Taking $F_{p_{4}}/12 = 2$ as the prior mean from \sref{model.par.fsex},
\eqref{eq:x.swr} simplifies to $C' = \frac{5}{6}\,C$.
Then, sampling $C_{p_{4}s_{1}i_{3}}$
from a gamma distribution with mean (95\%~CI) 8.4~(6.0,~11.0) from Table~\ref{tab:fsw.ratios},
and $\delta_{p_{4}}$ as specified in \sref{model.par.pdur},
I defined $Q_{p_{4}s_{1}i_{3}}$ and $K_{p_{4}s_{1}i_{3}}$
via \eqref{eq:x2Q} using $C'_{p_{4}s_{1}i_{3}}$ and $\gamma = 1/12$ year.
% TODO: (*) summarize distributions: Q, K, KF
For higher risk FSW, I sampled the \emph{relative} number/rate of regular clients from
1.5~(1.3,~1.7) (Table~\ref{tab:fsw.ratios}) as before.
%---------------------------------------------------------------------------------------------------
\paragraph{Clients}
Across Sub-Saharan Africa, data for clients of FSW on
the number of unique FSW visited and the frequency of sex is sparse.
Among 64 clients in Kenya,
the median number of sex work visits per week was 1.3 (68 per year);
most clients (68\%) had 1--3 regular FSW partners simultaneously, and
visited 0--3 new FSW per year \cite{Voeten2002}.
Among 261 truck drivers at sex work hotspots in Uganda,
the mean number of sexual partners was
7.4 in the past 30 days and 44.7 in the past year \cite{Matovu2012}.
\citet{Johnson2017} modelled yearly sex work visits among South African clients of FSW as
gamma-distributed with age over 10, peaking at 64 visits per year for clients aged 37.
To reflect these data, I specified clients overall to have
mean (95\%~CI) 60~(35,~90) sex acts with FSW per year
($K_{p_{34}s_{2}i_{34}}\,F_{p_{34}}/12$, gamma prior).
Then, the yearly sex acts among lower and higher risk clients are defined such that
higher risk have 2.0~(1.6,~2.5) times the number among lower risk.
Finally, since the distribution of sex acts between new \vs regular sex work partnerships
must match that among FSW, the specific values of $K_{p_{34}s_{2}i_{34}}$
were computed automatically.
% TODO: (*) summarize distributions: Q, K, KF
%---------------------------------------------------------------------------------------------------
\subsubsection{Main/Spousal \& Casual Partnerships}\label{model.par.pnum.msc}
Drawing on the results in \sref{model.par.wp.res},
I defined the numbers of main/spousal and casual partners
among each activity group as follows.
%---------------------------------------------------------------------------------------------------
\paragraph{Main/Spousal Partnerships}
To simplify model fitting, I sampled a common proportion of
individuals reporting a main/spousal partnership from a BAB distribution with 95\%~CI (25,~50)\%,
applied to all women and men in the lowest activity groups ($C_{p_{1}s_{12}i_{1}}$),
as well as all women in the medium activity group ($C_{p_{1}s_{1}i_{2}}$).
Then, \eqrefs{eq:x2Q}{eq:x2K} were used to define $Q$ and $K$, respectively.
Since FSW and clients had fewer main/spousal partnerships (see \sref{model.par.pnum.msc}),
I calculated the proportion of men in the medium activity group having main/spousal partnerships
$K_{p_{1}s_{2}i_{2}}$ to balance the total number of main/spousal partnerships among women and men.
%---------------------------------------------------------------------------------------------------
\paragraph{Casual Partnerships}
I similarly defined a common proportion of women and men in the lowest activity groups
reporting casual partnership $x_{p_{2}s_{12}i_{1}}$ with 95\%~CI (20,~55)\%.
However, the number of casual partnerships among $W_{2+}$ and $M_{2+}$ ramains uncertain.
The analysis above provides no information on these values,
but the number of partners in p12m for the medium activity groups must be at least about 1.5
to ensure these women and men actually have 2+ partners in p12m.
Thus, I sampled the number of casual partners reported by women in the medium activity group
$x_{p_{2}s_{1}i_{2}}$ from a gamma distribution with 95\%~CI (1.2,~2),
and computed $Q$ and $K$ via \eqrefs{eq:x2Q}{eq:x2K}.
As before, I calculated the numbers of casual partnerships among men in the medium activity group
$K_{p_{2}s_{2}i_{2}}$ to balance total casual partnerships.
%---------------------------------------------------------------------------------------------------
\paragraph{Main/Spousal \& Casual Partnerships among FSW \& Clients}
Among Swati FSW, the mean number of total non-paying partners in the past month was
approximately 1--1.5 (Table~\ref{tab:fsw.ratios}),
which may include both main/spousal partners and casual partners.
Among FSW in South Africa \cite{Wells2018} and Kenya \cite{Voeten2007},
while 54 and 72\% (respectively) reported being in a relationship, only 6 and 3\% were married,
although many non-marital partners may still constitute effectively ``main'' partnerships
with respect to condom use and duration.
Thus, I assumed that:
50\% of all FSW reported a main/spousal partner (\ie $x_{p_{1}s_{1}i_{34}} = 0.5$);
lower risk FSW reported $x_{p_{2}s_{1}i_{3}} = 0.5$ casual partners; and
higher risk FSW reported $x_{p_{2}s_{1}i_{4}} = 1.0$ casual partners, on average.
\par
Available data suggest that about half of clients also report non-sex work partners,
which are not always distinguished as main/spousal \vs casual partnerships
\cite{Lowndes2000,Santo2005}.
Non-paying partners of FSW are also often clients of other FSW \cite{Voeten2007,Godin2008}.
Yet, clients of FSW also tend to be younger and more likely to be
never/formerly married \vs non-client men \cite{Lowndes2000,Carael2006}.
So, I assumed that clients reported
half the numbers of main/spousal partnerships compared to lowest activity men:
$x_{p_{1}s_{2}i_{34}} = 0.5\,x_{p_{1}s_{2}i_{1}}$, and
25--100\% the numbers of casual partnerships compared to medium activity women (uniform prior).
As before, I computed $Q$ and $K$ via \eqrefs{eq:x2Q}{eq:x2K}.

%===================================================================================================
\subsection{Sex Frequency \& Partnership Duration}\label{model.par.sex}
%---------------------------------------------------------------------------------------------------
\subsubsection{Sex Frequency}\label{model.par.sex.freq}
The Eswatini general population data sources \cite{SDHS2006,SHIMS1,SHIMS2}
did not report on frequency of sex. % yes they did, SDHS2006: Table 6.10 TODO
In South Africa, average numbers of sex acts per week per partnership (non-sex work)
was reported as mean 2.5~(IQR: 1--3) \cite{Delva2013},
with consistent reports across main/spousal partnerships and casual partnerships.
Sex frequency among South Africans per month overall (not per-partnership)
is also summarized in \cite[Figure~3.15]{Shisana2005},
which is roughly consistent with \cite{Delva2013}, but motivates a smaller lower bound.
Median sex frequency per partnership-year in 1998 Rakai, Uganda was
approximately 90 acts with the ``more frequent'' of concurrent partners, and
approximately 20 acts with the ``less frequent'' \cite{Morris2010}.
Considering these data,
I sampled the number of sex acts per year in main/spousal partnerships
$F_{p_{1}}$ from a gamma prior distribution with 95\%~CI (13,~156),
and a relative rate for casual partnerships $F_{p_{2}}/F_{p_{1}} \sim \opname{Unif}(0.5,1)$.
As described in \sref{model.par.sw.part},
I defined $F_{p_{3}} = 12$ for occasional sex work partnerships,
and $F_{p_{4}} \sim \opname{Unif}(12,36)$ for regular sex work partnerships.
I also constrained samples of $F_{p_{4}}$ such that
higher risk FSW never have commercial sex more than twice daily, on average.
% Coital frequency is not thought to be influenced by concurrent partnerships \cite{Delva2013}.
%---------------------------------------------------------------------------------------------------
\subsubsection{Anal Sex}\label{model.par.sex.anal}
Among Eswatini data sources, only \cite{EswKP2014} (FSW, 2014)
counted sex acts separately for anal and vaginal sex.
Among all FSW, the proportion of ``average sex acts per week'' that were anal (vs vaginal) was 2.9\%.
However, a previous coital diary study in neighbouring KwaZulu-Natal suggested
much higher proportions were anal \cite{Ramjee1999},
and face-to-face interview survey design may result in under-reporting \cite{Owen2020}.
Owen et al. review studies of anal sex in South Africa, and estimate that
0.6--16.5\% of sex acts among the general population are anal \cite{Owen2017}, \vs
2.4--15.9\% among FSW \cite{Owen2020}.
To reflect this greater uncertainty, the proportions of sex acts which are anal
in all partnerships were sampled from a gamma prior distribution with 95\%~CI (0.6,~16.5)\%.
%---------------------------------------------------------------------------------------------------
\subsubsection{Partnership Duration}\label{model.par.sex.dur}
As explored in Chapter~\ref{foi}, the durations of sexual partnerships
can be key determinants of epidemic dynamics and intervention impact.%
\footnote{Chapter~\ref{foi} also discusses the related phenomenon of partnership concurrency,
  including how concurrency is represented in compartmental models.}
Eswatini-specific data on partnership duration are lacking.
Moreover, accurate estimation of partnership duration remains challenging even when data exist,
due to censoring, truncation, and sampling biases \cite{Burington2010}.
Similar to challenges in estimating sex work duration (\sref{app.model.math.xdur}),
we must distinguish the definition of an ``average partnership'' as
(a) among all partnerships in a population over a given \emph{time period}, \vs
(b) among all partnerships in a population \emph{cross-section}.
Case (b) will be biased by partnership duration,
so the estimated mean duration will longer,
while case (a) reflects an unbiased estimate.%
\footnote{If case (a) durations are exponentially distributed,
  the durations in case (b) will be gamma-distributed with $\alpha = 2, \beta = \lambda$;
  thus the mean duration in case (b) will be $\alpha/\beta = 2\lambda$ (twice as long).}
The difference between the exponential distribution mean and median
should also be kept in mind (see \sref{app.model.math.exp}).
%---------------------------------------------------------------------------------------------------
\paragraph{Main/Spousal Partnerships}
Detailed data on marriage in Eswatini was only captured in 2006 \cite[Table~6.1]{SDHS2006}.
The median age of first marriage was 24.3 among women and 27.7 among men (26.0 overall).
Approximately 64\% of women and 88\% of men (76\% overall) who were ever married or living together
were in a union at age 50--54.
However, no data indicated whether any respondents had remarried or entered into a secondary union.
Among women aged 40--49, the most recent data on
median age of first marriage and proportions ever remarried were
33 years old and 6.6\% in South Africa,
20.9 and 3.7\% in Lesotho, and 18.7 and 28.4\% in Mozambique \cite{John2022};
such data may not capture non-marital secondary unions.
Thus, I assumed $\rho = {}$5--20\% of unions among EmaSwati aged 50--54 were secondary.
Considering that the modelled population only includes ages 15--49,
I then defined the mean durations of main/spousal partnerships as
$\delta_{p_{1}} =  (0.76 - \rho)\,(49 - 26) \in (14.5, 18.5)$ years.
\par
In some models, partnership duration is used to define both
the total numbers of sex acts per partnership and the partnership change rate (see \sref{foi.prior}).
This change rate might be overestimated by the above definition,
since the rate should also consider whether and when
divorced/separated individuals form \emph{new} main/spousal partnerships.
The change rate could even be tied to the modelled baseline and HIV-attributable mortality,
given that the majority of Swati unions ended via spousal death
(83\% of unions among women and 56\% among men by age 50--54) \cite{SDHS2006}.
% TODO: (?) add exit from "k=1" state, rate proportional to HIV-attrib-mort of opposite sex
For simplicity and consistency with prior approaches,
I used the effective duration of 14.5--18.5 years throughout (uniform prior).
%---------------------------------------------------------------------------------------------------
\paragraph{Casual Partnerships}
No data is available regarding durations of non-marital sexual partnerships in Eswatini,
and regional data on are also limited.
I synthesized the available partnership duration data from
South Africa \cite{Harrison2008,Hargreaves2009,Nguyen2015},
Rural Tanzania \cite{Nnko2004},
and four cities in Kenya, Zambia, Benin, and Cameroon \cite{Ferry2001}.
Based on these data, I defined a gamma prior distribution for
the mean duration of casual partnerships $\delta_{p_{2}}$ with 95\%~CI (0.25,~1.5) years,
roughly consistent with prior models \cite{Johnson2009}.
A gamma distribution was chosen \vs uniform or normal
to reflect non-uniform belief while preventing negative values.
%---------------------------------------------------------------------------------------------------
\paragraph{Sex Work Partnerships}
As noted in \sref{model.par.sw.part}, duration of occasional sex work partnerships
is ill defined, but can be defined to comprise a single sex act with $F_{p_{3}}\delta_{p_{3}} = 1$.
Data on regular sex work partnerships is severely limited, and
sometimes regular paying clients later become
non-paying emotional partners \cite{Voeten2007,Mbonye2022}.
Based on \cite{Voeten2002}, I defined a gamma prior distribution for
the mean duration of regular sex work partnerships $\delta_{p_{4}}$ with 95\%~CI (0.5,~2.0) years.
% TODO: is this too long given FSW turnover?

%===================================================================================================
\subsection{Mixing}\label{model.par.mix}
In addition to more concentrated transmission
among FSW and their clients via regular and occasional sex work partnerships
--- which are \emph{only} formed among FSW and clients ---
other types of partnerships may be formed
preferentially between particular activity groups.
For example, FSW and clients may be more likely to form main or casual partnerships
with each other than with other activity groups.
Such preferences are captured in a ``mixing matrix'' $M$, where $M_{pii'}$ denotes
the total number of type-$p$ partnerships formed between groups $i$ and $i'$ in the population
(ignoring sex indices $s,s'$ temporarily)
--- \ie who has sex with whom.
The mixing matrix $M_{pii'}$ must be symmetric,
and have row/column sums equal to the total numbers of partnerships ``offered'' by any group:
$M_{pi} = P_{i} C_{pi}$ (group size $\times$ partnerships per-person).
%---------------------------------------------------------------------------------------------------
\subsubsection{Classic $\epsilon$ Mixing}\label{model.par.mix.eps}
In many risk/activity-stratified compartmental transmission models,
mixing is parameterized via a single parameter $\epsilon \in [0,1]$,
which controls the degree of like-with-like mixing \cite{Nold1980}.
This approach is often attributed to \cite{Garnett1994},
wherein a key adjustment for imbalanced partner numbers among women \vs men was introduced.
The approach defines the \emph{probability} of
someone from group $i$ forming a \emph{given} type-$p$ partnership with someone from group $i'$ as:
\begin{equation}\label{eq:mix.eps}
  \rho_{pii'} = (\epsilon)\,I_{ii'} + (1 - \epsilon)\,\pi_{ii'},
  \quad I_{ii'} = \begin{cases} ~1 & i = i'\\ ~0 & i \ne i' \end{cases},
  \quad \pi_{ii'} = \frac{M_{pi'}}{\sum_{j}M_{pj}}
\end{equation} where:
$I$ represents complete like-with-like mixing (an identity matrix),
$\pi$ represents random mixing (random but proportional to the number of partnerships ``offered''),
and $\epsilon$ effectively interpolates between these two extremes.
Thus, $\epsilon = 0$ reflects fully random mixing,
and $\epsilon = 1$ reflects fully like-with-like mixing.
Then, the total numbers of type-$p$ partnerships between groups $i$ and $i'$ can be
defined as $M_{pii'} = M_{pi}\,\rho_{pii'}$.
Three advantages of \eqref{eq:mix.eps} are:
(1) simplicity;
(2) $\epsilon$ can be directly interpreted as the proportion of partnerships
which are formed among like-with-like \vs randomly; and
(3) it guarantees that $M$ will be symmetric, even if $P$ and/or $C$ change.
Yet, the simplicity of this approach precludes implementation of more complex mixing patterns
--- such as preferential mixing among two of four total groups ---
although some modest extensions can be made,
such as asymetric age mixing among women and men \cite{Cremin2013}.
%---------------------------------------------------------------------------------------------------
\subsubsection{Log-Linear Mixing}\label{model.par.mix.ll}
A more general  approach to mixing is developed in \cite{Morris1991}.
This ``log-linear'' approach defines the mixing matrix elements $M_{pii'}$ as follows.
The expected total numbers of partnerships between risk groups under random mixing are defined as:
\begin{equation}\label{eq:mix.rand}
  \Pi_{pii'} = \frac{M_{pi} M_{pi'}}{\sum_{j} M_{pj}}
\end{equation}
Next, a matrix $\Phi_{pii'}$ is defined, representing the odds of
a type-$p$ partnership forming between groups $i$ and $i'$, compared to random mixing.
The matrix $\Phi$ must be symmetric,
and can be estimated directly from the right kind of data
(which is rarely available) \cite{Morris1991}.
Then, an initial estimate of $M_{pii'}$ is:
\begin{alignat}{1}
  M_{pii'}^{\,(0)} &= \exp{\left[\log{\left(\Pi_{pii'}\right)} + \Phi_{pii'} \right]} \nonumber\\
                 &= \Pi_{pii'} \exp{\left(\Phi_{pii'}\right)} \label{eq:mix.M0}
\end{alignat}
However, this estimate changes the total numbers of partnerships formed by each group:
$M_{pi}^{\,(0)} \ne \Pi_{pi}$, where
$M_{pi} = \sum_{i'} M_{pii'}$ and $\Pi_{pi} = \sum_{i'} \Pi_{pii'}$.
There is no \textit{a priori} definition of $M_{pii'}$ or adjustment to $\Phi_{pii'}$
that can guarantee the numbers of partnerships will not change.%
\footnote{I hypothesize that this lack of \textit{a priori} solution
  is the reason this approach has not been widely used.}
However, an iterative proportional fitting procedure \cite{Ruschendorf1995}
can resolve an estimate $M_{pii'}^{\,(n)}$ that maintains the total numbers of partnerships:
\begin{equation}\label{eq:mix.iter}
  M_{pii'}^{\,(n+1)} = M_{pii'}^{\,(n)} \frac{\Pi_{pf}}{M_{pf}^{\,(n)}}
  \qquad f = \begin{cases}
    ~i  & \txn{if $n$ is even} \\
    ~i' & \txn{if $n$ is odd}
  \end{cases}
\end{equation}
Each step of this procedure can be understood as
a re-scaling of the current estimate $M_{pii'}^{\,(n)}$
row-wise ($i$) or column-wise ($i'$) to match the numbers of partnerships
offered by individuals ($\Pi_{pi}$) or their partners ($\Pi_{pi'}$).
Each row-step re-introduces discrepancies in the columns, and vice versa,
but overall convergence is guaranteed \cite{Sinkhorn1964}.
\par
In practice, \eqref{eq:mix.iter} adds approximately
one decimal of precision per $2n$ for the $4\times4$ case,
thus 15--20 iterations is often sufficient to come within computational precision limits.
Since the partnerships matrix $M_{pii'}$ should adapt to reflect changes in
group sizes (\eg due to HIV mortality) or
numbers of partnerships offered (\eg see \sref{foi.prop}),
the matrix must be re-computed at every time point.
Thus, the procedure \eqref{eq:mix.iter} could be considered computationally expensive.
However, this approach provides great flexibility and interpretability
to specify complex mixing patterns via the odds matrix $\Phi_{pii'}$.
\par
Adding back the sex dimension indices $i \rightarrow si, ~ i' \rightarrow s'i'$,
two final adjustments are needed for the bipartite (\ie heterosexual) system.
First, I ensure that $M_{s=s'} = \Pi_{s=s'} = 0$.
Second, for the case when the total numbers of partnerships offered by women and men
do not balance ($\sum_j M_{ps_{1}j} \ne \sum_j M_{ps_{2}j}$),
I revise the denominator of \eqref{eq:mix.rand} to $\sum_{j} \omega_s M_{psj}$,
where $\omega_s$ are weights such that $\sum_s \omega_s = 1$.
Similar to the ``compromise'' parameter $\theta$ in \cite{Garnett1994},
if $\omega = \{1,0\}$, then women's partnership numbers are matched exactly
while men adapt their partner numbers to balance;
and conversely for $\omega = \{0,1\}$.
I fixed $\omega = \{0.5,0.5\}$ for equal adaptation among women and men.
%---------------------------------------------------------------------------------------------------
\subsubsection{Odds of Mixing}\label{model.par.mix.odds}
Despite the flexibility in the odds of mixing matrix $\Phi_{pii'}$,
limited data are available to inform specific elements,
especially for Eswatini in particular.
In Kenya \cite{Voeten2007}, Benin, Guinea, and Senegal \cite{Godin2008}, and Uganda \cite{Mbonye2022},
a disproportionate fraction of non-paying partners of FSW were former and/or current clients.
Given this fraction $\psi$ and the proportion of all men who are clients $\rho$,
the odds of these partnerships forming can be computed as:
\begin{equation}
  \Phi = \frac{\psi\,(1-\rho)}{(1-\psi)\,\rho}
\end{equation}
Taking $\psi \in (0.33,~0.70)$ \cite{Voeten2007,Godin2008}
and $\rho \in (5,~20)\%$ \cite{Hodgins2022}, we obtain $\Phi \in (2,~19)$.
As noted in \sref{model.par.pnum.msc}, its not clear whether such partnerships reflect
main/spousal or casual partnerships.
As such, I sampled a common value for both partnership types,
as well as for higher/lower risk FSW and clients:
$\Phi_{p_{12}i_{34}i'_{34}}$ from a gamma prior with 95\%~CI of (2,~19).
I further assumed that lowest activity women and men had
greater odds of forming main/spousal partnerships with each other,
based loosely on age cohorting effects \cite{Leclerc-Madlala2008},
observed like-with-like sexual mixing preferences in numerous other contexts
\cite{Morris1991,Garnett1993a,Admiraal2016},
and prior modelling work \cite{Knight2022sr}.
I sampled $\Phi_{p_{1}i_{1}i'_{1}}$ from a gamma prior with 95\%~CI of (1.5,~3).
I made no further assumptions about preferential mixing (\ie all other elements $\Phi = 1$).
Thus, I assumed that occasional and regular sex work partnerships form
randomly with respect to higher \vs lower FSW and their clients.

