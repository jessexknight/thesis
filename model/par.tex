\section{Parameterization}\label{model.par}
As described in \sref{intro.model.param}, model parameterization involves
specification of model parameter values, such as proportions, probabilities, rates, and ratios,
including stratified values to reflect heterogeneity,
and sampling distributions to reflect uncertainty.
Proportions and probabilities were generally modelled using
a beta approximation of the binomial distribution (BAB, see \sref{app.math.distr.bab}),
while rates and ratios were generally modelled using
a gamma, skewnormal, or inverse gaussian distribution.
\paragraph{Notation}
If $X$ is a parameter stratified by dimensions $a,b,c$,
then $X_{ab_{1}c_{23}}$ denotes the values of $X$ for
a particular but \emph{unspecified} stratum of $a$,
the \emph{specific} stratum $b = 1$,
and the \emph{aggregated} strata $c = 2,3$
(the aggregating operation is context-dependent, \eg sum for probabilities).
Additionally, the indices $sihc$ from Table~\ref{tab:model.dims} denote ``self'' strata,
whereas $s'i'h'c'$ denote ``other'' strata --- \ie individuals' partners.%
\footnote{\label{foot:code.note}%
  In the code: R uses one-based indexing, which match the notation here directly,
  while Python uses zero-based indexing, which therefore appear as $i \rightarrow i-1$ in the code.
  Also, the model code reorders states in the ART Cascade dimension for computational efficiency,
  with $c={}$1:~Undiagnosed; 2:~Diagnosed; 3:~Virally~Un-suppressed; 4:~On~ART; 5:~Virally~Suppressed.}
Finally, I re-use several dummy variables throughout the chapter:
$\rho$ for proportions, $\lambda$ for rates, $T$ for time periods, and $f$ for constants.
%===================================================================================================
\subsection{Risk Heterogeneity Among FSW}\label{model.par.fsw}
Existing HIV transmission models which include FSW
have rarely sub-stratified this population, such as to reflect
differential HIV risk or distinct typologies of sex work \cite{Blanchard2008,Scorgie2012};
yet such heterogeneities may influence transmission dynamics.
Among the studies identified in Chapter~\ref{sr},
only three sub-stratified FSW by risk-related factors:
\citet{Cremin2017} defined three levels of risk via regression analysis,
\citet{Low2015} distinguished between occasional and full-time FSW, while
\citet{Shannon2015} sub-stratified FSW by
work environment, violence exposure, and context-specific structural factors.
Seven other studies, reflecting two unique models \cite{Johnson2012,Maheu-Giroux2017},
employed age stratification of all activity groups, including FSW;
these models had several risk-related parameters which varied by age.
\par
The model structure here (Figure~\ref{fig:model.risk})
was designed to capture \emph{within}-FSW risk heterogeneity.
The objective of the following analysis was therefore to parameterize
lower \vs higher risk FSW.
I sought to define these groups based on biobehavioural and/or contextual factors
which are demonstrably associated with HIV risk,
and which can be mechanistically incorporated into a transmission model ---
\ie through the force of infection equation.
Later, the parameterization of these groups was validated through model fitting
to relative differences in HIV prevalence \sref{model.cal.targ.prev}.
\par
Many cross-sectional studies of HIV among FSW quantify
the association of risk factors with HIV serostatus
\cite{Aklilu2001,Dunkle2005,Scorgie2012,Jonas2020}.
However, serostatus reflects cumulative risk exposure,
whereas sexual risk behaviour is dynamic \cite{Watts2010,vanWees2020},
as is use of prevention resources \cite{Roberts2020}.
For example, while HIV prevalence often increases with age,
HIV incidence among women can peak before age 25 \cite{Dellar2015}.
Thus, risk factors associated with HIV serostatus are not necessarily
mechanistically related to HIV acquisition.
Indeed, FSW may reduce risk behaviours in response to seroconversion \cite{McClelland2006}.
Cohort studies that measure incidence
can help identify risk factors for HIV acquisition \cite{McKinnon2015,Nouaman2022},
but large sample sizes are often required to accurately estimate overall incidence rate,
let alone risk factors \cite{Priddy2011}.
%---------------------------------------------------------------------------------------------------
\subsubsection{FSW Survey Data}\label{model.par.fsw.data}
Three biobehavioural surveys, in
2011 \cite{Baral2014} (N = 325),
2014 \cite{EswKP2014} (N = 781), and
2021 \cite{EswIBBS2022} (N = 676)
provide HIV status and biobehavioural data on FSW in Eswatini.
The 2011 and 2021 surveys featured serologic HIV testing,
and employed respondent driven sampling (RDS, details in \cite{Yam2013}).
The 2014 survey relied on self-reported HIV status,
andd employed venue-based snowball sampling, based on the
Priorities for Local AIDS Control Efforts (PLACE) methodology,
which aims to identify areas of higher incidence \cite{Weir2005}.
More details about each study are given in \sref{intro.esw.hiv.data} and Table~\ref{tab:esw.data}.
I analyzed the individual-level data from 2011 and 2014 (data from 2021 not yet available)
to explore the potential association of biobehavioural factors with HIV risk,
so that such factors could then be used to distinguish between
lower risk \vs higher risk FSW.
% TODO: (*) add descriptive table
%---------------------------------------------------------------------------------------------------
\subsubsection{HIV Status}\label{model.par.fsw.hiv}
Only the 2011 and 2021 studies included serologic testing for HIV.
Among those tested in 2011 (N = 317, 98\%), 70\% were \hivp,
yielding RDS-adjusted prevalence estimate of 61\% (CI: 51--71\%) \cite{Baral2014}.
Among serologically \hivn, 11\% self-reported \hivp status (false positive), and
among serologically \hivp, 26\% self-reported \hivn status (false negative or undiagnosed).
Overall, self-reported HIV status underestimated HIV prevalence in 2011
by a factor of approximately 0.78 (55~vs~70\%).
Unadjusted HIV prevalence in 2021 was 58.8\%,
with 88\% (363/411) reporting previous awareness of \hivp status.
\par
In 2014, self-reported HIV prevalence was 38\% among respondents who reported (85\%).
This 38\% is surprisingly low considering that
the PLACE methodology explicitly aimed to sample venues
with higher HIV incidence \cite{Weir2005}, and 2014 \vs 2011 respondents
were older (median 27 \vs 25 years), % 2021 median: 28
had been selling sex longer (median 5 \vs 4 years), % 2021: 6
and tested more frequently (87 \vs 75\% tested at least once in the past year, % 2021: 75
82 \vs 63\% among self-reported \hivn).
Perhaps the differences are attributable to the sampling methodology.
Among respondents who self-reported \hivp status,
the 2014 survey also asked for age of HIV diagnosis (6\% missing).
Age of HIV diagnosis supports crude time-to-event analysis (next section),
which can account for confounding by age and censoring,
as compared to logistic regression on HIV status,
keeping in mind the limitations of self-reported HIV status.
%---------------------------------------------------------------------------------------------------
\subsubsection{Risk Factors for HIV}\label{model.par.fsw.fac}
Next, I explored the potential association of risk factors with HIV
via the following three models:%
\footnote{Logistic regression models were implemented using \texttt{lrm} from:
  \hreftt{cran.r-project.org/package=rms}.\\
Cox proportional hazards models were implemented using \texttt{coxaalen} from:
  \hreftt{cran.r-project.org/package=coxinterval}.}
\begin{enumerate}
  \item Logistic regression on serologic HIV status (2011 data)
  \item Logistic regression on self-reported HIV status (2014 data)
  \item Cox proportional hazards for interval-censored time to HIV infection,
    with interval from self-reported sex work debut 
    to either self-reported time of HIV diagnosis or survey date (2014 data);
    Figure~\ref{fig:fsw.tte.interval} illustrates
    the four potential censoring cases in this framework.
\end{enumerate}
An important limitation to all models is that
risk factors reported by FSW at the time of survey
are assumed to be fixed characteristics of the respondents,
rather than dynamic characteristics that vary over time.
Additionally, respondents with any missing variables for each individual model
were excluded from that model. % TODO: (%)
\begin{figure}
  \centering
  \includegraphics[scale=1]{diag.tte}
  \caption{Illustration of time-to-event analysis framework
    for cross-sectional FSW survey data}
  \label{fig:fsw.tte.interval}
  \floatfoot{
    $\bm{\times}$: HIV infection;
    SW: time of sex work debut;
    Dx: time of HIV diagnosis.}
\end{figure}
\par
Risk factors were selected based on
prior knowledge of plausible mechanistic influence on HIV incidence and/or prevalence.
The risk factors explored are summarized in Table~\ref{tab:fsw.stats},
including univariate and multivariable association under each model.
Variable selection for multivariable models
was performed using backward selection as described by \citet{Lawless1978},
using a $p \le 0.1$ (per variable) threshold for stepwise variable retention.
Estimated conditional effects of
variables retained in the multivariable logistic regression models
are illustrated in Figure~\ref{fig:fsw.lr}.
\begin{table}
  \centering
  \caption{Risk factors explored for association with \hivp status among FSW in Eswatini}
  \label{tab:fsw.stats}
  \centerline{%
\small%
\begin{tabular}{lcccccccccccc}
  \toprule
  & \multicolumn{4}{c}{2011 LR}
  & \multicolumn{4}{c}{2014 LR}
  & \multicolumn{4}{c}{2014 CPH} \\
  \cmidrule(rl){2-5}\cmidrule(rl){6-9}\cmidrule(rl){10-13}
  & \multicolumn{2}{c}{Univar} & \multicolumn{2}{c}{Multivar}
  & \multicolumn{2}{c}{Univar} & \multicolumn{2}{c}{Multivar}
  & \multicolumn{2}{c}{Univar} & \multicolumn{2}{c}{Multivar} \\
  \cmidrule(rl){2-3}\cmidrule(rl){4-5}\cmidrule(rl){6-7}\cmidrule(rl){8-9}\cmidrule(rl){10-11}\cmidrule(rl){12-13}
  Factor                          &  OR  &   p   &  OR  &   p    &  OR  &   p    &  OR  &   p    &  HR  &   p    &  HR  &   p    \\
  \midrule                        % 2011 LR uni  % 2011 LR multi % 2014 LR uni   % 2014 LR multi % 2014 CPH uni  % 2014 CPH multi
  Age\tn{a}                       & 1.11 & \vsig & ---  &  ---   & 1.14 & \vsig  & 1.15 & \vsig  & 1.09 & \vsig  & 1.09 & \vsig  \\
  Years selling sex\tn{a}         & 1.13 & \vsig & 1.13 & \vsig  & 1.12 & \vsig  & ---  &  ---   & 1.08 & \vsig  & ---  &  ---   \\
  Monthly sex work income\tn{b}   & 0.98 & 0.155 & ---  &  ---   & 0.98 & 0.097  & 0.97 & 0.084  & 0.98 & 0.019\s& 0.97 & 0.001\s\\[1ex]
  Non-paying partners\tn{c}       & 0.88 & 0.307 & ---  &  ---   & 1.07 & 0.233  & ---  &  ---   & 1.05 & 0.312  & ---  &  ---   \\
  Monthly new clients\tn{c}       & 1.01 & 0.412 & ---  &  ---   & 1.05 & \vsig  & 1.07 & \vsig  & 1.04 & \vsig  & 1.04 & \vsig  \\
  Monthly regular clients\tn{c}   & 1.01 & 0.351 & ---  &  ---   & 1.03 & 0.002  & ---  &  ---   & 1.02 & \vsig  & 1.02 & 0.034\s\\[1ex]
  Non-paying condom use\tn{d}     & 0.90 & 0.703 & ---  &  ---   & 0.90 & 0.673  & ---  &  ---   & 0.92 & 0.677  & ---  &  ---   \\
  New client condom use\tn{d}     & 0.60 & 0.100 & ---  &  ---   & 0.48 & 0.006\s& 1.25 & 0.599  & 0.56 & 0.004\s& ---  &  ---   \\
  Regular client condom use\tn{d} & 0.58 & 0.110 & ---  &  ---   & 0.39 & \vsig  & 0.35 & 0.004\s& 0.49 & \vsig  & 0.50 & \vsig  \\[1ex]
  Any anal sex past month         & 0.97 & 0.896 & ---  &  ---   & 1.89 & 0.015\s& ---  &  ---   & 1.57 & 0.015\s& 1.27 & 0.260  \\
  Any STI symptoms past year      & 2.29 & \vsig & 2.41 & \vsig  & 2.75 & \vsig  & 2.80 & \vsig  & 2.17 & \vsig  & 2.05 & \vsig  \\
  \bottomrule
\end{tabular}}
% TODO: HIV status?
\floatfoot{\raggedright
  \tnt[a]{OR per year};
  \tnt[b]{OR per Swazi lilangeni per month};
  \tnt[c]{OR per partner};
  \tnt[d]{2011: always vs not always, 2014: at last sex}.
  --- indicates variable was not selected in the multivariate model.
  LR: logistic regression on HIV$+/-$ status;
  CPH: Cox proportional hazards on time to self-reported HIV seroconversion.
  OR: odds ratio; HR: hazard ratio; p: p-value.
  2011 data based on serologic HIV test;
  2014 data based on self-reported HIV status, age of sex work debut, and age of HIV diagnosis.
}
\end{table}
\begin{figure}[h]
  \subcapoverlap
  \foreach \year/\var/\nvar in {2011/f/1,2011/c/2,2014/f/3,2014/c/3}{
  \begin{subfigure}{\nvar\linewidth/5+\linewidth/5}
    \includegraphics[scale=.7]{fsw.\year.lr.hiv.\var}
    \caption{\raggedright}
    \label{fig:fsw.lr.\year.\var}
  \end{subfigure}}
  \caption{Predicted conditional effects (probability)
    of variables in multivariable logistic regression models for HIV status}
  \label{fig:fsw.lr}
  \floatfoot{\fffsw{fig:fsw.lr}
    conditional probabilities shown for fixed covariates at arbitrary values.}
\end{figure}
\par
Following variable selection, each multivariable model was used to estimate
the total \hivp status odds ratio (logistic) or HIV incidence hazard ratio (Cox)
for each respondent in the respective survey ---
\ie $e^{X_i\,\beta}$ for respondent $i$ ---
representing an overall ``risk score'' under each model.
Respondents were then stratified into the top 20\% and bottom 80\% by these risk scores.
The values of each variable were compared between these two strata
using a test for the ratio of the means \cite{Tamhane2004} to support model parameterization;
these ratios are summarized in Table~\ref{tab:fsw.ratios},
and the distributions of variable values across the two strata
are illustrated in Figure~\ref{fig:fsw.f}.
\begin{table}
  \centering
  \caption{Ratios of HIV risk factor variables among higher \vs lower risk FSW in Eswatini}
  \label{tab:fsw.ratios}
  \centerline{\footnotesize%
\begin{tabular}{lcccccc}
  \toprule
  & \multicolumn{2}{c}{2011 LR}
  & \multicolumn{2}{c}{2014 LR}
  & \multicolumn{2}{c}{2014 CPH} \\
  \cmidrule(rl){2-3}\cmidrule(rl){4-5}\cmidrule(rl){6-7}
  Factor                            &  High / Low   &   Ratio (95\% CI)   &  High / Low   &   Ratio (95\% CI)   &   High / Low   &   Ratio (95\% CI)   \\
  \midrule
  Age                               & 31.8  / 24.7  & 1.29 (1.22, 1.36)\s & 32.6  / 26.2  & 1.24 (1.20, 1.28)\s &  33.5  / 26.6  & 1.26 (1.21, 1.31)\s \\
  Years selling sex                 & 11.3  /  4.03 & 2.81 (2.41, 3.25)\s & 10.0  /  5.47 & 1.83 (1.64, 2.03)\s &  10.2  /  5.83 & 1.75 (1.54, 1.98)\s \\
  Monthly sex work income\tn{a}     & 15.1  / 15.2  & 1.00 (0.86, 1.15)   &  6.77 /  7.06 & 0.96 (0.82, 1.11)   &   6.32 /  7.28 & 0.87 (0.73, 1.02)   \\[1ex]
  Non-paying partners               &  1.42 /  1.43 & 0.99 (0.81, 1.19)   &  1.56 /  1.11 & 1.40 (1.11, 1.72)\s &   1.53 /  1.19 & 1.29 (0.98, 1.62)   \\
  Monthly new clients               &  5.50 /  6.98 & 0.79 (0.49, 1.15)   &  8.39 /  4.15 & 2.02 (1.63, 2.44)\s &   8.36 /  4.41 & 1.90 (1.43, 2.39)\s \\
  Monthly regular clients           &  9.35 /  9.05 & 1.03 (0.69, 1.42)   & 11.1  /  8.25 & 1.35 (1.13, 1.57)\s &  12.4  /  8.61 & 1.44 (1.18, 1.71)\s \\[1ex]
  Non-paying condom use\tn{bc}      &  0.26 /  0.35 & 0.73 (0.40, 1.11)   &  0.77 /  0.81 & 0.95 (0.84, 1.06)   &   0.76 /  0.81 & 0.95 (0.81, 1.08)   \\
  New client condom use\tn{bc}      &  0.68 /  0.76 & 0.89 (0.73, 1.06)   &  0.79 /  0.91 & 0.86 (0.79, 0.94)\s &   0.74 /  0.94 & 0.79 (0.69, 0.88)\s \\
  Regular client condom use\tn{bc}  &  0.38 /  0.46 & 0.83 (0.45, 1.28)   &  0.67 /  0.91 & 0.74 (0.65, 0.82)\s &   0.60 /  0.92 & 0.65 (0.55, 0.75)\s \\[1ex]
  Any anal sex past month           &  0.59 /  0.41 & 1.41 (1.06, 1.84)\s &  0.17 /  0.07 & 2.43 (1.47, 3.85)\s &   0.23 /  0.07 & 3.24 (1.95, 5.34)\s \\
  Any STI symptoms past year\tn{c}  &  0.79 /  0.43 & 1.86 (1.54, 2.25)\s &  0.59 /  0.15 & 3.94 (3.15, 5.03)\s &   0.61 /  0.17 & 3.67 (2.87, 4.79)\s \\[1ex]
  HIV prevalence\tn{d}              &  0.94 /  0.64 & 1.46 (1.30, 1.63)\s &  0.66 /  0.29 & 2.29 (1.92, 2.75)\s &   0.71 /  0.31 & 2.32 (1.94, 2.80)\s \\
  \bottomrule
\end{tabular}}
% TODO: HIV status?
\floatfoot{\raggedright
  High / Low: mean variable value among higher / lower risk groups, as defined by
  the top 20\% / bottom 80\% in multivariable model-predicted risk score:
  odds ratio from logistic regression (LR);
  hazards ratio from Cox proportional hazards (CPH).
  \tnt[a]{Swati lilangeni per month};
  \tnt[b]{2011: always \vs not always, 2014: did use condom at last sex};
  \tnt[c]{proportion of respondents};
  \tnt[d]{2011: serologic HIV status; 2014: self-reported HIV status};
  \tnt[*]{statistically significant, $p < 0.05$}.
}
\end{table}

\section{Parameterization}\label{model.par}
As described in \sref{intro.model.param}, model parameterization involves
specification of model parameter values, such as proportions, probabilities, rates, and ratios,
including stratified values to reflect heterogeneity,
and sampling distributions to reflect uncertainty.
Proportions and probabilities were generally modelled using
a beta approximation of the binomial distribution (BAB, see \sref{app.math.distr.bab}),
while rates and ratios were generally modelled using
a gamma, skewnormal, or inverse gaussian distribution.
\paragraph{Notation}
If $X$ is a parameter stratified by dimensions $a,b,c$,
then $X_{ab_{1}c_{23}}$ denotes the values of $X$ for
a particular but \emph{unspecified} stratum of $a$,
the \emph{specific} stratum $b = 1$,
and the \emph{aggregated} strata $c = 2,3$
(the aggregating operation is context-dependent, \eg sum for probabilities).
Additionally, the indices $sihc$ from Table~\ref{tab:model.dims} denote ``self'' strata,
whereas $s'i'h'c'$ denote ``other'' strata --- \ie individuals' partners.%
\footnote{\label{foot:code.note}%
  In the code: R uses one-based indexing, which match the notation here directly,
  while Python uses zero-based indexing, which therefore appear as $i \rightarrow i-1$ in the code.
  Also, the model code reorders states in the ART Cascade dimension for computational efficiency,
  with $c={}$1:~Undiagnosed; 2:~Diagnosed; 3:~Virally~Un-suppressed; 4:~On~ART; 5:~Virally~Suppressed.}
Finally, I re-use several dummy variables throughout the chapter:
$\rho$ for proportions, $\lambda$ for rates, $T$ for time periods, and $f$ for constants.
%===================================================================================================
\subsection{Risk Heterogeneity Among FSW}\label{model.par.fsw}
Existing HIV transmission models which include FSW
have rarely sub-stratified this population, such as to reflect
differential HIV risk or distinct typologies of sex work \cite{Blanchard2008,Scorgie2012};
yet such heterogeneities may influence transmission dynamics.
Among the studies identified in Chapter~\ref{sr},
only three sub-stratified FSW by risk-related factors:
\citet{Cremin2017} defined three levels of risk via regression analysis,
\citet{Low2015} distinguished between occasional and full-time FSW, while
\citet{Shannon2015} sub-stratified FSW by
work environment, violence exposure, and context-specific structural factors.
Seven other studies, reflecting two unique models \cite{Johnson2012,Maheu-Giroux2017},
employed age stratification of all activity groups, including FSW;
these models had several risk-related parameters which varied by age.
\par
The model structure here (Figure~\ref{fig:model.risk})
was designed to capture \emph{within}-FSW risk heterogeneity.
The objective of the following analysis was therefore to parameterize
lower \vs higher risk FSW.
I sought to define these groups based on biobehavioural and/or contextual factors
which are demonstrably associated with HIV risk,
and which can be mechanistically incorporated into a transmission model ---
\ie through the force of infection equation.
Later, the parameterization of these groups was validated through model fitting
to relative differences in HIV prevalence \sref{model.cal.targ.prev}.
\par
Many cross-sectional studies of HIV among FSW quantify
the association of risk factors with HIV serostatus
\cite{Aklilu2001,Dunkle2005,Scorgie2012,Jonas2020}.
However, serostatus reflects cumulative risk exposure,
whereas sexual risk behaviour is dynamic \cite{Watts2010,vanWees2020},
as is use of prevention resources \cite{Roberts2020}.
For example, while HIV prevalence often increases with age,
HIV incidence among women can peak before age 25 \cite{Dellar2015}.
Thus, risk factors associated with HIV serostatus are not necessarily
mechanistically related to HIV acquisition.
Indeed, FSW may reduce risk behaviours in response to seroconversion \cite{McClelland2006}.
Cohort studies that measure incidence
can help identify risk factors for HIV acquisition \cite{McKinnon2015,Nouaman2022},
but large sample sizes are often required to accurately estimate overall incidence rate,
let alone risk factors \cite{Priddy2011}.
%---------------------------------------------------------------------------------------------------
\subsubsection{FSW Survey Data}\label{model.par.fsw.data}
Three biobehavioural surveys, in
2011 \cite{Baral2014} (N = 325),
2014 \cite{EswKP2014} (N = 781), and
2021 \cite{EswIBBS2022} (N = 676)
provide HIV status and biobehavioural data on FSW in Eswatini.
The 2011 and 2021 surveys featured serologic HIV testing,
and employed respondent driven sampling (RDS, details in \cite{Yam2013}).
The 2014 survey relied on self-reported HIV status,
andd employed venue-based snowball sampling, based on the
Priorities for Local AIDS Control Efforts (PLACE) methodology,
which aims to identify areas of higher incidence \cite{Weir2005}.
More details about each study are given in \sref{intro.esw.hiv.data} and Table~\ref{tab:esw.data}.
I analyzed the individual-level data from 2011 and 2014 (data from 2021 not yet available)
to explore the potential association of biobehavioural factors with HIV risk,
so that such factors could then be used to distinguish between
lower risk \vs higher risk FSW.
% TODO: (*) add descriptive table
%---------------------------------------------------------------------------------------------------
\subsubsection{HIV Status}\label{model.par.fsw.hiv}
Only the 2011 and 2021 studies included serologic testing for HIV.
Among those tested in 2011 (N = 317, 98\%), 70\% were \hivp,
yielding RDS-adjusted prevalence estimate of 61\% (CI: 51--71\%) \cite{Baral2014}.
Among serologically \hivn, 11\% self-reported \hivp status (false positive), and
among serologically \hivp, 26\% self-reported \hivn status (false negative or undiagnosed).
Overall, self-reported HIV status underestimated HIV prevalence in 2011
by a factor of approximately 0.78 (55~vs~70\%).
Unadjusted HIV prevalence in 2021 was 58.8\%,
with 88\% (363/411) reporting previous awareness of \hivp status.
\par
In 2014, self-reported HIV prevalence was 38\% among respondents who reported (85\%).
This 38\% is surprisingly low considering that
the PLACE methodology explicitly aimed to sample venues
with higher HIV incidence \cite{Weir2005}, and 2014 \vs 2011 respondents
were older (median 27 \vs 25 years), % 2021 median: 28
had been selling sex longer (median 5 \vs 4 years), % 2021: 6
and tested more frequently (87 \vs 75\% tested at least once in the past year, % 2021: 75
82 \vs 63\% among self-reported \hivn).
Perhaps the differences are attributable to the sampling methodology.
Among respondents who self-reported \hivp status,
the 2014 survey also asked for age of HIV diagnosis (6\% missing).
Age of HIV diagnosis supports crude time-to-event analysis (next section),
which can account for confounding by age and censoring,
as compared to logistic regression on HIV status,
keeping in mind the limitations of self-reported HIV status.
%---------------------------------------------------------------------------------------------------
\subsubsection{Risk Factors for HIV}\label{model.par.fsw.fac}
Next, I explored the potential association of risk factors with HIV
via the following three models:%
\footnote{Logistic regression models were implemented using \texttt{lrm} from:
  \hreftt{cran.r-project.org/package=rms}.\\
Cox proportional hazards models were implemented using \texttt{coxaalen} from:
  \hreftt{cran.r-project.org/package=coxinterval}.}
\begin{enumerate}
  \item Logistic regression on serologic HIV status (2011 data)
  \item Logistic regression on self-reported HIV status (2014 data)
  \item Cox proportional hazards for interval-censored time to HIV infection,
    with interval from self-reported sex work debut 
    to either self-reported time of HIV diagnosis or survey date (2014 data);
    Figure~\ref{fig:fsw.tte.interval} illustrates
    the four potential censoring cases in this framework.
\end{enumerate}
An important limitation to all models is that
risk factors reported by FSW at the time of survey
are assumed to be fixed characteristics of the respondents,
rather than dynamic characteristics that vary over time.
Additionally, respondents with any missing variables for each individual model
were excluded from that model. % TODO: (%)
\begin{figure}
  \centering
  \includegraphics[scale=1]{diag.tte}
  \caption{Illustration of time-to-event analysis framework
    for cross-sectional FSW survey data}
  \label{fig:fsw.tte.interval}
  \floatfoot{
    $\bm{\times}$: HIV infection;
    SW: time of sex work debut;
    Dx: time of HIV diagnosis.}
\end{figure}
\par
Risk factors were selected based on
prior knowledge of plausible mechanistic influence on HIV incidence and/or prevalence.
The risk factors explored are summarized in Table~\ref{tab:fsw.stats},
including univariate and multivariable association under each model.
Variable selection for multivariable models
was performed using backward selection as described by \citet{Lawless1978},
using a $p \le 0.1$ (per variable) threshold for stepwise variable retention.
Estimated conditional effects of
variables retained in the multivariable logistic regression models
are illustrated in Figure~\ref{fig:fsw.lr}.
\begin{table}
  \centering
  \caption{Risk factors explored for association with \hivp status among FSW in Eswatini}
  \label{tab:fsw.stats}
  \centerline{%
\small%
\begin{tabular}{lcccccccccccc}
  \toprule
  & \multicolumn{4}{c}{2011 LR}
  & \multicolumn{4}{c}{2014 LR}
  & \multicolumn{4}{c}{2014 CPH} \\
  \cmidrule(rl){2-5}\cmidrule(rl){6-9}\cmidrule(rl){10-13}
  & \multicolumn{2}{c}{Univar} & \multicolumn{2}{c}{Multivar}
  & \multicolumn{2}{c}{Univar} & \multicolumn{2}{c}{Multivar}
  & \multicolumn{2}{c}{Univar} & \multicolumn{2}{c}{Multivar} \\
  \cmidrule(rl){2-3}\cmidrule(rl){4-5}\cmidrule(rl){6-7}\cmidrule(rl){8-9}\cmidrule(rl){10-11}\cmidrule(rl){12-13}
  Factor                          &  OR  &   p   &  OR  &   p    &  OR  &   p    &  OR  &   p    &  HR  &   p    &  HR  &   p    \\
  \midrule                        % 2011 LR uni  % 2011 LR multi % 2014 LR uni   % 2014 LR multi % 2014 CPH uni  % 2014 CPH multi
  Age\tn{a}                       & 1.11 & \vsig & ---  &  ---   & 1.14 & \vsig  & 1.15 & \vsig  & 1.09 & \vsig  & 1.09 & \vsig  \\
  Years selling sex\tn{a}         & 1.13 & \vsig & 1.13 & \vsig  & 1.12 & \vsig  & ---  &  ---   & 1.08 & \vsig  & ---  &  ---   \\
  Monthly sex work income\tn{b}   & 0.98 & 0.155 & ---  &  ---   & 0.98 & 0.097  & 0.97 & 0.084  & 0.98 & 0.019\s& 0.97 & 0.001\s\\[1ex]
  Non-paying partners\tn{c}       & 0.88 & 0.307 & ---  &  ---   & 1.07 & 0.233  & ---  &  ---   & 1.05 & 0.312  & ---  &  ---   \\
  Monthly new clients\tn{c}       & 1.01 & 0.412 & ---  &  ---   & 1.05 & \vsig  & 1.07 & \vsig  & 1.04 & \vsig  & 1.04 & \vsig  \\
  Monthly regular clients\tn{c}   & 1.01 & 0.351 & ---  &  ---   & 1.03 & 0.002  & ---  &  ---   & 1.02 & \vsig  & 1.02 & 0.034\s\\[1ex]
  Non-paying condom use\tn{d}     & 0.90 & 0.703 & ---  &  ---   & 0.90 & 0.673  & ---  &  ---   & 0.92 & 0.677  & ---  &  ---   \\
  New client condom use\tn{d}     & 0.60 & 0.100 & ---  &  ---   & 0.48 & 0.006\s& 1.25 & 0.599  & 0.56 & 0.004\s& ---  &  ---   \\
  Regular client condom use\tn{d} & 0.58 & 0.110 & ---  &  ---   & 0.39 & \vsig  & 0.35 & 0.004\s& 0.49 & \vsig  & 0.50 & \vsig  \\[1ex]
  Any anal sex past month         & 0.97 & 0.896 & ---  &  ---   & 1.89 & 0.015\s& ---  &  ---   & 1.57 & 0.015\s& 1.27 & 0.260  \\
  Any STI symptoms past year      & 2.29 & \vsig & 2.41 & \vsig  & 2.75 & \vsig  & 2.80 & \vsig  & 2.17 & \vsig  & 2.05 & \vsig  \\
  \bottomrule
\end{tabular}}
% TODO: HIV status?
\floatfoot{\raggedright
  \tnt[a]{OR per year};
  \tnt[b]{OR per Swazi lilangeni per month};
  \tnt[c]{OR per partner};
  \tnt[d]{2011: always vs not always, 2014: at last sex}.
  --- indicates variable was not selected in the multivariate model.
  LR: logistic regression on HIV$+/-$ status;
  CPH: Cox proportional hazards on time to self-reported HIV seroconversion.
  OR: odds ratio; HR: hazard ratio; p: p-value.
  2011 data based on serologic HIV test;
  2014 data based on self-reported HIV status, age of sex work debut, and age of HIV diagnosis.
}
\end{table}
\begin{figure}[h]
  \subcapoverlap
  \foreach \year/\var/\nvar in {2011/f/1,2011/c/2,2014/f/3,2014/c/3}{
  \begin{subfigure}{\nvar\linewidth/5+\linewidth/5}
    \includegraphics[scale=.7]{fsw.\year.lr.hiv.\var}
    \caption{\raggedright}
    \label{fig:fsw.lr.\year.\var}
  \end{subfigure}}
  \caption{Predicted conditional effects (probability)
    of variables in multivariable logistic regression models for HIV status}
  \label{fig:fsw.lr}
  \floatfoot{\fffsw{fig:fsw.lr}
    conditional probabilities shown for fixed covariates at arbitrary values.}
\end{figure}
\par
Following variable selection, each multivariable model was used to estimate
the total \hivp status odds ratio (logistic) or HIV incidence hazard ratio (Cox)
for each respondent in the respective survey ---
\ie $e^{X_i\,\beta}$ for respondent $i$ ---
representing an overall ``risk score'' under each model.
Respondents were then stratified into the top 20\% and bottom 80\% by these risk scores.
The values of each variable were compared between these two strata
using a test for the ratio of the means \cite{Tamhane2004} to support model parameterization;
these ratios are summarized in Table~\ref{tab:fsw.ratios},
and the distributions of variable values across the two strata
are illustrated in Figure~\ref{fig:fsw.f}.
\begin{table}
  \centering
  \caption{Ratios of HIV risk factor variables among higher \vs lower risk FSW in Eswatini}
  \label{tab:fsw.ratios}
  \centerline{\footnotesize%
\begin{tabular}{lcccccc}
  \toprule
  & \multicolumn{2}{c}{2011 LR}
  & \multicolumn{2}{c}{2014 LR}
  & \multicolumn{2}{c}{2014 CPH} \\
  \cmidrule(rl){2-3}\cmidrule(rl){4-5}\cmidrule(rl){6-7}
  Factor                            &  High / Low   &   Ratio (95\% CI)   &  High / Low   &   Ratio (95\% CI)   &   High / Low   &   Ratio (95\% CI)   \\
  \midrule
  Age                               & 31.8  / 24.7  & 1.29 (1.22, 1.36)\s & 32.6  / 26.2  & 1.24 (1.20, 1.28)\s &  33.5  / 26.6  & 1.26 (1.21, 1.31)\s \\
  Years selling sex                 & 11.3  /  4.03 & 2.81 (2.41, 3.25)\s & 10.0  /  5.47 & 1.83 (1.64, 2.03)\s &  10.2  /  5.83 & 1.75 (1.54, 1.98)\s \\
  Monthly sex work income\tn{a}     & 15.1  / 15.2  & 1.00 (0.86, 1.15)   &  6.77 /  7.06 & 0.96 (0.82, 1.11)   &   6.32 /  7.28 & 0.87 (0.73, 1.02)   \\[1ex]
  Non-paying partners               &  1.42 /  1.43 & 0.99 (0.81, 1.19)   &  1.56 /  1.11 & 1.40 (1.11, 1.72)\s &   1.53 /  1.19 & 1.29 (0.98, 1.62)   \\
  Monthly new clients               &  5.50 /  6.98 & 0.79 (0.49, 1.15)   &  8.39 /  4.15 & 2.02 (1.63, 2.44)\s &   8.36 /  4.41 & 1.90 (1.43, 2.39)\s \\
  Monthly regular clients           &  9.35 /  9.05 & 1.03 (0.69, 1.42)   & 11.1  /  8.25 & 1.35 (1.13, 1.57)\s &  12.4  /  8.61 & 1.44 (1.18, 1.71)\s \\[1ex]
  Non-paying condom use\tn{bc}      &  0.26 /  0.35 & 0.73 (0.40, 1.11)   &  0.77 /  0.81 & 0.95 (0.84, 1.06)   &   0.76 /  0.81 & 0.95 (0.81, 1.08)   \\
  New client condom use\tn{bc}      &  0.68 /  0.76 & 0.89 (0.73, 1.06)   &  0.79 /  0.91 & 0.86 (0.79, 0.94)\s &   0.74 /  0.94 & 0.79 (0.69, 0.88)\s \\
  Regular client condom use\tn{bc}  &  0.38 /  0.46 & 0.83 (0.45, 1.28)   &  0.67 /  0.91 & 0.74 (0.65, 0.82)\s &   0.60 /  0.92 & 0.65 (0.55, 0.75)\s \\[1ex]
  Any anal sex past month           &  0.59 /  0.41 & 1.41 (1.06, 1.84)\s &  0.17 /  0.07 & 2.43 (1.47, 3.85)\s &   0.23 /  0.07 & 3.24 (1.95, 5.34)\s \\
  Any STI symptoms past year\tn{c}  &  0.79 /  0.43 & 1.86 (1.54, 2.25)\s &  0.59 /  0.15 & 3.94 (3.15, 5.03)\s &   0.61 /  0.17 & 3.67 (2.87, 4.79)\s \\[1ex]
  HIV prevalence\tn{d}              &  0.94 /  0.64 & 1.46 (1.30, 1.63)\s &  0.66 /  0.29 & 2.29 (1.92, 2.75)\s &   0.71 /  0.31 & 2.32 (1.94, 2.80)\s \\
  \bottomrule
\end{tabular}}
% TODO: HIV status?
\floatfoot{\raggedright
  High / Low: mean variable value among higher / lower risk groups, as defined by
  the top 20\% / bottom 80\% in multivariable model-predicted risk score:
  odds ratio from logistic regression (LR);
  hazards ratio from Cox proportional hazards (CPH).
  \tnt[a]{Swati lilangeni per month};
  \tnt[b]{2011: always \vs not always, 2014: did use condom at last sex};
  \tnt[c]{proportion of respondents};
  \tnt[d]{2011: serologic HIV status; 2014: self-reported HIV status};
  \tnt[*]{statistically significant, $p < 0.05$}.
}
\end{table}

\section{Parameterization}\label{model.par}
As described in \sref{intro.model.param}, model parameterization involves
specification of model parameter values, such as proportions, probabilities, rates, and ratios,
including stratified values to reflect heterogeneity,
and sampling distributions to reflect uncertainty.
Proportions and probabilities were generally modelled using
a beta approximation of the binomial distribution (BAB, see \sref{app.math.distr.bab}),
while rates and ratios were generally modelled using
a gamma, skewnormal, or inverse gaussian distribution.
\paragraph{Notation}
If $X$ is a parameter stratified by dimensions $a,b,c$,
then $X_{ab_{1}c_{23}}$ denotes the values of $X$ for
a particular but \emph{unspecified} stratum of $a$,
the \emph{specific} stratum $b = 1$,
and the \emph{aggregated} strata $c = 2,3$
(the aggregating operation is context-dependent, \eg sum for probabilities).
Additionally, the indices $sihc$ from Table~\ref{tab:model.dims} denote ``self'' strata,
whereas $s'i'h'c'$ denote ``other'' strata --- \ie individuals' partners.%
\footnote{\label{foot:code.note}%
  In the code: R uses one-based indexing, which match the notation here directly,
  while Python uses zero-based indexing, which therefore appear as $i \rightarrow i-1$ in the code.
  Also, the model code reorders states in the ART Cascade dimension for computational efficiency,
  with $c={}$1:~Undiagnosed; 2:~Diagnosed; 3:~Virally~Un-suppressed; 4:~On~ART; 5:~Virally~Suppressed.}
Finally, I re-use several dummy variables throughout the chapter:
$\rho$ for proportions, $\lambda$ for rates, $T$ for time periods, and $f$ for constants.
%===================================================================================================
\subsection{Risk Heterogeneity Among FSW}\label{model.par.fsw}
Existing HIV transmission models which include FSW
have rarely sub-stratified this population, such as to reflect
differential HIV risk or distinct typologies of sex work \cite{Blanchard2008,Scorgie2012};
yet such heterogeneities may influence transmission dynamics.
Among the studies identified in Chapter~\ref{sr},
only three sub-stratified FSW by risk-related factors:
\citet{Cremin2017} defined three levels of risk via regression analysis,
\citet{Low2015} distinguished between occasional and full-time FSW, while
\citet{Shannon2015} sub-stratified FSW by
work environment, violence exposure, and context-specific structural factors.
Seven other studies, reflecting two unique models \cite{Johnson2012,Maheu-Giroux2017},
employed age stratification of all activity groups, including FSW;
these models had several risk-related parameters which varied by age.
\par
The model structure here (Figure~\ref{fig:model.risk})
was designed to capture \emph{within}-FSW risk heterogeneity.
The objective of the following analysis was therefore to parameterize
lower \vs higher risk FSW.
I sought to define these groups based on biobehavioural and/or contextual factors
which are demonstrably associated with HIV risk,
and which can be mechanistically incorporated into a transmission model ---
\ie through the force of infection equation.
Later, the parameterization of these groups was validated through model fitting
to relative differences in HIV prevalence \sref{model.cal.targ.prev}.
\par
Many cross-sectional studies of HIV among FSW quantify
the association of risk factors with HIV serostatus
\cite{Aklilu2001,Dunkle2005,Scorgie2012,Jonas2020}.
However, serostatus reflects cumulative risk exposure,
whereas sexual risk behaviour is dynamic \cite{Watts2010,vanWees2020},
as is use of prevention resources \cite{Roberts2020}.
For example, while HIV prevalence often increases with age,
HIV incidence among women can peak before age 25 \cite{Dellar2015}.
Thus, risk factors associated with HIV serostatus are not necessarily
mechanistically related to HIV acquisition.
Indeed, FSW may reduce risk behaviours in response to seroconversion \cite{McClelland2006}.
Cohort studies that measure incidence
can help identify risk factors for HIV acquisition \cite{McKinnon2015,Nouaman2022},
but large sample sizes are often required to accurately estimate overall incidence rate,
let alone risk factors \cite{Priddy2011}.
%---------------------------------------------------------------------------------------------------
\subsubsection{FSW Survey Data}\label{model.par.fsw.data}
Three biobehavioural surveys, in
2011 \cite{Baral2014} (N = 325),
2014 \cite{EswKP2014} (N = 781), and
2021 \cite{EswIBBS2022} (N = 676)
provide HIV status and biobehavioural data on FSW in Eswatini.
The 2011 and 2021 surveys featured serologic HIV testing,
and employed respondent driven sampling (RDS, details in \cite{Yam2013}).
The 2014 survey relied on self-reported HIV status,
andd employed venue-based snowball sampling, based on the
Priorities for Local AIDS Control Efforts (PLACE) methodology,
which aims to identify areas of higher incidence \cite{Weir2005}.
More details about each study are given in \sref{intro.esw.hiv.data} and Table~\ref{tab:esw.data}.
I analyzed the individual-level data from 2011 and 2014 (data from 2021 not yet available)
to explore the potential association of biobehavioural factors with HIV risk,
so that such factors could then be used to distinguish between
lower risk \vs higher risk FSW.
% TODO: (*) add descriptive table
%---------------------------------------------------------------------------------------------------
\subsubsection{HIV Status}\label{model.par.fsw.hiv}
Only the 2011 and 2021 studies included serologic testing for HIV.
Among those tested in 2011 (N = 317, 98\%), 70\% were \hivp,
yielding RDS-adjusted prevalence estimate of 61\% (CI: 51--71\%) \cite{Baral2014}.
Among serologically \hivn, 11\% self-reported \hivp status (false positive), and
among serologically \hivp, 26\% self-reported \hivn status (false negative or undiagnosed).
Overall, self-reported HIV status underestimated HIV prevalence in 2011
by a factor of approximately 0.78 (55~vs~70\%).
Unadjusted HIV prevalence in 2021 was 58.8\%,
with 88\% (363/411) reporting previous awareness of \hivp status.
\par
In 2014, self-reported HIV prevalence was 38\% among respondents who reported (85\%).
This 38\% is surprisingly low considering that
the PLACE methodology explicitly aimed to sample venues
with higher HIV incidence \cite{Weir2005}, and 2014 \vs 2011 respondents
were older (median 27 \vs 25 years), % 2021 median: 28
had been selling sex longer (median 5 \vs 4 years), % 2021: 6
and tested more frequently (87 \vs 75\% tested at least once in the past year, % 2021: 75
82 \vs 63\% among self-reported \hivn).
Perhaps the differences are attributable to the sampling methodology.
Among respondents who self-reported \hivp status,
the 2014 survey also asked for age of HIV diagnosis (6\% missing).
Age of HIV diagnosis supports crude time-to-event analysis (next section),
which can account for confounding by age and censoring,
as compared to logistic regression on HIV status,
keeping in mind the limitations of self-reported HIV status.
%---------------------------------------------------------------------------------------------------
\subsubsection{Risk Factors for HIV}\label{model.par.fsw.fac}
Next, I explored the potential association of risk factors with HIV
via the following three models:%
\footnote{Logistic regression models were implemented using \texttt{lrm} from:
  \hreftt{cran.r-project.org/package=rms}.\\
Cox proportional hazards models were implemented using \texttt{coxaalen} from:
  \hreftt{cran.r-project.org/package=coxinterval}.}
\begin{enumerate}
  \item Logistic regression on serologic HIV status (2011 data)
  \item Logistic regression on self-reported HIV status (2014 data)
  \item Cox proportional hazards for interval-censored time to HIV infection,
    with interval from self-reported sex work debut 
    to either self-reported time of HIV diagnosis or survey date (2014 data);
    Figure~\ref{fig:fsw.tte.interval} illustrates
    the four potential censoring cases in this framework.
\end{enumerate}
An important limitation to all models is that
risk factors reported by FSW at the time of survey
are assumed to be fixed characteristics of the respondents,
rather than dynamic characteristics that vary over time.
Additionally, respondents with any missing variables for each individual model
were excluded from that model. % TODO: (%)
\begin{figure}
  \centering
  \includegraphics[scale=1]{diag.tte}
  \caption{Illustration of time-to-event analysis framework
    for cross-sectional FSW survey data}
  \label{fig:fsw.tte.interval}
  \floatfoot{
    $\bm{\times}$: HIV infection;
    SW: time of sex work debut;
    Dx: time of HIV diagnosis.}
\end{figure}
\par
Risk factors were selected based on
prior knowledge of plausible mechanistic influence on HIV incidence and/or prevalence.
The risk factors explored are summarized in Table~\ref{tab:fsw.stats},
including univariate and multivariable association under each model.
Variable selection for multivariable models
was performed using backward selection as described by \citet{Lawless1978},
using a $p \le 0.1$ (per variable) threshold for stepwise variable retention.
Estimated conditional effects of
variables retained in the multivariable logistic regression models
are illustrated in Figure~\ref{fig:fsw.lr}.
\begin{table}
  \centering
  \caption{Risk factors explored for association with \hivp status among FSW in Eswatini}
  \label{tab:fsw.stats}
  \input{model/tab.fsw.factor.stats}
\end{table}
\begin{figure}[h]
  \subcapoverlap
  \foreach \year/\var/\nvar in {2011/f/1,2011/c/2,2014/f/3,2014/c/3}{
  \begin{subfigure}{\nvar\linewidth/5+\linewidth/5}
    \includegraphics[scale=.7]{fsw.\year.lr.hiv.\var}
    \caption{\raggedright}
    \label{fig:fsw.lr.\year.\var}
  \end{subfigure}}
  \caption{Predicted conditional effects (probability)
    of variables in multivariable logistic regression models for HIV status}
  \label{fig:fsw.lr}
  \floatfoot{\fffsw{fig:fsw.lr}
    conditional probabilities shown for fixed covariates at arbitrary values.}
\end{figure}
\par
Following variable selection, each multivariable model was used to estimate
the total \hivp status odds ratio (logistic) or HIV incidence hazard ratio (Cox)
for each respondent in the respective survey ---
\ie $e^{X_i\,\beta}$ for respondent $i$ ---
representing an overall ``risk score'' under each model.
Respondents were then stratified into the top 20\% and bottom 80\% by these risk scores.
The values of each variable were compared between these two strata
using a test for the ratio of the means \cite{Tamhane2004} to support model parameterization;
these ratios are summarized in Table~\ref{tab:fsw.ratios},
and the distributions of variable values across the two strata
are illustrated in Figure~\ref{fig:fsw.f}.
\begin{table}
  \centering
  \caption{Ratios of HIV risk factor variables among higher \vs lower risk FSW in Eswatini}
  \label{tab:fsw.ratios}
  \input{model/tab.fsw.factor.ratios}
\end{table}

\section{Parameterization}\label{model.par}
As described in \sref{intro.model.param}, model parameterization involves
specification of model parameter values, such as proportions, probabilities, rates, and ratios,
including stratified values to reflect heterogeneity,
and sampling distributions to reflect uncertainty.
Proportions and probabilities were generally modelled using
a beta approximation of the binomial distribution (BAB, see \sref{app.math.distr.bab}),
while rates and ratios were generally modelled using
a gamma, skewnormal, or inverse gaussian distribution.
\paragraph{Notation}
If $X$ is a parameter stratified by dimensions $a,b,c$,
then $X_{ab_{1}c_{23}}$ denotes the values of $X$ for
a particular but \emph{unspecified} stratum of $a$,
the \emph{specific} stratum $b = 1$,
and the \emph{aggregated} strata $c = 2,3$
(the aggregating operation is context-dependent, \eg sum for probabilities).
Additionally, the indices $sihc$ from Table~\ref{tab:model.dims} denote ``self'' strata,
whereas $s'i'h'c'$ denote ``other'' strata --- \ie individuals' partners.%
\footnote{\label{foot:code.note}%
  In the code: R uses one-based indexing, which match the notation here directly,
  while Python uses zero-based indexing, which therefore appear as $i \rightarrow i-1$ in the code.
  Also, the model code reorders states in the ART Cascade dimension for computational efficiency,
  with $c={}$1:~Undiagnosed; 2:~Diagnosed; 3:~Virally~Un-suppressed; 4:~On~ART; 5:~Virally~Suppressed.}
Finally, I re-use several dummy variables throughout the chapter:
$\rho$ for proportions, $\lambda$ for rates, $T$ for time periods, and $f$ for constants.
\input{model/par.fsw}
\input{model/par.beta}
\input{model/par.hiv}
\input{model/par.popsex}
%===================================================================================================
\subsection{HIV Progression \& Mortality}\label{model.par.hiv}
%---------------------------------------------------------------------------------------------------
\subsubsection{HIV Progression}\label{model.par.hiv.dur}
The length of time spent in each HIV stage is related to
rates of progression between stages $\eta_{h}$,
rates of additional HIV-attributable mortality by stage $\mu_{\textsc{hiv},h}$,
and treatment via antiretroviral therapy (ART).
\citet{Lodi2011} estimate median times from seroconversion to
CD4 $<$ 500, $<$ 350, and $<$ 200 cells/mm\tsup{3}, while
\citet{Mangal2017} directly estimate the rates of progression between CD4 states $\eta_{h}$
in a simple compartmental model.
Based on these data, I modelled mean durations ($1/\eta_{h}$) of:%
\footnote{Assuming exponential distributions for durations in each CD4 state
  (see \sref{app.model.math.comp} for more details).}
0.142 years in acute infection ($h=2$, from \sref{model.par.beta.hiv});
3.35 years in CD4~$>$~500 ($h=3$);
3.74 years in 350~$<$~CD4~$<$~500 ($h=4$); and
5.26 years in 200~$<$~CD4~$<$~350 ($h=5$); plus
the remaining time until death in CD4~$<$~200 ($h=6$, AIDS).
Since the duration in acute infection ($h=2$) is randomly sampled,
the remaining duration in CD4~$>$~500 ($h=3$) is adjusted accordingly.
%---------------------------------------------------------------------------------------------------
\subsubsection{HIV Mortality}\label{model.par.hiv.mort}
Mortality rates by CD4-count in the absence of ART were estimated in
multiple African studies \cite{Badri2006,Anglaret2012,Mangal2017};
based on these data, I estimated yearly HIV-attributable mortality rates $\mu_{\textsc{hiv},h}$ as:
0 during acute phase ($h=2$);
0.4\% during CD4~$>$~500 ($h=3$);
2\% during 350~$<$~CD4~$<$~500 ($h=4$);
4\% during 200~$<$~CD4~$<$~350 ($h=5$); and 
20\% during CD4~$<$~200 ($h=6$, AIDS).
%===================================================================================================
\subsection{Antiretroviral Therapy}\label{model.par.art}
Viral suppression via antiretroviral therapy (ART) influences
the probability of HIV transmission, as well as rates of HIV progression and HIV-related mortality.
The model considers individuals on ART before ($c=4$) and after ($c=5$)
achieving full viral load suppression (VLS), as defined by undetectable HIV RNA in blood samples.
Among retained patients initiating ART, time to VLS
is usually described as ``within 6 months'' \cite{Thompson2012}.
More specifically, \citet{Mujugira2016} estimate the median time to VLS as 3 months,
yielding an estimated \emph{mean} duration for $c=4$ of 4.3 months (see \sref{app.model.math.comp}).
%---------------------------------------------------------------------------------------------------
\subsubsection{Probability of HIV Transmission}\label{model.par.art.beta}
All available evidence suggests that viral suppression by ART to undetectable levels
prevents HIV transmission, \ie undetectable = untransmittable (``U=U'') \cite{Eisinger2019}.
Thus, I assumed zero HIV transmission from individuals with VLS ($c=5$).
However, HIV transmission may still occur
during the period between ART initiation to viral suppression ($c=4$) \cite{Mujugira2016}.
\citet{Donnell2010} estimate an adjusted incidence ratio of 0.08~(0.0,~0.57) for all individuals on ART.
However, in \cite{Donnell2010} and \cite{Cohen2016}, the 1 and 4 (respectively)
genetically linked infections from individuals on ART all occurred within 90 days of ART initiation,
suggesting that risk of transmission only persists before viral suppression.
Adjusting the incidence denominator (person-time)
to 90 days per individual who initiated ART in \cite{Donnell2010}
results in approximately 3.13 times higher estimated incidence ratio: 0.25 for this specific period.%
\footnote{In \cite{Donnell2010}, individuals who initiated ART contributed
  approximately 9.4 months per-person (273 persons / 349 person-years, Tables~2~and~3);
  thus the first 3 months of each individual represent
  3/9.4 = 0.319 fewer person-months of follow-up.}
Thus, I sampled relative infectiousness on ART but before viral suppression ($c=4$)
from a beta distribution with mean (95\%~CI) of 0.25~(0.01,~0.67).
%---------------------------------------------------------------------------------------------------
\subsubsection{HIV Progression \& Mortality}\label{model.par.art.hiv}
\def\hunprog{$h = 6 \rightarrow 5 \rightarrow 4 \rightarrow 3$\xspace}
Effective ART stops CD4 cell decline and results in some CD4 recovery \cite{Battegay2006,Lawn2006}.
Most CD4 recovery occurs within the first year of treatment \cite{Battegay2006}.
Due to the limited number of modelled treatment states,
I model this initial recovery to be associated with the 4.3-month pre-VLS ART state ($c=4$).
\citet{Lawn2006,Gabillard2013} estimate an increase of between 25--39 cells/mm\tsup{3} per month
during the first 3 months of treatment.
Since HIV states $h=4,5,6$ correspond to 150, 150, and 200-wide CD4 strata,
I model rates of movement along \hunprog during pre-VLS ART ($c=4$) as
0.20, 0.20, 0.17 per month, respectively.
After initial increases, CD4 recovery is modest and plateaus.
\citet{Battegay2006} report approximate increases of
22.4 cells/mm\tsup{3} per year between years 1 and 5 on ART.
Thus, I model rates of movement along \hunprog after VLS ($c=5$) as 0.15 per year.
\par
Since higher CD4 states are modelled to have lower mortality rates (see \sref{model.par.hiv.mort}),
the modelled recovery of CD4 cells via ART described above implicitly affords a mortality benefit.
However, HIV infection is associated with increased risk of death by non-AIDS causes
--- \ie unrelated to CD4 count ---
including cardiovascular disease and renal disease \cite{Phillips2008}.
\citet{Lundgren2015} estimated 61\% reduction in non-AIDS life-threatening events due to ART.
For the same CD4 strata, \citet{Gabillard2013} also report approximately 2-times higher
mortality rates within the first year of ART versus thereafter,
suggesting that VLS is associated with 50\% mortality reduction independent of CD4 increase.
Thus, I modelled an additional 50\% reduction in mortality among individuals with VLS ($c=5$),
and half this (25\%) reduction before achieving VLS ($c=4$).
% TODO: rates of diagnosis, testing, vls
\section{Parameterization}\label{model.par}
As described in \sref{intro.model.param}, model parameterization involves
specification of model parameter values, such as proportions, probabilities, rates, and ratios,
including stratified values to reflect heterogeneity,
and sampling distributions to reflect uncertainty.
Proportions and probabilities were generally modelled using
a beta approximation of the binomial distribution (BAB, see \sref{app.math.distr.bab}),
while rates and ratios were generally modelled using
a gamma, skewnormal, or inverse gaussian distribution.
\paragraph{Notation}
If $X$ is a parameter stratified by dimensions $a,b,c$,
then $X_{ab_{1}c_{23}}$ denotes the values of $X$ for
a particular but \emph{unspecified} stratum of $a$,
the \emph{specific} stratum $b = 1$,
and the \emph{aggregated} strata $c = 2,3$
(the aggregating operation is context-dependent, \eg sum for probabilities).
Additionally, the indices $sihc$ from Table~\ref{tab:model.dims} denote ``self'' strata,
whereas $s'i'h'c'$ denote ``other'' strata --- \ie individuals' partners.%
\footnote{\label{foot:code.note}%
  In the code: R uses one-based indexing, which match the notation here directly,
  while Python uses zero-based indexing, which therefore appear as $i \rightarrow i-1$ in the code.
  Also, the model code reorders states in the ART Cascade dimension for computational efficiency,
  with $c={}$1:~Undiagnosed; 2:~Diagnosed; 3:~Virally~Un-suppressed; 4:~On~ART; 5:~Virally~Suppressed.}
Finally, I re-use several dummy variables throughout the chapter:
$\rho$ for proportions, $\lambda$ for rates, $T$ for time periods, and $f$ for constants.
\input{model/par.fsw}
\input{model/par.beta}
\input{model/par.hiv}
\input{model/par.popsex}
%===================================================================================================
\subsection{HIV Progression \& Mortality}\label{model.par.hiv}
%---------------------------------------------------------------------------------------------------
\subsubsection{HIV Progression}\label{model.par.hiv.dur}
The length of time spent in each HIV stage is related to
rates of progression between stages $\eta_{h}$,
rates of additional HIV-attributable mortality by stage $\mu_{\textsc{hiv},h}$,
and treatment via antiretroviral therapy (ART).
\citet{Lodi2011} estimate median times from seroconversion to
CD4 $<$ 500, $<$ 350, and $<$ 200 cells/mm\tsup{3}, while
\citet{Mangal2017} directly estimate the rates of progression between CD4 states $\eta_{h}$
in a simple compartmental model.
Based on these data, I modelled mean durations ($1/\eta_{h}$) of:%
\footnote{Assuming exponential distributions for durations in each CD4 state
  (see \sref{app.model.math.comp} for more details).}
0.142 years in acute infection ($h=2$, from \sref{model.par.beta.hiv});
3.35 years in CD4~$>$~500 ($h=3$);
3.74 years in 350~$<$~CD4~$<$~500 ($h=4$); and
5.26 years in 200~$<$~CD4~$<$~350 ($h=5$); plus
the remaining time until death in CD4~$<$~200 ($h=6$, AIDS).
Since the duration in acute infection ($h=2$) is randomly sampled,
the remaining duration in CD4~$>$~500 ($h=3$) is adjusted accordingly.
%---------------------------------------------------------------------------------------------------
\subsubsection{HIV Mortality}\label{model.par.hiv.mort}
Mortality rates by CD4-count in the absence of ART were estimated in
multiple African studies \cite{Badri2006,Anglaret2012,Mangal2017};
based on these data, I estimated yearly HIV-attributable mortality rates $\mu_{\textsc{hiv},h}$ as:
0 during acute phase ($h=2$);
0.4\% during CD4~$>$~500 ($h=3$);
2\% during 350~$<$~CD4~$<$~500 ($h=4$);
4\% during 200~$<$~CD4~$<$~350 ($h=5$); and 
20\% during CD4~$<$~200 ($h=6$, AIDS).
%===================================================================================================
\subsection{Antiretroviral Therapy}\label{model.par.art}
Viral suppression via antiretroviral therapy (ART) influences
the probability of HIV transmission, as well as rates of HIV progression and HIV-related mortality.
The model considers individuals on ART before ($c=4$) and after ($c=5$)
achieving full viral load suppression (VLS), as defined by undetectable HIV RNA in blood samples.
Among retained patients initiating ART, time to VLS
is usually described as ``within 6 months'' \cite{Thompson2012}.
More specifically, \citet{Mujugira2016} estimate the median time to VLS as 3 months,
yielding an estimated \emph{mean} duration for $c=4$ of 4.3 months (see \sref{app.model.math.comp}).
%---------------------------------------------------------------------------------------------------
\subsubsection{Probability of HIV Transmission}\label{model.par.art.beta}
All available evidence suggests that viral suppression by ART to undetectable levels
prevents HIV transmission, \ie undetectable = untransmittable (``U=U'') \cite{Eisinger2019}.
Thus, I assumed zero HIV transmission from individuals with VLS ($c=5$).
However, HIV transmission may still occur
during the period between ART initiation to viral suppression ($c=4$) \cite{Mujugira2016}.
\citet{Donnell2010} estimate an adjusted incidence ratio of 0.08~(0.0,~0.57) for all individuals on ART.
However, in \cite{Donnell2010} and \cite{Cohen2016}, the 1 and 4 (respectively)
genetically linked infections from individuals on ART all occurred within 90 days of ART initiation,
suggesting that risk of transmission only persists before viral suppression.
Adjusting the incidence denominator (person-time)
to 90 days per individual who initiated ART in \cite{Donnell2010}
results in approximately 3.13 times higher estimated incidence ratio: 0.25 for this specific period.%
\footnote{In \cite{Donnell2010}, individuals who initiated ART contributed
  approximately 9.4 months per-person (273 persons / 349 person-years, Tables~2~and~3);
  thus the first 3 months of each individual represent
  3/9.4 = 0.319 fewer person-months of follow-up.}
Thus, I sampled relative infectiousness on ART but before viral suppression ($c=4$)
from a beta distribution with mean (95\%~CI) of 0.25~(0.01,~0.67).
%---------------------------------------------------------------------------------------------------
\subsubsection{HIV Progression \& Mortality}\label{model.par.art.hiv}
\def\hunprog{$h = 6 \rightarrow 5 \rightarrow 4 \rightarrow 3$\xspace}
Effective ART stops CD4 cell decline and results in some CD4 recovery \cite{Battegay2006,Lawn2006}.
Most CD4 recovery occurs within the first year of treatment \cite{Battegay2006}.
Due to the limited number of modelled treatment states,
I model this initial recovery to be associated with the 4.3-month pre-VLS ART state ($c=4$).
\citet{Lawn2006,Gabillard2013} estimate an increase of between 25--39 cells/mm\tsup{3} per month
during the first 3 months of treatment.
Since HIV states $h=4,5,6$ correspond to 150, 150, and 200-wide CD4 strata,
I model rates of movement along \hunprog during pre-VLS ART ($c=4$) as
0.20, 0.20, 0.17 per month, respectively.
After initial increases, CD4 recovery is modest and plateaus.
\citet{Battegay2006} report approximate increases of
22.4 cells/mm\tsup{3} per year between years 1 and 5 on ART.
Thus, I model rates of movement along \hunprog after VLS ($c=5$) as 0.15 per year.
\par
Since higher CD4 states are modelled to have lower mortality rates (see \sref{model.par.hiv.mort}),
the modelled recovery of CD4 cells via ART described above implicitly affords a mortality benefit.
However, HIV infection is associated with increased risk of death by non-AIDS causes
--- \ie unrelated to CD4 count ---
including cardiovascular disease and renal disease \cite{Phillips2008}.
\citet{Lundgren2015} estimated 61\% reduction in non-AIDS life-threatening events due to ART.
For the same CD4 strata, \citet{Gabillard2013} also report approximately 2-times higher
mortality rates within the first year of ART versus thereafter,
suggesting that VLS is associated with 50\% mortality reduction independent of CD4 increase.
Thus, I modelled an additional 50\% reduction in mortality among individuals with VLS ($c=5$),
and half this (25\%) reduction before achieving VLS ($c=4$).
% TODO: rates of diagnosis, testing, vls
\section{Parameterization}\label{model.par}
As described in \sref{intro.model.param}, model parameterization involves
specification of model parameter values, such as proportions, probabilities, rates, and ratios,
including stratified values to reflect heterogeneity,
and sampling distributions to reflect uncertainty.
Proportions and probabilities were generally modelled using
a beta approximation of the binomial distribution (BAB, see \sref{app.math.distr.bab}),
while rates and ratios were generally modelled using
a gamma, skewnormal, or inverse gaussian distribution.
\paragraph{Notation}
If $X$ is a parameter stratified by dimensions $a,b,c$,
then $X_{ab_{1}c_{23}}$ denotes the values of $X$ for
a particular but \emph{unspecified} stratum of $a$,
the \emph{specific} stratum $b = 1$,
and the \emph{aggregated} strata $c = 2,3$
(the aggregating operation is context-dependent, \eg sum for probabilities).
Additionally, the indices $sihc$ from Table~\ref{tab:model.dims} denote ``self'' strata,
whereas $s'i'h'c'$ denote ``other'' strata --- \ie individuals' partners.%
\footnote{\label{foot:code.note}%
  In the code: R uses one-based indexing, which match the notation here directly,
  while Python uses zero-based indexing, which therefore appear as $i \rightarrow i-1$ in the code.
  Also, the model code reorders states in the ART Cascade dimension for computational efficiency,
  with $c={}$1:~Undiagnosed; 2:~Diagnosed; 3:~Virally~Un-suppressed; 4:~On~ART; 5:~Virally~Suppressed.}
Finally, I re-use several dummy variables throughout the chapter:
$\rho$ for proportions, $\lambda$ for rates, $T$ for time periods, and $f$ for constants.
%===================================================================================================
\subsection{Risk Heterogeneity Among FSW}\label{model.par.fsw}
Existing HIV transmission models which include FSW
have rarely sub-stratified this population, such as to reflect
differential HIV risk or distinct typologies of sex work \cite{Blanchard2008,Scorgie2012};
yet such heterogeneities may influence transmission dynamics.
Among the studies identified in Chapter~\ref{sr},
only three sub-stratified FSW by risk-related factors:
\citet{Cremin2017} defined three levels of risk via regression analysis,
\citet{Low2015} distinguished between occasional and full-time FSW, while
\citet{Shannon2015} sub-stratified FSW by
work environment, violence exposure, and context-specific structural factors.
Seven other studies, reflecting two unique models \cite{Johnson2012,Maheu-Giroux2017},
employed age stratification of all activity groups, including FSW;
these models had several risk-related parameters which varied by age.
\par
The model structure here (Figure~\ref{fig:model.risk})
was designed to capture \emph{within}-FSW risk heterogeneity.
The objective of the following analysis was therefore to parameterize
lower \vs higher risk FSW.
I sought to define these groups based on biobehavioural and/or contextual factors
which are demonstrably associated with HIV risk,
and which can be mechanistically incorporated into a transmission model ---
\ie through the force of infection equation.
Later, the parameterization of these groups was validated through model fitting
to relative differences in HIV prevalence \sref{model.cal.targ.prev}.
\par
Many cross-sectional studies of HIV among FSW quantify
the association of risk factors with HIV serostatus
\cite{Aklilu2001,Dunkle2005,Scorgie2012,Jonas2020}.
However, serostatus reflects cumulative risk exposure,
whereas sexual risk behaviour is dynamic \cite{Watts2010,vanWees2020},
as is use of prevention resources \cite{Roberts2020}.
For example, while HIV prevalence often increases with age,
HIV incidence among women can peak before age 25 \cite{Dellar2015}.
Thus, risk factors associated with HIV serostatus are not necessarily
mechanistically related to HIV acquisition.
Indeed, FSW may reduce risk behaviours in response to seroconversion \cite{McClelland2006}.
Cohort studies that measure incidence
can help identify risk factors for HIV acquisition \cite{McKinnon2015,Nouaman2022},
but large sample sizes are often required to accurately estimate overall incidence rate,
let alone risk factors \cite{Priddy2011}.
%---------------------------------------------------------------------------------------------------
\subsubsection{FSW Survey Data}\label{model.par.fsw.data}
Three biobehavioural surveys, in
2011 \cite{Baral2014} (N = 325),
2014 \cite{EswKP2014} (N = 781), and
2021 \cite{EswIBBS2022} (N = 676)
provide HIV status and biobehavioural data on FSW in Eswatini.
The 2011 and 2021 surveys featured serologic HIV testing,
and employed respondent driven sampling (RDS, details in \cite{Yam2013}).
The 2014 survey relied on self-reported HIV status,
andd employed venue-based snowball sampling, based on the
Priorities for Local AIDS Control Efforts (PLACE) methodology,
which aims to identify areas of higher incidence \cite{Weir2005}.
More details about each study are given in \sref{intro.esw.hiv.data} and Table~\ref{tab:esw.data}.
I analyzed the individual-level data from 2011 and 2014 (data from 2021 not yet available)
to explore the potential association of biobehavioural factors with HIV risk,
so that such factors could then be used to distinguish between
lower risk \vs higher risk FSW.
% TODO: (*) add descriptive table
%---------------------------------------------------------------------------------------------------
\subsubsection{HIV Status}\label{model.par.fsw.hiv}
Only the 2011 and 2021 studies included serologic testing for HIV.
Among those tested in 2011 (N = 317, 98\%), 70\% were \hivp,
yielding RDS-adjusted prevalence estimate of 61\% (CI: 51--71\%) \cite{Baral2014}.
Among serologically \hivn, 11\% self-reported \hivp status (false positive), and
among serologically \hivp, 26\% self-reported \hivn status (false negative or undiagnosed).
Overall, self-reported HIV status underestimated HIV prevalence in 2011
by a factor of approximately 0.78 (55~vs~70\%).
Unadjusted HIV prevalence in 2021 was 58.8\%,
with 88\% (363/411) reporting previous awareness of \hivp status.
\par
In 2014, self-reported HIV prevalence was 38\% among respondents who reported (85\%).
This 38\% is surprisingly low considering that
the PLACE methodology explicitly aimed to sample venues
with higher HIV incidence \cite{Weir2005}, and 2014 \vs 2011 respondents
were older (median 27 \vs 25 years), % 2021 median: 28
had been selling sex longer (median 5 \vs 4 years), % 2021: 6
and tested more frequently (87 \vs 75\% tested at least once in the past year, % 2021: 75
82 \vs 63\% among self-reported \hivn).
Perhaps the differences are attributable to the sampling methodology.
Among respondents who self-reported \hivp status,
the 2014 survey also asked for age of HIV diagnosis (6\% missing).
Age of HIV diagnosis supports crude time-to-event analysis (next section),
which can account for confounding by age and censoring,
as compared to logistic regression on HIV status,
keeping in mind the limitations of self-reported HIV status.
%---------------------------------------------------------------------------------------------------
\subsubsection{Risk Factors for HIV}\label{model.par.fsw.fac}
Next, I explored the potential association of risk factors with HIV
via the following three models:%
\footnote{Logistic regression models were implemented using \texttt{lrm} from:
  \hreftt{cran.r-project.org/package=rms}.\\
Cox proportional hazards models were implemented using \texttt{coxaalen} from:
  \hreftt{cran.r-project.org/package=coxinterval}.}
\begin{enumerate}
  \item Logistic regression on serologic HIV status (2011 data)
  \item Logistic regression on self-reported HIV status (2014 data)
  \item Cox proportional hazards for interval-censored time to HIV infection,
    with interval from self-reported sex work debut 
    to either self-reported time of HIV diagnosis or survey date (2014 data);
    Figure~\ref{fig:fsw.tte.interval} illustrates
    the four potential censoring cases in this framework.
\end{enumerate}
An important limitation to all models is that
risk factors reported by FSW at the time of survey
are assumed to be fixed characteristics of the respondents,
rather than dynamic characteristics that vary over time.
Additionally, respondents with any missing variables for each individual model
were excluded from that model. % TODO: (%)
\begin{figure}
  \centering
  \includegraphics[scale=1]{diag.tte}
  \caption{Illustration of time-to-event analysis framework
    for cross-sectional FSW survey data}
  \label{fig:fsw.tte.interval}
  \floatfoot{
    $\bm{\times}$: HIV infection;
    SW: time of sex work debut;
    Dx: time of HIV diagnosis.}
\end{figure}
\par
Risk factors were selected based on
prior knowledge of plausible mechanistic influence on HIV incidence and/or prevalence.
The risk factors explored are summarized in Table~\ref{tab:fsw.stats},
including univariate and multivariable association under each model.
Variable selection for multivariable models
was performed using backward selection as described by \citet{Lawless1978},
using a $p \le 0.1$ (per variable) threshold for stepwise variable retention.
Estimated conditional effects of
variables retained in the multivariable logistic regression models
are illustrated in Figure~\ref{fig:fsw.lr}.
\begin{table}
  \centering
  \caption{Risk factors explored for association with \hivp status among FSW in Eswatini}
  \label{tab:fsw.stats}
  \input{model/tab.fsw.factor.stats}
\end{table}
\begin{figure}[h]
  \subcapoverlap
  \foreach \year/\var/\nvar in {2011/f/1,2011/c/2,2014/f/3,2014/c/3}{
  \begin{subfigure}{\nvar\linewidth/5+\linewidth/5}
    \includegraphics[scale=.7]{fsw.\year.lr.hiv.\var}
    \caption{\raggedright}
    \label{fig:fsw.lr.\year.\var}
  \end{subfigure}}
  \caption{Predicted conditional effects (probability)
    of variables in multivariable logistic regression models for HIV status}
  \label{fig:fsw.lr}
  \floatfoot{\fffsw{fig:fsw.lr}
    conditional probabilities shown for fixed covariates at arbitrary values.}
\end{figure}
\par
Following variable selection, each multivariable model was used to estimate
the total \hivp status odds ratio (logistic) or HIV incidence hazard ratio (Cox)
for each respondent in the respective survey ---
\ie $e^{X_i\,\beta}$ for respondent $i$ ---
representing an overall ``risk score'' under each model.
Respondents were then stratified into the top 20\% and bottom 80\% by these risk scores.
The values of each variable were compared between these two strata
using a test for the ratio of the means \cite{Tamhane2004} to support model parameterization;
these ratios are summarized in Table~\ref{tab:fsw.ratios},
and the distributions of variable values across the two strata
are illustrated in Figure~\ref{fig:fsw.f}.
\begin{table}
  \centering
  \caption{Ratios of HIV risk factor variables among higher \vs lower risk FSW in Eswatini}
  \label{tab:fsw.ratios}
  \input{model/tab.fsw.factor.ratios}
\end{table}

\section{Parameterization}\label{model.par}
As described in \sref{intro.model.param}, model parameterization involves
specification of model parameter values, such as proportions, probabilities, rates, and ratios,
including stratified values to reflect heterogeneity,
and sampling distributions to reflect uncertainty.
Proportions and probabilities were generally modelled using
a beta approximation of the binomial distribution (BAB, see \sref{app.math.distr.bab}),
while rates and ratios were generally modelled using
a gamma, skewnormal, or inverse gaussian distribution.
\paragraph{Notation}
If $X$ is a parameter stratified by dimensions $a,b,c$,
then $X_{ab_{1}c_{23}}$ denotes the values of $X$ for
a particular but \emph{unspecified} stratum of $a$,
the \emph{specific} stratum $b = 1$,
and the \emph{aggregated} strata $c = 2,3$
(the aggregating operation is context-dependent, \eg sum for probabilities).
Additionally, the indices $sihc$ from Table~\ref{tab:model.dims} denote ``self'' strata,
whereas $s'i'h'c'$ denote ``other'' strata --- \ie individuals' partners.%
\footnote{\label{foot:code.note}%
  In the code: R uses one-based indexing, which match the notation here directly,
  while Python uses zero-based indexing, which therefore appear as $i \rightarrow i-1$ in the code.
  Also, the model code reorders states in the ART Cascade dimension for computational efficiency,
  with $c={}$1:~Undiagnosed; 2:~Diagnosed; 3:~Virally~Un-suppressed; 4:~On~ART; 5:~Virally~Suppressed.}
Finally, I re-use several dummy variables throughout the chapter:
$\rho$ for proportions, $\lambda$ for rates, $T$ for time periods, and $f$ for constants.
\input{model/par.fsw}
\input{model/par.beta}
\input{model/par.hiv}
\input{model/par.popsex}
%===================================================================================================
\subsection{HIV Progression \& Mortality}\label{model.par.hiv}
%---------------------------------------------------------------------------------------------------
\subsubsection{HIV Progression}\label{model.par.hiv.dur}
The length of time spent in each HIV stage is related to
rates of progression between stages $\eta_{h}$,
rates of additional HIV-attributable mortality by stage $\mu_{\textsc{hiv},h}$,
and treatment via antiretroviral therapy (ART).
\citet{Lodi2011} estimate median times from seroconversion to
CD4 $<$ 500, $<$ 350, and $<$ 200 cells/mm\tsup{3}, while
\citet{Mangal2017} directly estimate the rates of progression between CD4 states $\eta_{h}$
in a simple compartmental model.
Based on these data, I modelled mean durations ($1/\eta_{h}$) of:%
\footnote{Assuming exponential distributions for durations in each CD4 state
  (see \sref{app.model.math.comp} for more details).}
0.142 years in acute infection ($h=2$, from \sref{model.par.beta.hiv});
3.35 years in CD4~$>$~500 ($h=3$);
3.74 years in 350~$<$~CD4~$<$~500 ($h=4$); and
5.26 years in 200~$<$~CD4~$<$~350 ($h=5$); plus
the remaining time until death in CD4~$<$~200 ($h=6$, AIDS).
Since the duration in acute infection ($h=2$) is randomly sampled,
the remaining duration in CD4~$>$~500 ($h=3$) is adjusted accordingly.
%---------------------------------------------------------------------------------------------------
\subsubsection{HIV Mortality}\label{model.par.hiv.mort}
Mortality rates by CD4-count in the absence of ART were estimated in
multiple African studies \cite{Badri2006,Anglaret2012,Mangal2017};
based on these data, I estimated yearly HIV-attributable mortality rates $\mu_{\textsc{hiv},h}$ as:
0 during acute phase ($h=2$);
0.4\% during CD4~$>$~500 ($h=3$);
2\% during 350~$<$~CD4~$<$~500 ($h=4$);
4\% during 200~$<$~CD4~$<$~350 ($h=5$); and 
20\% during CD4~$<$~200 ($h=6$, AIDS).
%===================================================================================================
\subsection{Antiretroviral Therapy}\label{model.par.art}
Viral suppression via antiretroviral therapy (ART) influences
the probability of HIV transmission, as well as rates of HIV progression and HIV-related mortality.
The model considers individuals on ART before ($c=4$) and after ($c=5$)
achieving full viral load suppression (VLS), as defined by undetectable HIV RNA in blood samples.
Among retained patients initiating ART, time to VLS
is usually described as ``within 6 months'' \cite{Thompson2012}.
More specifically, \citet{Mujugira2016} estimate the median time to VLS as 3 months,
yielding an estimated \emph{mean} duration for $c=4$ of 4.3 months (see \sref{app.model.math.comp}).
%---------------------------------------------------------------------------------------------------
\subsubsection{Probability of HIV Transmission}\label{model.par.art.beta}
All available evidence suggests that viral suppression by ART to undetectable levels
prevents HIV transmission, \ie undetectable = untransmittable (``U=U'') \cite{Eisinger2019}.
Thus, I assumed zero HIV transmission from individuals with VLS ($c=5$).
However, HIV transmission may still occur
during the period between ART initiation to viral suppression ($c=4$) \cite{Mujugira2016}.
\citet{Donnell2010} estimate an adjusted incidence ratio of 0.08~(0.0,~0.57) for all individuals on ART.
However, in \cite{Donnell2010} and \cite{Cohen2016}, the 1 and 4 (respectively)
genetically linked infections from individuals on ART all occurred within 90 days of ART initiation,
suggesting that risk of transmission only persists before viral suppression.
Adjusting the incidence denominator (person-time)
to 90 days per individual who initiated ART in \cite{Donnell2010}
results in approximately 3.13 times higher estimated incidence ratio: 0.25 for this specific period.%
\footnote{In \cite{Donnell2010}, individuals who initiated ART contributed
  approximately 9.4 months per-person (273 persons / 349 person-years, Tables~2~and~3);
  thus the first 3 months of each individual represent
  3/9.4 = 0.319 fewer person-months of follow-up.}
Thus, I sampled relative infectiousness on ART but before viral suppression ($c=4$)
from a beta distribution with mean (95\%~CI) of 0.25~(0.01,~0.67).
%---------------------------------------------------------------------------------------------------
\subsubsection{HIV Progression \& Mortality}\label{model.par.art.hiv}
\def\hunprog{$h = 6 \rightarrow 5 \rightarrow 4 \rightarrow 3$\xspace}
Effective ART stops CD4 cell decline and results in some CD4 recovery \cite{Battegay2006,Lawn2006}.
Most CD4 recovery occurs within the first year of treatment \cite{Battegay2006}.
Due to the limited number of modelled treatment states,
I model this initial recovery to be associated with the 4.3-month pre-VLS ART state ($c=4$).
\citet{Lawn2006,Gabillard2013} estimate an increase of between 25--39 cells/mm\tsup{3} per month
during the first 3 months of treatment.
Since HIV states $h=4,5,6$ correspond to 150, 150, and 200-wide CD4 strata,
I model rates of movement along \hunprog during pre-VLS ART ($c=4$) as
0.20, 0.20, 0.17 per month, respectively.
After initial increases, CD4 recovery is modest and plateaus.
\citet{Battegay2006} report approximate increases of
22.4 cells/mm\tsup{3} per year between years 1 and 5 on ART.
Thus, I model rates of movement along \hunprog after VLS ($c=5$) as 0.15 per year.
\par
Since higher CD4 states are modelled to have lower mortality rates (see \sref{model.par.hiv.mort}),
the modelled recovery of CD4 cells via ART described above implicitly affords a mortality benefit.
However, HIV infection is associated with increased risk of death by non-AIDS causes
--- \ie unrelated to CD4 count ---
including cardiovascular disease and renal disease \cite{Phillips2008}.
\citet{Lundgren2015} estimated 61\% reduction in non-AIDS life-threatening events due to ART.
For the same CD4 strata, \citet{Gabillard2013} also report approximately 2-times higher
mortality rates within the first year of ART versus thereafter,
suggesting that VLS is associated with 50\% mortality reduction independent of CD4 increase.
Thus, I modelled an additional 50\% reduction in mortality among individuals with VLS ($c=5$),
and half this (25\%) reduction before achieving VLS ($c=4$).
% TODO: rates of diagnosis, testing, vls
\section{Parameterization}\label{model.par}
As described in \sref{intro.model.param}, model parameterization involves
specification of model parameter values, such as proportions, probabilities, rates, and ratios,
including stratified values to reflect heterogeneity,
and sampling distributions to reflect uncertainty.
Proportions and probabilities were generally modelled using
a beta approximation of the binomial distribution (BAB, see \sref{app.math.distr.bab}),
while rates and ratios were generally modelled using
a gamma, skewnormal, or inverse gaussian distribution.
\paragraph{Notation}
If $X$ is a parameter stratified by dimensions $a,b,c$,
then $X_{ab_{1}c_{23}}$ denotes the values of $X$ for
a particular but \emph{unspecified} stratum of $a$,
the \emph{specific} stratum $b = 1$,
and the \emph{aggregated} strata $c = 2,3$
(the aggregating operation is context-dependent, \eg sum for probabilities).
Additionally, the indices $sihc$ from Table~\ref{tab:model.dims} denote ``self'' strata,
whereas $s'i'h'c'$ denote ``other'' strata --- \ie individuals' partners.%
\footnote{\label{foot:code.note}%
  In the code: R uses one-based indexing, which match the notation here directly,
  while Python uses zero-based indexing, which therefore appear as $i \rightarrow i-1$ in the code.
  Also, the model code reorders states in the ART Cascade dimension for computational efficiency,
  with $c={}$1:~Undiagnosed; 2:~Diagnosed; 3:~Virally~Un-suppressed; 4:~On~ART; 5:~Virally~Suppressed.}
Finally, I re-use several dummy variables throughout the chapter:
$\rho$ for proportions, $\lambda$ for rates, $T$ for time periods, and $f$ for constants.
\input{model/par.fsw}
\input{model/par.beta}
\input{model/par.hiv}
\input{model/par.popsex}
%===================================================================================================
\subsection{HIV Progression \& Mortality}\label{model.par.hiv}
%---------------------------------------------------------------------------------------------------
\subsubsection{HIV Progression}\label{model.par.hiv.dur}
The length of time spent in each HIV stage is related to
rates of progression between stages $\eta_{h}$,
rates of additional HIV-attributable mortality by stage $\mu_{\textsc{hiv},h}$,
and treatment via antiretroviral therapy (ART).
\citet{Lodi2011} estimate median times from seroconversion to
CD4 $<$ 500, $<$ 350, and $<$ 200 cells/mm\tsup{3}, while
\citet{Mangal2017} directly estimate the rates of progression between CD4 states $\eta_{h}$
in a simple compartmental model.
Based on these data, I modelled mean durations ($1/\eta_{h}$) of:%
\footnote{Assuming exponential distributions for durations in each CD4 state
  (see \sref{app.model.math.comp} for more details).}
0.142 years in acute infection ($h=2$, from \sref{model.par.beta.hiv});
3.35 years in CD4~$>$~500 ($h=3$);
3.74 years in 350~$<$~CD4~$<$~500 ($h=4$); and
5.26 years in 200~$<$~CD4~$<$~350 ($h=5$); plus
the remaining time until death in CD4~$<$~200 ($h=6$, AIDS).
Since the duration in acute infection ($h=2$) is randomly sampled,
the remaining duration in CD4~$>$~500 ($h=3$) is adjusted accordingly.
%---------------------------------------------------------------------------------------------------
\subsubsection{HIV Mortality}\label{model.par.hiv.mort}
Mortality rates by CD4-count in the absence of ART were estimated in
multiple African studies \cite{Badri2006,Anglaret2012,Mangal2017};
based on these data, I estimated yearly HIV-attributable mortality rates $\mu_{\textsc{hiv},h}$ as:
0 during acute phase ($h=2$);
0.4\% during CD4~$>$~500 ($h=3$);
2\% during 350~$<$~CD4~$<$~500 ($h=4$);
4\% during 200~$<$~CD4~$<$~350 ($h=5$); and 
20\% during CD4~$<$~200 ($h=6$, AIDS).
%===================================================================================================
\subsection{Antiretroviral Therapy}\label{model.par.art}
Viral suppression via antiretroviral therapy (ART) influences
the probability of HIV transmission, as well as rates of HIV progression and HIV-related mortality.
The model considers individuals on ART before ($c=4$) and after ($c=5$)
achieving full viral load suppression (VLS), as defined by undetectable HIV RNA in blood samples.
Among retained patients initiating ART, time to VLS
is usually described as ``within 6 months'' \cite{Thompson2012}.
More specifically, \citet{Mujugira2016} estimate the median time to VLS as 3 months,
yielding an estimated \emph{mean} duration for $c=4$ of 4.3 months (see \sref{app.model.math.comp}).
%---------------------------------------------------------------------------------------------------
\subsubsection{Probability of HIV Transmission}\label{model.par.art.beta}
All available evidence suggests that viral suppression by ART to undetectable levels
prevents HIV transmission, \ie undetectable = untransmittable (``U=U'') \cite{Eisinger2019}.
Thus, I assumed zero HIV transmission from individuals with VLS ($c=5$).
However, HIV transmission may still occur
during the period between ART initiation to viral suppression ($c=4$) \cite{Mujugira2016}.
\citet{Donnell2010} estimate an adjusted incidence ratio of 0.08~(0.0,~0.57) for all individuals on ART.
However, in \cite{Donnell2010} and \cite{Cohen2016}, the 1 and 4 (respectively)
genetically linked infections from individuals on ART all occurred within 90 days of ART initiation,
suggesting that risk of transmission only persists before viral suppression.
Adjusting the incidence denominator (person-time)
to 90 days per individual who initiated ART in \cite{Donnell2010}
results in approximately 3.13 times higher estimated incidence ratio: 0.25 for this specific period.%
\footnote{In \cite{Donnell2010}, individuals who initiated ART contributed
  approximately 9.4 months per-person (273 persons / 349 person-years, Tables~2~and~3);
  thus the first 3 months of each individual represent
  3/9.4 = 0.319 fewer person-months of follow-up.}
Thus, I sampled relative infectiousness on ART but before viral suppression ($c=4$)
from a beta distribution with mean (95\%~CI) of 0.25~(0.01,~0.67).
%---------------------------------------------------------------------------------------------------
\subsubsection{HIV Progression \& Mortality}\label{model.par.art.hiv}
\def\hunprog{$h = 6 \rightarrow 5 \rightarrow 4 \rightarrow 3$\xspace}
Effective ART stops CD4 cell decline and results in some CD4 recovery \cite{Battegay2006,Lawn2006}.
Most CD4 recovery occurs within the first year of treatment \cite{Battegay2006}.
Due to the limited number of modelled treatment states,
I model this initial recovery to be associated with the 4.3-month pre-VLS ART state ($c=4$).
\citet{Lawn2006,Gabillard2013} estimate an increase of between 25--39 cells/mm\tsup{3} per month
during the first 3 months of treatment.
Since HIV states $h=4,5,6$ correspond to 150, 150, and 200-wide CD4 strata,
I model rates of movement along \hunprog during pre-VLS ART ($c=4$) as
0.20, 0.20, 0.17 per month, respectively.
After initial increases, CD4 recovery is modest and plateaus.
\citet{Battegay2006} report approximate increases of
22.4 cells/mm\tsup{3} per year between years 1 and 5 on ART.
Thus, I model rates of movement along \hunprog after VLS ($c=5$) as 0.15 per year.
\par
Since higher CD4 states are modelled to have lower mortality rates (see \sref{model.par.hiv.mort}),
the modelled recovery of CD4 cells via ART described above implicitly affords a mortality benefit.
However, HIV infection is associated with increased risk of death by non-AIDS causes
--- \ie unrelated to CD4 count ---
including cardiovascular disease and renal disease \cite{Phillips2008}.
\citet{Lundgren2015} estimated 61\% reduction in non-AIDS life-threatening events due to ART.
For the same CD4 strata, \citet{Gabillard2013} also report approximately 2-times higher
mortality rates within the first year of ART versus thereafter,
suggesting that VLS is associated with 50\% mortality reduction independent of CD4 increase.
Thus, I modelled an additional 50\% reduction in mortality among individuals with VLS ($c=5$),
and half this (25\%) reduction before achieving VLS ($c=4$).
% TODO: rates of diagnosis, testing, vls
\section{Parameterization}\label{model.par}
As described in \sref{intro.model.param}, model parameterization involves
specification of model parameter values, such as proportions, probabilities, rates, and ratios,
including stratified values to reflect heterogeneity,
and sampling distributions to reflect uncertainty.
Proportions and probabilities were generally modelled using
a beta approximation of the binomial distribution (BAB, see \sref{app.math.distr.bab}),
while rates and ratios were generally modelled using
a gamma, skewnormal, or inverse gaussian distribution.
\paragraph{Notation}
If $X$ is a parameter stratified by dimensions $a,b,c$,
then $X_{ab_{1}c_{23}}$ denotes the values of $X$ for
a particular but \emph{unspecified} stratum of $a$,
the \emph{specific} stratum $b = 1$,
and the \emph{aggregated} strata $c = 2,3$
(the aggregating operation is context-dependent, \eg sum for probabilities).
Additionally, the indices $sihc$ from Table~\ref{tab:model.dims} denote ``self'' strata,
whereas $s'i'h'c'$ denote ``other'' strata --- \ie individuals' partners.%
\footnote{\label{foot:code.note}%
  In the code: R uses one-based indexing, which match the notation here directly,
  while Python uses zero-based indexing, which therefore appear as $i \rightarrow i-1$ in the code.
  Also, the model code reorders states in the ART Cascade dimension for computational efficiency,
  with $c={}$1:~Undiagnosed; 2:~Diagnosed; 3:~Virally~Un-suppressed; 4:~On~ART; 5:~Virally~Suppressed.}
Finally, I re-use several dummy variables throughout the chapter:
$\rho$ for proportions, $\lambda$ for rates, $T$ for time periods, and $f$ for constants.
%===================================================================================================
\subsection{Risk Heterogeneity Among FSW}\label{model.par.fsw}
Existing HIV transmission models which include FSW
have rarely sub-stratified this population, such as to reflect
differential HIV risk or distinct typologies of sex work \cite{Blanchard2008,Scorgie2012};
yet such heterogeneities may influence transmission dynamics.
Among the studies identified in Chapter~\ref{sr},
only three sub-stratified FSW by risk-related factors:
\citet{Cremin2017} defined three levels of risk via regression analysis,
\citet{Low2015} distinguished between occasional and full-time FSW, while
\citet{Shannon2015} sub-stratified FSW by
work environment, violence exposure, and context-specific structural factors.
Seven other studies, reflecting two unique models \cite{Johnson2012,Maheu-Giroux2017},
employed age stratification of all activity groups, including FSW;
these models had several risk-related parameters which varied by age.
\par
The model structure here (Figure~\ref{fig:model.risk})
was designed to capture \emph{within}-FSW risk heterogeneity.
The objective of the following analysis was therefore to parameterize
lower \vs higher risk FSW.
I sought to define these groups based on biobehavioural and/or contextual factors
which are demonstrably associated with HIV risk,
and which can be mechanistically incorporated into a transmission model ---
\ie through the force of infection equation.
Later, the parameterization of these groups was validated through model fitting
to relative differences in HIV prevalence \sref{model.cal.targ.prev}.
\par
Many cross-sectional studies of HIV among FSW quantify
the association of risk factors with HIV serostatus
\cite{Aklilu2001,Dunkle2005,Scorgie2012,Jonas2020}.
However, serostatus reflects cumulative risk exposure,
whereas sexual risk behaviour is dynamic \cite{Watts2010,vanWees2020},
as is use of prevention resources \cite{Roberts2020}.
For example, while HIV prevalence often increases with age,
HIV incidence among women can peak before age 25 \cite{Dellar2015}.
Thus, risk factors associated with HIV serostatus are not necessarily
mechanistically related to HIV acquisition.
Indeed, FSW may reduce risk behaviours in response to seroconversion \cite{McClelland2006}.
Cohort studies that measure incidence
can help identify risk factors for HIV acquisition \cite{McKinnon2015,Nouaman2022},
but large sample sizes are often required to accurately estimate overall incidence rate,
let alone risk factors \cite{Priddy2011}.
%---------------------------------------------------------------------------------------------------
\subsubsection{FSW Survey Data}\label{model.par.fsw.data}
Three biobehavioural surveys, in
2011 \cite{Baral2014} (N = 325),
2014 \cite{EswKP2014} (N = 781), and
2021 \cite{EswIBBS2022} (N = 676)
provide HIV status and biobehavioural data on FSW in Eswatini.
The 2011 and 2021 surveys featured serologic HIV testing,
and employed respondent driven sampling (RDS, details in \cite{Yam2013}).
The 2014 survey relied on self-reported HIV status,
andd employed venue-based snowball sampling, based on the
Priorities for Local AIDS Control Efforts (PLACE) methodology,
which aims to identify areas of higher incidence \cite{Weir2005}.
More details about each study are given in \sref{intro.esw.hiv.data} and Table~\ref{tab:esw.data}.
I analyzed the individual-level data from 2011 and 2014 (data from 2021 not yet available)
to explore the potential association of biobehavioural factors with HIV risk,
so that such factors could then be used to distinguish between
lower risk \vs higher risk FSW.
% TODO: (*) add descriptive table
%---------------------------------------------------------------------------------------------------
\subsubsection{HIV Status}\label{model.par.fsw.hiv}
Only the 2011 and 2021 studies included serologic testing for HIV.
Among those tested in 2011 (N = 317, 98\%), 70\% were \hivp,
yielding RDS-adjusted prevalence estimate of 61\% (CI: 51--71\%) \cite{Baral2014}.
Among serologically \hivn, 11\% self-reported \hivp status (false positive), and
among serologically \hivp, 26\% self-reported \hivn status (false negative or undiagnosed).
Overall, self-reported HIV status underestimated HIV prevalence in 2011
by a factor of approximately 0.78 (55~vs~70\%).
Unadjusted HIV prevalence in 2021 was 58.8\%,
with 88\% (363/411) reporting previous awareness of \hivp status.
\par
In 2014, self-reported HIV prevalence was 38\% among respondents who reported (85\%).
This 38\% is surprisingly low considering that
the PLACE methodology explicitly aimed to sample venues
with higher HIV incidence \cite{Weir2005}, and 2014 \vs 2011 respondents
were older (median 27 \vs 25 years), % 2021 median: 28
had been selling sex longer (median 5 \vs 4 years), % 2021: 6
and tested more frequently (87 \vs 75\% tested at least once in the past year, % 2021: 75
82 \vs 63\% among self-reported \hivn).
Perhaps the differences are attributable to the sampling methodology.
Among respondents who self-reported \hivp status,
the 2014 survey also asked for age of HIV diagnosis (6\% missing).
Age of HIV diagnosis supports crude time-to-event analysis (next section),
which can account for confounding by age and censoring,
as compared to logistic regression on HIV status,
keeping in mind the limitations of self-reported HIV status.
%---------------------------------------------------------------------------------------------------
\subsubsection{Risk Factors for HIV}\label{model.par.fsw.fac}
Next, I explored the potential association of risk factors with HIV
via the following three models:%
\footnote{Logistic regression models were implemented using \texttt{lrm} from:
  \hreftt{cran.r-project.org/package=rms}.\\
Cox proportional hazards models were implemented using \texttt{coxaalen} from:
  \hreftt{cran.r-project.org/package=coxinterval}.}
\begin{enumerate}
  \item Logistic regression on serologic HIV status (2011 data)
  \item Logistic regression on self-reported HIV status (2014 data)
  \item Cox proportional hazards for interval-censored time to HIV infection,
    with interval from self-reported sex work debut 
    to either self-reported time of HIV diagnosis or survey date (2014 data);
    Figure~\ref{fig:fsw.tte.interval} illustrates
    the four potential censoring cases in this framework.
\end{enumerate}
An important limitation to all models is that
risk factors reported by FSW at the time of survey
are assumed to be fixed characteristics of the respondents,
rather than dynamic characteristics that vary over time.
Additionally, respondents with any missing variables for each individual model
were excluded from that model. % TODO: (%)
\begin{figure}
  \centering
  \includegraphics[scale=1]{diag.tte}
  \caption{Illustration of time-to-event analysis framework
    for cross-sectional FSW survey data}
  \label{fig:fsw.tte.interval}
  \floatfoot{
    $\bm{\times}$: HIV infection;
    SW: time of sex work debut;
    Dx: time of HIV diagnosis.}
\end{figure}
\par
Risk factors were selected based on
prior knowledge of plausible mechanistic influence on HIV incidence and/or prevalence.
The risk factors explored are summarized in Table~\ref{tab:fsw.stats},
including univariate and multivariable association under each model.
Variable selection for multivariable models
was performed using backward selection as described by \citet{Lawless1978},
using a $p \le 0.1$ (per variable) threshold for stepwise variable retention.
Estimated conditional effects of
variables retained in the multivariable logistic regression models
are illustrated in Figure~\ref{fig:fsw.lr}.
\begin{table}
  \centering
  \caption{Risk factors explored for association with \hivp status among FSW in Eswatini}
  \label{tab:fsw.stats}
  \centerline{%
\small%
\begin{tabular}{lcccccccccccc}
  \toprule
  & \multicolumn{4}{c}{2011 LR}
  & \multicolumn{4}{c}{2014 LR}
  & \multicolumn{4}{c}{2014 CPH} \\
  \cmidrule(rl){2-5}\cmidrule(rl){6-9}\cmidrule(rl){10-13}
  & \multicolumn{2}{c}{Univar} & \multicolumn{2}{c}{Multivar}
  & \multicolumn{2}{c}{Univar} & \multicolumn{2}{c}{Multivar}
  & \multicolumn{2}{c}{Univar} & \multicolumn{2}{c}{Multivar} \\
  \cmidrule(rl){2-3}\cmidrule(rl){4-5}\cmidrule(rl){6-7}\cmidrule(rl){8-9}\cmidrule(rl){10-11}\cmidrule(rl){12-13}
  Factor                          &  OR  &   p   &  OR  &   p    &  OR  &   p    &  OR  &   p    &  HR  &   p    &  HR  &   p    \\
  \midrule                        % 2011 LR uni  % 2011 LR multi % 2014 LR uni   % 2014 LR multi % 2014 CPH uni  % 2014 CPH multi
  Age\tn{a}                       & 1.11 & \vsig & ---  &  ---   & 1.14 & \vsig  & 1.15 & \vsig  & 1.09 & \vsig  & 1.09 & \vsig  \\
  Years selling sex\tn{a}         & 1.13 & \vsig & 1.13 & \vsig  & 1.12 & \vsig  & ---  &  ---   & 1.08 & \vsig  & ---  &  ---   \\
  Monthly sex work income\tn{b}   & 0.98 & 0.155 & ---  &  ---   & 0.98 & 0.097  & 0.97 & 0.084  & 0.98 & 0.019\s& 0.97 & 0.001\s\\[1ex]
  Non-paying partners\tn{c}       & 0.88 & 0.307 & ---  &  ---   & 1.07 & 0.233  & ---  &  ---   & 1.05 & 0.312  & ---  &  ---   \\
  Monthly new clients\tn{c}       & 1.01 & 0.412 & ---  &  ---   & 1.05 & \vsig  & 1.07 & \vsig  & 1.04 & \vsig  & 1.04 & \vsig  \\
  Monthly regular clients\tn{c}   & 1.01 & 0.351 & ---  &  ---   & 1.03 & 0.002  & ---  &  ---   & 1.02 & \vsig  & 1.02 & 0.034\s\\[1ex]
  Non-paying condom use\tn{d}     & 0.90 & 0.703 & ---  &  ---   & 0.90 & 0.673  & ---  &  ---   & 0.92 & 0.677  & ---  &  ---   \\
  New client condom use\tn{d}     & 0.60 & 0.100 & ---  &  ---   & 0.48 & 0.006\s& 1.25 & 0.599  & 0.56 & 0.004\s& ---  &  ---   \\
  Regular client condom use\tn{d} & 0.58 & 0.110 & ---  &  ---   & 0.39 & \vsig  & 0.35 & 0.004\s& 0.49 & \vsig  & 0.50 & \vsig  \\[1ex]
  Any anal sex past month         & 0.97 & 0.896 & ---  &  ---   & 1.89 & 0.015\s& ---  &  ---   & 1.57 & 0.015\s& 1.27 & 0.260  \\
  Any STI symptoms past year      & 2.29 & \vsig & 2.41 & \vsig  & 2.75 & \vsig  & 2.80 & \vsig  & 2.17 & \vsig  & 2.05 & \vsig  \\
  \bottomrule
\end{tabular}}
% TODO: HIV status?
\floatfoot{\raggedright
  \tnt[a]{OR per year};
  \tnt[b]{OR per Swazi lilangeni per month};
  \tnt[c]{OR per partner};
  \tnt[d]{2011: always vs not always, 2014: at last sex}.
  --- indicates variable was not selected in the multivariate model.
  LR: logistic regression on HIV$+/-$ status;
  CPH: Cox proportional hazards on time to self-reported HIV seroconversion.
  OR: odds ratio; HR: hazard ratio; p: p-value.
  2011 data based on serologic HIV test;
  2014 data based on self-reported HIV status, age of sex work debut, and age of HIV diagnosis.
}
\end{table}
\begin{figure}[h]
  \subcapoverlap
  \foreach \year/\var/\nvar in {2011/f/1,2011/c/2,2014/f/3,2014/c/3}{
  \begin{subfigure}{\nvar\linewidth/5+\linewidth/5}
    \includegraphics[scale=.7]{fsw.\year.lr.hiv.\var}
    \caption{\raggedright}
    \label{fig:fsw.lr.\year.\var}
  \end{subfigure}}
  \caption{Predicted conditional effects (probability)
    of variables in multivariable logistic regression models for HIV status}
  \label{fig:fsw.lr}
  \floatfoot{\fffsw{fig:fsw.lr}
    conditional probabilities shown for fixed covariates at arbitrary values.}
\end{figure}
\par
Following variable selection, each multivariable model was used to estimate
the total \hivp status odds ratio (logistic) or HIV incidence hazard ratio (Cox)
for each respondent in the respective survey ---
\ie $e^{X_i\,\beta}$ for respondent $i$ ---
representing an overall ``risk score'' under each model.
Respondents were then stratified into the top 20\% and bottom 80\% by these risk scores.
The values of each variable were compared between these two strata
using a test for the ratio of the means \cite{Tamhane2004} to support model parameterization;
these ratios are summarized in Table~\ref{tab:fsw.ratios},
and the distributions of variable values across the two strata
are illustrated in Figure~\ref{fig:fsw.f}.
\begin{table}
  \centering
  \caption{Ratios of HIV risk factor variables among higher \vs lower risk FSW in Eswatini}
  \label{tab:fsw.ratios}
  \centerline{\footnotesize%
\begin{tabular}{lcccccc}
  \toprule
  & \multicolumn{2}{c}{2011 LR}
  & \multicolumn{2}{c}{2014 LR}
  & \multicolumn{2}{c}{2014 CPH} \\
  \cmidrule(rl){2-3}\cmidrule(rl){4-5}\cmidrule(rl){6-7}
  Factor                            &  High / Low   &   Ratio (95\% CI)   &  High / Low   &   Ratio (95\% CI)   &   High / Low   &   Ratio (95\% CI)   \\
  \midrule
  Age                               & 31.8  / 24.7  & 1.29 (1.22, 1.36)\s & 32.6  / 26.2  & 1.24 (1.20, 1.28)\s &  33.5  / 26.6  & 1.26 (1.21, 1.31)\s \\
  Years selling sex                 & 11.3  /  4.03 & 2.81 (2.41, 3.25)\s & 10.0  /  5.47 & 1.83 (1.64, 2.03)\s &  10.2  /  5.83 & 1.75 (1.54, 1.98)\s \\
  Monthly sex work income\tn{a}     & 15.1  / 15.2  & 1.00 (0.86, 1.15)   &  6.77 /  7.06 & 0.96 (0.82, 1.11)   &   6.32 /  7.28 & 0.87 (0.73, 1.02)   \\[1ex]
  Non-paying partners               &  1.42 /  1.43 & 0.99 (0.81, 1.19)   &  1.56 /  1.11 & 1.40 (1.11, 1.72)\s &   1.53 /  1.19 & 1.29 (0.98, 1.62)   \\
  Monthly new clients               &  5.50 /  6.98 & 0.79 (0.49, 1.15)   &  8.39 /  4.15 & 2.02 (1.63, 2.44)\s &   8.36 /  4.41 & 1.90 (1.43, 2.39)\s \\
  Monthly regular clients           &  9.35 /  9.05 & 1.03 (0.69, 1.42)   & 11.1  /  8.25 & 1.35 (1.13, 1.57)\s &  12.4  /  8.61 & 1.44 (1.18, 1.71)\s \\[1ex]
  Non-paying condom use\tn{bc}      &  0.26 /  0.35 & 0.73 (0.40, 1.11)   &  0.77 /  0.81 & 0.95 (0.84, 1.06)   &   0.76 /  0.81 & 0.95 (0.81, 1.08)   \\
  New client condom use\tn{bc}      &  0.68 /  0.76 & 0.89 (0.73, 1.06)   &  0.79 /  0.91 & 0.86 (0.79, 0.94)\s &   0.74 /  0.94 & 0.79 (0.69, 0.88)\s \\
  Regular client condom use\tn{bc}  &  0.38 /  0.46 & 0.83 (0.45, 1.28)   &  0.67 /  0.91 & 0.74 (0.65, 0.82)\s &   0.60 /  0.92 & 0.65 (0.55, 0.75)\s \\[1ex]
  Any anal sex past month           &  0.59 /  0.41 & 1.41 (1.06, 1.84)\s &  0.17 /  0.07 & 2.43 (1.47, 3.85)\s &   0.23 /  0.07 & 3.24 (1.95, 5.34)\s \\
  Any STI symptoms past year\tn{c}  &  0.79 /  0.43 & 1.86 (1.54, 2.25)\s &  0.59 /  0.15 & 3.94 (3.15, 5.03)\s &   0.61 /  0.17 & 3.67 (2.87, 4.79)\s \\[1ex]
  HIV prevalence\tn{d}              &  0.94 /  0.64 & 1.46 (1.30, 1.63)\s &  0.66 /  0.29 & 2.29 (1.92, 2.75)\s &   0.71 /  0.31 & 2.32 (1.94, 2.80)\s \\
  \bottomrule
\end{tabular}}
% TODO: HIV status?
\floatfoot{\raggedright
  High / Low: mean variable value among higher / lower risk groups, as defined by
  the top 20\% / bottom 80\% in multivariable model-predicted risk score:
  odds ratio from logistic regression (LR);
  hazards ratio from Cox proportional hazards (CPH).
  \tnt[a]{Swati lilangeni per month};
  \tnt[b]{2011: always \vs not always, 2014: did use condom at last sex};
  \tnt[c]{proportion of respondents};
  \tnt[d]{2011: serologic HIV status; 2014: self-reported HIV status};
  \tnt[*]{statistically significant, $p < 0.05$}.
}
\end{table}

\section{Parameterization}\label{model.par}
As described in \sref{intro.model.param}, model parameterization involves
specification of model parameter values, such as proportions, probabilities, rates, and ratios,
including stratified values to reflect heterogeneity,
and sampling distributions to reflect uncertainty.
Proportions and probabilities were generally modelled using
a beta approximation of the binomial distribution (BAB, see \sref{app.math.distr.bab}),
while rates and ratios were generally modelled using
a gamma, skewnormal, or inverse gaussian distribution.
\paragraph{Notation}
If $X$ is a parameter stratified by dimensions $a,b,c$,
then $X_{ab_{1}c_{23}}$ denotes the values of $X$ for
a particular but \emph{unspecified} stratum of $a$,
the \emph{specific} stratum $b = 1$,
and the \emph{aggregated} strata $c = 2,3$
(the aggregating operation is context-dependent, \eg sum for probabilities).
Additionally, the indices $sihc$ from Table~\ref{tab:model.dims} denote ``self'' strata,
whereas $s'i'h'c'$ denote ``other'' strata --- \ie individuals' partners.%
\footnote{\label{foot:code.note}%
  In the code: R uses one-based indexing, which match the notation here directly,
  while Python uses zero-based indexing, which therefore appear as $i \rightarrow i-1$ in the code.
  Also, the model code reorders states in the ART Cascade dimension for computational efficiency,
  with $c={}$1:~Undiagnosed; 2:~Diagnosed; 3:~Virally~Un-suppressed; 4:~On~ART; 5:~Virally~Suppressed.}
Finally, I re-use several dummy variables throughout the chapter:
$\rho$ for proportions, $\lambda$ for rates, $T$ for time periods, and $f$ for constants.
%===================================================================================================
\subsection{Risk Heterogeneity Among FSW}\label{model.par.fsw}
Existing HIV transmission models which include FSW
have rarely sub-stratified this population, such as to reflect
differential HIV risk or distinct typologies of sex work \cite{Blanchard2008,Scorgie2012};
yet such heterogeneities may influence transmission dynamics.
Among the studies identified in Chapter~\ref{sr},
only three sub-stratified FSW by risk-related factors:
\citet{Cremin2017} defined three levels of risk via regression analysis,
\citet{Low2015} distinguished between occasional and full-time FSW, while
\citet{Shannon2015} sub-stratified FSW by
work environment, violence exposure, and context-specific structural factors.
Seven other studies, reflecting two unique models \cite{Johnson2012,Maheu-Giroux2017},
employed age stratification of all activity groups, including FSW;
these models had several risk-related parameters which varied by age.
\par
The model structure here (Figure~\ref{fig:model.risk})
was designed to capture \emph{within}-FSW risk heterogeneity.
The objective of the following analysis was therefore to parameterize
lower \vs higher risk FSW.
I sought to define these groups based on biobehavioural and/or contextual factors
which are demonstrably associated with HIV risk,
and which can be mechanistically incorporated into a transmission model ---
\ie through the force of infection equation.
Later, the parameterization of these groups was validated through model fitting
to relative differences in HIV prevalence \sref{model.cal.targ.prev}.
\par
Many cross-sectional studies of HIV among FSW quantify
the association of risk factors with HIV serostatus
\cite{Aklilu2001,Dunkle2005,Scorgie2012,Jonas2020}.
However, serostatus reflects cumulative risk exposure,
whereas sexual risk behaviour is dynamic \cite{Watts2010,vanWees2020},
as is use of prevention resources \cite{Roberts2020}.
For example, while HIV prevalence often increases with age,
HIV incidence among women can peak before age 25 \cite{Dellar2015}.
Thus, risk factors associated with HIV serostatus are not necessarily
mechanistically related to HIV acquisition.
Indeed, FSW may reduce risk behaviours in response to seroconversion \cite{McClelland2006}.
Cohort studies that measure incidence
can help identify risk factors for HIV acquisition \cite{McKinnon2015,Nouaman2022},
but large sample sizes are often required to accurately estimate overall incidence rate,
let alone risk factors \cite{Priddy2011}.
%---------------------------------------------------------------------------------------------------
\subsubsection{FSW Survey Data}\label{model.par.fsw.data}
Three biobehavioural surveys, in
2011 \cite{Baral2014} (N = 325),
2014 \cite{EswKP2014} (N = 781), and
2021 \cite{EswIBBS2022} (N = 676)
provide HIV status and biobehavioural data on FSW in Eswatini.
The 2011 and 2021 surveys featured serologic HIV testing,
and employed respondent driven sampling (RDS, details in \cite{Yam2013}).
The 2014 survey relied on self-reported HIV status,
andd employed venue-based snowball sampling, based on the
Priorities for Local AIDS Control Efforts (PLACE) methodology,
which aims to identify areas of higher incidence \cite{Weir2005}.
More details about each study are given in \sref{intro.esw.hiv.data} and Table~\ref{tab:esw.data}.
I analyzed the individual-level data from 2011 and 2014 (data from 2021 not yet available)
to explore the potential association of biobehavioural factors with HIV risk,
so that such factors could then be used to distinguish between
lower risk \vs higher risk FSW.
% TODO: (*) add descriptive table
%---------------------------------------------------------------------------------------------------
\subsubsection{HIV Status}\label{model.par.fsw.hiv}
Only the 2011 and 2021 studies included serologic testing for HIV.
Among those tested in 2011 (N = 317, 98\%), 70\% were \hivp,
yielding RDS-adjusted prevalence estimate of 61\% (CI: 51--71\%) \cite{Baral2014}.
Among serologically \hivn, 11\% self-reported \hivp status (false positive), and
among serologically \hivp, 26\% self-reported \hivn status (false negative or undiagnosed).
Overall, self-reported HIV status underestimated HIV prevalence in 2011
by a factor of approximately 0.78 (55~vs~70\%).
Unadjusted HIV prevalence in 2021 was 58.8\%,
with 88\% (363/411) reporting previous awareness of \hivp status.
\par
In 2014, self-reported HIV prevalence was 38\% among respondents who reported (85\%).
This 38\% is surprisingly low considering that
the PLACE methodology explicitly aimed to sample venues
with higher HIV incidence \cite{Weir2005}, and 2014 \vs 2011 respondents
were older (median 27 \vs 25 years), % 2021 median: 28
had been selling sex longer (median 5 \vs 4 years), % 2021: 6
and tested more frequently (87 \vs 75\% tested at least once in the past year, % 2021: 75
82 \vs 63\% among self-reported \hivn).
Perhaps the differences are attributable to the sampling methodology.
Among respondents who self-reported \hivp status,
the 2014 survey also asked for age of HIV diagnosis (6\% missing).
Age of HIV diagnosis supports crude time-to-event analysis (next section),
which can account for confounding by age and censoring,
as compared to logistic regression on HIV status,
keeping in mind the limitations of self-reported HIV status.
%---------------------------------------------------------------------------------------------------
\subsubsection{Risk Factors for HIV}\label{model.par.fsw.fac}
Next, I explored the potential association of risk factors with HIV
via the following three models:%
\footnote{Logistic regression models were implemented using \texttt{lrm} from:
  \hreftt{cran.r-project.org/package=rms}.\\
Cox proportional hazards models were implemented using \texttt{coxaalen} from:
  \hreftt{cran.r-project.org/package=coxinterval}.}
\begin{enumerate}
  \item Logistic regression on serologic HIV status (2011 data)
  \item Logistic regression on self-reported HIV status (2014 data)
  \item Cox proportional hazards for interval-censored time to HIV infection,
    with interval from self-reported sex work debut 
    to either self-reported time of HIV diagnosis or survey date (2014 data);
    Figure~\ref{fig:fsw.tte.interval} illustrates
    the four potential censoring cases in this framework.
\end{enumerate}
An important limitation to all models is that
risk factors reported by FSW at the time of survey
are assumed to be fixed characteristics of the respondents,
rather than dynamic characteristics that vary over time.
Additionally, respondents with any missing variables for each individual model
were excluded from that model. % TODO: (%)
\begin{figure}
  \centering
  \includegraphics[scale=1]{diag.tte}
  \caption{Illustration of time-to-event analysis framework
    for cross-sectional FSW survey data}
  \label{fig:fsw.tte.interval}
  \floatfoot{
    $\bm{\times}$: HIV infection;
    SW: time of sex work debut;
    Dx: time of HIV diagnosis.}
\end{figure}
\par
Risk factors were selected based on
prior knowledge of plausible mechanistic influence on HIV incidence and/or prevalence.
The risk factors explored are summarized in Table~\ref{tab:fsw.stats},
including univariate and multivariable association under each model.
Variable selection for multivariable models
was performed using backward selection as described by \citet{Lawless1978},
using a $p \le 0.1$ (per variable) threshold for stepwise variable retention.
Estimated conditional effects of
variables retained in the multivariable logistic regression models
are illustrated in Figure~\ref{fig:fsw.lr}.
\begin{table}
  \centering
  \caption{Risk factors explored for association with \hivp status among FSW in Eswatini}
  \label{tab:fsw.stats}
  \input{model/tab.fsw.factor.stats}
\end{table}
\begin{figure}[h]
  \subcapoverlap
  \foreach \year/\var/\nvar in {2011/f/1,2011/c/2,2014/f/3,2014/c/3}{
  \begin{subfigure}{\nvar\linewidth/5+\linewidth/5}
    \includegraphics[scale=.7]{fsw.\year.lr.hiv.\var}
    \caption{\raggedright}
    \label{fig:fsw.lr.\year.\var}
  \end{subfigure}}
  \caption{Predicted conditional effects (probability)
    of variables in multivariable logistic regression models for HIV status}
  \label{fig:fsw.lr}
  \floatfoot{\fffsw{fig:fsw.lr}
    conditional probabilities shown for fixed covariates at arbitrary values.}
\end{figure}
\par
Following variable selection, each multivariable model was used to estimate
the total \hivp status odds ratio (logistic) or HIV incidence hazard ratio (Cox)
for each respondent in the respective survey ---
\ie $e^{X_i\,\beta}$ for respondent $i$ ---
representing an overall ``risk score'' under each model.
Respondents were then stratified into the top 20\% and bottom 80\% by these risk scores.
The values of each variable were compared between these two strata
using a test for the ratio of the means \cite{Tamhane2004} to support model parameterization;
these ratios are summarized in Table~\ref{tab:fsw.ratios},
and the distributions of variable values across the two strata
are illustrated in Figure~\ref{fig:fsw.f}.
\begin{table}
  \centering
  \caption{Ratios of HIV risk factor variables among higher \vs lower risk FSW in Eswatini}
  \label{tab:fsw.ratios}
  \input{model/tab.fsw.factor.ratios}
\end{table}

\section{Parameterization}\label{model.par}
As described in \sref{intro.model.param}, model parameterization involves
specification of model parameter values, such as proportions, probabilities, rates, and ratios,
including stratified values to reflect heterogeneity,
and sampling distributions to reflect uncertainty.
Proportions and probabilities were generally modelled using
a beta approximation of the binomial distribution (BAB, see \sref{app.math.distr.bab}),
while rates and ratios were generally modelled using
a gamma, skewnormal, or inverse gaussian distribution.
\paragraph{Notation}
If $X$ is a parameter stratified by dimensions $a,b,c$,
then $X_{ab_{1}c_{23}}$ denotes the values of $X$ for
a particular but \emph{unspecified} stratum of $a$,
the \emph{specific} stratum $b = 1$,
and the \emph{aggregated} strata $c = 2,3$
(the aggregating operation is context-dependent, \eg sum for probabilities).
Additionally, the indices $sihc$ from Table~\ref{tab:model.dims} denote ``self'' strata,
whereas $s'i'h'c'$ denote ``other'' strata --- \ie individuals' partners.%
\footnote{\label{foot:code.note}%
  In the code: R uses one-based indexing, which match the notation here directly,
  while Python uses zero-based indexing, which therefore appear as $i \rightarrow i-1$ in the code.
  Also, the model code reorders states in the ART Cascade dimension for computational efficiency,
  with $c={}$1:~Undiagnosed; 2:~Diagnosed; 3:~Virally~Un-suppressed; 4:~On~ART; 5:~Virally~Suppressed.}
Finally, I re-use several dummy variables throughout the chapter:
$\rho$ for proportions, $\lambda$ for rates, $T$ for time periods, and $f$ for constants.
\input{model/par.fsw}
\input{model/par.beta}
\input{model/par.hiv}
\input{model/par.popsex}
%===================================================================================================
\subsection{HIV Progression \& Mortality}\label{model.par.hiv}
%---------------------------------------------------------------------------------------------------
\subsubsection{HIV Progression}\label{model.par.hiv.dur}
The length of time spent in each HIV stage is related to
rates of progression between stages $\eta_{h}$,
rates of additional HIV-attributable mortality by stage $\mu_{\textsc{hiv},h}$,
and treatment via antiretroviral therapy (ART).
\citet{Lodi2011} estimate median times from seroconversion to
CD4 $<$ 500, $<$ 350, and $<$ 200 cells/mm\tsup{3}, while
\citet{Mangal2017} directly estimate the rates of progression between CD4 states $\eta_{h}$
in a simple compartmental model.
Based on these data, I modelled mean durations ($1/\eta_{h}$) of:%
\footnote{Assuming exponential distributions for durations in each CD4 state
  (see \sref{app.model.math.comp} for more details).}
0.142 years in acute infection ($h=2$, from \sref{model.par.beta.hiv});
3.35 years in CD4~$>$~500 ($h=3$);
3.74 years in 350~$<$~CD4~$<$~500 ($h=4$); and
5.26 years in 200~$<$~CD4~$<$~350 ($h=5$); plus
the remaining time until death in CD4~$<$~200 ($h=6$, AIDS).
Since the duration in acute infection ($h=2$) is randomly sampled,
the remaining duration in CD4~$>$~500 ($h=3$) is adjusted accordingly.
%---------------------------------------------------------------------------------------------------
\subsubsection{HIV Mortality}\label{model.par.hiv.mort}
Mortality rates by CD4-count in the absence of ART were estimated in
multiple African studies \cite{Badri2006,Anglaret2012,Mangal2017};
based on these data, I estimated yearly HIV-attributable mortality rates $\mu_{\textsc{hiv},h}$ as:
0 during acute phase ($h=2$);
0.4\% during CD4~$>$~500 ($h=3$);
2\% during 350~$<$~CD4~$<$~500 ($h=4$);
4\% during 200~$<$~CD4~$<$~350 ($h=5$); and 
20\% during CD4~$<$~200 ($h=6$, AIDS).
%===================================================================================================
\subsection{Antiretroviral Therapy}\label{model.par.art}
Viral suppression via antiretroviral therapy (ART) influences
the probability of HIV transmission, as well as rates of HIV progression and HIV-related mortality.
The model considers individuals on ART before ($c=4$) and after ($c=5$)
achieving full viral load suppression (VLS), as defined by undetectable HIV RNA in blood samples.
Among retained patients initiating ART, time to VLS
is usually described as ``within 6 months'' \cite{Thompson2012}.
More specifically, \citet{Mujugira2016} estimate the median time to VLS as 3 months,
yielding an estimated \emph{mean} duration for $c=4$ of 4.3 months (see \sref{app.model.math.comp}).
%---------------------------------------------------------------------------------------------------
\subsubsection{Probability of HIV Transmission}\label{model.par.art.beta}
All available evidence suggests that viral suppression by ART to undetectable levels
prevents HIV transmission, \ie undetectable = untransmittable (``U=U'') \cite{Eisinger2019}.
Thus, I assumed zero HIV transmission from individuals with VLS ($c=5$).
However, HIV transmission may still occur
during the period between ART initiation to viral suppression ($c=4$) \cite{Mujugira2016}.
\citet{Donnell2010} estimate an adjusted incidence ratio of 0.08~(0.0,~0.57) for all individuals on ART.
However, in \cite{Donnell2010} and \cite{Cohen2016}, the 1 and 4 (respectively)
genetically linked infections from individuals on ART all occurred within 90 days of ART initiation,
suggesting that risk of transmission only persists before viral suppression.
Adjusting the incidence denominator (person-time)
to 90 days per individual who initiated ART in \cite{Donnell2010}
results in approximately 3.13 times higher estimated incidence ratio: 0.25 for this specific period.%
\footnote{In \cite{Donnell2010}, individuals who initiated ART contributed
  approximately 9.4 months per-person (273 persons / 349 person-years, Tables~2~and~3);
  thus the first 3 months of each individual represent
  3/9.4 = 0.319 fewer person-months of follow-up.}
Thus, I sampled relative infectiousness on ART but before viral suppression ($c=4$)
from a beta distribution with mean (95\%~CI) of 0.25~(0.01,~0.67).
%---------------------------------------------------------------------------------------------------
\subsubsection{HIV Progression \& Mortality}\label{model.par.art.hiv}
\def\hunprog{$h = 6 \rightarrow 5 \rightarrow 4 \rightarrow 3$\xspace}
Effective ART stops CD4 cell decline and results in some CD4 recovery \cite{Battegay2006,Lawn2006}.
Most CD4 recovery occurs within the first year of treatment \cite{Battegay2006}.
Due to the limited number of modelled treatment states,
I model this initial recovery to be associated with the 4.3-month pre-VLS ART state ($c=4$).
\citet{Lawn2006,Gabillard2013} estimate an increase of between 25--39 cells/mm\tsup{3} per month
during the first 3 months of treatment.
Since HIV states $h=4,5,6$ correspond to 150, 150, and 200-wide CD4 strata,
I model rates of movement along \hunprog during pre-VLS ART ($c=4$) as
0.20, 0.20, 0.17 per month, respectively.
After initial increases, CD4 recovery is modest and plateaus.
\citet{Battegay2006} report approximate increases of
22.4 cells/mm\tsup{3} per year between years 1 and 5 on ART.
Thus, I model rates of movement along \hunprog after VLS ($c=5$) as 0.15 per year.
\par
Since higher CD4 states are modelled to have lower mortality rates (see \sref{model.par.hiv.mort}),
the modelled recovery of CD4 cells via ART described above implicitly affords a mortality benefit.
However, HIV infection is associated with increased risk of death by non-AIDS causes
--- \ie unrelated to CD4 count ---
including cardiovascular disease and renal disease \cite{Phillips2008}.
\citet{Lundgren2015} estimated 61\% reduction in non-AIDS life-threatening events due to ART.
For the same CD4 strata, \citet{Gabillard2013} also report approximately 2-times higher
mortality rates within the first year of ART versus thereafter,
suggesting that VLS is associated with 50\% mortality reduction independent of CD4 increase.
Thus, I modelled an additional 50\% reduction in mortality among individuals with VLS ($c=5$),
and half this (25\%) reduction before achieving VLS ($c=4$).
% TODO: rates of diagnosis, testing, vls
\section{Parameterization}\label{model.par}
As described in \sref{intro.model.param}, model parameterization involves
specification of model parameter values, such as proportions, probabilities, rates, and ratios,
including stratified values to reflect heterogeneity,
and sampling distributions to reflect uncertainty.
Proportions and probabilities were generally modelled using
a beta approximation of the binomial distribution (BAB, see \sref{app.math.distr.bab}),
while rates and ratios were generally modelled using
a gamma, skewnormal, or inverse gaussian distribution.
\paragraph{Notation}
If $X$ is a parameter stratified by dimensions $a,b,c$,
then $X_{ab_{1}c_{23}}$ denotes the values of $X$ for
a particular but \emph{unspecified} stratum of $a$,
the \emph{specific} stratum $b = 1$,
and the \emph{aggregated} strata $c = 2,3$
(the aggregating operation is context-dependent, \eg sum for probabilities).
Additionally, the indices $sihc$ from Table~\ref{tab:model.dims} denote ``self'' strata,
whereas $s'i'h'c'$ denote ``other'' strata --- \ie individuals' partners.%
\footnote{\label{foot:code.note}%
  In the code: R uses one-based indexing, which match the notation here directly,
  while Python uses zero-based indexing, which therefore appear as $i \rightarrow i-1$ in the code.
  Also, the model code reorders states in the ART Cascade dimension for computational efficiency,
  with $c={}$1:~Undiagnosed; 2:~Diagnosed; 3:~Virally~Un-suppressed; 4:~On~ART; 5:~Virally~Suppressed.}
Finally, I re-use several dummy variables throughout the chapter:
$\rho$ for proportions, $\lambda$ for rates, $T$ for time periods, and $f$ for constants.
\input{model/par.fsw}
\input{model/par.beta}
\input{model/par.hiv}
\input{model/par.popsex}
%===================================================================================================
\subsection{HIV Progression \& Mortality}\label{model.par.hiv}
%---------------------------------------------------------------------------------------------------
\subsubsection{HIV Progression}\label{model.par.hiv.dur}
The length of time spent in each HIV stage is related to
rates of progression between stages $\eta_{h}$,
rates of additional HIV-attributable mortality by stage $\mu_{\textsc{hiv},h}$,
and treatment via antiretroviral therapy (ART).
\citet{Lodi2011} estimate median times from seroconversion to
CD4 $<$ 500, $<$ 350, and $<$ 200 cells/mm\tsup{3}, while
\citet{Mangal2017} directly estimate the rates of progression between CD4 states $\eta_{h}$
in a simple compartmental model.
Based on these data, I modelled mean durations ($1/\eta_{h}$) of:%
\footnote{Assuming exponential distributions for durations in each CD4 state
  (see \sref{app.model.math.comp} for more details).}
0.142 years in acute infection ($h=2$, from \sref{model.par.beta.hiv});
3.35 years in CD4~$>$~500 ($h=3$);
3.74 years in 350~$<$~CD4~$<$~500 ($h=4$); and
5.26 years in 200~$<$~CD4~$<$~350 ($h=5$); plus
the remaining time until death in CD4~$<$~200 ($h=6$, AIDS).
Since the duration in acute infection ($h=2$) is randomly sampled,
the remaining duration in CD4~$>$~500 ($h=3$) is adjusted accordingly.
%---------------------------------------------------------------------------------------------------
\subsubsection{HIV Mortality}\label{model.par.hiv.mort}
Mortality rates by CD4-count in the absence of ART were estimated in
multiple African studies \cite{Badri2006,Anglaret2012,Mangal2017};
based on these data, I estimated yearly HIV-attributable mortality rates $\mu_{\textsc{hiv},h}$ as:
0 during acute phase ($h=2$);
0.4\% during CD4~$>$~500 ($h=3$);
2\% during 350~$<$~CD4~$<$~500 ($h=4$);
4\% during 200~$<$~CD4~$<$~350 ($h=5$); and 
20\% during CD4~$<$~200 ($h=6$, AIDS).
%===================================================================================================
\subsection{Antiretroviral Therapy}\label{model.par.art}
Viral suppression via antiretroviral therapy (ART) influences
the probability of HIV transmission, as well as rates of HIV progression and HIV-related mortality.
The model considers individuals on ART before ($c=4$) and after ($c=5$)
achieving full viral load suppression (VLS), as defined by undetectable HIV RNA in blood samples.
Among retained patients initiating ART, time to VLS
is usually described as ``within 6 months'' \cite{Thompson2012}.
More specifically, \citet{Mujugira2016} estimate the median time to VLS as 3 months,
yielding an estimated \emph{mean} duration for $c=4$ of 4.3 months (see \sref{app.model.math.comp}).
%---------------------------------------------------------------------------------------------------
\subsubsection{Probability of HIV Transmission}\label{model.par.art.beta}
All available evidence suggests that viral suppression by ART to undetectable levels
prevents HIV transmission, \ie undetectable = untransmittable (``U=U'') \cite{Eisinger2019}.
Thus, I assumed zero HIV transmission from individuals with VLS ($c=5$).
However, HIV transmission may still occur
during the period between ART initiation to viral suppression ($c=4$) \cite{Mujugira2016}.
\citet{Donnell2010} estimate an adjusted incidence ratio of 0.08~(0.0,~0.57) for all individuals on ART.
However, in \cite{Donnell2010} and \cite{Cohen2016}, the 1 and 4 (respectively)
genetically linked infections from individuals on ART all occurred within 90 days of ART initiation,
suggesting that risk of transmission only persists before viral suppression.
Adjusting the incidence denominator (person-time)
to 90 days per individual who initiated ART in \cite{Donnell2010}
results in approximately 3.13 times higher estimated incidence ratio: 0.25 for this specific period.%
\footnote{In \cite{Donnell2010}, individuals who initiated ART contributed
  approximately 9.4 months per-person (273 persons / 349 person-years, Tables~2~and~3);
  thus the first 3 months of each individual represent
  3/9.4 = 0.319 fewer person-months of follow-up.}
Thus, I sampled relative infectiousness on ART but before viral suppression ($c=4$)
from a beta distribution with mean (95\%~CI) of 0.25~(0.01,~0.67).
%---------------------------------------------------------------------------------------------------
\subsubsection{HIV Progression \& Mortality}\label{model.par.art.hiv}
\def\hunprog{$h = 6 \rightarrow 5 \rightarrow 4 \rightarrow 3$\xspace}
Effective ART stops CD4 cell decline and results in some CD4 recovery \cite{Battegay2006,Lawn2006}.
Most CD4 recovery occurs within the first year of treatment \cite{Battegay2006}.
Due to the limited number of modelled treatment states,
I model this initial recovery to be associated with the 4.3-month pre-VLS ART state ($c=4$).
\citet{Lawn2006,Gabillard2013} estimate an increase of between 25--39 cells/mm\tsup{3} per month
during the first 3 months of treatment.
Since HIV states $h=4,5,6$ correspond to 150, 150, and 200-wide CD4 strata,
I model rates of movement along \hunprog during pre-VLS ART ($c=4$) as
0.20, 0.20, 0.17 per month, respectively.
After initial increases, CD4 recovery is modest and plateaus.
\citet{Battegay2006} report approximate increases of
22.4 cells/mm\tsup{3} per year between years 1 and 5 on ART.
Thus, I model rates of movement along \hunprog after VLS ($c=5$) as 0.15 per year.
\par
Since higher CD4 states are modelled to have lower mortality rates (see \sref{model.par.hiv.mort}),
the modelled recovery of CD4 cells via ART described above implicitly affords a mortality benefit.
However, HIV infection is associated with increased risk of death by non-AIDS causes
--- \ie unrelated to CD4 count ---
including cardiovascular disease and renal disease \cite{Phillips2008}.
\citet{Lundgren2015} estimated 61\% reduction in non-AIDS life-threatening events due to ART.
For the same CD4 strata, \citet{Gabillard2013} also report approximately 2-times higher
mortality rates within the first year of ART versus thereafter,
suggesting that VLS is associated with 50\% mortality reduction independent of CD4 increase.
Thus, I modelled an additional 50\% reduction in mortality among individuals with VLS ($c=5$),
and half this (25\%) reduction before achieving VLS ($c=4$).
% TODO: rates of diagnosis, testing, vls
\section{Parameterization}\label{model.par}
As described in \sref{intro.model.param}, model parameterization involves
specification of model parameter values, such as proportions, probabilities, rates, and ratios,
including stratified values to reflect heterogeneity,
and sampling distributions to reflect uncertainty.
Proportions and probabilities were generally modelled using
a beta approximation of the binomial distribution (BAB, see \sref{app.math.distr.bab}),
while rates and ratios were generally modelled using
a gamma, skewnormal, or inverse gaussian distribution.
\paragraph{Notation}
If $X$ is a parameter stratified by dimensions $a,b,c$,
then $X_{ab_{1}c_{23}}$ denotes the values of $X$ for
a particular but \emph{unspecified} stratum of $a$,
the \emph{specific} stratum $b = 1$,
and the \emph{aggregated} strata $c = 2,3$
(the aggregating operation is context-dependent, \eg sum for probabilities).
Additionally, the indices $sihc$ from Table~\ref{tab:model.dims} denote ``self'' strata,
whereas $s'i'h'c'$ denote ``other'' strata --- \ie individuals' partners.%
\footnote{\label{foot:code.note}%
  In the code: R uses one-based indexing, which match the notation here directly,
  while Python uses zero-based indexing, which therefore appear as $i \rightarrow i-1$ in the code.
  Also, the model code reorders states in the ART Cascade dimension for computational efficiency,
  with $c={}$1:~Undiagnosed; 2:~Diagnosed; 3:~Virally~Un-suppressed; 4:~On~ART; 5:~Virally~Suppressed.}
Finally, I re-use several dummy variables throughout the chapter:
$\rho$ for proportions, $\lambda$ for rates, $T$ for time periods, and $f$ for constants.
%===================================================================================================
\subsection{Risk Heterogeneity Among FSW}\label{model.par.fsw}
Existing HIV transmission models which include FSW
have rarely sub-stratified this population, such as to reflect
differential HIV risk or distinct typologies of sex work \cite{Blanchard2008,Scorgie2012};
yet such heterogeneities may influence transmission dynamics.
Among the studies identified in Chapter~\ref{sr},
only three sub-stratified FSW by risk-related factors:
\citet{Cremin2017} defined three levels of risk via regression analysis,
\citet{Low2015} distinguished between occasional and full-time FSW, while
\citet{Shannon2015} sub-stratified FSW by
work environment, violence exposure, and context-specific structural factors.
Seven other studies, reflecting two unique models \cite{Johnson2012,Maheu-Giroux2017},
employed age stratification of all activity groups, including FSW;
these models had several risk-related parameters which varied by age.
\par
The model structure here (Figure~\ref{fig:model.risk})
was designed to capture \emph{within}-FSW risk heterogeneity.
The objective of the following analysis was therefore to parameterize
lower \vs higher risk FSW.
I sought to define these groups based on biobehavioural and/or contextual factors
which are demonstrably associated with HIV risk,
and which can be mechanistically incorporated into a transmission model ---
\ie through the force of infection equation.
Later, the parameterization of these groups was validated through model fitting
to relative differences in HIV prevalence \sref{model.cal.targ.prev}.
\par
Many cross-sectional studies of HIV among FSW quantify
the association of risk factors with HIV serostatus
\cite{Aklilu2001,Dunkle2005,Scorgie2012,Jonas2020}.
However, serostatus reflects cumulative risk exposure,
whereas sexual risk behaviour is dynamic \cite{Watts2010,vanWees2020},
as is use of prevention resources \cite{Roberts2020}.
For example, while HIV prevalence often increases with age,
HIV incidence among women can peak before age 25 \cite{Dellar2015}.
Thus, risk factors associated with HIV serostatus are not necessarily
mechanistically related to HIV acquisition.
Indeed, FSW may reduce risk behaviours in response to seroconversion \cite{McClelland2006}.
Cohort studies that measure incidence
can help identify risk factors for HIV acquisition \cite{McKinnon2015,Nouaman2022},
but large sample sizes are often required to accurately estimate overall incidence rate,
let alone risk factors \cite{Priddy2011}.
%---------------------------------------------------------------------------------------------------
\subsubsection{FSW Survey Data}\label{model.par.fsw.data}
Three biobehavioural surveys, in
2011 \cite{Baral2014} (N = 325),
2014 \cite{EswKP2014} (N = 781), and
2021 \cite{EswIBBS2022} (N = 676)
provide HIV status and biobehavioural data on FSW in Eswatini.
The 2011 and 2021 surveys featured serologic HIV testing,
and employed respondent driven sampling (RDS, details in \cite{Yam2013}).
The 2014 survey relied on self-reported HIV status,
andd employed venue-based snowball sampling, based on the
Priorities for Local AIDS Control Efforts (PLACE) methodology,
which aims to identify areas of higher incidence \cite{Weir2005}.
More details about each study are given in \sref{intro.esw.hiv.data} and Table~\ref{tab:esw.data}.
I analyzed the individual-level data from 2011 and 2014 (data from 2021 not yet available)
to explore the potential association of biobehavioural factors with HIV risk,
so that such factors could then be used to distinguish between
lower risk \vs higher risk FSW.
% TODO: (*) add descriptive table
%---------------------------------------------------------------------------------------------------
\subsubsection{HIV Status}\label{model.par.fsw.hiv}
Only the 2011 and 2021 studies included serologic testing for HIV.
Among those tested in 2011 (N = 317, 98\%), 70\% were \hivp,
yielding RDS-adjusted prevalence estimate of 61\% (CI: 51--71\%) \cite{Baral2014}.
Among serologically \hivn, 11\% self-reported \hivp status (false positive), and
among serologically \hivp, 26\% self-reported \hivn status (false negative or undiagnosed).
Overall, self-reported HIV status underestimated HIV prevalence in 2011
by a factor of approximately 0.78 (55~vs~70\%).
Unadjusted HIV prevalence in 2021 was 58.8\%,
with 88\% (363/411) reporting previous awareness of \hivp status.
\par
In 2014, self-reported HIV prevalence was 38\% among respondents who reported (85\%).
This 38\% is surprisingly low considering that
the PLACE methodology explicitly aimed to sample venues
with higher HIV incidence \cite{Weir2005}, and 2014 \vs 2011 respondents
were older (median 27 \vs 25 years), % 2021 median: 28
had been selling sex longer (median 5 \vs 4 years), % 2021: 6
and tested more frequently (87 \vs 75\% tested at least once in the past year, % 2021: 75
82 \vs 63\% among self-reported \hivn).
Perhaps the differences are attributable to the sampling methodology.
Among respondents who self-reported \hivp status,
the 2014 survey also asked for age of HIV diagnosis (6\% missing).
Age of HIV diagnosis supports crude time-to-event analysis (next section),
which can account for confounding by age and censoring,
as compared to logistic regression on HIV status,
keeping in mind the limitations of self-reported HIV status.
%---------------------------------------------------------------------------------------------------
\subsubsection{Risk Factors for HIV}\label{model.par.fsw.fac}
Next, I explored the potential association of risk factors with HIV
via the following three models:%
\footnote{Logistic regression models were implemented using \texttt{lrm} from:
  \hreftt{cran.r-project.org/package=rms}.\\
Cox proportional hazards models were implemented using \texttt{coxaalen} from:
  \hreftt{cran.r-project.org/package=coxinterval}.}
\begin{enumerate}
  \item Logistic regression on serologic HIV status (2011 data)
  \item Logistic regression on self-reported HIV status (2014 data)
  \item Cox proportional hazards for interval-censored time to HIV infection,
    with interval from self-reported sex work debut 
    to either self-reported time of HIV diagnosis or survey date (2014 data);
    Figure~\ref{fig:fsw.tte.interval} illustrates
    the four potential censoring cases in this framework.
\end{enumerate}
An important limitation to all models is that
risk factors reported by FSW at the time of survey
are assumed to be fixed characteristics of the respondents,
rather than dynamic characteristics that vary over time.
Additionally, respondents with any missing variables for each individual model
were excluded from that model. % TODO: (%)
\begin{figure}
  \centering
  \includegraphics[scale=1]{diag.tte}
  \caption{Illustration of time-to-event analysis framework
    for cross-sectional FSW survey data}
  \label{fig:fsw.tte.interval}
  \floatfoot{
    $\bm{\times}$: HIV infection;
    SW: time of sex work debut;
    Dx: time of HIV diagnosis.}
\end{figure}
\par
Risk factors were selected based on
prior knowledge of plausible mechanistic influence on HIV incidence and/or prevalence.
The risk factors explored are summarized in Table~\ref{tab:fsw.stats},
including univariate and multivariable association under each model.
Variable selection for multivariable models
was performed using backward selection as described by \citet{Lawless1978},
using a $p \le 0.1$ (per variable) threshold for stepwise variable retention.
Estimated conditional effects of
variables retained in the multivariable logistic regression models
are illustrated in Figure~\ref{fig:fsw.lr}.
\begin{table}
  \centering
  \caption{Risk factors explored for association with \hivp status among FSW in Eswatini}
  \label{tab:fsw.stats}
  \input{model/tab.fsw.factor.stats}
\end{table}
\begin{figure}[h]
  \subcapoverlap
  \foreach \year/\var/\nvar in {2011/f/1,2011/c/2,2014/f/3,2014/c/3}{
  \begin{subfigure}{\nvar\linewidth/5+\linewidth/5}
    \includegraphics[scale=.7]{fsw.\year.lr.hiv.\var}
    \caption{\raggedright}
    \label{fig:fsw.lr.\year.\var}
  \end{subfigure}}
  \caption{Predicted conditional effects (probability)
    of variables in multivariable logistic regression models for HIV status}
  \label{fig:fsw.lr}
  \floatfoot{\fffsw{fig:fsw.lr}
    conditional probabilities shown for fixed covariates at arbitrary values.}
\end{figure}
\par
Following variable selection, each multivariable model was used to estimate
the total \hivp status odds ratio (logistic) or HIV incidence hazard ratio (Cox)
for each respondent in the respective survey ---
\ie $e^{X_i\,\beta}$ for respondent $i$ ---
representing an overall ``risk score'' under each model.
Respondents were then stratified into the top 20\% and bottom 80\% by these risk scores.
The values of each variable were compared between these two strata
using a test for the ratio of the means \cite{Tamhane2004} to support model parameterization;
these ratios are summarized in Table~\ref{tab:fsw.ratios},
and the distributions of variable values across the two strata
are illustrated in Figure~\ref{fig:fsw.f}.
\begin{table}
  \centering
  \caption{Ratios of HIV risk factor variables among higher \vs lower risk FSW in Eswatini}
  \label{tab:fsw.ratios}
  \input{model/tab.fsw.factor.ratios}
\end{table}

\section{Parameterization}\label{model.par}
As described in \sref{intro.model.param}, model parameterization involves
specification of model parameter values, such as proportions, probabilities, rates, and ratios,
including stratified values to reflect heterogeneity,
and sampling distributions to reflect uncertainty.
Proportions and probabilities were generally modelled using
a beta approximation of the binomial distribution (BAB, see \sref{app.math.distr.bab}),
while rates and ratios were generally modelled using
a gamma, skewnormal, or inverse gaussian distribution.
\paragraph{Notation}
If $X$ is a parameter stratified by dimensions $a,b,c$,
then $X_{ab_{1}c_{23}}$ denotes the values of $X$ for
a particular but \emph{unspecified} stratum of $a$,
the \emph{specific} stratum $b = 1$,
and the \emph{aggregated} strata $c = 2,3$
(the aggregating operation is context-dependent, \eg sum for probabilities).
Additionally, the indices $sihc$ from Table~\ref{tab:model.dims} denote ``self'' strata,
whereas $s'i'h'c'$ denote ``other'' strata --- \ie individuals' partners.%
\footnote{\label{foot:code.note}%
  In the code: R uses one-based indexing, which match the notation here directly,
  while Python uses zero-based indexing, which therefore appear as $i \rightarrow i-1$ in the code.
  Also, the model code reorders states in the ART Cascade dimension for computational efficiency,
  with $c={}$1:~Undiagnosed; 2:~Diagnosed; 3:~Virally~Un-suppressed; 4:~On~ART; 5:~Virally~Suppressed.}
Finally, I re-use several dummy variables throughout the chapter:
$\rho$ for proportions, $\lambda$ for rates, $T$ for time periods, and $f$ for constants.
\input{model/par.fsw}
\input{model/par.beta}
\input{model/par.hiv}
\input{model/par.popsex}
%===================================================================================================
\subsection{HIV Progression \& Mortality}\label{model.par.hiv}
%---------------------------------------------------------------------------------------------------
\subsubsection{HIV Progression}\label{model.par.hiv.dur}
The length of time spent in each HIV stage is related to
rates of progression between stages $\eta_{h}$,
rates of additional HIV-attributable mortality by stage $\mu_{\textsc{hiv},h}$,
and treatment via antiretroviral therapy (ART).
\citet{Lodi2011} estimate median times from seroconversion to
CD4 $<$ 500, $<$ 350, and $<$ 200 cells/mm\tsup{3}, while
\citet{Mangal2017} directly estimate the rates of progression between CD4 states $\eta_{h}$
in a simple compartmental model.
Based on these data, I modelled mean durations ($1/\eta_{h}$) of:%
\footnote{Assuming exponential distributions for durations in each CD4 state
  (see \sref{app.model.math.comp} for more details).}
0.142 years in acute infection ($h=2$, from \sref{model.par.beta.hiv});
3.35 years in CD4~$>$~500 ($h=3$);
3.74 years in 350~$<$~CD4~$<$~500 ($h=4$); and
5.26 years in 200~$<$~CD4~$<$~350 ($h=5$); plus
the remaining time until death in CD4~$<$~200 ($h=6$, AIDS).
Since the duration in acute infection ($h=2$) is randomly sampled,
the remaining duration in CD4~$>$~500 ($h=3$) is adjusted accordingly.
%---------------------------------------------------------------------------------------------------
\subsubsection{HIV Mortality}\label{model.par.hiv.mort}
Mortality rates by CD4-count in the absence of ART were estimated in
multiple African studies \cite{Badri2006,Anglaret2012,Mangal2017};
based on these data, I estimated yearly HIV-attributable mortality rates $\mu_{\textsc{hiv},h}$ as:
0 during acute phase ($h=2$);
0.4\% during CD4~$>$~500 ($h=3$);
2\% during 350~$<$~CD4~$<$~500 ($h=4$);
4\% during 200~$<$~CD4~$<$~350 ($h=5$); and 
20\% during CD4~$<$~200 ($h=6$, AIDS).
%===================================================================================================
\subsection{Antiretroviral Therapy}\label{model.par.art}
Viral suppression via antiretroviral therapy (ART) influences
the probability of HIV transmission, as well as rates of HIV progression and HIV-related mortality.
The model considers individuals on ART before ($c=4$) and after ($c=5$)
achieving full viral load suppression (VLS), as defined by undetectable HIV RNA in blood samples.
Among retained patients initiating ART, time to VLS
is usually described as ``within 6 months'' \cite{Thompson2012}.
More specifically, \citet{Mujugira2016} estimate the median time to VLS as 3 months,
yielding an estimated \emph{mean} duration for $c=4$ of 4.3 months (see \sref{app.model.math.comp}).
%---------------------------------------------------------------------------------------------------
\subsubsection{Probability of HIV Transmission}\label{model.par.art.beta}
All available evidence suggests that viral suppression by ART to undetectable levels
prevents HIV transmission, \ie undetectable = untransmittable (``U=U'') \cite{Eisinger2019}.
Thus, I assumed zero HIV transmission from individuals with VLS ($c=5$).
However, HIV transmission may still occur
during the period between ART initiation to viral suppression ($c=4$) \cite{Mujugira2016}.
\citet{Donnell2010} estimate an adjusted incidence ratio of 0.08~(0.0,~0.57) for all individuals on ART.
However, in \cite{Donnell2010} and \cite{Cohen2016}, the 1 and 4 (respectively)
genetically linked infections from individuals on ART all occurred within 90 days of ART initiation,
suggesting that risk of transmission only persists before viral suppression.
Adjusting the incidence denominator (person-time)
to 90 days per individual who initiated ART in \cite{Donnell2010}
results in approximately 3.13 times higher estimated incidence ratio: 0.25 for this specific period.%
\footnote{In \cite{Donnell2010}, individuals who initiated ART contributed
  approximately 9.4 months per-person (273 persons / 349 person-years, Tables~2~and~3);
  thus the first 3 months of each individual represent
  3/9.4 = 0.319 fewer person-months of follow-up.}
Thus, I sampled relative infectiousness on ART but before viral suppression ($c=4$)
from a beta distribution with mean (95\%~CI) of 0.25~(0.01,~0.67).
%---------------------------------------------------------------------------------------------------
\subsubsection{HIV Progression \& Mortality}\label{model.par.art.hiv}
\def\hunprog{$h = 6 \rightarrow 5 \rightarrow 4 \rightarrow 3$\xspace}
Effective ART stops CD4 cell decline and results in some CD4 recovery \cite{Battegay2006,Lawn2006}.
Most CD4 recovery occurs within the first year of treatment \cite{Battegay2006}.
Due to the limited number of modelled treatment states,
I model this initial recovery to be associated with the 4.3-month pre-VLS ART state ($c=4$).
\citet{Lawn2006,Gabillard2013} estimate an increase of between 25--39 cells/mm\tsup{3} per month
during the first 3 months of treatment.
Since HIV states $h=4,5,6$ correspond to 150, 150, and 200-wide CD4 strata,
I model rates of movement along \hunprog during pre-VLS ART ($c=4$) as
0.20, 0.20, 0.17 per month, respectively.
After initial increases, CD4 recovery is modest and plateaus.
\citet{Battegay2006} report approximate increases of
22.4 cells/mm\tsup{3} per year between years 1 and 5 on ART.
Thus, I model rates of movement along \hunprog after VLS ($c=5$) as 0.15 per year.
\par
Since higher CD4 states are modelled to have lower mortality rates (see \sref{model.par.hiv.mort}),
the modelled recovery of CD4 cells via ART described above implicitly affords a mortality benefit.
However, HIV infection is associated with increased risk of death by non-AIDS causes
--- \ie unrelated to CD4 count ---
including cardiovascular disease and renal disease \cite{Phillips2008}.
\citet{Lundgren2015} estimated 61\% reduction in non-AIDS life-threatening events due to ART.
For the same CD4 strata, \citet{Gabillard2013} also report approximately 2-times higher
mortality rates within the first year of ART versus thereafter,
suggesting that VLS is associated with 50\% mortality reduction independent of CD4 increase.
Thus, I modelled an additional 50\% reduction in mortality among individuals with VLS ($c=5$),
and half this (25\%) reduction before achieving VLS ($c=4$).
% TODO: rates of diagnosis, testing, vls
\section{Parameterization}\label{model.par}
As described in \sref{intro.model.param}, model parameterization involves
specification of model parameter values, such as proportions, probabilities, rates, and ratios,
including stratified values to reflect heterogeneity,
and sampling distributions to reflect uncertainty.
Proportions and probabilities were generally modelled using
a beta approximation of the binomial distribution (BAB, see \sref{app.math.distr.bab}),
while rates and ratios were generally modelled using
a gamma, skewnormal, or inverse gaussian distribution.
\paragraph{Notation}
If $X$ is a parameter stratified by dimensions $a,b,c$,
then $X_{ab_{1}c_{23}}$ denotes the values of $X$ for
a particular but \emph{unspecified} stratum of $a$,
the \emph{specific} stratum $b = 1$,
and the \emph{aggregated} strata $c = 2,3$
(the aggregating operation is context-dependent, \eg sum for probabilities).
Additionally, the indices $sihc$ from Table~\ref{tab:model.dims} denote ``self'' strata,
whereas $s'i'h'c'$ denote ``other'' strata --- \ie individuals' partners.%
\footnote{\label{foot:code.note}%
  In the code: R uses one-based indexing, which match the notation here directly,
  while Python uses zero-based indexing, which therefore appear as $i \rightarrow i-1$ in the code.
  Also, the model code reorders states in the ART Cascade dimension for computational efficiency,
  with $c={}$1:~Undiagnosed; 2:~Diagnosed; 3:~Virally~Un-suppressed; 4:~On~ART; 5:~Virally~Suppressed.}
Finally, I re-use several dummy variables throughout the chapter:
$\rho$ for proportions, $\lambda$ for rates, $T$ for time periods, and $f$ for constants.
\input{model/par.fsw}
\input{model/par.beta}
\input{model/par.hiv}
\input{model/par.popsex}