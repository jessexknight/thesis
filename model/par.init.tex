%===================================================================================================
\subsection{Initialization}\label{model.par.init}
The first cases of HIV and AIDS in Eswatini
were diagnosed in 1986 and 1987, respectively \cite{Whiteside2007},
although HIV may have been present several years earlier \cite{Iliffe2005}.
As such, I initialize the model in 1980 with no HIV,
and simulate introduction of HIV at a random year between 1980 and 1985 (uniform prior).
HIV introduction is modelled as
exogenous infection of 0.01\% (\ttilde\,24) individuals in the model,%
\footnote{No further import/export of HIV to/from Eswatini is considered thereafter in the model.
  HIV transmission between Eswatini and neighbouring countries,
  including South Africa and Mozambique,
  has likely continued throughout the epidemic
  due to labour migration and other factors \cite{Iliffe2005}.
  However, I assume that such transmissions have low overall influence on epidemic dynamics.}
  % TODO: (?) discuss possible effects
distributed across activity groups in proportion to their size, comprising:
% TODO: (?) maybe among highest activity instead
5\% acute HIV ($h=2$), 65\% with \cdf{500}{} ($h=3$) and 30\% with \cdf{350}{500} ($h=4$),
all undiagnosed ($c=1$).%
\footnote{In compartmental models, the numbers of individuals in each state (compartment)
  need not be whole numbers.}
The population size of EmaSwati aged 15--49 in 1980
was defined as 243,000 from \cite{WorldBank}.
