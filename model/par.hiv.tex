%===================================================================================================
\subsection{HIV Progression \& Mortality}\label{model.par.hiv}
%---------------------------------------------------------------------------------------------------
\subsubsection{HIV Progression}\label{model.par.hiv.dur}
The length of time spent in each HIV stage is related to
rates of progression between stages $\eta_{h}$,
rates of additional HIV-attributable mortality by stage $\mu_{\textsc{hiv},h}$,
and treatment via antiretroviral therapy (ART).
\citet{Lodi2011} estimate median times from seroconversion to
\cdf{}{500}, $<$\,350, and $<$\,200 cells/mm\tsup{3}, while
\citet{Mangal2017} directly estimate the rates of progression between CD4 states $\eta_{h}$
in a simple compartmental model.
Based on these data, I modelled mean durations ($1/\eta_{h}$) of:%
\footnote{Assuming exponential distributions for durations in each CD4 state
  (see \sref{app.model.math.exp} for more details).}
0.142 years in acute infection ($h=2$, from \sref{model.par.beta.hiv});
3.35 years in \cdf{500}{} ($h=3$);
3.74 years in \cdf{350}{500} ($h=4$); and
5.26 years in \cdf{200}{350} ($h=5$); plus
the remaining time until death in \cdf{}{200} ($h=6$, AIDS).
Since the duration in acute infection ($h=2$) is randomly sampled,
the remaining duration in \cdf{500}{} ($h=3$) is adjusted accordingly.
%---------------------------------------------------------------------------------------------------
\subsubsection{HIV Mortality}\label{model.par.hiv.mort}
Mortality rates by CD4-count in the absence of ART were estimated in
multiple African studies \cite{Badri2006,Anglaret2012,Mangal2017};
based on these data, I estimated yearly HIV-attributable mortality rates $\mu_{\textsc{hiv},h}$ as:
0 during acute phase ($h=2$);
0.4\% during \cdf{500}{} ($h=3$);
2\% during \cdf{350}{500} ($h=4$);
4\% during \cdf{200}{350} ($h=5$); and
20\% during \cdf{}{200} ($h=6$, AIDS).
%===================================================================================================
\subsection{Antiretroviral Therapy}\label{model.par.art}
Viral suppression via antiretroviral therapy (ART) influences
the probability of HIV transmission, as well as rates of HIV progression and HIV-related mortality.
The model considers individuals on ART before ($c=3$) and after ($c=4$)
achieving full viral load suppression (VLS), as defined by undetectable HIV RNA in blood samples.
Among retained patients initiating ART, time to VLS
is usually described as ``within 6 months'' \cite{Thompson2012}.
More specifically, \citet{Mujugira2016} estimate the median time to VLS as 3 months,
yielding an estimated \emph{mean} duration for $c=3$ of 4.3 months (see \sref{app.model.math.exp}),
and thus $\sigma \approx 2.77$ per year.
%---------------------------------------------------------------------------------------------------
\subsubsection{Probability of HIV Transmission on ART}\label{model.par.art.beta}
All available evidence suggests that viral suppression by ART to undetectable levels
prevents HIV transmission, \ie undetectable = untransmittable (``U=U'') \cite{Eisinger2019}.
Thus, I assumed zero HIV transmission from individuals with VLS ($c=4$).
However, HIV transmission may still occur
during the period between ART initiation to viral suppression ($c=3$) \cite{Mujugira2016}.
\citet{Donnell2010} estimate an adjusted incidence ratio of 0.08~(0.0,~0.57) for all individuals on ART.
However, in \cite{Donnell2010} and \cite{Cohen2016}, the 1 and 4 (respectively)
genetically linked infections from individuals on ART all occurred within 90 days of ART initiation,
suggesting that risk of transmission only persists before viral suppression.
Adjusting the incidence denominator (person-time)
to 90 days per individual who initiated ART in \cite{Donnell2010}
results in approximately 3.13 times higher estimated incidence ratio: 0.25 for this specific period.%
\footnote{In \cite{Donnell2010}, individuals who initiated ART contributed
  approximately 9.4 months per-person (273 persons / 349 person-years, Tables~2~and~3);
  thus the first 3 months of each individual represent
  3/9.4 = 0.319 fewer person-months of follow-up.}
Thus, I sampled relative infectiousness on ART but before viral suppression ($c=3$)
from a beta distribution with mean (95\%~CI) of 0.25~(0.01,~0.67).
Finally, I assumed that the virally un-suppressed state ($c=5$) had
half the reduced infectiousness of $c=3$, yielding 95\%~CI: (0.50,~0.83).
%---------------------------------------------------------------------------------------------------
\subsubsection{HIV Progression \& Mortality on ART}\label{model.par.art.hiv}
\def\hunprog{$h = 6 \rightarrow 5 \rightarrow 4 \rightarrow 3$\xspace}
Effective ART stops CD4 cell decline and results in some CD4 recovery \cite{Battegay2006,Lawn2006}.
Most CD4 recovery occurs within the first year of treatment \cite{Battegay2006}.
Due to the limited number of modelled treatment states,
I model this initial recovery to be associated with the 4.3-month pre-VLS ART state ($c=3$).
\citet{Lawn2006,Gabillard2013} estimate an increase of between 25--39 cells/mm\tsup{3} per month
during the first 3 months of treatment.
Since HIV states $h=4,5,6$ correspond to 150, 150, and 200-wide CD4 strata,
I model rates of movement along \hunprog during pre-VLS ART ($c=3$) as
0.20, 0.20, 0.17 per month, respectively.
After initial increases, CD4 recovery is modest and plateaus.
\citet{Battegay2006} report approximate increases of
22.4 cells/mm\tsup{3} per year between years 1 and 5 on ART.
Thus, I model rates of movement along \hunprog after VLS ($c=4$) as 0.15 per year.
\par
Since higher CD4 states are modelled to have lower mortality rates (see \sref{model.par.hiv.mort}),
the modelled recovery of CD4 cells via ART described above implicitly affords a mortality benefit.
However, HIV infection is associated with increased risk of death by non-AIDS causes
--- \ie unrelated to CD4 count ---
including cardiovascular disease and renal disease \cite{Phillips2008}.
\citet{Lundgren2015} estimated 61\% reduction in non-AIDS life-threatening events due to ART.
For the same CD4 strata, \citet{Gabillard2013} also report approximately 2-times higher
mortality rates within the first year of ART versus thereafter,
suggesting that VLS is associated with 50\% mortality reduction independent of CD4 increase.
Thus, I modelled an additional 50\% reduction in mortality among individuals with VLS ($c=4$),
and half this (25\%) reduction before achieving VLS ($c=3$).
% TODO: rates of diagnosis, testing, vls