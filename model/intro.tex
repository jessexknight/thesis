This chapter details the development of
a deterministic compartmental model of heterosexual HIV transmission in Eswatini,
to address downstream research questions in Chapters \ref{foi}~and~\ref{art}.
Drawing on the insights from Chapter~\ref{sr},
the model aims to capture key determinants of heterosexual transmission dynamics, including
sex work, numbers of sexual partners, levels of condom use, anal sex, and ART scale-up.
The model was implemented in Python~v3.8.10 with Numpy~v1.22.2,
and solved numerically using 4th order Runge-Kutta \cite{Gill1951} using a timestep of 0.05 years.
Post-hoc analysis was conducted in R~v3.6.3.
All code and selected results are available on GitHub.%
\footnote{\label{foot:github}\hreftt{github.com/mishra-lab/hiv-fsw-art}}
\par
The remainder of this chapter is organized as follows:
\newcommand{\iref}[1]{\item \sref{#1}}
\begin{itemize}
  \iref{model.str} outlines the model \emph{structure} and notation, including
  the population stratifications and sexual partnership types considered.
  \iref{model.par} details the data, analyses, and assumptions used for model \emph{parameterization}
  --- \ie estimating prior distributions for model parameters, including:
  HIV initialization in Eswatini;
  HIV transmission probabilities and modifiers thereof;
  progression rates through HIV stages and the ART cascade of care;
  analysis of primary survey data to distinguish higher \vs lower risk Swati FSW;
  risk group sizes, rates of turnover, sexual partnership numbers,
  sex frequencies, partnership durations, and mixing.
  \iref{model.cal} details the data, analyses, and assumptions used for model \emph{calibration}
  --- \ie inferring parameter posterior distributions,
  under which model outputs best match calibration targets, including:
  overall and (where possible) group-specific HIV prevalence, incidence, and ART cascade attainment.
  \iref{model.res} presents the \emph{results} of model calibration, including:
  posterior parameter distributions, comparison of model outputs \vs calibration targets, and
  a descriptive summary of modelled transmission dynamics over time.
  \iref{model.disc} provides \emph{discussion} on the chapter methods and results,
  and notes the limitations of the model.
\end{itemize}
\pagebreak % TEMP
Notably, \sref{model.par} features several methodological contributions
to support model parameterization, including:
\begin{itemize}
  \item\sref{model.par.fsw}:
  original analysis of primary data to distinguish higher \vs lower risk FSW
  \item\sref{model.par.wp}:
  an adjustment for reporting bias in partner numbers data using polling booth data
  \item\sref{model.par.turn.act}
  an adjustment for censoring in risk group turnover/duration data
  \item\sref{model.par.pnum.adj}:
  an adjustment for bias due to partnership duration in partner numbers data
  \item\sref{model.par.mix}
  a balancing approach to support more flexible mixing patterns
\end{itemize}
