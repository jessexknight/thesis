%===================================================================================================
\subsection{Probability of HIV Transmission}\label{model.par.beta}
I parameterized the overall probability of transmission per sex act $\beta$ as
the product of a base rate $\beta_0$,
and independent relative effects corresponding to multiple factors.
Such factors (indexed $f$) included:
sex act type $a$, condom use, prevalence of circumcision among susceptible men,
partner HIV infection stage $h'$ and viral suppression via ART $c'$,
as well as prevalence of STI co-infection/symptoms among both partners.
Thus, $\beta$ was defined as:
\begin{equation}\label{eq:model.beta}
  \beta_{asis'i'h'c'} = \beta_0 \, R_{\beta,f_1} \dots R_{\beta,f_N}
\end{equation}
The impact of each factor (except ART) on the probability of HIV transmission
is described in the following subsections,
while the prevalence of each factor is given in \sref{model.par.tm}.
The impact of ART on transmission is described in \sref{model.par.art.beta}.
%---------------------------------------------------------------------------------------------------
\subsubsection{HIV Infection Stage}\label{model.par.beta.hiv}
\citet{Boily2009} synthesized per-act transmission probability in the absence of ART
from 43 studies in 25 populations.
Among 7 studies reporting stage of HIV infection (early, asymptomatic, late),
infection stage explained 95\% of variance
in per-act probability of transmission in \cite{Boily2009}.
Such differences in transmission are most likely due to differences in viral load,
which is associated with HIV stage \cite{Saag1996,Donnell2010}.
% These differences also motivated in part stratification of
% HIV infection by disease stage in the benchmark model.
The probability of transmission during the middle asymptomatic period,
was reported as mean (95\%~CI) 0.072~(0.053,~0.097)\% per act, reflecting $\beta_0$.
To improve model fit (see \sref{model.cal}), the 95\%~CI was increased to (0.053,~0.15)\%,
which was used to define a gamma prior distribution for $\beta_0$.
This probability was assumed to apply to vaginal intercourse,
based on the studies considered.
\par
For early infection ($h=2$), \citet{Boily2009} estimated
the relative infectiousness of the first 5 months of infection
as 9.2~(4.5,~18.8) times higher than the asymptomatic period.
However, both the duration and infectiousness of the acute phase
have been long debated \cite{Hollingsworth2008,Cohen2011ahi,Cohen2012}.
In a recent reanalysis of the Rakai cohort data, \citet{Bellan2015} estimate
a much smaller contribution of the acute phase to overall infection,
summarized as 8.4~(0,~63) ``excess hazard-months'' (EHM).
This excess risk represents the joint uncertainty and collinearity in the estimated
duration of 1.7~(.55,~6.8) months and relative infectiousness of 5.3~(.79,~57).
Thus, I sampled the duration $\delta_{h=2}$ from
a gamma prior with mean (95\%~CI) 1.7~(.55,~6) months,
and relative infectiousness $R_{\beta,h'=2}$ from
a gamma prior with 5.3~(1,~15) times the asymptomatic period
(confidence intervals were adjusted to fit the gamma distributions, and to ensure 1 $<$ EHM $<$ 63).
\par
For late-stage disease, defined as 6-15 months before death in \cite{Boily2009},
\citeauthor{Boily2009} estimated the relative rate of transmission as 7.3~(4.5,~11.9).
However, I defined later HIV stages by CD4 count, including
\cdf{200}{350} ($h=5$) and \cdf{}{200} ($h=6$, AIDS),
which reflects closer to 50 and 18 months before death in the absence of ART, respectively.
Therefore, I combined estimates from several sources
\cite{Wawer2005,Boily2009,Donnell2010} to define two gamma prior distributions
with mean (95~CI\%) 1.6~(1.3,~1.9) and 8.3~(4.5,~13),
for the relative rate of HIV transmission in these two stages ($h=5,6$), respectively.
For \cdf{350}{} ($h=3,4$), I assumed no change from the baseline probability $\beta_0$.
%---------------------------------------------------------------------------------------------------
\subsubsection{Sex Act Types}\label{model.par.beta.sex}
% TODO: Hughes2012: no diff mtf/ftm after adjustment (but age gaps?)
The model considers vaginal and anal intercourse,
further stratified by sex (male-to-female/insertive \vs female-to-male/receptive).
For vaginal intercourse, evidence for differential risk by sex is mixed,
with some studies reporting no difference \cite{Wawer2005,Hughes2012},
and others reporting up to 2-times higher male-to-female ($s'=2,s=1$) transmission
\vs female-to-male ($s'=1,s=2$) \cite{Gregson2002a,Boily2009}.
% TODO: is this supported by Boily2009 for LIC ?? no...
To reflect this uncertainty, I sampled
the relative rate of male-to-female \vs female-to-male transmission from $\opname{Unif}[1,~2]$;
in applying this relative rate, both male-to-female and female-to-male transmission probabilities
were adjusted such that the overall mean was preserved.
\par
\citet{Baggaley2018} synthesized the per-act transmission probability for anal intercourse,
with most data from MSM studies.
Analyses in \cite{Baggaley2018} were not stratified by HIV stage,
so I assumed the same relative rates derived in \sref{model.par.fsw}
applied equally to vaginal and anal intercourse.
Overall female-to-male (insertive) per-act transmission probabilities were similar for
anal intercourse \cite{Baggaley2013} (without ART): 0.14~(0.04,~0.29)\% \vs
vaginal intercourse \cite{Boily2009} (without commercial sex exposure): 0.164~(0.056,~0.481)\%;
thus I assumed that female-to-male (insertive) transmission probabilities
for anal \vs vaginal intercourse were equal.
By contrast, male-to-female (receptive) per-act transmission probabilities were approximately 10 higher
in anal intercourse \cite{Baggaley2018} (without ART): 1.67~(0.44,~3.67)\% \vs
vaginal intercourse \cite{Boily2009} (without commercial sex exposure): 0.143~(0.088,~0.233)\%;
thus I assumed a fixed 10-fold increase in male-to-female transmission probability
for anal \vs vaginal intercourse.
See \sref{model.par.sex} for sex act frequency within each partnership type.
%---------------------------------------------------------------------------------------------------
\subsubsection{Circumcision}\label{model.par.beta.circ}
Relative risk in per-act HIV female-to-male transmission for circumcised \vs uncircumcised men
via vaginal intercourse has consistently been estimated as
approximately 0.50, with 95\%~CI spanning (0.29,~0.96) \cite{Boily2009,Hughes2012,Patel2014}.
Since circumcision status is unrelated to the research question,
I fixed this effect at 50\% relative risk.
For anal intercourse, \citet{Wiysonge2011} estimated that circumcision resulted in
.27~(.17,~.44) the odds of HIV acquisition for the insertive partner.
It can be shown that relative reduction in incidence represents a lower bound
on relative reduction in per-act transmission probability.%
\footnote{See \sref{model.par.fsw} for more discussion.}
Thus, for anal intercourse, I similarly fixed the per-act effect at 27\%.
Finally, there is inconclusive evidence to suggest that circumcision status affects
male-to-female/receptive transmission \cite{Weiss2009,Wiysonge2011}, so I assumed no effect.
See \sref{model.par.tm.circ} for prevalence of circumcision in Eswatini over time.
%---------------------------------------------------------------------------------------------------
\subsubsection{Condoms}\label{model.par.beta.condom}
The most recent meta-analysis of condom effectiveness in heterosexual couples by \citet{Giannou2016}
estimated a relative risk of approximately 0.26~(0.13,~0.43).
No significant differences were noted between female-to-male \vs male-to-female transmission.
A recent study among men who have sex with men found
a similar effect for anal sex \cite{Smith2015}.
Thus, condom effectiveness was fixed at 74\%.
See \sref{model.par.tm.condom} for levels of condom use in Eswatini over time.
%---------------------------------------------------------------------------------------------------
\subsubsection{Genital Ulcer Disease}\label{model.par.beta.gud}
% TODO: Hughes2012: GUD only +sus not +inf
Genital ulcer disease (GUD)
is another another established risk factor for HIV transmission \cite{Plummer1991,Fleming1999}.
Some, but not all GUD is associated with sexually transmitted infections (STIs),
and some, but not all STIs can cause GUD \cite{Fleming1999}.
GUD is thought to increase both HIV susceptibility and infectiousness
through a variety of mechanisms \cite{Fleming1999,Sheffield2007,Fox2010},
but HIV may also facilitate transmission of various STIs
through immunosuppression \cite{Wasserheit1992}.
The meta-analysis by \citet{Boily2009} found that
presence of STI alone was not associated with increased HIV transmission: RR 1.11~(0.30,~4.14),
but GUD was: RR 5.29~(1.43,~19.6),
with most studies examining GUD among the HIV-susceptible partner.
One study \cite{Gray2001} estimated RR 2.58~(1.03,~5.69) of transmission
for GUD among the HIV-positive partner.
Most studies defined GUD status as any experience of symptoms during the study period
(\eg past 12 months, p12m),
since precise delineation of GUD episodes is challenging.
Morover, individuals may take action to reduce onward STI transmission,
such as accessing treatment, having less sex, and using condoms \cite{SDHS2006}.
Thus, the true effect of GUD on HIV transmission
via unprotected sex during active GUD episodes may be larger.
However, if estimates of GUD prevalence and GUD effect (on HIV transmission)
use consistent definitions (\eg any GUD in p12m),
then the time-averaged effect can be applied without need to estimate GUD episode duration.
On the other hand, association of GUD and HIV transmission may not reflect \emph{causation},
but rather \emph{confounding} by uncontrolled exposure risk.
As such, I applied factors for increased susceptibility and infectiousness due to GUD
in accordance with group-specific p12m GUD prevalence (see \sref{model.par.tm.gud}),
with median 95\%~CI (1.2,~7.0) and (1.2,~3.4) (gamma priors), respectively.
%===================================================================================================
\subsection{Prevalence of Transmission Modifiers}\label{model.par.tm}
%---------------------------------------------------------------------------------------------------
\subsubsection{Circumcision}\label{model.par.tm.circ}
Traditional (non-medical) circumcision in Eswatini is rare,
reported as approximately 0.7\% of men aged 15-49 in 2016 \cite{SHIMS2}.
Voluntary medical male circumcision (VMMC) increased circumcision coverage to 8.2\% by 2007,
following demand for mainly hygienic reasons \cite{SDHS2006}.
In 2007, the government further increased scale-up of VMMC services
as part of HIV prevention efforts \cite{SDHS2006}, leading to
17.1\% coverage in 2011 \cite{SHIMS1},
30.0\% in 2017 \cite{SHIMS2}, and
37\% in 2021 \cite{EswCOP21}.
Since VMMC continues to be a key element of Eswatini's HIV response \cite{EswCOP21},
I assumed that coverage could reach and plateau at 50--90\% (95\%~CI) by 2050.
There is minimal evidence of differential condom use by circumcision status \cite{SHIMS1},
so I assumed no differences.
Similarly, while circumcision differed by union status in \cite{SHIMS2}
(\eg 22.1\% circumcised among men in a union \vs 31.7\% among men not in a union),
differences did not persist after re-stratifying these men
into groups with 0-1 \vs 2+ partners per year, as described in \sref{model.par.nsw}.
In Zambia, circumcision status was not associated with paying for sex \cite{Carrasco2020}.
%---------------------------------------------------------------------------------------------------
\subsubsection{Condom Use}\label{model.par.tm.condom}
Condom use is typically reported as either
categorical for a recent period, usually 30 days,
\eg \shortquote{never, rarely, sometimes, often, always}; or
binary for the most recent sex act.
Both report types may be subject to reporting bias,
but the ``last sex'' more directly translates into a proportion of sex acts.
The direction of reporting bias may vary with social context, with
\cite{Cordero-Coma2012} suggesting over-reporting of consom use, and
\cite{Behanzin2013} suggesting under-reporting of conddom use.
As such, I made no systemic adjustments to the available condom use data.
Table~\ref{tab:esw.condom.data} summarizes the available condom use data for Eswatini.
\begin{table}
  \centering
  \caption{Estimates of condom use in Eswatini}
  \label{tab:esw.condom.data}
  \small
\begin{tabular}{lrlllcll}
  \toprule
  Partnership Type     & Year & Population & Type      & \%   &  (95\%~CI)   & Ref                & Notes \\
  \midrule
  Main                 & 2006 & Women      & last sex  & 23.5 & (23.2,~23.9) & \cite{SDHS2006}    & \tn{a} \\
                       &      & Men        & last sex  & 23.1 & (19.4,~26.9) & \cite{SDHS2006}    & \tn{a} \\
                       & 2016 & Women      & last sex  & 52.7 & (52.5,~52.9) & \cite{SHIMS2}      & \tn{a} \\
                       &      & Men        & last sex  & 33.7 & (30.8,~36.7) & \cite{SHIMS2}      & \tn{a} \\[1ex]
  Main or Casual       & 1988 & Women      & currently &\d0.6 &  (0.4,~1.3)  & \cite{SFHS1988}    & \tn{b} \\
                       &      & Men        & currently &\d7.3 & (5.9,~12.1)  & \cite{SFHS1988}    & \tn{b} \\
                       & 2002 & FSW        & last sex  & 60   &     ---      & \cite{EswSBSS2002} & \tn{cd} \\
                       &      &            & always    & 45.8 &     ---      & \cite{EswSBSS2002} & \tn{cd} \\
                       & 2006 & Women      & last sex  & 36.5 &     ---      & \cite{SDHS2006}    & \\
                       &      & Men        & last sex  & 47.2 &     ---      & \cite{SDHS2006}    & \\
                       & 2011 & Women      & always    & 30   &     ---      & \cite{SDHS2006}    & \\
                       &      & Men        & always    & 34   &     ---      & \cite{SDHS2006}    & \\
                       &      & FSW        & last sex  & 51.1 & (41.8,~60.4) & \cite{Baral2014}   & \tn{de} \\
                       &      &            & always    & 20.8 & (14.7,~26.9) & \cite{Baral2014}   & \tn{de} \\
                       & 2014 & FSW        & last sex  & 80.6 & (64.7,~89.6) & \cite{EswKP2014}   & \tn{g} \\
                       & 2016 & Women      & last sex  & 58.3 &     ---      & \cite{SHIMS2}      & \\
                       &      & Men        & last sex  & 53.1 &     ---      & \cite{SHIMS2}      & \\[1ex]
  Casual               & 2006 & Women      & last sex  & 53.5 &     ---      & \cite{SDHS2006}    & \\
                       &      & Men        & last sex  & 66.0 &     ---      & \cite{SDHS2006}    & \\
                       & 2016 & Women      & last sex  & 64.9 &     ---      & \cite{SHIMS2}      & \\
                       &      & Men        & last sex  & 73.7 &     ---      & \cite{SHIMS2}      & \\[1ex]
  Sex Work Unspecified & 2002 & FSW        & last sex  & 90   &     ---      & \cite{EswSBSS2002} & \tn{d} \\
                       &      &            & always    & 74.4 &     ---      & \cite{EswSBSS2002} & \tn{d} \\
                       & 2020 & FSW        & always    & 50   &     ---      & \cite{EswIBBS2022} & \\[1ex]
  New Sex Work         & 2011 & FSW        & last sex  & 84.8 & (57.9,~92.4) & \cite{Baral2014}   & \tn{ef} \\
                       &      &            & always    & 56.7 & (47.8,~65.6) & \cite{Baral2014}   & \tn{d} \\
                       & 2014 & FSW        & last sex  & 88.5 & (54.9,~95.9) & \cite{EswKP2014}   & \tn{g} \\[1ex]
  Regular Sex Work     & 2011 & FSW        & last sex  & 82.9 & (56.8,~90.0) & \cite{Baral2014}   & \tn{ef} \\
                       &      &            & always    & 38.6 & (29.5,~47.7) & \cite{Baral2014}   & \tn{e} \\
                       & 2014 & FSW        & last sex  & 85.6 & (47.9,~95.0) & \cite{EswKP2014}   & \tn{g} \\
  \bottomrule
\end{tabular}
\floatfoot{
  \tnt[a]{Back-calculated as described in \sref{model.par.tm.condom}};
  \tnt[b]{95\%~CI from urban \& rural data};
  \tnt[c]{Described as ``non-paying partners'' in the survey};
  \tnt[d]{Two major cities only (Manzini \& Mbambane)};
  \tnt[e]{RDS-adjusted};
  \tnt[f]{95\%~CI lower bound reduced by 25\% due to possible reporting bias};
  \tnt[g]{95\%~CI bounds from regions with lowest and highest reported condom use}.
}
% TODO: add EswIBBS2022
\end{table}
%---------------------------------------------------------------------------------------------------
\paragraph{Main/Spousal \& Casual}
No direct estimates of condom use in main/spousal partnerships are available;
condom use at last sex (with a non-paying partner)
was either reported overall or for casual partners only.%
\footnote{``Higher risk'' partners were defined in \cite{SDHS2006} as:
  \shortquote{Sexual intercourse with a partner
    who was neither a spouse nor lived with the respondent},
    effectively matching the model definition of ``casual'' partnerships.}
However, the proportions of individuals with various relationship statuses
(\eg polygynous union, non-polygynous union, not in a union, see \sref{model.par.nsw})
can be used to back-calculate condom use in main/spousal partnerships
for both 2006 \cite{SDHS2006} and 2016 \cite{SHIMS2}.
To do so, I assumed whether ``last sex'' among individuals in unions with 2+ partners
was with their main/spousal partner or with a casual partner;
or more generally, what proportion of most recent sex acts was with a casual partner.
I repeated the back-calculation assuming 5\% and 95\%,
yielding the confidence intervals shown in Table~\ref{tab:esw.condom.data}.
Estimates of condom use in non-paying partners were
lower among FSW \vs the wider population in 2011 (20.8\% vs \ttilde32\% ``always''), but
higher in 2014-16 (80.1\% vs \ttilde55.7\% ``last sex'').
Therefore, I assumed no differences in condom use
among FSW \vs the wider population for main/spousal or casual partnerships.
%---------------------------------------------------------------------------------------------------
\paragraph{Sex Work}
All data on sex work partnerships in Eswatini is from FSW (\ie not their clients).
A 2001 study in Ghana \cite{Cote2004} suggested that
FSW were more likely than their clients to report having used a condom.
As such, I adjusted the lower bound of 95\%~CI for condom use in sex work partnerships ($p=3,4$)
as either 75\% of the reported lower bound, or the lowest reported region-specific estimate.
Estimates for 2002 \cite{EswSBSS2002} were obtained from two major cities only (Manzini and Mbambane);
since early condom availability was mainly urban,
treated these estimates as 95\%~CI upper-bounds,
and defined the lower bound as 20\% of the reported values.
%---------------------------------------------------------------------------------------------------
\paragraph{Anal Sex}
\citet{Owen2020} estimate that among FSW globally,
condom use in anal sex is approximately 79~(66,~94)\% that of condom use in vaginal sex.%
\footnote{I integrated the reported confidence intervals using the delta method
  after assuming binomial-distributed proportions.}
In Eswatini \cite{Baral2014,EswKP2014}, relative condom use in anal sex \vs vaginal sex
ranged from 44\% among new clients in 2011 to 88\% among regular clients in 2014.
So, I sampled relative condom use in anal \vs vaginal sex from a BAB prior distribution
with 95\%~CI: (50,~95)\%.
%---------------------------------------------------------------------------------------------------
\paragraph{Sampling \& Trends}
While levels of condom use reported by men and women do not always agree,
the levels should agree in simulated partnerships.
To reflect uncertainty due to the discrepancy,
I sampled condom use for each year and partnership type
from BAB prior distributions having 95\%~CI
that spans the range of estimates from men and women (where applicable),
including the widest points of all confidence intervals.
I further expanded the confidence intervals in some cases
by enforcing a maximum value of $N = 100$ for the BAB distribution.
I assume that condom use was effectively zero in 1980 \cite{SFHS1988}.
I also assume andd enforce two conditions that:
condom use must be monotonic increasing over time; and
condom use must be highest in new sex work partnerships, and lowest in main partnerships,
for all sampled parameter values.
For each available year, I simultaneously sample condom use for all partnership types,
and samples failing the condition are discarded.
As illustrated in \sref{app.math.distr.jsam}, this sampling strategy
minimizes differences between the prior and sampled-with-constraint distributions.
For each partnership type, I then smoothly interpolate
between sampled levels of condom use over the available years
using monotone piecewise cubic interpolation \cite{Fritsch1980}.
%---------------------------------------------------------------------------------------------------
\subsubsection{Genital Ulcer Disease}\label{model.par.tm.gud}
% TODO: update for 2023-03-02
Self-reported prevalence of GUD in p12m among sexually active women and men aged 15--49
was approximately 7\% in 2006 \cite[Table~13.14]{SDHS2006};
this prevalence was not stratified by numbers of partners, so I assumed it was equal across
sexually active individuals in lowest and medium activity groups.
However, approximately 19~and~32\% of the lowest activity women and men
were not sexually active during p12m (see \sref{model.par.nsw});
thus I reduced GUD prevalence by 19~and~32\%, respectively, among these groups.
\par
The 2011 and 2014 FSW surveys did not ask respondents about GUD specifically,
but about any STI symptoms in p12m.%
\footnote{The survey question about STI symptoms was:
  \shortquote{In the last 12 months, have you had symptoms of a sexually transmitted infection
    including discharge from your vagina or sores on or around your vagina or anus}.}
In the wider population \cite{SDHS2006},
approximately 60\% of women self-reporting any STI symptoms specifically reported GUD in p12m;
thus, self-reported STI symptoms among FSW may overestimate p12m GUD prevalence.
Approximately 50\% and 25\% of FSW reported STI symptoms in 2011 and 2014, respectively.
Reflecting uncertainty related to self-reported estimates, STI \vs GUD, and sampling bias,
I sampled p12m GUD prevalence among lower risk FSW from
a BAB distribution with 95\%~CI (10,~40)\%.
Per analysis in \sref{model.par.fsw}, I assumed that STI (and thus GUD) prevalence was
approximately 3~(1.5,~5) times higher among higher risk FSW (gamma prior),
with an upper bound of 100\%.
%\par
FSW data also suggest declining STI prevalence between 2011 and 2014.
%this decline could reflect scale-up of differentiated health services delivery for key populations,
%including STI testing and treatment for FSW \cite{TODO}.
However, STI prevalence among Swati youth in 2017--18 remained high \cite{Jasumback2020}.
Thus, to reflect uncertainty in STI/GUD prevalence trends,
I sampled a relative reduction in GUD prevalence for all populations between 2020 and 2050
from a uniform distribution spanning [0.2,~1].
\par
Finally, no Eswatini-specific data are available for clients of FSW,
but studies in Zimbabwe \cite{Cowan2005}, Senegal \cite{Santo2005} and Zambia \cite{Carrasco2020}
have found 2.5--3.7 (95\% CI span 1.4--5.0) the odds
of STI symptoms during the past 6--12 months among clients \vs non-clients.
Yet, I assumed that even higher risk clients
could not have greater GUD prevalence than lower risk FSW.
Thus, I sampled higher risk client GUD prevalence uniformly between 7\% and that of lower risk FSW,
and sampled lower risk client GUD prevalence uniformly between 7\% and that of higher risk clients.