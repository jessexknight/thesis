%===================================================================================================
\subsection{Approach}\label{model.cal.appr}
I used a Bayesian approach for model calibration \cite{Menzies2017}.
Let $\theta$ denote the complete set of 73 calibrated model parameters (Table~\ref{tab:par.defs}),
and $T$ the complete set of 69 calibration targets.
The goal of calibration is to obtain samples from
the posterior distribution of parameters given the targets $p(\theta \mid T)$.
This posterior distribution can be defined via Bayes' rule as:
\begin{equation}
  p(\theta \mid T) = \frac{p(T \mid \theta)\,p(\theta)}{p(T)}
\end{equation}
The posterior distribution was characterized empirically via Monte Carlo simulation
--- \ie by randomly sampling parameter sets $\theta_s \sim p(\theta)$,
and for each set computing the likelihood $p(T \mid \theta_s)$.
This likelihood was defined via
independent uncertainty distributions for each calibration target $T_i$.
For example, overall HIV prevalence in Eswatini was estimated as
27.2\%, 95\%~CI: (25.8,~28.7) in 2016 \cite{SHIMS2};
using this information, I defined a BAB distribution
as the likelihood function for this calibration target $p(T_i \mid \theta)$.
Thus, a parameter set $\theta_{s_1}$ which yields
model-estimated overall HIV prevalence $T_i(\theta_{s_1}) = 25\%$ in 2016
would have a higher likelihood for this target than
a parameter set $\theta_{s_2}$ which yielsds $T_i(\theta_{s_2}) = 20\%$ in 2016.
The independent likelihoods for each target were aggregated on logarithmic scale
to give the overall likelihood:
\begin{equation}
  \log p(T \mid \theta_s) = \sum_i \log p(T_i \mid \theta_s)
\end{equation}
Any individual log-likelihood which was beyond computational precision was replaced with
an arbitrarily large negative number ($-{10}^6$).
In order to obtain good coverage of the sampling space,
most (58) calibrated parameters were sampled using Latin hypercube sampling \cite{Stein1987}.
The remaining 15 calibrated parameters were sampled randomly and iteratively
until they satisfied a set of relational constraints (see \sref{app.model.cal.constr}).
\par
Although several iterative methods exist to
update the sampling distributions based on the likelihoods,
and thereby characterize the posterior distribution more efficiently \cite{Menzies2017},
I did not update the sampling distributions.
Rather, I simply took the top 1\% of parameter sets $\theta_s$ by likelihood,
and assume these are approximately representative of the posterior distribution.
Within the top 1\%, I also did not weight parameter sets by likelihood.
I sampled 100,000 parameter sets, yielding $N_j = 1000$ posterior samples and
corresponding plausible epidemic simulations or ``model fits''.
