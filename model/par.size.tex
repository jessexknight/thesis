%===================================================================================================
\subsection{Activity Group Sizes}\label{model.par.size}
Population sizes of all activity groups are modelled as proportions of the total population,
which are assumed to remain roughly constant.
Individuals can, however, move between groups (see \sref{model.par.turn.act})
--- \ie groups are open populations ---
and disproportionate mortality due to HIV between groups
may cause higher risk groups to shrink over time.
Overall population growth is discussed in \sref{model.par.turn.bd}.
%---------------------------------------------------------------------------------------------------
\subsubsection{Female Sex Workers}\label{model.par.size.fsw}
The proportion of women who report sex work in national demographic and health surveys
is generally considered unreliable due to social desirability bias,
particularly if the survey is face-to-face and household-based
\cite{Konings1995,Gregson2002,Gregson2004,Lowndes2012,Behanzin2013}.
Therefore, FSW population size estimates require
targeted surveys and unique methodologies \cite{UNAIDS2010kps,Abdul-Quader2014}.
In both \cite{EswKP2014} and \cite{EswIBBS2022}, the Swati FSW population size
was estimated using a combination of
unique object method, service multiplier method, prior survey participation,
and network scale-up method (NSUM) \cite{UNAIDS2010kps}.
In 2011 \cite{EswKP2014}, regional FSW population size estimates
ranged from 0.7\% to 6.5\% of all women,
with overall population-weighted mean across regions of 2.9\%;
in 2021 \cite{EswIBBS2022}, the mean (95\%~CI) estimates were 2.43~(1.17,~5.02)\%.
To reflect this uncertainty in the model, a BAB distribution was fitted
such that 95\% of the probability fell between 0.7\% and 6.5\%,
and used as the prior distribution for the proportion of women who are FSW:
$P_{s_{1}i_{34}} / P_{s_{1}}$.
Then, following the analysis in \sref{model.par.fsw},
the proportion of all FSW in the higher risk FSW group was fixed at 20\%,
and likewise the lower risk group at 80\%.
%---------------------------------------------------------------------------------------------------
\subsubsection{Clients of FSW}\label{model.par.size.cli}
Similar to FSW, household-based surveys are not considered reliable data sources
for estimating the population size of clients of FSW \cite{Behanzin2013}.
However, few surveys are designed to reach clients of FSW,
and no direct estimates of FSW size exist for Eswatini.
So, I use a common approach for inferring the FSW client size \cite{Cote2004},
similar to the ``multiplier method'' \cite{Morison2001}.
Given the FSW population proportion $P_{s_{1}i_{34}}$,
the number of average yearly new and regular sex work clients per FSW $Q_{p_{34}s_{1}i_{34}}$,
the frequency of sex per partnership-year $F_{p_{34}}$, and
the total number of yearly commercial sex acts per client year $Q_{p_{34}s_{2}i_{34}}\,F_{p_{34}}$,
the total client population $P_{s_{2}i_{34}}$ is defined as:
\begin{equation}
  {\textstyle\sum_{i}} P_{s_{2}i_{34}} =
  \frac{\sum_{pi} P_{s_{1}i}\,Q_{p_{34}s_{1}i_{34}}\,F_{p_{34}}}
       {\sum_{pi}             Q_{p_{34}s_{2}i_{34}}\,F_{p_{34}}}
  \label{eq:model.fsw.cli.tot}
\end{equation}
Then, as with FSW, the proportion of total clients in the higher risk client group
is defined as 20\% of all clients, and likewise for the lower risk group at 80\%.
Using $Q_{p_{34}s_{1}i_{34}}$, $Q_{p_{34}s_{2}i_{34}}$, and $F_{p_{34}}$
as defined below in \sref{model.par.pnum.sw}, the client population size $P_{s_{2}i_{34}}$
estimated by this method was 14.5~(5.2,~33.9)\% of men.
%---------------------------------------------------------------------------------------------------
\subsubsection{Wider Population}\label{model.par.size.wp}
Based on the results of \sref{model.par.wp},
I defined the sizes of the modelled lower and medium activity groups,
and the average numbers of main/spousal partnerships per person.
I assumed that $W'_{2+}$ and $M'_{2+}$ included FSW and client population sizes, respectively.
Thus, the populations size of medium activity women was defined as
$P_{s_{1}i_{2}} = W'_{2+} - P_{s_{1}i_{34}}$.
Sampling $W'_{2+}$ from a BAB distribution with 95\%~CI (10,~27)\%,
the resulting 95\%~CI for medium activity women $P_{s_{1}i_{2}}$ was (6,~25)\% of women.
The lowest activity women population size was then defined as $1 - P_{s_{1}i_{234}}$,
representing (73,~90)\% of women.
Since there is greater uncertainty in the client population size,
the same approach for the medium activity men population size $P_{s_{2}i_{2}}$
could yield negative values.
Instead, I sampled $P_{s_{2}i_{2}}$ directly from
a BAB distribution with 95\%~CI (10,~17)\%, yielding
95\%~CI for $P_{s_{2}i_{234}}$ of (15,~50)\% of men,
which is close to (15,~44)\% from $M_{2+}$.
The lowest activity men were then then defined as $1 - P_{s_{2}i_{234}}$,
representing (50,~85)\% of men.
