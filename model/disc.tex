\section{Discussion}\label{model.disc}
% TODO: (*) compare vs other models of Eswatini
Model design, parameterization, and calibration are key steps in applied transmission modelling,
each step comprising numerous assumptions and analyses.
As I illustrate throughout this thesis,
these assumptions and analyses can be strong determinants of model outputs.
Yet, the full details of these steps
are often relegated to the supplementary materials of published articles --- if available at all ---
with varying notation, terminology, and organization \cite{Delva2012,Knight2022sr}.
It's not clear whether these supplementary materials
are subject to the same level of peer review as the main text.
This chapter gives the complete details of the Eswatini model development, which,
in combination with the online code,\footref{foot:github}
aims to provide full transparency and opportunity for peer review.
%===================================================================================================
\subsection{Methodological Contributions for Model Parameterization}\label{model.disc.par}
The analyses required to support model parameterization depend heavily on the available data.
Standardized approaches will likely remain less fruitful \vs
careful consideration of the data at hand with respect to potential biases and precise interpretation.
This chapter presents several novel methodologies for model parameterization,
which may be useful to modellers, epidemiologists, and others.
%---------------------------------------------------------------------------------------------------
\subsubsection{Quantifying Sexual Behaviour}\label{model.disc.par.sex}
Quantifying sexual behaviour has long been challenging due to
issues of representative sampling, non-response, recall bias, and reporting bias \cite{Fenton2001}.
Such challenges may be magnified by the HIV epidemic itself,
and intersect with issues of stigma and marginalization.
Household-based face-to-face surveys, such as the demographic and health surveys \cite{DHS}
are typically one of few context-specific data sources for HIV transmission models;
yet sexual behaviour data from these surveys have a high risk of bias.
For example, household-based surveys likely miss populations at higher risk of HIV,
including those who are highly mobile, homeless, or who live in institutions
like brothels, prisons, and barracks \cite{GarciaCalleja2005,Mishra2008,Cassels2014,Camlin2016}.
As such, household-based surveys are generally not recommended to collect data on
stigmatized behaviour and/or key populations \cite{UNAIDS2010kps,Abdul-Quader2014}.
\par
Moreover, comparison of survey delivery modes, including
self-administered questionnaires, computer-assisted tools, and anonymous polling booth methods
suggests that face-to-face surveys likely induce strong social desirability reporting bias
\cite{Langhaug2010,Lowndes2012}.
For example, reporting of extramarital sex in p12m among men and women
was estimated via polling booth \vs face-to-face survey to be
3--7 times higher in Benin \cite{Behanzin2013} and 6--8 times higher in India \cite{Lowndes2012};
reporting of genital ulcers was likewise 2--5 and \emph{14--35} times higher, respectively;
buying and selling sex was similarly biased \cite{Behanzin2013,Lowndes2012}.
Qualitative data from Eswatini \cite{Ruark2014,Fielding-Miller2016,Ruark2019,Pulerwitz2021}
reinforce the possibility of prevalent unreported sexual partnerships.
% TODO: (?) something about non-mutually exclusive partnership types
% TODO: (~) reiterate that gen-pop surveys are not enough: compare KP pop sizes & data vs DHS
%---------------------------------------------------------------------------------------------------
\paragraph{Reporting Bias Adjustment}
In \sref{model.par.wp}, I proposed a framework to incorporate reporting biases
when estimating the proportions of women and/or men who report stigmatized behaviour
--- in this case, numbers of sexual partners.
This framework formalizes ad hoc adjustments often made by modellers
to reconcile relatively small numbers of reported partners
with high levels of observed transmission, \eg \cite{Anderson2014}.
Unlike ad hoc adjustments, the proposed framework explicitly uses
a specified ratio between adjusted \vs reported population proportions engaging in the behaviour,
and further supports uncertainty in this ratio via Monte Carlo sampling.
The framework also allows estimation of
internally consistent population proportions (\ie sum to 100\%) for more than 2 strata
through constructing and solving a system of constrained, nonlinear equations.
%---------------------------------------------------------------------------------------------------
\paragraph{Recall Period Adjustment}
A related issue concerns how to derive a partnership formation rate
(or number of concurrent parthers) from survey data.
Sexual health surveys will typically ask questions like
\shortquote{How many different people have you had sex with in the past 12 months?} \cite{DHS}
or \shortquote{past 1 month}, etc. \cite{Baral2014,EswKP2014,EswBSS2002}.
Then, it's not obvious whether the reported partner numbers should be interpreted as
a rate per recall period, or simply a number of concurrent partners.
Indeed, the former interpretation of these data has likely contributed to
a common practice of capping modelled partnership durations at 1 year (see \sref{foi.prior.dur}),
with notable influence on model outputs.
In \sref{model.par.pnum.adj}, I showed that the correct interpretation
is somewhere in between these extremes,
and derived expressions for both
the partnership formation rate \eqref{eq:x2Q}
and numbers of concurrent partners \eqref{eq:x2K}, given a partnership duration.
While partnership duration can also be challenging to measure \cite{Burington2010},
these expressions can help conceptualize survey responses and,
at minimum, support more precise assumptions when analyzing the data.
Future work should explore the influence of heterogeneous partnership duration on these equations.
%---------------------------------------------------------------------------------------------------
\subsubsection{Log-Linear Mixing}\label{model.disc.par.mix}
Mixing patterns --- \ie who contacts whom --- are
a well-established determinant of epidemic dynamics \cite{Nold1980,Jacquez1988,Garnett1993}.
Like-with-like (``assortative'') mixing generally acts to
compound the effects of risk heterogeneity:
increasing the initial rate of epidemic growth (reproduction number)
and decreasing the equilibrium prevalence \cite{Jacquez1988}.
Despite the recognized importance of mixing,
there are surprisingly limited data to inform sexual mixing patterns among risk groups, and
many compartmental HIV models continue to use a 1-parameter ($\epsilon$) approach
\cite{Nold1980,Knight2022sr}.%
\footnote{Stratification of partnership types,
  where different risk groups form different numbers of each partnership type,
  also contributes to overall mixing patterns.}
This approach assumes that a minimum proportion of partners ($\epsilon$)
are guaranteed to be from the same risk group,
and that this proportion is fixed and equal for all risk groups.
\par
By contrast, the log-linear approach proposed in \cite{Morris1991}
provides greater flexibility in conceptualizing and implementing mixing via
the \emph{odds} of any two groups mixing \vs random mixing.
However, \cite{Morris1991} does not provide a method to
maintain fixed partnership numbers / formation rates
--- which are typically assumed constant based on the available data ---
for arbitrary mixing patterns;
I hypothesize that this limitation has prevented widespread adoption of the log-linear approach.
In \sref{model.par.mix.ll}, I developed a method to maintain
fixed partnership formation rates for arbitrary mixing patterns
using an iterative proportional fitting procedure \cite{Ruschendorf1995}.
This method therefore allows specification of more complex mixing patterns
to reflect emerging data and/or modelling hypotheses,
while maintaining fixed overall sexual activity.
The log-linear approach also defines mixing patterns at the population-level
(\vs partnerships per-person, or conditional probability of a given partnership),
making it easy to verify and/or enforce that partnerships ``balance'',
as population-level mixing matrices should be symmetric \cite{Knight2023chap}.
%---------------------------------------------------------------------------------------------------
\subsubsection{Duration of Risk Exposure}\label{model.disc.par.dur}
In addition to risk heterogeneity and mixing, recent work has shown that
``turnover'' among risk groups, also called ``episodic risk'', is another key determinant of
epidemic dynamics and intervention impact \cite{Henry2015,Knight2020}.
Risk group turnover acts to reduce risk heterogeneity via
net movement of infections from higher risk groups into lower risk groups;
thus, calibrating a model to a given prevalence ratio with \vs without turnover
requires an even larger \emph{incidence} ratio \cite{Knight2020}.
Turnover can be parameterized using, among other things,
the average duration within a given risk group \cite{Knight2020}.
Within the model, these durations are implicitly assumed to be exponentially distributed,
which appears reasonable for Swati FSWs \cite{Baral2014,EswKP2014} (Figure~\ref{fig:fsw.yss.adj}).
However, such durations are often estimated from survey data using the difference between
the respondent's current age and the age they reported first selling sex.
As discussed in \sref{model.par.turn.act},
this definition of sex work duration can be biased by up to three factors:
right censoring, as FSW continue selling sex after the survey (duration underestimated);
difficulties reaching new FSW \cite{Cheuk2020} (duration overestimated);
and intermittent engagement in sex work (duration overestimated).
Although I have tried to explicitly account for such biases,
future work could explore and compare alternate methods of estimating
duration in sex work (or other epidemiologically relevant ``states'') \cite{Fazito2012}.
Indeed, in the absence of age stratification,
the conceptualization and implementation of turnover in the current model is somewhat simplistic
(see also \sref{model.disc.lim.str}), ignoring
unique vulnerabilities faced by young sex workers \cite{Cheuk2020}, and
whether paid sex is driven by supply \vs demand \cite{Garnett1993,Steen2019}.
%---------------------------------------------------------------------------------------------------
\subsubsection{Within-Group Heterogeneity}\label{model.disc.par.het}
Key populations are usually assumed to be homogeneous in compartmental models
--- \ie heterogeneity \emph{within} key populations is not considered.
Yet, there is substantial variability in
the structural, behavioural, and network-level HIV risk factors experienced by FSW,
within and between epidemic ontexts \cite{Blanchard2008,Scorgie2012,Shannon2015,Willcox2021}.
The Eswatini model aims to represent this heterogeneity --- albeit simply ---
by stratifying FSW into higher and lower risk groups.
As explored in \sref{model.par.fsw}, I parameterized these groups using individual-level data from
FSW surveys in 2011 \cite{Baral2014} and 2014 \cite{EswKP2014} in Eswatini.
Although these data were not ideal for inferring mechanistic risk (see below),
the data-driven stratification and parameterization of risk groups used
--- drawing on risk score methodology \cite{Ayton2020,Willcox2021} ---
may be useful elsewhere.
\par
Unfortunately, these parameterization analyses are limited by
the suitability of the available data, namely:
the cross-sectional nature of both surveys, and
the availability of only self-reported HIV status in 2014 \cite{Baral2014,EswKP2014}.
While cross-sectional data can be used to estimate associations of factors with HIV status,
which may be directly useful for recommending HIV testing \cite{Moucheraud2022},
the same factors may not be associated with HIV \emph{acquisition} risk,
since risk is dynamic but HIV status reflects cumulative risk.
Factors associated with acquisition risk would be more useful as mechanistic model inputs,
and can be estimated from longitudinal data,
as in the case of risk scores used to support PrEP initiation \cite{Ayton2020,Willcox2021}.
Regarding HIV status, self-reported status was historically considered unreliable due to
low rates of HIV diagnosis and high rates of incidence among FSW \cite{Mountain2014sr,Hakim2018};
however, recent scale-up of HIV testing and incidence declines in Eswatini
may render self-reported HIV status a reasonable proxy for serological HIV status~\cite{Xia2021}.
%---------------------------------------------------------------------------------------------------
% TODO: (?) \subsubsection{Partnerships Types}\label{model.disc.par.part}
% Although not a novel contribution of this model, the explicit representation of both
% occasional (one-off) and regular (repeat) sex work partnerships \dots
% and main/spousal partnerships \emph{between} FSW and their clients.
% \cite{Watts2010}
% longer & lower condom use
%===================================================================================================
\subsection{Limitations of the Model}\label{model.disc.lim}
Despite the advacements in model parameterization described above,
the model developed here still has several limitations.
This section describes these limitation and their potential influence on model outputs.
%---------------------------------------------------------------------------------------------------
\subsubsection{Model Structure}\label{model.disc.lim.str}
The model structure includes $2 \times 4 \times (1 + 5 \times 5) = 208$ compartments in total,
reflecting sex, activity level, and HIV / treatment dimensions,
as well as four distinct partnership types, and vaginal \vs anal sex.
Yet even these stratifications omit several important aspects of HIV epidemiology in Eswatini.
%---------------------------------------------------------------------------------------------------
\paragraph{Men Who Have Sex with Men}
Men who have sex with men (MSM) experience disproportionate HIV risk globally
due to multiple factors, including
increased probability of transmission via anal sex and differences in sexual network density
\cite{Beyrer2012,Hessou2019}.
Pooled HIV prevalence among MSM is estimated to be
3--9 times higher \vs men overall in Sub-Saharan Africa (SSA);
however, prevalence ratios are generally smaller in larger epidemics \cite{Hessou2019}.
In fact, HIV prevalence among Swati MSM has been estimated to be
similar to among men aged 15--49 overall (approximately 20\%) \cite{Hessou2019,EswIBBS2022,SHIMS2},
likely because Swati MSM tend to skew younger, while HIV prevalence increases with age
\cite{SHIMS2,EswIBBS2022}
The population size of MSM in Eswatini is estimated to be
1--2\% of men aged 15--49 \cite{EswKP2014,EswIBBS2022,WorldBank}.
Thus, although unmet needs of MSM in other SSA countries
are estimated to drive overall transmission \cite{Stone2021,Silhol2021},
including via overlapping MSM and heterosexual networks \cite{Beyrer2010},
the same may be less true in the Eswatini due to high overall HIV prevalence.
Therefore, the influence of omitting MSM on modelled HIV transmission dyanmics
would likely be relatively small in Eswatini \vs in contexts with lower overall HIV prevalence.
%---------------------------------------------------------------------------------------------------
\paragraph{Age Stratification \& Transactional Partnerships}
HIV prevalence in Eswatini, as elsewhere, continues to be strongly associated with age,
increasing from $<$5\% at age 15 to approximately 50\% between ages 30--50,
and declining thereafter \cite{SDHS2006,SHIMS1,SHIMS2}.
While HIV risk likely accumulates with age due to sexual activity,
older generations would have experienced lower cumulative risk
if their sexual activity peaked before widespread HIV transmission.
The age of peak prevalence is also shifting older as incidence declines, suggesting that
younger generations are experiencing lower cumulative risk by a given age \vs older generations;
yet, prevalence continues to peak earlier among women \vs men, suggesting that
women experience more risk earlier \cite{SDHS2006,SHIMS1,SHIMS2}.
\par
Indeed, adolescent girls and young women are increasingly recognized as another key population
in the HIV epidemic response \cite{Dellar2015}, whose vulnerabilities include:
higher biological susceptibility, gender-based violence, food/economic insecurity, and
transactional relationships --- defined in \cite{Stoebenau2016} as:
\shortquote{non-commercial, non-marital sexual relationships motivated by
the implicit assumption that sex will be exchanged for material support or other benefits}
\cite{Yi2013,Dellar2015,Wamoyi2016}.
Qualitative data highlight the prevalence of such factors in Eswatini,
with roots in patriarchal norms and broader social pressures
\cite{Jones2009,Ruark2014,Fielding-Miller2016,Ruark2019,Pulerwitz2021}.
\par
By omitting age stratification,
and not explicitly modelling transactional relationships as distinct from casual partnerships,
the model may fail to capture two key epidemiological phenomena:
1) declining incidence due to age-cohorting effects ---
since true overall age mixing is likely assortative with moderate age disparities
\cite{Harling2014,Fielding-Miller2016,Pulerwitz2021}
some infections can become ``trapped'' within age cohorts \cite{Beauclair2018},
whereas the model without age stratification implicitly assumes
random age mixing throughout the population;%
\footnote{The importance of age-disparate partnerships for prevention remains controversial
  \cite{Leclerc-Madlala2008,Ott2011,Harling2014}.}
2) mechanistic contributions of transactional partnerships and associated factors ---
the importance of transmission drivers that are not modelled evidently cannot be inferred,
and may instead be mis-attributed to factors that \emph{are} modelled,
such as the relative susceptibility of women \vs men (see \sref{model.par.beta.sex}).
%---------------------------------------------------------------------------------------------------
\subsubsection{Calibration}\label{model.disc.lim.cal}
My goals for model parameterization and calibration were to:
a) obtain parameter sets which yielded plausible HIV epidemic trajectories for
Eswatini and/or Southern Africa in general; and
b) favor additional uncertainty over assumption-driven parameter constraints.
Thus, I opted to consider uncertainty in a large number of parameters ($N = 73$),
despite having only a similar number of calibration targets ($N = 69$).
I represented this uncertainty mainly via univariate parameter priors,
which could permit implausible \emph{combinations} of parameter values,
although I enforced a few joint parameter constraints (\sref{app.model.cal.constr}).
Additionally, the calibration approach used
(selecting the top 1\% of 100,000 parameter sets by likelihood)
was similar to prior approaches \cite{Johnson2010}, but still relatively simple and ad hoc.
As a result, my ability to infer model parameter values through calibration was limited,
and many parameter posterior distributions did not differ significantly from their priors
(Figure~\ref{fig:post.cor}).
The quality of parameter inference and model agreement with the calibration targets
might be improved using more efficient calibration techniques \cite{Menzies2017}, such as
Sampling Importance Resampling (SIR) \cite{Rubin1987} or
Incremental Mixture Importance Sampling (IMIS) \cite{Raftery2010}.
However, the quality of model-based evidence likely depends
more on appropriate specification of model structure and unbiased parameter priors,
than on more efficient calibration techniques.
%---------------------------------------------------------------------------------------------------
\subsubsection{Evolving Context \& Interventions}\label{model.disc.lim.evo}
A final group of limitations relate to the evolving epidemic context and interventions
which are not captured by the model, including
the \covid pandemic, growing civil unrest, and emerging interventions.
These conditions have largely developed since 2018, and thus
are unlikely to influence the retrospective modelling analyses of later chapters.
Moreover, the model applications in Chapters \ref{foi}~and~\ref{art} are mainly illustrative,
rather than directly tied to specific policy questions for Eswatini.
%---------------------------------------------------------------------------------------------------
\paragraph{\covid}
Many health systems were disrupted by the \covid pandemic, including in Eswatini \cite{EswCOP21}.
Prevention programs were particularly impacted during 2020, including scale-up of
pre-exposure prophylaxis (PrEP) and voluntary medical male circumcision (VMMC),
as well as HIV and viral load testing services \cite{EswCOP21}.
While rapid interventions designed to minimize ART interruption were largely successful,
\covid mortality was high among PLHIV in Eswatini \cite{EswCOP21}.
Additionally, government restrictions aimed at reducing \covid transmission --- including
closing bars, clubs, etc., imposing a nighttime curfew, and travel restrictions \cite{EswIBBS2022}
--- likely also reduced HIV incidence, especially among FSW \cite{Stone2023}.
%---------------------------------------------------------------------------------------------------
\paragraph{Civil Unrest}
As noted in \sref{intro.esw},
democratic freedoms in Eswatini are severely limited by the absolute monarchy \cite{Mthembu2022},
and socioeconomic inequality remains high \cite{Kali2023}.
An ongoing financial crisis and frustration with the political conditions led to
growing pro-democratic protests since 2018,
which grew further with \covid restrictions \cite{Kali2023,Mthembu2022}.
Such protests have been met with violence, including the assassination of
prominent human rights lawyer and activist Thulani Maseko in 2023 January \cite{Maseko2023}.
The trajectory of this unrest is not clear \cite{Maphalala2021},
but the implications for HIV service delivery in the coming years could be substantial.
%---------------------------------------------------------------------------------------------------
\paragraph{New Interventions}
In 2017, Eswatini began a PrEP demonstration project
in 6 rural primary care clinics \cite{NERCHA2018,Barnighausen2019},
and aims to expand PrEP access nationally in the coming years,
with a focus on adolescent girls and young women (AGYW) and FSW \cite{EswCOP21}.
An estimated 25\% of FSW (N = 264) and 8\% of MSM (N = 303) were on PrEP by 2021 \cite{EswIBBS2022}.
The success of these efforts will likely be further improved
with the addition of long-acting injectable PrEP options \cite{Clement2020,Smith2023}.
Similar improvements in viral suppression may also be gained in the coming years
via long-acting injectable ART \cite{Thoueille2022}.
Injectable PrEP and ART can help overcome
many of the structural barriers associated with oral --- \ie daily pill --- regimens,
such as high population mobility and familial power structures
\cite{Dlamini-Simelane2017,Horter2019,Becker2020,Barnighausen2020}.
The current PrEP expansion is also part of the national roll-out of the DREAMS
(Determined, Resilient, Empowered, AIDSFree, Mentored, and Safe) package,
which aims to address multiple HIV vulnerabilities among AGYWs \cite{Saul2018,EswCOP21}.
The current model does not currently include any of these emerging interventions,
but they should be feasible to integrate in the future.
