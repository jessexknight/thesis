\section{Durations}\label{app.math.dur}
%===================================================================================================
\subsection{Exponential Duration Assumption in Compartmental Modedls}\label{app.math.dur.exp}
Let $\lambda$ be the fixed exit rate from compartment $A$, which is assumed to be homogeneous.
Then $D \sim \lambda e^{-\lambda D}$ is % TODO: double check this
the exponentially distributed duration time in the group.
\paragraph{Mean \& Median Duration}
The mean duration is $\mu = 1/\lambda$ and the median is $m = \log(2)/\lambda \approx 0.69\,\mu$.
Thus, if 50\% of individuals progress from compartment $A$ to $B$ by time $\tau$ (median duration),
the exit rate $\lambda$ is given by $\log(2)/\tau$.
\paragraph{Collapsing Compartments in Series}
Let compartments $A$ and $B$ be in series, with exit rates $\lambda_A$ and $\lambda_B$ respectively.
Collapsing $A$ and $B$ into $AB$ will sum the mean durations: $D_{AB} = 1/\lambda_A + 1/\lambda_B$;
thus, the exit rate from $AB$ will be $\lambda_{AB} = 1/(1/\lambda_A + 1/\lambda_B)$.
\paragraph{Collapsing Compartments in Parallel}
Let compartments $A$ and $B$ be in parallel, with exit rates $\lambda_A$ and $\lambda_B$ respectively.
Collapsing $A$ and $B$ into $AB$ will sum the exit rates: $\lambda_{AB} = \lambda_A + \lambda_B$;
thus, the mean duration in $AB$ will be $D_{AB} = 1/(\lambda_A + \lambda_B)$.
%===================================================================================================
\subsection{Estimating Duration in Sex Work from Cross Sectional Data}\label{app.math.dur.xs}
Cross sectional sex work surveys will often ask respondents about their duration in sex work.
These durations might then be taken to be the average durations in sex work;
however, this will be an underestimate,
because respondents will continue selling sex after the survey \cite{Fazito2012}.%
\footnote{An alternate example would be
  to take the mean age of a population as the life expectancy!
  Thanks to Saulius Simcikas and Dr. Jarle Tufto
  for help identifying and discussing this bias:
  \hreftt{stats.stackexchange.com/questions/298828}.}
% This bias parallels challenges to estimating sexual partnership duration \cite{Burington2010}.
\par
Figure~\ref{fig:diagram.dur.xs} illustrates a steady-state population
with 7 women selling sex at any given time.
The steady-state assumption implies that a women leaving sex work after $\delta$ years
will be immediately replaced by a women entering sex work
whose eventual duration will also be $\delta$ years.
Let $\delta$ be this true duration, and $\delta_s$ be the duration reported in the survey.
If we assume that the survey reaches women at a random time point during the duration $\delta$,
then $\delta_s \sim \opname{Unif}(0,\delta)$,
and the mean reported duration is $E(\delta_s) = \frac{1}{2}E(\delta)$.
Thus, $E(\delta) = 2 E(\delta_s)$ would be an estimate of the true mean duration from the sample.
In reality, sex work surveys may be more likely to reach
women who have already been selling sex for several months or years,
due to delayed self-identification as sex worker \cite{Cheuk2020}.
Thus, we would expect that $f = E(\delta) / E(\delta_s) \in (1,2)$,
which we can use to compute the mean exit rate as described in \sref{app.model.math.exp}.
\begin{figure}[h]
  \centering
  \includegraphics[width=\linewidth]{diagram.dur.xs}
  \caption{Illustrative steady-state population of 7 FSW,
    with varying true durations in sex work $\delta$,
    versus the observed durations in sex work $\delta_s$ via cross-sectional survey.}
  \label{fig:diagram.dur.xs}
\end{figure}
\par
Another observation we can make from Figure~\ref{fig:diagram.dur.xs} is that
women who sell sex longer are more likely to be captured in the survey.
That is, while the sampled durations are representative of women who \emph{currently} sell sex,
these durations are biased high versus the population of women who \emph{ever} sell sex.
It's not clear whether this observation is widely understood
and kept in mind when interpreting sex work survey data.