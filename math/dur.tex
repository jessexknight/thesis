\section{Durations}\label{app.math.dur}
%===================================================================================================
\subsection{Exponential Duration Assumption in Compartmental Modedls}\label{app.math.dur.exp}
Let $\lambda$ be the fixed exit rate from compartment $A$, which is assumed to be homogeneous.
Then $D \sim \lambda e^{-\lambda D}$ is % TODO: double check this
the exponentially distributed duration time in the group.
\paragraph{Mean \& Median Duration}
The mean duration is $\mu = 1/\lambda$ and the median is $m = \log(2)/\lambda \approx 0.69\,\mu$.
Thus, if 50\% of individuals progress from compartment $A$ to $B$ by time $\tau$ (median duration),
the exit rate $\lambda$ is given by $\log(2)/\tau$.
\paragraph{Collapsing Compartments in Series}
Let compartments $A$ and $B$ be in series, with exit rates $\lambda_A$ and $\lambda_B$ respectively.
Collapsing $A$ and $B$ into $AB$ will sum the mean durations: $D_{AB} = 1/\lambda_A + 1/\lambda_B$;
thus, the exit rate from $AB$ will be $\lambda_{AB} = 1/(1/\lambda_A + 1/\lambda_B)$.
\paragraph{Collapsing Compartments in Parallel}
Let compartments $A$ and $B$ be in parallel, with exit rates $\lambda_A$ and $\lambda_B$ respectively.
Collapsing $A$ and $B$ into $AB$ will sum the exit rates: $\lambda_{AB} = \lambda_A + \lambda_B$;
thus, the mean duration in $AB$ will be $D_{AB} = 1/(\lambda_A + \lambda_B)$.
%===================================================================================================
\subsection{Estimating Duration in Sex Work from Cross Sectional Data}\label{app.math.dur.xs}
Cross sectional sex work surveys will often ask respondents about their duration in sex work.
These durations might then be taken to be the average durations in sex work;
however, this will be an underestimate,
because respondents will continue selling sex after the survey \cite{Fazito2012}.%
\footnote{An alternate example would be
  to take the mean age of a population as the life expectancy!
  Thanks to Saulius Simcikas and Dr. Jarle Tufto
  for help identifying and discussing this bias:
  \hreftt{stats.stackexchange.com/questions/298828}.}
% This bias parallels challenges to estimating sexual partnership duration \cite{Burington2010}.
\par
Figure~\ref{fig:durs.pop} illustrates a steady-state population
with 7 women selling sex at any given time.
The steady-state assumption implies that a women leaving sex work after $\delta$ years
will be immediately replaced by a women entering sex work
whose eventual duration will also be $\delta$ years.
Let $\delta$ be this true duration, and $\delta_s$ be the duration reported in the survey.
If we assume that the survey reaches women at a random time point during the duration $\delta$,
then $\delta_s \sim \opname{Unif}(0,\delta)$,
and the mean reported duration is $E(\delta_s) = \frac{1}{2}E(\delta)$.
Thus, $E(\delta) = 2 E(\delta_s)$ would be an estimate of the true mean duration from the sample.
In reality, sex work surveys may be more likely to reach
women who have already been selling sex for several months or years,
due to delayed self-identification as sex worker \cite{Cheuk2020}.
Thus, we would expect that $f = E(\delta) / E(\delta_s) \in (1,2)$,
which we can use to compute the mean exit rate as described in \sref{app.math.dur.exp}.
\begin{figure}[h]
  \centering
  \includegraphics[scale=1]{durs.pop}
  \caption{Illustrative steady-state population of 7 FSW,
    with varying true durations in sex work $\delta$,
    \vs the observed durations in sex work $\delta_s$ via cross-sectional survey.}
  \label{fig:durs.pop}
\end{figure}
\par
Another observation we can make from Figure~\ref{fig:durs.pop} is that
women who sell sex longer are more likely to be captured in the survey.
That is, while the sampled durations are representative of women who \emph{currently} sell sex,
these durations are biased high \vs the population of women who \emph{ever} sell sex.
It's not clear whether this observation is widely understood
and kept in mind when interpreting sex work survey data.
%===================================================================================================
\subsection{Quantifying Partnerships}\label{app.math.dur.qp}
% TODO: simplify this as Q = x / (D + \omega), etc...
Similar to \sref{app.math.dur.xs},
sexual partnerships are often quantified using cross-sectional surveys.
In this case, respondents are typically asked to report the numbers of unique partners
during a standardized recall period $\omega$ --- \eg
\shortquote{How many different people have you had sex with during the past year?}
Such data can then be used to inform modelled
rates of partnership change $Q$ and/or numbers of concurrrent partnerships $K$.
\par
If partnership duration is long and the recall period is short
--- including $\omega \approx 0$ for
\shortquote{Are you currently in a long-term sexual partnership?} ---
the reported partnerships mostly reflect \emph{ongoing} partnerships,
and thus $x \approx K$.
If partnership duration is short and the recall period is long,
--- including $D \approx 0$ for
\shortquote{How many one-off sexual partners have you had during the past year?} ---
the reported partnerships mostly reflect \emph{complete} partnerships,
and thus $x/\omega \approx Q$.
However, if partnership duration and recall period are similar in length,
the reported partnerships reflect a mixture of tail-ends, complete, and ongoing partnerships,
and thus $x$ overestimates $K$, but $x/\omega$ also overestimates $Q$.
In summary:
\begin{itemize}
  \item $\omega \ll D$: mostly ongoing partnerships;
  $x \approx K$ (concurrrent)
  \item $\omega \gg D$: mostly complete partnerships;
  $x/\omega \approx Q$ (change rate)
  \item $\omega\,\approx\,D$: some tail-ends, some complete, some ongoing;
  $x > K$, $x/\omega > Q$ (neither)
\end{itemize}
\paragraph{Conceptual Framework}
I developed an approach to estimate $Q$ and $K$ from $x$ and $\omega$.
The approach requires the following assumptions
(although future work could explore more complex/realistic assumptions):
\begin{itemize}
  \item partnership duration $D$ is known
  \item any potential gap $G$ between successive partnerships is known
  \item partnerships are at steady state and exactly equal, including fixed values for $D$ and $G$
  \item timing of the survey/recall period is effectively random with respect to partnerships
\end{itemize}
Let $T = D + G$ be the total period corresponding to one partnership,
such that one ``series'' of back-to-back partnerships contributes, on average,
$D/T$ to the mean number of concurrrent partnerships $K$.
The number of concurrent series $S$ is a conceptual tool for the population overall,
which may include fractional values;
\eg Figure~\ref{fig:durs.obs} illustrates a case with $S = 3.5$.
We can then define:
\begin{figure}
  \centering
  \includegraphics[scale=1]{durs.obs}
  \caption{Illustration of conceptual framework for quantifying partnerships
    from the number reported during a given recall period}
  \label{fig:durs.obs}
  \floatfoot{
    Circle: partnership start; line: ongoing partnership; square: partnership end;
    black: reported partnership; grey: partnership not reported;
    dotted line: fractional partnership for population-level average;
    $s$: partnership series index;
    $\alpha$: partnership series weight for population-level average;
    $\omega$/red: recall period;
    $x$: number of reported partnerships for $\omega$.}
\end{figure}
\begin{alignat}{1}
  Q &= S / T        \label{eq:Q.dur}\\
  K &= S D / T = QD \label{eq:K.dur}
\end{alignat}
Thus, if the number of concurrent series $S$ can be estimated from
the reported partnerships $x$ during recall period $\omega$,
then $Q$ and $K$ can be resolved as in \eqrefs{eq:Q.dur}{eq:K.dur}.
\par
For a single series $s$, the expected number of reported partnerships for the recall period $\omega$
(including tail-ends, complete, and ongoing partnerships)
is defined as $e = \mathbb{E}\,[x\mid s]$.
Drawing on $p(x \mid s,\omega)$ (Figure~\ref{fig:durs.plot}, derivation not shown), we can obtain:
\begin{equation}\label{eq:e.dur}
  e = \mathbb{E}\,[x\mid s] = \frac{D + \omega}{T}
\end{equation}
which is linear with intercept $D/T$ and slope $\omega/T$.
The number of concurrrent series is then estimated as the ratio of reported partnerships
to those expected per series: $S = x/e$.
\begin{figure}
  \centering
  \includegraphics[scale=1]{durs.plot}
  \caption{Probability of reporting $x$ partnerships for a given recall period $\omega$,
    fixed partnership duration $D$, gap between partnerships $G$, and partnership in one series.}
  \label{fig:durs.plot}
  \floatfoot{%
    The function is evidently periodic with period $T = D + G$ and offset $G$;
    the expected value $\mathbb{E}[x \mid s]$ is given in \eqref{eq:e.dur}.}
\end{figure}
\paragraph{Example}
In Figure~\ref{fig:durs.obs},
partnership duration $D = 3$ years, and gap length $G = 1$ year.
An example recall period of $\omega = 6$ years is illustrated,
for which an average of $x = 8$ partnerships are reported.
The expected number of reported partnerships per series is $e = (3+6)/(3+1) = 2.25$;
thus $S = x/e \approx 3.5$,%
\footnote{In fact, the illustrated value of $x = 8$ is slightly higher than
  the expected value $\mathbb{E}[x \mid \omega = 6] = 7.875$,
  due to the arbitrary relative timing of series' shown;
  taking $x = 7.875$, we obtain $S = 3.5$ exactly.}
$Q = S/T = 0.875$, and $K = QD = 2.625$.
Such results can be verified by careful examination of Figure~\ref{fig:durs.obs}.
