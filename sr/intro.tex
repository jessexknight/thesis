\section{Introduction}
As of 2019, two thirds (25.7 million) of all people living with HIV globally
were in Sub-Saharan Africa (SSA), where
an estimated one million new HIV infections were acquired in 2019 \cite{AIDSinfo}.
In SSA and elsewhere, HIV treatment via antiretroviral therapy (ART)
remains a key element of combination HIV prevention \cite{WHO2016art}.
\par
Eligibility to initiate ART has seen continued expansion over the years
--- \ie earlier and earlier initiation during HIV disease ---
following evidence of individual-level health benefits \cite{Lundgren2015,Danel2015}
and partner-level prevention benefits \cite{Anglemyer2013,Cohen2016}.
Expansion cumulated with the current recommendation of
immediate ART following HIV diagnosis, or ``universal test-and-treat'' \cite{WHO2016art}.
Parallel to ART expansion, interest has grown in estimating
the population-level prevention benefits of ART, via both
model-based studies \cite{Granich2009,Eaton2012,Delva2012,Cori2014} and
recent large-scale community-based trials \cite{Havlir2019,Hayes2019,Iwuji2018}.
Mixed results from these trials \cite{Havlir2019,Hayes2019,Iwuji2018}
have renewed interest in understanding potential determinants of
population-level ART prevention benefits \cite{Baral2019,Havlir2020}.
\par
Risk heterogeneity is a well-established determinant of
HIV epidemic emergence and persistence \cite{Anderson1986,Boily1997}.
Such heterogeneity is defined by various factors affecting acquisition and onward transmission risk.
Systematic model comparison studies have found that projected prevention impacts of ART scale-up
were smaller when more heterogeneity was captured in the model \cite{Hontelez2013,Rozhnova2016}.
Moreover, populations experiencing barriers to viral suppression
may be at highest risk for acquisition and onward transmission, including key populations such as
women and men who sell sex, and men who have sex with men \cite{Hakim2018,Nyato2019}.
Data also suggest that subgroups not formally described as key populations,
such as youth, men who have sex with women, including clients of sex workers, may also
experience barriers to engagement in ART care \cite{Arnesen2017,Chikwari2018,Quinn2019}.
Indeed, data suggest that ART scale-up in practice
has not reached subgroups equally \cite{Green2020}.
\par
Given the critical role of transmission modelling
in estimating the prevention impacts of ART \cite{Eaton2012,Delva2012},
we sought to examine how heterogeneity in risk and ART uptake has been represented
in mathematical models used to assess the prevention impacts of ART scale-up in SSA.
We conducted a scoping review and ecological regression with the following objectives.
Among non-linear compartmental models of sexual HIV transmission
used to simulate the prevention impacts of ART in SSA:
\begin{enumerate}
  \item\label{sr.rq.1}
    In which epidemic contexts (geographies, populations, epidemic phases)
    have these models been applied?
  \item\label{sr.rq.2}
    How was the model structured to represent key factors of risk heterogeneity?
  \item\label{sr.rq.3}
    What are the potential influences of representations of risk heterogeneity
    on the projected prevention benefits of ART in the overall population?
\end{enumerate}