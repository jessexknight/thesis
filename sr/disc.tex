\section{Discussion}\label{sr.disc}
Model-based evidence continues to support
evaluation and mechanistic understanding of ART prevention impacts.
Such evidence may be sensitive to modelling assumptions about risk heterogeneity.
Via scoping review, we found that stratification by sexual activity and key population(s)
was considered in approximately two-thirds and two-fifths of studies to date, respectively;
one-third considered risk group turnover and one-quarter considered differential ART cascade by any risk group.
In multivariable ecological analysis, we found that
projected incidence reductions and proportions of infections averted were influenced by
risk heterogeneity when risk group turnover and differential ART cascade were also considered.
\par
Our findings suggest that the proportion of onward transmission prevented through ART
may be reduced via turnover.
Data suggest considerable within-person variability in sexual risk among key populations,
including MSM, FSW, and clients of FSW \cite{Fazito2012,Romero-Severson2012,Roberts2020},
as well as in the wider population \cite{Houle2018}.
This risk variability is often reflected in compartmental models as risk group turnover.
Previous modelling suggested that
turnover could \emph{increase} the prevention benefits of treatment \cite{Henry2015};
however, the model in \cite{Henry2015} was calibrated to overall equilibrium prevalence,
allowing the reproduction number to decrease with increasing turnover.
By contrast, when calibrating to group-specific prevalence with turnover,
greater risk heterogeneity is inferred with \vs without turnover,
and the reproduction number may actually increase \cite{Knight2020}.
Turnover of higher risk groups can also reduce ART coverage in those groups through
net outflow of treated individuals, and net inflow of susceptible individuals,
some of whom then become infected \cite{Knight2020}.
Thus, mechanistically, turnover could reduce the transmission benefits of ART.
These findings suggest that turnover is important to capture as part of modeling risk heterogeneity,
and as such, models would benefit from surveys, cohorts, and repeated population size estimates
that can provide data on individual-level trajectories of sexual risk,
such as duration in sex work \cite{Watts2010}.
\par
Most models assumed equal ART cascade transition rates across subgroups,
including diagnosis, ART initiation, and retention.
However, recent data suggest differential ART cascade by sex, age, and key populations
\cite{Lancaster2016,Stannah2019,Ma2020,Green2020}.
These differences may stem from the unique needs of subgroups
and is one reason why differentiated ART services are a core component of HIV programs
\cite{Chikwari2018,Ehrenkranz2019}.
Moreover, barriers to ART may intersect with transmission risk, particularly among key populations,
due to issues of stigma, discrimination, and criminalization \cite{Ortblad2019,Baral2019}.
Our ecological analysis estimated that
differences in cascade by sex (lower among men) or risk (key populations prioritized)
had a large influence on projected ART prevention benefits.
Thus, opportunities exist to incorporate differentiated cascade data,
examine the intersections of intervention and risk heterogeneity, and
to consider the impact of HIV services as delivered on the ground.
Similar opportunities were noted regarding modelling of pre-exposure prophylaxis in SSA \cite{Case2019}.
Depending on the research question, the modelled treatment cascade may need
to include more cascade steps and states related to treatment failure/discontinuation.
\par
The next generation of ART prevention impact modelling can be advanced by leveraging
rapid growth in data on risk heterogeneity and its intersection with intervention heterogeneity
\cite{Beyrer2012,Baral2012,Mishra2016}.
Key populations often reflect intersections of risk heterogeneity with turnover,
and intervention heterogeneity (cascade differences),
which together suggest the unmet needs of key populations
play an important role in the overall dynamics of HIV transmission in SSA \cite{Bekker2015,Stone2021}.
Although none of the models in the review considered a lower ART cascade among key populations,
data suggest large cascade differences, most notably lower proportions across the cascade,
among key populations in SSA \cite{Mountain2014sr,Hakim2018,Stannah2019}.
Similarly, we found that the number of modelled clients per female sex worker, and
the relative rate of partnership formation among female sex workers \vs other women
did not always reflect available data syntheses for sex work \cite{Watts2010,Scorgie2012}.
Among studies with different partnership types, only 1/5 modelled
main/spousal partnerships---with more sex acts/lower condom use---between two higher risk individuals,
while 4/5 modelled only casual/commercial partnerships among higher risk individuals.
However, data suggest that female sex workers form main/spousal partnerships
with regular clients and boyfriends/spouses from higher risk groups \cite{Scorgie2012}.
Improved modelling and prioritization of sevices designed to reach key populations
will rely on continued investment in community-led data collection for hard-to-reach populations.
\par
Our scoping review has several limitations.
First, we examined key populations based as traditionally defined \cite{WHO2016kp},
based on social and economic marginalization and criminalization in SSA,
and future work would benefit from examing risk heterogenetiy across more subgroups,
such as mobile populations and adolescent girls and young women,
where data suggest cascade disparities and risk heterogeneity \cite{Tanser2015,Dellar2015}.
Second, our conceptual framework for risk heterogeneity did not explicitly examine
heterogeneity related to anal sex, which is associated with higher probability of HIV transmission;
nor did we examine structural risk factors like violence \cite{Silverman2011,Baggaley2013}.
Third, we did not extract whether models were calibrated,
and if so, which parameters were fixed \vs fitted.
If certain parameters were fitted, it could explain some counterintuitive effect estimates.
For example, models with \vs without increased infectiousness in late-stage HIV
might infer lower earlier-stage infectiousness through model fitting,
such that overall infectiousness is roughly the same.
Then, when simulating ART scale-up to individuals with earlier-stage HIV,
the estimated prevention benefit could be relatively lower.
A similar mechanism could explain increased ART prevention impacts when including acute infection.
Importantly, we conducted an ecological analysis,
and within-model comparisons like \cite{Dodd2010,Hontelez2013} that explore
the influence of each key factor identified in this review would be an important next step.
\par
In conclusion, model-based evidence of ART prevention impacts could likely be improved by:
1) capturinig risk heterogeneity with risk group turnover,
   as a determinant of inferred risk heterogeneity during model calibration, and
   to reflect challenges to maintaining ART coverage among risk groups with high turnover;
2) integrating data on differences in ART cascade between sexual risk groups,
   to reflect services as delivered on the ground; and
3) capturing heterogenetiy related to key populations,
   to reflect intersections of transmission risk and barriers to HIV services
   that may undermine the prevention benefits of ART.