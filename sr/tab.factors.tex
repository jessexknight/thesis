\begin{tabular}{llp{.33\linewidth}p{.4\linewidth}}
  \toprule
  \textbf{Factor}
& \textbf{MP\tn{a}}
& \textbf{Definition}
& \textbf{Possible mechanism(s) of influence on ART prevention impact}
\\
\midrule
  Acute Infection
& $\beta_i$
& Increased infectiousness immediately following infection \citep{Hollingsworth2008,Boily2009}
& \textbf{Biological}: transmissions during acute infection are unlikely to be prevented by ART
\\
  Late-Stage Infection
& $\beta_i$
& Increased infectiousness during late-stage infection \citep{Hollingsworth2008,Boily2009}
& \textbf{Biological}: transmissions during late-stage are more likely to be prevented by ART
\\
  Drug Resistance
& $\beta_i$
& Transmitted factor that requires regimen switch to achieve viral suppression \citep{DeWaal2018}
& \textbf{Biological}: transmissions during longer delay to achieving viral suppression will not be prevented by ART
\\
\midrule
  HIV Morbidity
& $Q$; $A$
& Reduced sexual activity during late-stage disease \citep{Myer2010,McGrath2013}
& \textbf{Behaviour Change}: reduced morbidity via ART could increase HIV prevalence among the sexually active population
\\
  HIV Counselling
& $Q$; $A$; $\alpha$
& Reduced sexual activity and/or increased condom use after HIV diagnosis \citep{Tiwari2020}
& \textbf{Behaviour Change}: increased HIV testing with ART scale up can contribute to prevention even before viral suppression is achieved
\\
\midrule
  Activity Groups
& $Q$; $\alpha$
& Any stratification by rate of partnership formation \citep{Anderson1991}
& \textbf{Network}: higher transmission risk among higher activity
\\
  Age Groups
& $Q$; $\alpha$
& Any stratification by age
& \textbf{Network \& Cascade}: higher transmission risk and barriers to viral suppression among youth \citep{Birdthistle2019,Green2020}
\\
  Key Populations
& $Q$; $\alpha$
& Any epidemiologically defined higher risk groups \citep{WHO2016KP}
& \textbf{Network \& Cascade} higher transmission risk and barriers to viral suppression among key populations \citep{Hakim2018}
\\
  Group Turnover
& $\theta$
& Individuals move between activity groups and/or key populations reflecting sexual lifecourse \citep{Watts2010}
& \textbf{Network \& Cascade}: counteract effect of stratification due to shorter periods in higher risk \citep{Knight2020};
  viral suppression may be achieved only after periods of higher risk
\\
  Assortative Mixing
& $\Phi$
& Any degree of assortative mixing (like-with-like) by age, activity, and/or key populations
& \textbf{Network}: assortative sexual networks compound effect of stratification \citep{Anderson1991}
\\
  Partnership Types
& $A$; $\alpha$
& Different partnership types are simulated, with different numbers of sex acts and/or condom usage \citep{Scorgie2012}
& \textbf{Network}: longer duration and lower condom use among main versus casual/sex work partnerships
  counteracts effect of stratification
\\
  ART Cascade Gaps
& $\delta$; $\tau$
& Slower ART cascade transitions among higher activity groups or key populations \citep{Hakim2018,Green2020}
& \textbf{Cascade}: ART prevention benefits may be allocated differentially among risk groups
\\
\bottomrule
\end{tabular}
\floatfoot{
  \tnt[a]{MP: Model Parameters ---
  $\beta_i, \beta_s$: transmission probability per act (infectiousness, susceptibility);
  $A$:         number of sex acts of each type per partnership;
  $\alpha$:    proportion of sex acts unprotected by a condom;
  $Q$:         partnership formation rate;
  $\Phi$:      mixing matrix (probability of partnership formation);
  $\mu$:       mortality rate;
  $\nu$:       entry rate;
  $\theta$:    internal turnover between activity groups;
  $\delta$:    diagnosis rate;
  $\tau$:      ART initiation rate (and retention-related factors).
}}