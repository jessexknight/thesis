I owe many thanks at the end of this PhD adventure:
\begin{acknowlist}
\ack{to Sharmistha}
    {for teaching me the art and science of transmission modelling,
     for supporting my personal and professional growth in countless ways, and
     for having nothing but enthusiasm even and especially when I notice another potential problem in prior models}
\ack{to Kristy}
    {for keeping our team sane and supported, amidst two pandemics}
\ack{to Huiting and Linwei}
    {for tutoring me in biostatistics, epidemiology, and the habit of getting things~done}
\ack{to Korryn and Alex}
    {for sharing with me in the delights and despairs of coding the world in a box}
\ack{to Ekta}
    {for reminding me of the many paths we may choose between to make the world a better place}
\bigskip
\ack{to Stef, Amrita, Sheree, Kate, and many others}
    {for tireless, fearless, data-driven advocacy for and with marginalized communities globally}
\ack{to Bheki, Laura, Sindy, and Zandi}
    {for sharing with us your expertise and guidance to understand sex~work and HIV within Eswatini}
\ack{to Mike}
    {for introducing me to the Bayesian tools to meld imperfect data with expert knowledge}
\ack{to Rupert}
    {for passing on your calm wisdom, and plugging the rabbit holes so I could graduate on~time}
\bigskip
\ack{to the HIV modelling community, including Marie-Claude, Mathieu, Romain, Mike, Leigh, and many others}
    {for constant ambition to generate high-quality model-based evidence, at the intersection of several challenging disciplines}
\ack{to the Linux, \LaTeX, Python, and R open source communities}
    {for building nearly every tool I have used}
\ack{to the many Stack Exchange communities and SciNet}
    {for teaching me how to actually use these tools}
\ack{but not to my keyboard}
    {whose ``a'' and ``d'' keys failed me, in the final days of writing}
\bigskip
\ack{to the Natural Sciences and Engineering Research Council of Canada, the Government of Ontario, and the University of Toronto}
    {for directly funding this work}
\ack{to the MAP Centre for Urban Health Solutions, Unity Health Toronto, and the University}
    {for supporting the work and my personal growth in numerous ways}
% TODO: RTP ?
\bigskip
\ack{to Noam Chomsky, Amy Goodman, Chris Hedges, John Mearsheimer, Louis Rossmann, and Rollie Williams}
    {for showing me where to look for evil, when it hides in plain sight}
\ack{to Andrew Huberman}
    {for sharing ``science-based strategies'' to control my monkey brain}
\ack{to Attia}
    {for converting me from an instant coffee barbarian to an espresso connoisseur}
\ack{to Aaron and Andrea}
    {for sharing with me baby brews and byzantine board games}
\ack{to Brayden}
    {for sharing with me several homes, and a monk when I needed it}
\ack{to my parents}
    {for showing me love, discipline, and the Language of the Universe}
\ack{and to Ali}
    {for showing me love, patience, and a vision for a meaningful life, in and outside of work.}
\end{acknowlist}
\clearpage
I would also like to acknowledge the lands on which I worked to produce this thesis.
These lands include the traditional, ancestral, and unceded territories of the
Mississaugas of the Credit, Chippewa, Huron-Wendat, Seneca, % Toronto
Anishinaabe, Haudenosaunee, % Kingston
and Musqueam peoples. % Vancouver
These places are still the home to many Indigenous people from across Turtle Island,
and I am grateful to have the opportunity to live, learn, and work on these lands.
\par
Finally, I would like to reflect and comment on my positionality with respect to HIV research and practice.
I benefit from many dimensions of privilege, being:
white, male, cisgender, heterosexual, English-speaking, neurotypical, able-bodied, and athiest.
I was born and raised in a highly supportive upper middle class Canadian family,
and I have had numerous opportunities to travel, learn, and follow my interests.
I recognize that my lived experiences are, in fact, quite distant from those that
afford a deeper understanding of the HIV epidemic, in Eswatini and elsewhere.
I have never experienced poverty, stigma, oppression, or violence.
I have never been to Eswatini and only travelled to Sub-Saharan Africa once,
for two months in 2015 as part of the Engineering World Health Summer Institute in Rwanda.
I have never delivered health services,
though I am fortunate to work with many who have.
\par
Nevertheless I am inspired by
people living with HIV, families, friends,
brave leaders and advocates of marginalized communities,
healthcare providers and public health professionals
who have been fighting for decades against HIV.
I believe that these efforts can end the HIV epidemic in our lifetime,
provided they remain responsive to emergent risks and persistent prevention gaps.
I believe that these risks and gaps can be difficult to identify and are easily overlooked,
but that they often derive from intersecting upstream inequities.
I believe that quantitative data about these inequities
cannot be understood without qualitative insights,
and that the benefits of addressing these inequities
extend beyond biomedical definitions of wellness.
Thus, while my thesis highlights the importance of
HIV prevention for female sex workers and their clients in Eswatini,
I perceive this work as part of broader efforts to
maintain health equity at the fore of the HIV epidemic response.
