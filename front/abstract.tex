\par % intro
Compartmental HIV transmission models help guide global, national, and local epidemic responses
via detailed projections, hypothetical scenarios, and mechanistic inference.
While such models must make simplifying assumptions due to time and data constraints,
prior work has shown that these assumptions can sometimes bias model outputs.
In this thesis, I re-examine common assumptions used in compartmental HIV transmission models,
and explore their potential influence on key model outputs.
I focus on disproportionate risk among female sex workers,
within the high-prevalence HIV epidemic of Eswatini.
\par % methods + results
First, I systematically review how differential risk has been captured in
prior models exploring HIV treatment scale-up across Sub-Saharan Africa.
Among 94 studies, I find that only 2/5 explicitly included sex work,
and only 1/4 considered differences in treatment access across risk groups.
Next, I design, parameterize, and calibrate a model of heterosexual HIV transmission in Eswatini.
The model features 4 partnership types and 8 risk groups, including
higher/lower risk female sex workers and their clients.
While parameterizing the model, I develop several new adjustments
for common sources of bias in sexual behaviour data.
I also critically review existing ``force of infection equation'' approaches,
used to model HIV transmission via sexual partnerships within compartmental models,
with respect to their implicit assumptions, data needs, strengths, and limitations.
Drawing on this review, I develop a new approach
--- the \emph{Effective Partnerships Adjustment} ---
which can overcome some of the key limitations of existing approaches.
Comparing the new and existing approaches in the Eswatini model,
I show that some existing approaches might be systematically
overestimating the impact of prevention in longer partnerships, and
underestimating the impact of prevention in shorter partnerships.
Finally, I apply the Eswatini model to further explore
scenarios with different HIV treatment scale-up across risk groups.
I find that the prevention impacts of treatment could be substantially reduced
if higher risk groups are ``left behind''.
\par % conclusion + significance
Taken together, these results suggest that existing compartmental HIV transmission models
may underestimate the importance of prioritizing resources to
populations at highest risk of HIV acquisition and/or transmission,
including female sex workers, even within a high prevalence epidemic such as Eswatini.
