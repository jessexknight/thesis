\section{Introduction}\label{art.intro}
Early HIV treatment via antiretroviral therapy (ART) offers
numerous individual-level health benefits \cite{Gabillard2013,Maartens2014,Danel2015,Lundgren2015},
and can prevent transmission in serodiscordant partnerships \cite{Anglemyer2013,Cohen2016,Rodger2019}.
As such, immediate initiation of ART has been recommended by WHO since 2016 \cite{WHO2016art}.
Alongside expanding ART eligibility,
interest has grown in the potential population-level prevention impacts of ART, motivating
numerous modelling studies \cite{Granich2009,Eaton2012,Eaton2014art,Knight2022sr} and
several large community-based trials \cite{Makhema2019,Havlir2019,Hayes2019,Iwuji2018}
of ART scale-up as ``treatment as prevention'', especially within Sub-Saharan Africa.
In general, the prevention impacts estimated via these trials
have not met expectations from modelling, prompting questions about
the potential influence of modelling assumptions on predictions \cite{Baral2019}.
\par
Within these modelling studies, ART prevention impacts are usually quantified as
incidence rate reduction or cumulative infections averted
in scenarios with higher cascade attainment \vs scenarios with lower attainment \cite{Knight2022sr}.
Modelled populations are often stratified by risk,
including key populations like FSW and their clients,
to capture important epidemic dynamics related to risk heterogeneity
\cite{Stigum1994,Garnett1996,Watts2010}.
However, these studies almost always assume that cascade
attainment (\ie proportions of PLHIV who are diagnosed, treated, and virally suppressed) or
progression (\ie rates of diagnosis, treatment initiation, and treatment failure/discontinuation)
are equal across modelled risk groups.
For example, among the modelling studies reviewed in Chapter~\ref{sr},
key populations were usually assumed to have ``average'' cascade progression,
or ``above average'' progression in some scenarios, but never ``below average''.
\par
Yet, there is growing evidence of differential ART cascade across population strata,
including age, gender, mobility, and risk \cite{Hakim2018,Green2020}.
These differences can be driven by
unique barriers to engagement in care faced by vulnerable populations,
which intersect with drivers of HIV risk \cite{Wanyenze2016,Schwartz2017,Schmidt-Sane2022}.
Moreover, the lowest cascades likely remain unmeasured \cite{Hakim2018,Boothe2021}.
These intersections of risk and cascade heterogeneity
could potentially undercut the prevention impacts of ART scale-up
anticipated from model-based evidence \cite{Baral2019}.
Therefore, I sought to examine the following questions
in an illustrative modelling analysis:
\begin{enumerate}
  \item\label{obj:art.1} How are projections of ART prevention impacts
    influenced by differences in ART cascade across risk groups?
  \item\label{obj:art.2} Under which epidemic conditions
    do such differences have the largest influence?
\end{enumerate}
I examined these questions using the Eswatini model from Chapter~\ref{model},
focusing on differential risk related to sex work.
Eswatini has recently achieved outstanding cascade gains, surpassing 95-95-95
(see \sref{model.par.cascade}) \cite{Walsh2020,AIDSinfo}.
As such, I used observed ART scale-up in Eswatini as a \emph{base case}
reflecting evidently attainable scale-up,
and explored \emph{counterfactual} scenarios in which scale-up was slower,
and where specific risk groups could have been ``left behind''.
