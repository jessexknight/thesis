\section{Discussion}\label{art.disc}
I sought to explore how intersections of risk heterogeneity and differential ART coverage
may influence model-estimated prevention impacts of ART.
I found that ART scale-up that ``leaves behind'' higher risk groups,
such as female sex workers (FSW) and their clients,
can result in substantially more HIV infections,
even for the same population-overall coverage.
I also found that the transmission impact of leaving behind higher risk groups
generally increased with:
the size of the risk group,
the size of their predominant partner group (\ie clients for FSW and FSW for clients),
and the rate of turnover among their predominant partner group.
\par
% TODO: (!) add paragraph re. significance for Eswatini context before broadening
Although my analysis only considered Eswatini,
my findings are likely generalizeable to other epidemic contexts.
In fact, HIV prevalence ratios between key populations and the population overall
are relatively low in Eswatini \vs elsewhere \cite{Baral2012,Hessou2019};
thus, the transmission impact of cascade gaps among key populations in other contexts
would likely be even greater than I found for Eswatini.
Moreover, as HIV incidence declines in many settings,
epidemics may become re-concentrated among key populations \cite{Brown2019,Garnett2021},
further magnifying the transmission impact of cascade disparities.
\par
To my knowledge, this study is the first to explore the transmission impact of
heterogeneity in ART coverage across risk groups, within consistent population-overall coverage.
In my review of mathematical modelling of ART scale-up in Sub-Saharan Africa (Chapter~\ref{sr}),
I found that few studies have considered any cascade differences by risk group,
but that such differences likely mediate ART prevention impacts \cite{Knight2022sr}.
Cascade gaps have been observed among men \vs women \cite{Quinn2019,Green2020},
younger \vs older populations \cite{Green2020,Lebelonyane2021},
key populations \vs the population overall \cite{Hakim2018},
and within key populations themselves \cite{Mayanja2018,Jaffer2022}.
Moreover, unmeasured cascades
--- such as among populations who have not been reached by programs and interventions ---
are likely lowest \cite{Hakim2018,Boothe2021}.
Consistent integration of these data going forward could
improve the quality of model-based evidence for HIV resource prioritization.
\par
Global ART scale-up has many benefits, including for
individual-level health outcomes \cite{Gabillard2013,Lundgren2015},
prevention in serodiscordant relationships \cite{Cohen2016},
and contributing to population-level incidence declines \cite{Havlir2020}.
However, efforts to maximize cascade coverage should not overlook
populations that may be harder to reach,
where barriers to engagement in HIV care often intersect with drivers of HIV risk
\cite{Wanyenze2016,Schwartz2017,Schmidt-Sane2022,Baral2019}.
Such populations can be reached effectively through
tailored services to meet their unique needs \cite{Ehrenkranz2019},
services which can be designed and refined with ongoing community engagement
\cite{Chikwari2018,Mlambo2019,Comins2022}.
As I have shown, an equity-focused approach to ART scale-up can maximize prevention impacts,
and accelerate the end of the HIV epidemic.
\par
My analysis above has three major strengths.
First, drawing on my conceptual framework for risk heterogeneity (Table~\ref{tab:sr.factors})
and multiple sources of context-specific data \cite{SDHS2006,SHIMS1,Justman2016,Baral2014,EswKP2014},
I captured several dimensions of risk heterogeneity, including:
heterosexual anal sex,
four types of sexual partnerships,
sub-stratification of FSW and clients into higher and lower risk,
and risk group turnover.
Second, whereas most modelling studies of ART scale-up
project hypothetical future scenarios which may not be achievable,
my base case scenario reflects observed scale-up in Eswatini.
Finally, my analytic approach to objective \ref{obj:art.2},
in which epidemic conditions are conceptualized as potential effect modifiers,
represents a unique methodological contribution to the HIV modelling literature.
\par
My analysis above has three main limitations.
First, I only considered heterosexual HIV transmission in Eswatini,
and mainly explored risk heterogeneity related to sex work.
However, my findings would likely generalize
to other transmission networks and determinants of risk heterogeneity,
including risk groups not always recognized as key populations,
such as mobile populations and young women \cite{Tanser2015,Cheuk2020}.
Second, I did not consider transmitted drug resistance.
Drug resistance is more likely to emerge
in the context of barriers to viral suppression \cite{Pham2014};
thus, cascade gaps among those at higher risk
would likely accelerate emergence of transmitted drug resistance, and amplify impacts of such gaps.
Third, even among the top 1\% of model fits,
substantial uncertainty remained in the values of inferred parameters,
yielding wide confidence intervals in the outputs of interest (additional infections and incidence).
In the absence of additional data,
such intervals reflect true uncertainty in the magnitude of these outputs,
though more advanced calibration techniques could potentially improve precision.
\par
In conclusion, HIV prevention efforts should be rooted in
context-specific understandings of prevention gaps.
In the ``treatment as prevention'' era, prevention gaps include cascade gaps.
Thus, differential cascades within and between populations at higher risk of HIV
must be described, modelled, and ultimately addressed
to fully realize the anticipated prevention impacts of ART.
