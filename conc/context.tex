\section{Results in Context}\label{conc.context}
Global HIV incidence has been declining since the late 1990s,
with the largest reductions coming across Sub-Saharan Africa (SSA) \cite{Joshi2021,AIDSinfo}.
Global HIV prevalence has simultaneously plateaued,
as PLHIV continue to live longer with advances in ART and clinical care \cite{AIDSinfo}.
Yet, these trends are not consistent within or between contexts,
and service gaps threaten to prolong the epidemic and sustain health inequities
\cite{Akullian2017,Tanser2018,Ortblad2019,Barr2021,Joshi2021,AIDSinfo}.
These realities underscore the importance of
identifying, understanding, and ultimately addressing differences in HIV risk,
including through models with precise assumptions.
%---------------------------------------------------------------------------------------------------
\paragraph{Growing \& Shrinking Epidemics}
While HIV incidence is decreasing across much of SSA, new infections are estimated to be
increasing in the Middle East, North Africa, Eastern Europe, and Central Asia \cite{AIDSinfo}.
Several modelling studies have highlighted how
prioritizing resources to populations at highest risk of HIV
is most efficient at reducing infections,
especially during periods of epidemic growth \cite{Boily2002,Silhol2021,Stone2021}.
Results throughout this thesis further support this conclusion.
Thus, the ability of models to accurately reflect these populations
and quantify the relative impact of their unmet needs
will be essential for guiding efficient response within these growing epidemics.
Moreover, in contexts with decreasing overall transmission,
new infections are expected to ``re-concentrate'' among populations at highest risk
\cite{Akullian2017,Ortblad2019,Barr2021,Joshi2021}
--- reflecting analogous conditions to growing epidemics.
Thus, within declining epidemics too,
modelling assumptions must explicitly reflect pockets of residual risk,
and remain responsive to emerging data on prevention gaps.
%---------------------------------------------------------------------------------------------------
\paragraph{Treatment as Prevention}
The additional role of ART in preventing HIV transmission offers numerous benefits
at the individual, partnership, and community levels.
However, the results of major trials and recent model validation studies suggest that
anticipated community-level prevention benefits of ART have potentially been overestimated
\cite{Eaton2015,Baral2019,Havlir2020}.
Chapters \ref{sr}~and~\ref{art} offer evidence that this overestimation could arise from
failure to consider intersections of HIV risk and barriers to care.
Such intersections may derive from upstream structural factors
\cite{Beyrer2012,Baral2013,Shannon2015,McBride2021},
but also from inherent limitations of treatment as prevention \vs other prevention tools.
For example, short delays in HIV diagnosis, ART initiation, and viral suppression may be inevitable,
during which time some risk of onward transmission remains,
notably in the context of acute infection \cite{Cohen2012}.
Thus, rates of HIV testing and movement along the ART cascade will need to remain high
--- especially among populations at highest risk of onward transmission ---
in order to maximize ART prevention benefits.
These efforts should then be complemented by continued investment in prevention tools,
including condom provision, risk communication, structural interventions, and multiple PrEP options
\cite{Eisinger2019tk}.
%---------------------------------------------------------------------------------------------------
\paragraph{HIV in Eswatini}
Efforts to maximize ART prevention impacts could be guided by
the highly successful \emph{MaxART} programs in Eswatini,
which featured comprehensive and community-led initiatives to reach all populations,
especially youth and men, whose cascades are often lower \cite{MaxART1,MaxART2,Green2020}.
Notably, these and/or other programs appear to have reached
Swati FSW and MSM living with HIV too \cite{EswIBBS2022}
--- whose ART cascades might now be among the highest in the world \cite{AIDSinfo}.
Alongside \emph{MaxART}, Eswatini has also expanded
the number and coverage of prevention tools available \cite{NERCHA2014,NERCHA2018},
informed by context-specific epidemiological drivers of transmission,
including multiple structural factors and unmet prevention needs among key populations.%
\footnote{Though sex work, same-sex sex, and posessing drugs
  regrettably remain illegal \cite{UNAIDS2022lpa}.}
As a result, HIV incidence in Eswatini is estimated to have declined substantially in recent years
\cite{SHIMS1,SHIMS2,AIDSinfo}.
I am hopeful that these efforts will continue to rapidly and equitably end
the HIV epidemic in Eswatini and around the world.
