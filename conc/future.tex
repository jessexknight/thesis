\section{Recommendations \& Future Work}\label{conc.future}
This thesis contributes new insights about the influence of several common modelling assumptions,
as well as methods to facilitate alternate assumptions which can be more plausible and/or flexible.
These contributions can therefore support development of
best practices for HIV transmission modelling, while also
opening new avenues of research to validate and further explore the alternate assumptions.
%===================================================================================================
\subsection{Validation \& Model Comparison}\label{conc.future.val}
Chapter~\ref{model} presents several
incremental variations on existing methods for model design and parameterization,
but does not explore the influence on model outputs of each variation \vs conventional methods.
Similarly, Chapter~\ref{foi} includes
preliminary comparison of different force of infection approaches,
but the proposed \emph{Effective Partnerships Adjustment} approach should be further validated.
This section offers hypotheses about
the potential influence of these incremental model developments,
and an outline potential analyses to test these hypotheses.
%---------------------------------------------------------------------------------------------------
\paragraph{Substratified Key Populations}
The Eswatini model substratified FSW and clients into higher \vs lower risk,
whereas FSW and clients are each typically modelled as a single risk group.
This additional substratification is then expected to
more fully capture effects of risk heterogeneity (see \sref{intro.model.het}).
Future work could compare models with \vs without this substratification
--- plus different definitions of substrata ---
across different epidemic contexts and model applications,
similar to analyses in \cite{Hontelez2013,Bernard2017,Suen2017}.
These results could then guide the design of new models and model updates,
especially for multi-context/multi-purpose models such as
Goals \cite{Stover2014}, Optima \cite{Kerr2015}, and the like.
%---------------------------------------------------------------------------------------------------
\paragraph{Parntership Types}
Similar work could explore the influence of defining partnership types
by the risk groups involved \vs as overlapping types
(\eg whether main/spousal partnerships can form between FSW and clients),
and which partnership types are considered
(\eg whether repeat \vs one-off sex work partnerships are distinguished).
Failure to model main/spousal partnerships between FSW and clients
could underestimate core group dynamics, while
failure to distinguish repeat \vs one-off sex work partnerships could have varied effects,
since repeat sex work partnerships are likely to have
lower condom use but more post-transmission sex acts.
Future work should also explore explicit representations of transactional sex
--- and different typologies thereof \cite{Fielding-Miller2016} ---
given the hypothesized importance of these partnerships
for HIV transmission in SSA \cite{Wamoyi2016}.
%---------------------------------------------------------------------------------------------------
\paragraph{Force of Infection}
% TODO: (?)
%===================================================================================================
\subsection{Data Considerations}\label{conc.future.data}
Measuring sexual behaviour is notoriously difficult due to
sampling, participation, recall, and reporting biases \cite{Fenton2001},
as well as intricacies in the interpretation of responses.
Yet, because sexual behaviour data are often
mechanistically incorporated into HIV transmission models,
accurate estimation and interpretation of these data remains a critical step in model development.
%---------------------------------------------------------------------------------------------------
\paragraph{Data Sources}
In many contexts, household-based surveys, such as
Demographic and Health Surveys (DHS) \cite{DHS} and
Population-based HIV Impact Assessment (PHIA) surveys \cite{PHIA},
are the main source of HIV epidemiological data to support transmission modelling.
Such surveys are likely to underestimate
the sizes of communities and prevalence of behaviours that are stigmatized and/or criminalized
\cite{Fenton2001,Mishra2008,Lowndes2012,Behanzin2013}.
While unique sampling methods have helped
reach conventional key populations with tailored surveys \cite{UNAIDS2010kps,UNAIDSKPA},
additional surveys are likely needed among other priority populations,
including mobile populations and young women \cite{Akullian2017,Camlin2019,Cheuk2020}.
Simultaneously, complementary general population surveys with alternate delivery modes
--- \eg polling booth designs \cite{Lowndes2012,Behanzin2013} and others \cite{Langhaug2010} ---
could help quantify potential biases associated with household-based survey data.
% TODO: (?) something about programmatic data
%---------------------------------------------------------------------------------------------------
\paragraph{Dynamic Risk}
Adding to previous work \cite{Henry2015,Knight2020},
this thesis highlights the importance of modelling activity group ``turnover''
--- \ie dynamic individual-level risk.
However, the conceptualization of turnover in the Eswatini model
is limited by the lack of age stratification, particularly considering
the growing data on unique HIV vulnerabilities among young women \cite{Wamoyi2016,Cheuk2020,Ma2020}.
Therefore, future models could explore age-related trajectories of sexual risk
for key and priority populations, similar to modelling of MSM in \cite{Rozhnova2019,Basten2021}.
Such models could also help quantify, for example,
the proportions of prevalent infections among lower activity groups
that were actually acquired during priods of higher activity.
Data to support these models could be obtained from
cohort studies and/or cross-sectional surveys with appropriate questions
\cite{McKinnon2014,McKinnon2015,Olawore2018}.
Such data could also allow improved analysis of HIV risk factors
beyond associations with prevalent infection as in \sref{model.par.fsw}.
%---------------------------------------------------------------------------------------------------
\paragraph{Duration Data}
This thesis also highlights the importance of quantifying ``durations''
--- for both activity groups and sexual partnerships.
As noted in \sref{model.par.turn.act}~and~\ref{model.par.pnum.adj},
quantifying these durations requires
careful design of survey questions and interpretation of responses, to account for:
left censoring (\eg whether a partnership started before the recall period),
right censoring (\eg whether a partnership is ongoing), and
possible ``gaps'' within reported durations
(\eg whether a respondent ever stopped selling sex, for how long, etc.).
\citet{Burington2010} review considerations for estimating sexual partnership durations, while
\citet{Fazito2012} review similar considerations for key populations.
\clearpage % TEMP
%===================================================================================================
\subsection{Model Design \& Parameterization \vs Calibration}\label{conc.future.cal}
% TODO: (?) something about data driven machine learing \vs mechanistic modelling, Baker2018
It has been argued that model calibration can overcome
suboptimal model structure and/or parameterization, implying that
many of the considerations above may be inconsequential.
A related argument is that modellers should avoid ``non-identifiability'' among parameters,
whereby multiple parameters have similar roles, such that
changes in one parameter can be compensated by opposite changes in another parameter, and
inference on parameter values becomes difficult.%
\footnote{An analogous problem in statistical models is collinearity \cite{Harrell2015}.}
That is, a simple calibrated model with a few generic parameters is desirable.
In the context of HIV transmission modelling,
I respectfully dispute these arguments for the following three reasons.
\par
First, the number of available HIV calibration targets is typically small,
and often similar to the number of uncertain model parameters
--- \ie much smaller than the ``15:1'' rule for classic statistical inference \cite{Harrell2015}.
Moreover, the majority of calibration targets, including HIV prevalence and cascade data,
reflect cumulative measures for recent years,
which could be reproduced by a wide range of simulations.
Indeed, prior work has shown how diverse model structures can fit the same data similarly well,
but yield varied model outputs \cite{Hontelez2013,Eaton2015,Bernard2017,Suen2017}.%
\footnote{Although sensitivity analyses are often used to quantify the influence of
  specific model parameters on model outputs \cite{Blower1994,Johnson2016cc},
  it could be interesting to similarly quantify the influence of
  specific calibration targets on inferred parameter values.}
Thus, the ability of a model to reproduce these limited data
is often a necessary but not sufficient condition for a plausible model.
\par
Second, uncertainty in model outputs due to non-identifiability of parameters
can be desirable for transparency.
Unless the goal of model calibration is to infer \emph{individual} parameter values,
it is often sufficient to identify \emph{combinations} of parameter values
which yield plausible epidemics.
Selecting parameters to combine and/or fix may reduce the spread of model outputs,
but this can imply a false degree of certainty.
That is, simpler models may make ``fewer'' assumptions, but each assumption is often ``bigger''.
For example, a model may include no risk stratification and a single partnership type,
reflecting two (big) assumptions; whereas
another model may include 8 risk groups and 4 partnership types, each with unique parameters,
reflecting numerous (smaller) assumptions.
Additional model complexity is often added based on the particular research question
using new data or expert knowledge.
Thus, I would argue that \emph{simplifying} assumptions
should require stronger justification than \emph{complicating} assumptions.
\par
Third, I and others have shown how modelling assumptions can engender
hidden biases in model outputs which may/not be addressed via model calibration
\cite{Hontelez2013,Johnson2016mf,Bernard2017,Knight2020}.
Such biases are most likely to affect
the relative contributions of various transmission pathways,
for which fewer calibration targets are often available.
For example, there may be estimates of overall HIV prevalence,
but not HIV prevalence ratios among groups.
The risk of bias is amplified when a model designed for one research question
is repurposed to answer other questions,
without necessarily updating the model structure and parameterization.
Thus, models exploring a range of intervention strategies
should incorporate the best available knowledge about existing prevention gaps,
while the potential harms of misjudging these gaps should be kept top-of-mind.
