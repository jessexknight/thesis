Compartmental models of HIV transmission
continue to guide epidemic response in numerous ways and at multiple scales
\cite{Stover2021,Phillips2022,TenBrink2022}.
Such models must make simplifying assumptions due to time and data constraints.
However, prior work has shown that these simplifying assumptions
can sometimes influence key model outputs
\cite{Hontelez2013,Johnson2016mf,Suen2017,Knight2020}.
Thus, this thesis sought to explore the overarching research question:
\begin{quote}
  How do modelling assumptions influence outputs of compartmental HIV transmission models?
\end{quote}
Modelling assumptions were conceptualized to include:
the design of model structure, such as which risk groups are explicitly represented;
the parameterization of model inputs, such as how potential biases in data are adjusted; and
the implementation of the model, such as which equations are used to define the force of infection.
\par
I explored these assumptions in the context of heterosexual HIV transmission within Eswatini,
focusing on assumptions related to sex work.
The key results of each chapter can be summarized as follows:
\newcommand{\chpar}[1]{\paragraph{Chapter~\ref{#1}: \nameref*{#1}}}
\chpar{sr}
Among 94 compartmental HIV transmission models
applied to assess ART prevention impacts:
(2.1) approximately 2/3 and 2/5 stratified the modelled population
by sexual activity and key populations, respectively, while
approximately 1/3 and 1/4 considered
any turnover and differential ART cascade among risk groups, respectively;
(2.2) model-estimated ART prevention impacts were generally
lower when risk group turnover was considered, and
higher when one or more cascade steps were prioritized to key populations; and
(2.3) no models considered reduced cascade among key populations.
\chpar{model}
(3.1) I proposed several incremental changes to conventional parameterization steps, including:
a data-driven approach to stratifying higher \vs lower risk FSW,
an adjustment for reporting bias in partner numbers data using polling booth data,
an adjustment for bias due to partnership duration in partner numbers data,
an adjustment for censoring in risk group turnover/duration data, and
a balancing approach to support more flexible ``log-linear'' mixing patterns;
(3.2) I developed and calibrated a model of heterosexual HIV transmission in Eswatini
that integrated these changes and can be applied to downstream research questions.
\chpar{foi}
(4.1) I critically reviewed existing approaches to
modelling HIV transmission via sexual partnerships within compartmental models,
including their implicit assumptions, data needs, strengths, and limitations;
(4.2) I developed a new approach --- the \emph{Effective Partnerships Adjustment} ---
which overcomes many of the limitations of existing approaches; and
(4.3) through comparison with the new approach, I inferred that
some existing approaches might be systematically
overestimating the impact of prevention in longer partnerships, and
underestimating the impact of prevention in shorter partnerships.
\chpar{art}
(5.1) I showed how ART prevention impacts could be substantially reduced
if higher risk groups are ``left behind'' during scale-up
--- \ie filling the research gap identified in (2.3); and
(5.2) that this reduction increases with the higher risk group
population size, relative incidence, and turnover among their predominant partner group.
\par
Taken together, these results suggest that existing compartmental HIV transmission models
may underestimate the importance of prioritizing resources to
populations at highest risk of HIV acquisition and/or transmission.
Such populations include classically defined key populations like female sex workers,
but also emergent reconceptualizations of ``higher risk'' contexts and periods,
such as transactional sex among young women \cite{Wamoyi2016,Cheuk2020},
and highly mobile populations \cite{Akullian2017}.
Indeed, as HIV treatment and prevention programs
increasingly meet the needs of well-studied populations,
efforts to characterize and address residual risks at the peripheries of program reach
will be needed to fully prevent \emph{all} new infections.
\par
This chapter aims to place these results in context with respect to
broader trends in HIV epidemiology and response,
and offer specific recommendations for future work.
